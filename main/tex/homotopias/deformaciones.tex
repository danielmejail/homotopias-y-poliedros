\theoremstyle{plain}
\newtheorem{teoExtensionDeHomotopiasYDeformaciones}{Teorema}[section]
\newtheorem{coroExtensionDeHomotopiasRetractoEquivaleARetractoDebil}%
	[teoExtensionDeHomotopiasYDeformaciones]{Corolario}

\theoremstyle{remark}
\newtheorem{obsDeformableEquivaleATieneInversaHomotopicaADerecha}%
	{Observaci\'{o}n}[section]

%-------------

Sea $X$ un espacio topol\'{o}gico. En la secci\'{o}n anterior consideramos
subespacios $A\subset X$ tales que la inclusi\'{o}n $\inc[A]:\,A\rightarrow X$
admite una inversa a izquierda en la categor\'{\i}a de espacios topol\'{o}gicos
y funciones continuas, o bien en la categor\'{\i}a homot\'{o}pica. Dado un
subespacio $X'\subset X$, una \emph{deformaci\'{o}n} de $X'$ en (dentro de)
$X$ es una homotop\'{\i}a $D:\,X'\times\intervalo\rightarrow X$ tal que
$D(x',0)=x'$ para todo $x'\in X'$. Si $D\big(X'\times\{1\}\big)\subset A$
para cierto subespacio $A$ de $X$, se dice que $X'$ es \emph{deformable en %
$X$ en $A$}. Si $X'=X$ se dice, simplemente, que $X$ es deformable en $A$.

\begin{obsDeformableEquivaleATieneInversaHomotopicaADerecha}%
	\label{obs:deformableequivaleatieneinversahomotopicaaderecha}
	Sea $X$ un espacio topol\'{o}gico y sea $A\subset X$ un subespacio.
	Entonces $X$ es deformable en $A$, si y s\'{o}lo si la inclusi\'{o}n
	$\inc[A]:\,A\rightarrow X$ admite una inversa homot\'{o}pica
	\emph{a derecha}. Veamos que esto es as\'{\i}. Si existe
	$f:\,X\rightarrow A$ y una homotop\'{\i}a
	$F:\,X\times\intervalo\rightarrow X$ tal que
	\begin{align*}
		F(x,0) & \,=\,x \quad\text{y} \\
		F(x,1) & \,=\,\inc[A]\circ f(x)
	\end{align*}
	%
	para todo $x\in X$ (existe una inversa a derecha de $\inc[A]$ en la
	categor\'{\i}a homot\'{o}pica), entonces $F$ es una deformaci\'{o}n
	de $X$ (dentro de $X$) y
	\begin{align*}
		F\big(X\times\{1\}\big) & \,=\,\inc[A]\circ f(X) \,\subset\,A
	\end{align*}
	%
	con lo que, por definici\'{o}n, $X$ es deformable en $A$.
	Rec\'{\i}procamente, si $X$ es deformable en $A$ y
	$D:\,X\times\intervalo\rightarrow X$ es una deformaci\'{o}n de $X$
	(una homotop\'{\i}a que cumple $D(x,0)=x$ en $X$) tal que
	$D\big(X\times\{1\}\big)\subset A$, entonces existe una \'{u}nica
	funci\'{o}n (de conjuntos) $f:\,X\rightarrow A$ tal que
	\begin{align*}
		\inc[A]\circ f(x) & \,=\,D(x,1)
	\end{align*}
	%
	para todo $x\in X$. Como $A$ es un subespacio, $f$ es continua.
	Como $D$ es una deformaci\'{o}n de $X$, $D$ es una homotop\'{\i}a de
	la identidad $\id[X]$ en $\inc[A]\circ f$, es decir, $\inc[A]$
	admite una inversa a derecha en la categr\'{\i}a homot\'{o}pica.
\end{obsDeformableEquivaleATieneInversaHomotopicaADerecha}

En la secci\'{o}n anterior consideramos dos tipos de retractos: aquellos
subespacios $A\subset X$ para los cuales existe una funci\'{o}n
$r:\,X\rightarrow A$ tal que $r\circ\inc[A]=\id[A]$ y aquellos subespacios
para los cuales existe una funci\'{o}n $r:\,X\rightarrow A$ tal que
$r\circ\inc[A]\simeq\id[A]$. En esta secci\'{o}n, dado un subespacio
$A\subset X$, nos preguntamos si existe una funci\'{o}n $f:\,X\rightarrow A$
tal que $\inc[A]\circ f\simeq\id[X]$, es decir, si la inclusi\'{o}n
$\inc[A]:\,A\rightarrow X$ admite una inversa homot\'{o}pica a derecha.
Notemos que el problema de la existencia de inversas a derecha de $\inc[A]$ en
la categor\'{\i}a de espacios topol\'{o}gicos y funciones continuas es trivial,
ya que, si $\inc[A]\circ f=\id[X]$, entonces $\inc[A]$ es sobre y $A=X$.

Dado un subespacio $A\subset X$, se dice que $A$ es un \emph{retracto por %
deformaci\'{o}n d\'{e}bil de $X$}, si $\inc[A]$ es una equivalencia
homot\'{o}pica: esto quiere decir que existen $f,r:\,X\rightarrow A$ tales que
$\inc[A]\circ f\simeq\id[X]$ y $r\circ\inc[A]\simeq\id[A]$ (necesariamente,
$r\simeq f$). Seg\'{u}n lo mencionado en la observaci\'{o}n
\ref{obs:deformableequivaleatieneinversahomotopicaaderecha},
$\inc[A]\circ f\simeq\id[X]$ equivale a que $X$ sea deformable en $A$ y
$r\circ\inc[A]\simeq\id[A]$ significa que $A$ es un retracto d\'{e}bil de $X$.
(Un retracto por deformaci\'{o}n d\'{e}bil es un retracto d\'{e}bil que es,
adem\'{a}s, deformaci\'{o}n).

Sea $A\subset X$ un subespacio. Si $A$ es un retracto de $X$ (existe
$r:\,X\rightarrow A$ tal que $r\circ\inc[A]=\id[A]$) y la inclusi\'{o}n
admite una inversa homot\'{o}pica a derecha, $f:\,X\rightarrow A$, entonces
decimos que $A$ es un \emph{retracto por deformaci\'{o}n de $X$}. Notemos que
$\inc[A]$ es una equivalencia homot\'{o}pica (y, necesariamente, $r\simeq f$),
con lo cual todo retracto por deformaci\'{o}n es un retracto por
deformaci\'{o}n d\'{e}bil. (Un retracto por deformaci\'{o}n es un retracto que
es deformaci\'{o}n).

Un \emph{retracto por deformaci\'{o}n fuerte} de un espacio $X$ es un
subespacio $A\subset X$ que es retracto y existe una funci\'{o}n
$f:\,X\rightarrow A$ tal que $\inc[A]\circ f\simeq\id[X]\,\rel{A}$.

para el cual existe una retracci\'{o}n
$r:\,X\rightarrow A$ tal que $\inc[A]\circ r \simeq\,\rel{A}$. En particular,
tomando $f=r$, se ve que todo retracto por deformaci\'{o}n fuerte es un
retracto por deformaci\'{o}n.


En estas definiciones, podemos asumir que $r=f$. En el caso de la primera
definici\'{o}n, si $\inc[A]\circ f\simeq\id[X]$ y $r\circ\inc[A]\simeq\id[A]$,
entonces
\begin{align*}
	\clase{\inc[A]}\,\clase{f} & \,=\,\clase{\id[X]}\quad\text{y} \\
	\clase{r}\,\clase{\inc[A]} & \,=\,\clase{\id[A]}
	\text{ .}
\end{align*}
%
Componiendo y cancelando, se deduce que $\clase{f}=\clase{r}$, es decir,
$f\simeq r$. Por lo tanto, $r$ verifica $\inc[A]\circ r\simeq\id[X]$,
tambi\'{e}n. Con respecto a la segunda definici\'{o}n el mismo argumento
muestra que la retracci\'{o}n $r$ verifica la condici\'{o}n sobre $f$.
En la definici\'{o}n de retracto por deformaci\'{o}n fuerte, el argumento debe
ser distinto, pues la relaci\'{o}n de homotop\'{\i}a se asume relativa al
subespacio $A$. Aun as\'{\i}, si
\begin{align*}
	\inc[A]\circ f & \,\simeq\,\id[X]\,\rel{A}\quad\text{y} \\
	r\circ\inc[A] & \,=\,\id[A]
	\text{ ,}
\end{align*}
%
entonces $r$ verifica
\begin{align*}
	\inc[A]\circ r & \,=\,\inc[A]\circ r\circ\id[X]
		\,\simeq_{(\rel{A})}\,\inc[A]\circ r\circ\inc[A]\circ f
		\,=\,\inc[A]\circ f
	\text{ .}
\end{align*}
%
Pero entonces $r$ verifica la condici\'{o}n sobre $f$ y podemos asumir que
$f=r$ en este caso tambi\'{e}n.


%----------
Si en lugar de concentrarnos en las propiedades de un espacio miramos
lo que pasa con las funciones con propiedades como las de la retracci\'{o}n
de la inclusi\'{o}n,\dots

En la secci\'{o}n anterior consideramos dos tipos de retracciones:
aquellas que admiten una inversa continua a derecha y aquellas que admiten
una inversa homot\'{o}pica a derecha, aquellas que son, respectivamente,
inversas a izquierda de la inclusi\'{o}n de un subespacio en la categor\'{\i}a
de espacios topol\'{o}gicos y funciones continuas y aquellas que son
inversas a izquierda de la inclusi\'{o}n en la categor\'{\i}a homot\'{o}pica.

%----------

\begin{coroExtensionDeHomotopiasRetractoEquivaleARetractoDebil}
	Si $(X,A)$ tiene la propiedad de extensi\'{o}n de homotop\'{\i}as
	con erspecto a $A$, entonces $A$ es un retracto por deformaci\'{o}n
	d\'{e}bil de $X$, si y s\'{o}lo si es un retracto por deformaci\'{o}n.
\end{coroExtensionDeHomotopiasRetractoEquivaleARetractoDebil}

\begin{proof}
	Este corolario es consecuencia del teorema
	\ref{thm:extensiondehomotopiasretractoequivalearetractodebil}.
\end{proof}

\begin{teoExtensionDeHomotopiasYDeformaciones}%
	\label{thm:extensiondehomotopiasydeformaciones}
	Sea $X$ un espacio topol\'{o}gico y sea $A\subset X$ un subespacio
	cerrado tal que el par $(X\times\intervalo,L)$, donde
	\begin{align*}
		L & \,=\,(X\times\{0\})\,\cup\,(A\times\intervalo)\,\cup\,
			(X\times\{1\})
		\text{ ,}
	\end{align*}
	%
	tiene la propiedad de extensi\'{o}n de homotop\'{\i}as con respecto a
	$X$. Entonces $A$ es un retracto por deformaci\'{o}n de $X$, si y
	s\'{o}lo si es un retracto por deformaci\'{o}n fuerte.
\end{teoExtensionDeHomotopiasYDeformaciones}

\begin{proof}
	Supongamos que $A$ es un retracto por deformaci\'{o}n de $X$. Sea
	$r:\,X\rightarrow A$ una retracci\'{o}n de $X$ en $A$ y sea
	$F:\,X\times\intervalo\rightarrow X$ una homotop\'{\i}a de $\id[X]$
	en $\inc[A]\circ r$. Lo que hay que ver es que existe una
	homotop\'{\i}a relativa a $A$. Sea $G:\,L\times\intervalo\rightarrow X$
	la funci\'{o}n definida por
	\begin{equation}
		\label{eq:extensiondehomotopiasydeformaciones}
	\begin{aligned}
		G((x,0),t') & \,=\, x \quad\text{en }
			(X\times\{0\})\times\intervalo \text{ ,} \\
		G((x,t),t') & \,=\, F(x,(1-t')\,t) \quad\text{en }
			(A\times\intervalo)\times\intervalo \quad\text{y} \\
		G((x,1),t') & \,=\,F(\inc[A]\circ r(x),1-t') \quad\text{en }
			(X\times\{1\})\times\intervalo
		\text{ .}
	\end{aligned}
	\end{equation}
	%
	Como $A$ es cerrado en $X$, $G$ es continua. Ahora bien, si
	$(x,t)\in L$, a tiempo $t'=0$,
	\begin{align*}
		G((x,t),0) & \,=\,F(x,t)
		\text{ ,}
	\end{align*}
	%
	pues
	\begin{equation}
		\label{eq:extensiondehomotopiasydeformacionesatiempocero}
	\begin{aligned}
		G((x,0),0) & \,=\,x\,=\,F(x,0) \text{ ,} \\
		G((x,t),0) & \,=\,F(x,t) \quad\text{y} \\
		G((x,1),0) & \,=\,F(\inc[A]\circ r(x),1) \,=\,
			(\inc[A]\circ r)\circ (\inc[A]\circ r) (x) \\
		& \,=\,\inc[A]\circ r(x) \,=\, F(x,1)
		\text{ .}
	\end{aligned}
	\end{equation}
	%
	Sea $f_{0}=G|_{L\times\{0\}}$ y sea $f_{1}=G|_{L\times\{1\}}$.
	Entonces las igualdades
	\eqref{eq:extensiondehomotopiasydeformacionesatiempocero},
	implican que $g:\,(X\times\intervalo)\times\{0\}$ definida por
	\begin{align*}
		g((x,t),0) & \,=\, F(x,t)	
	\end{align*}
	%
	es una extensi\'{o}n de $f_{0}$. Por la propiedad de extensi\'{o}n de
	homotop\'{\i}as (ver la observaci\'{o}n
	\ref{obs:extensiondehomotopias}), $f_{1}$ admite una extensi\'{o}n a
	$(X\times\intervalo)\times\{1\}$. Sea
	\begin{align*}
		G' & \,:\,(X\times\intervalo)\times\{1\}\,\rightarrow\,X
	\end{align*}
	%
	una extensi\'{o}n de $f_{1}=G|_{L\times\{1\}}$ y sea
	\begin{align*}
		H & \,:\,X\times\intervalo\,\rightarrow\,X
	\end{align*}
	%
	La funci\'{o}n
	\begin{align*}
		H(x,t) & \,=\,G'((x,t),1)
		\text{ .}
	\end{align*}
	%
	Entonces $H$ verifica
	\begin{equation}
		\label{eq:extensiondehomotopiasydeformacionesfuerte}
	\begin{aligned}
		H(x,0) & \,=\,G'((x,0),1)\,=\,G((x,0),1) \,=\,x
			\quad\text{si }x\in X \text{ ,} \\
		H(x,1) & \,=\,G'((x,1),1)\,=\,G((x,1),1) \\
		& \,=\,F(\inc[A]\circ r(x),0) \,=\,\inc[A]\circ r(x)
			\quad\text{si }x\in X\quad\text{y} \\
		H(x,t) & \,=\,G'((x,t),1)\,=\,G((x,t),1) \\
		& \,=\,F(x,0)\,=\,x
			\quad\text{si }x\in A,\,t\in\intervalo
		\text{ .}
	\end{aligned}
	\end{equation}
	%
	Entonces $H$ es una homotop\'{\i}a de $\id[X]$ en $\inc[A]\circ r$.
	relativa a $A$.
\end{proof}
