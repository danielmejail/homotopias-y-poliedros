\theoremstyle{plain}
\newtheorem{teoDeLaEsferaAlDisco}{Teorema}[section]

\theoremstyle{remark}
\newtheorem{obsHomotopiaEsEquivalencia}[teoDeLaEsferaAlDisco]{Observaci\'{o}n}
\newtheorem{obsComposicionesDeHomotopicasSonHomotopicas}[teoDeLaEsferaAlDisco]%
	{Observaci\'{o}n}
\newtheorem{obsEnContractilSonHomotopicos}[teoDeLaEsferaAlDisco]%
	{Observaci\'{o}n}
\newtheorem{obsDeLaEsferaAlDisco}[teoDeLaEsferaAlDisco]{Observaci\'{o}n}

%-------------

\begin{quote}
	``Many of the computable functors, because they are computable, are
	invariant under continuous deformation. Therefore they cannot
	distiguish between spaces (or maps) that can be continuously deformed
	from one to the other; the most that can be hoped for from such
	functors is that they characterize the space (or map) up to
	continuous deformation.''
\end{quote}

Un \emph{par topol\'{o}gico} es un par de espacios $(X',X)$ junto con
una funci\'{o}n continua $f:\,X\rightarrow X'$ entre ellos (un par
topol\'{o}gico es una funci\'{o}n continua). Un morfismo de pares de
$f:\,X\rightarrow X'$ en $g:\,Y\rightarrow Y'$ consiste en un par de funciones
continuas $h:\,X\rightarrow Y$ y $h':\,X'\rightarrow Y'$ tales que
$h'\circ f= g\circ h$. Nos concentraremos en pares $(X,A)$, donde $A\subset X$
es un subespacio (y $f:\,A\rightarrow X$ es la inclusi\'{o}n).

Sean $(X,A)$ e $(Y,B)$ dos pares topol\'{o}gicos ($A\subset X$ y $B\subset Y$)
y sean $f_{0},f_{1}:\,(X,A)\rightarrow (Y,B)$ morfismos de pares que coinciden
en alg\'{u}n subespacio $X'\subset X$ (independiente de $A$ y de $B$).
Decimos que $f_{0}$ y $f_{1}$ son \emph{homot\'{o}picas relativas a $X'$},
si existe un morfismo de pares
\begin{align*}
	F & \,:\,(X\times\intervalo,A\times\intervalo)\,\rightarrow\,
		(Y,B)
\end{align*}
%
(es decir, una homotop\'{\i}a $X\times\intervalo\rightarrow Y$ tal que
$F(A\times\intervalo)\subset B$ (los puntos de $A$, que van a parar a $B$,
permanecen en $B$ durante la deformaci\'{o}n)) tal que
\begin{align*}
	F(x,0) & \,=\,f_{0}(x) \text{ ,} \\
	F(x,1) & \,=\,f_{1}(x) \quad\text{y} \\
	F(x,t) & \,=\,f_{0}(x)\,=\,f_{1}(x) \quad\text{si }x\in X'
	\text{ .}
\end{align*}
%
Denotamos esta relaci\'{o}n entre $f_{0}$ y $f_{1}$ por
$f_{0}\simeq f_{1}\,\rel{X'}$ y decimos que $F$ es una homotop\'{\i}a (de pares)
relativa a $X'$ de $f_{0}$ en $f_{1}$.

Dado $t\in\intervalo$, sea $h_{t}:\,(X,A)\rightarrow (X,A)\times\intervalo$ la
funci\'{o}n
\begin{align*}
	h_{t}(x) & \,=\,(x,t)
	\text{ .}
\end{align*}
%
Si $F$ es una homotop\'{\i}a relativa a $X'$ de $f_{0}$ en $f_{1}$, entonces
\begin{align*}
	F\circ h_{0} & \,=\,f_{0} \text{ ,} \\
	F\circ h_{1} & \,=\,f_{1} \quad\text{y} \\
	F\circ h_{t}|_{X'} & \,=\,f_{0}|_{X'} \,=\,f_{1}|_{X'}
	\text{ ,}
\end{align*}
%
para todo $t\in\intervalo$. Una \emph{familia a un par\'{a}metro} es una
familia de funciones $\{f_{t}\}_{t\in\intervalo}$ parametrizada por el
intervalo unitario (o $\bb{R}$, o\dots). En el contexto de pares
topol\'{o}gicos, se requiere que, para todo $t$, la funci\'{o}n $f_{t}$
sea un morfismo (continuo) de pares $f_{t}:\,(X,A)\rightarrow (Y,B)$ (dominio
y codominio independientes de $t$). Una familia a un par\'{a}metro
$\{f_{t}:\,(X,A)\rightarrow (Y,B)\}_{t\in\intervalo}$ se dice \emph{continua},
si el morfismo de pares
\begin{align*}
	\big((x,t)\mapsto f_{t}(x)\big) & \,:\,
		(X\times\intervalo,A\times\intervalo)\,\rightarrow (Y,B)
\end{align*}
%
es continuo. En tal caso, la aplicaci\'{o}n $F(x,t)=f_{t}(x)$ define una
homotop\'{\i}a de $f_{0}$ en $f_{1}$. Toda familia a un par\'{a}metro
tiene asociada una funci\'{o}n
\begin{align*}
	(t\mapsto f_{t}) & \,:\,\intervalo\,\rightarrow\,
		\continuas{(X,A)}{(Y,B)}
	\text{ .}
\end{align*}
%
Si $\intervalo$ tiene la topolog\'{\i}a usual y a $\continuas{(X,A)}{(Y,B)}$
se le da la topolog\'{\i}a compacto abierta, entonces la funci\'{o}n
asociada a una familia a un par\'{a}metro continua es continua.
Rec\'{\i}procamente, si $X$ es localmente compacto Hausdorff, entonces
dada una funci\'{o}n continua
\begin{math}
	\phi:\,\intervalo\rightarrow\continuas{(X,A)}{(Y,B)}
\end{math}~,
la familia a un par\'{a}metro $\{\phi(t)\}_{t}$ es continua y define una
homotop\'{\i}a de $\varphi(0)$ en $\varphi(1)$.

\begin{obsHomotopiaEsEquivalencia}\label{obs:homotopiaesequivalencia}
	La relaci\'{o}n de homotop\'{\i}a relativa a un subespacio es de una
	relaci\'{o}n de equivalencia. Si $f:\,(X,A)\rightarrow (Y,B)$ es
	continua y $X'\subset X$ es un subespacio, entonces la homotop\'{\i}a
	constante $F(x,t)=f(x)$ realiza $f\simeq f\,\rel{X'}$; si
	$f_{0},f_{1}:\,(X,A)\rightarrow (Y,B)$ y $f_{0}\simeq f_{1}\,\rel{X'}$
	y $F$ es una homotop\'{\i}a de $f_{0}$ en $f_{1}$ relativa a $X'$,
	entonces la homotop\'{\i}a inversa $(x,t)\mapsto F(x,1-t)$ realiza
	$f_{1}\simeq f_{0}\,\rel{X'}$; finalmente, si
	$f_{2}:\,(X,A)\rightarrow (Y,B)$ es un tercer morfismo de pares y
	$G$ es una homotop\'{\i}a de $f_{1}$ en $f_{2}$ relativa a $X'$,
	entonces la funci\'{o}n
	\begin{align*}
		H(x,t) & \,=\,
			\begin{cases}
				F(x,2t) & \quad\text{si }t\leq 1/2 \\
				G(x,2t-1) & \quad\text{si }t\geq 1/2
			\end{cases}
	\end{align*}
	%
	es continua, por ser continua en cada uno de los cerrados en donde
	es definida, verifica $F(A\times\intervalo)\subset B$ y es una
	homotop\'{\i}a de $f_{0}$ en $f_{2}$ relativa a $X'$.

	Dados pares  $(X,A)$ e $(Y,B)$ definimos el \emph{conjunto de clases %
	de homotop\'{\i}a de $(X,A)$ en $(Y,B)$ relativas a $X'$} como el
	conjunto de clases de equivalencia respecto de esta relaci\'{o}n.
	Denotamos este conjunto por $\homotopicas[X']{X,A}{Y,B}$ y la clase
	de un morfismo $f:\,(X,A)\rightarrow (Y,B)$ por $\clase[X']{f}$.
\end{obsHomotopiaEsEquivalencia}

\begin{obsComposicionesDeHomotopicasSonHomotopicas}%
	\label{obs:composicionesdehomotopicassonhomotopicas}
	Sean $f_{0},f_{1}:\,(X,A)\rightarrow (Y,B)$ y sean
	$g_{0},g_{1}:\,(Y,B)\rightarrow (Z,C)$. Sea $X'\subset X$ y sea
	$Y'\subset Y$. Supongamos que $f_{0}|_{X'}=f_{1}|_{X'}$, que
	$g_{0}|_{Y'}=g_{1}|_{Y'}$ y que $f_{0}(X')=f_{1}(X')\subset Y'$.
	Si $f_{0}\simeq f_{1}\,\rel{X'}$ y $g_{0}\simeq g_{1}\,\rel{Y'}$
	entonces
	\begin{align*}
		g_{0}\circ f_{0} & \,\simeq\, g_{1}\circ f_{1}\quad\rel{X'}
		\text{ .}
	\end{align*}
	%
	Si $F$ realiza la homotop\'{\i}a entre $f_{0}$ y $f_{1}$ y $G$ realiza
	la homotop\'{\i}a entre $g_{0}$ y $g_{1}$, entonces
	\begin{align*}
		g_{0}\circ F & \,:\,(X\times\intervalo,A\times\intervalo)
			\,\rightarrow\,(Y,B)\,\rightarrow\,(Z,C)
	\end{align*}
	%
	es una homotop\'{\i}a de $g_{0}\circ f_{0}$ en $g_{0}\circ f_{1}$
	relativa a $X'$ y
	\begin{align*}
		G\circ (f_{1}\times\id[\intervalo]) & \,:\,
			(X\times\intervalo,A\times\intervalo)\,\rightarrow\,
			(Y\times\intervalo,B\times\intervalo)\,\rightarrow\,
			(Z,C)
	\end{align*}
	%
	es una homotop\'{\i}a de $g_{0}\circ f_{1}$ en $g_{1}\circ f_{1}$
	relativa a $f_{1}^{-1}(Y')$. Pero $f_{1}(X')\subset Y'$ implica que
	esta homotop\'{\i}a tambi\'{e}n es relativa a $X'$. Por la
	observaci\'{o}n \ref{obs:HomotopiaEsEquivalencia}, la relaci\'{o}n de
	homotop\'{\i}a es transitiva y se deduce que $g_{0}\circ f_{0}$ es
	homot\'{o}pica a $g_{1}\circ f_{1}$ relativa a $X'$.
\end{obsComposicionesDeHomotopicasSonHomotopicas}

Teniendo en cuenta estas observaciones, podemos definir la
\emph{categor\'{\i}a homot\'{o}pica de pares} como la categor\'{\i}a cuyos
objetos son pares de espacios topol\'{o}gicos, al igual que en la
categor\'{\i}a de pares, $(X,A)$, $(Y,B)$, pero cuyos morfismos son las clases
de homotop\'{\i}a de morfismos de pares relativas al conjunto vac\'{\i}o,
$\homotopicas{X,A}{Y,B}=\homotopicas[\varnothing]{X,A}{Y,B}$. La clase de un
morfismo de pares relativa al conjunto vac\'{\i}o la denotamos simplemente
$\clase{f}$. Diremos que un diagrama
\begin{center}
	\begin{tikzcd}
		(X,A) \arrow[r] \arrow[d] & (Y,B) \arrow[d] \\
		(X',A') \arrow[r] & (Y'B')
	\end{tikzcd}
\end{center}
\emph{conmuta m\'{o}dulo homotop\'{\i}as}, si el diagrama correspondiente en
la categor\'{\i}a homoto\'{o}pica de pares es conmutativo. Una
\emph{equivalencia homot\'{o}pica} es un morfismo de pares
$f:\,(X,A)\rightarrow (Y,B)$ tal que su clase de homotop\'{\i}a
$\clase{f}\in\homotopicas{X,A}{Y,B}$ sea un isomorfismo en la categor\'{\i}a
homot\'{o}pica, es decir, existiese $g:\,(Y,B)\rightarrow (X,B)$ tal que
\begin{align*}
	\clase{g\circ f} & \,=\,\clase{g}\circ\clase{f} \,=\,
		\clase{\id[(X,A)]} \quad\text{y} \\
	\clase{f\circ g} & \,=\,\clase{f}\circ\clase{g} \,=\,
		\clase{\id[(Y,B)]}
	\text{ .}
\end{align*}
%
Tambi\'{e}n diremos que $\clase{f}$ es una equivalencia homot\'{o}pica y que
$g$, o que $\clase{g}$ es su \emph{inversa homot\'{o}pica}. Decimos que dos
pares $(X,A)$ e $(Y,B)$ \emph{tienen el mismo tipo homot\'{o}pico}, si
existe una equivalencia homot\'{o}pica entre ellos, es decir, si son isomorfos
en la categor\'{\i}a homot\'{o}pica.

Un espacio topol\'{o}gico $X$ se dice \emph{contr\'{a}ctil}, si la identidad
$\id[X]$ es homot\'{o}pica a una funci\'{o}n $X\rightarrow X$ constante, es
decir, si existe $x_{0}\in X$ tal que $\id[X]\simeq (x\mapsto x_{0})$. Una
\emph{contracci\'{o}n} de $X$ en $x_{0}\in X$ es una homotop\'{\i}a
de $F$ de $\id[X]$ en a funci\'{o}n constante $x\mapsto x_{0}$.

\begin{obsEnContractilSonHomotopicos}\label{obs:encontractilsonhomotopicos}
	Sea $Y$ un espacio topol\'{o}gico contr\'{a}ctil. Sea $y_{0}\in Y$
	y sea $c:\,Y\rightarrow Y$ la funci\'{o}n constante $c(y)=y_{0}$.
	Supongamos que $\id[Y]\simeq c$. Si $f_{0},f_{1}:\,X\rightarrow Y$
	son funciones continuas, entonces
	\begin{align*}
		f_{0} & \,=\,\id[Y]\circ f_{0} \,\simeq\,c\circ f_{0}
			\quad\text{y} \\
		f_{1} & \,=\,\id[Y]\circ f_{1} \,\simeq\,c\circ f_{1}
		\text{ .}
	\end{align*}
	%
	Pero $c\circ f_{0}=c\circ f_{1}$. Por transitividad,
	$f_{0}\simeq f_{1}$. En definitiva, el conjunto $\homotopicas{X}{Y}$
	consiste en una \'{u}nica clase cualquiera sea el espacio $X$, dos
	funciones continuas de $X$ en $Y$ son homot\'{o}picas.

	Si definimos $\tilde{c}:\,Y\rightarrow \{y_{0}\}$ (si identificamos $c$
	con la funci\'{o}n cuyo codominio es el conjunto puntual $\{y_{0}\}$),
	entonces $\tilde{c}$ es una equivalencia homot\'{o}pica: su inversa
	homot\'{o}pica est\'{a} dada por la inclusi\'{o}n
	$\inc:\,\{y_{0}\}\rightarrow Y$. Pero entonces $Y$ tiene el tipo
	homot\'{o}pico del punto $\{y_{0}\}$. En particular,
	\begin{align*}
		\homotopicas{X}{Y} & \,\simeq\,\homotopicas{X}{\{y_{0}\}}
	\end{align*}
	%
	v\'{\i}a
	\begin{math}
		\clase{f}\mapsto
			\clase{\tilde{c}}\circ\clase{f}=
			\clase{\tilde{c}\circ f}
	\end{math}~.
	Pero el conjunto $\homotopicas{X}{\{y_{0}\}}$ posee una \'{u}nica
	clase, la clase de la \'{u}nica funci\'{o}n $X\rightarrow \{y_{0}\}$.
	Concluimos, de esta manera tambi\'{e}n, que todo par de funciones
	$f_{0},f_{1}:\,X\rightarrow Y$ cuyo codominio es contr\'{a}ctil son
	homot\'{o}picas.

	Notemos que la noci\'{o}n de ser contr\'{a}ctil equivale a tener el
	tipo homot\'{o}pico del punto: ya vimos que
	$\id[\{y_{0}\}]=\tilde{c}\circ\inc$ y que
	$\id[Y]\simeq\inc\circ\tilde{c}$, con lo que $Y$ y el conjunto puntual
	$\{y_{0}\}$ tienen el mismo tipo homot\'{o}pico. Rec\'{\i}procamente,
	si $Y$ tiene el tipo homot\'{o}pico de un conjunto puntual $\{y_{0}\}$,
	entonces existen $f:\,Y\rightarrow \{y_{0}\}$ y
	$g:\,\{y_{0}\}\rightarrow Y$ tales que $f\circ g=\id[\{y_{0}\}]$
	(pues no hay otra funci\'{o}n $\{y_{0}\}\rightarrow\{y_{0}\}$) y
	$g\circ f\simeq\id[Y]$. En particular, la segunda equivalencia
	implica que $Y$ es contr\'{a}ctil.
\end{obsEnContractilSonHomotopicos}

La existencia de homotop\'{\i}as est\'{a} relacionada con la posibilidad de
extender funciones.

\begin{teoDeLaEsferaAlDisco}\label{thm:delaesferaaldisco}
	Sea $p_{0}\in\esfera{n}$ un punto arbitrario de la esfera de
	dimensi\'{o}n $n$ ($n\geq 1$, $n\geq 0$) y sea
	$f:\,\esfera{n}\rightarrow Y$ una funci\'{o}n continua. Las
	siguientes afirmaciones son equivalentes:
	\begin{itemize}
		\item[(i)] $f$ es homot\'{o}pica a una constante;
		\item[(ii)] $f$ se puede extender de manera continua al
			disco $\disco{n+1}$;
		\item[(iii)] $f$ es homot\'{o}pica a una constante relativa
			al punto $\{p_{0}\}$.
	\end{itemize}
	%
\end{teoDeLaEsferaAlDisco}

La \'{u}ltima afirmaci\'{o}n es equivalente a que $f\simeq (x\mapsto p_{0})$
v\'{\i}a una homotop\'{\i}a que deje quieto el punto $p_{0}$.

\begin{proof}
	\emph{(i) implica (ii):} sea $F$ una homotop\'{\i}a de $f$ en una
	funci\'{o}n constante $c:\,\esfera{n}\rightarrow Y$ y sea $y_{0}\in Y$
	tal que $c(x)=y_{0}$ para todo $x\in\esfera{n}$. Sea
	$f':\,\disco{n+1}\rightarrow y$ la funci\'{o}n
	\begin{align*}
		f'(x) & \,=\,
			\begin{cases}
				y_{0} & \quad\text{si }
					0\leq |x|\leq 1/2 \\[10pt]
				F\Big(\dfrac{x}{|x|},2-2|x|\Big)
					& \quad\text{si }1/2\leq |x|\leq 1
			\end{cases}
		\text{ .}
	\end{align*}
	%
	Entonces $f'$ est\'{a} bien definida y es continua en $\disco{n+1}$.
	Adem\'{a}s, como $F(x,0)=f(x)$ para $x\in\esfera{n}$, concluimos que
	$f'$ es una extensi\'{o}n continua de $f$ al disco.

	\emph{(ii) implica (iii):} Si $f':\,\disco{n+1}\rightarrow Y$ es una
	extensi\'{o}n continua de $f$, la expresi\'{o}n
	\begin{align*}
		F(x,t) & \,=\,f'((1-t)\,x)
	\end{align*}
	%
	define una homotop\'{\i}a de $f$ en la funci\'{o}n constante
	$x\mapsto f'(0)$. Pero esta homotop\'{\i}a no es necesariamente
	relativa a un punto de la esfera. Entonces, dado $p_{0}\in\esfera{n}$,
	sea $F:\,\esfera{n}\times\intervalo\rightarrow Y$ la funci\'{o}n
	\begin{align*}
		F(x,t) & \,=\,f'((1-t)\,x+t\,p_{0})
		\text{ .}
	\end{align*}
	%
	Esta funci\'{o}n es continua, porque $f'$ lo es, y verifica:
	\begin{align*}
		F(x,0) & \,=\,f'(x) \text{ ,} \\
		F(x,1) & \,=\,f'(p_{0}) \quad\text{y} \\
		F(p_{0},t) & \,=\,f'(p_{0})
		\text{ .}
	\end{align*}
	%
	Entonces, como $f'|_{\esfera{n}}=f$, $F$ es una homotop\'{\i}a de
	$f$ en la funci\'{o}n constante $x\mapsto f(p_{0})$ relativa a
	$\{p_{0}\}$.
\end{proof}

\begin{obsDeLaEsferaAlDisco}\label{obs:delaesferaaldisco}
	De la observaci\'{o}n \ref{obs:encontractilsonhomotopicos} y del
	teorema \ref{thm:delaesferaaldisco}, se deduce que toda funci\'{o}n de
	$\esfera{n}$ en un espacio contr\'{a}ctil admite una extensi\'{o}n
	continua al disco $\disco{n+1}$.
\end{obsDeLaEsferaAlDisco}
