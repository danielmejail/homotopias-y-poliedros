\theoremstyle{definition}
\newtheorem{ejemploIdentidadDeEspacioEuclideo}{Ejemplo}[section]
\newtheorem{ejemploIdentidadDelIntervalo}[ejemploIdentidadDeEspacioEuclideo]%
	{Ejemplo}
\newtheorem{ejemploReflexionEnElDisco}[ejemploIdentidadDeEspacioEuclideo]%
	{Ejemplo}
\newtheorem{ejemploHomotopiaLinealEnUnConvexo}%
	[ejemploIdentidadDeEspacioEuclideo]{Ejemplo}
\newtheorem{ejemploElPeine}[ejemploIdentidadDeEspacioEuclideo]{Ejemplo}
\newtheorem{ejemploElPeineYElCuadrado}[ejemploIdentidadDeEspacioEuclideo]%
	{Ejemplo}
\newtheorem{ejemploPuntoDeformacionFuerteDeConvexos}%
	[ejemploIdentidadDeEspacioEuclideo]{Ejemplo}
\newtheorem{ejemploEsferaDeformacionFuerteDelEspacioSinUnPunto}%
	[ejemploIdentidadDeEspacioEuclideo]{Ejemplo}
\newtheorem{ejemploElPeineYElCuadradoNoEsRetracto}%
	[ejemploIdentidadDeEspacioEuclideo]{Ejemplo}
\newtheorem{ejemploElPuntoYElPeine}%
	[ejemploIdentidadDeEspacioEuclideo]{Ejemplo}

%-------------

\begin{ejemploIdentidadDeEspacioEuclideo}%
	\label{ejemplo:identidaddeespacioeuclideo}
	Sean $X=Y=\bb{R}^{n}$ y sean $f_{0}(x)=x$ y $f_{1}(x)=0$ para todo
	punto $x\in\bb{R}^{n}$. Sea
	$F:\,\bb{R}^{n}\times\intervalo\rightarrow\bb{R}^{n}$ la funci\'{o}n
	\begin{align*}
		F(x,t) & \,=\,(1-t)\,x
		\text{ .}
	\end{align*}
	%
	Entonces $F$ es una homotop\'{\i}a de $f_{0}=\id[\bb{R}^{n}]$ en la
	funci\'{o}n constante $f_{1}=0$ relativa al subespacio $\{0\}$.
\end{ejemploIdentidadDeEspacioEuclideo}

\begin{ejemploIdentidadDelIntervalo}\label{ejemplo:identidaddelintervalo}
	Si ahora $X=Y=\intervalo$, $f_{0}(t)=t$ y $f_{1}(t)=0$, para todo
	instante $t\in\intervalo$, y
	$F:\,\intervalo\times\intervalo\rightarrow\intervalo$ es la funci\'{o}n
	\begin{align*}
		F(t,t') & \,=\,(1-t')\,t
		\text{ ,}
	\end{align*}
	%
	entonces $F$ es una homotop\'{\i}a de $f_{0}=\id[\intervalo]$ en
	la funci\'{o}n constante $f_{1}=0$ relativa al subespacio $\{0\}$.
\end{ejemploIdentidadDelIntervalo}

\begin{ejemploReflexionEnElDisco}\label{ejemplo:reflexioneneldisco}
	Sean $X=Y=\disco{2}$ y sean $A=B=\esfera{1}$. Sea
	\begin{math}
		f_{0},f_{1}:\,(\disco{2},\esfera{1})\rightarrow
			(\disco{2},\esfera{1})
	\end{math}
	las funciones dadas por
	\begin{align*}
		f_{0}(re^{i\theta}) & \,=\,re^{i\theta} \\
		f_{1}(re^{i\theta}) & \,=\,re^{i(\theta+\pi)}
		\text{ ,}
	\end{align*}
	%
	es decir, $f_{0}=\id[\disco{2}]$ y $f_{1}$ es hacer medio giro, o bien
	reflejar un punto en el origen. Sean $F,F'$ las funciones definidas por
	\begin{align*}
		F(re^{i\theta},t) & \,=\,re^{i(\theta+t\pi)} \\
		F'(re^{i\theta},t) & \,=\,re^{i(\theta-t\pi)}
		\text{ .}
	\end{align*}
	%
	Entonces $f_{0}\simeq f_{1}\,\rel{\{0\}}$, tanto v\'{\i}a $F$ como
	v\'{\i}a $F'$.
\end{ejemploReflexionEnElDisco}

\begin{ejemploHomotopiaLinealEnUnConvexo}\label{ejemplo:contraccionenunconvexo}
	Sea $X$ un espacio topol\'{o}gico arbitrario y sea $Y\subset\bb{R}^{n}$
	un subespacio convexo. Sean $f_{0},f_{1}:\,X\rightarrow Y$ funciones
	continuas y sea $X'\subset X$ un subespacio en donde $f_{0}$ y $f_{1}$
	coinciden. Si $F:\,X\times\intervalo\rightarrow Y$ es la funci\'{o}n
	definida por
	\begin{align*}
		F(x,t) & \,=\,t\,f_{1}(x) + (1-t)\,f_{0}(x)
		\text{ ,}
	\end{align*}
	%
	entonces $F$ es una homotop\'{\i}a de $f_{0}$ en $f_{1}$ relativa a
	$X'$.
\end{ejemploHomotopiaLinealEnUnConvexo}

\begin{ejemploElPeine}\label{ejemplo:elpeine}
	Sea $Y\subset\bb{R}^{2}$ el subespacio del plano dado por
	\begin{align*}
		Y & \,=\,\big\{(x,y)\in\bb{R}^{n}\,:\,
			(0\leq y\leq 1 \wedge x=1/n)\vee
			(y=0\wedge 0\leq x\leq 1)\big\}
		\text{ .}
	\end{align*}
	%
	Si $F:\,Y\times\intervalo\rightarrow Y$ es la funci\'{o}n
	\begin{align*}
		F((x,y),t) & \,=\,(x,(1-t)\,y)
		\text{ ,}
	\end{align*}
	%
	entonces $F$ es una homotop\'{\i}a de $\id[Y]$ en la proyecci\'{o}n
	$\pi:\,(x,y)\mapsto (x,0)$ (relativa al subespacio
	$\{(x,0)\,:\,0\leq x\leq 1\}$). Como este subespacio es contr\'{a}ctil,
	por transitividad, se deduce que $Y$ es contr\'{a}ctil. En particular,
	si $c:\,Y\rightarrow Y$ es la funci\'{o}n constante
	$c(x,y)=(1,0)$, entonces $\id[Y]\simeq c$ y coinciden en $(1,0)$.
	Pero, como la topolog\'{\i}a de $Y$ es la de subespacio del plano,
	no existe una homotop\'{\i}a de $\id[Y]$ en $c$ relativa al
	punto $\{(1,0)\}$, es decir, que deje fijo el punto.
\end{ejemploElPeine}

\begin{ejemploElPeineYElCuadrado}\label{ejemplo:elpeineyelcuadrado}
	Sea $X=\intervalo^{2}\subset\bb{R}^{2}$ y sea $A$ el espacio del
	ejemplo \ref{ejemplo:elpeine}. Entonces $A$ y $X$ son contr\'{a}ctiles.
	De acuerdo con la observaci\'{o}n \ref{obs:encontractilsonhomotopicos},
	la inclusi\'{o}n $\inc:\,A\rightarrow X$ es una equivalencia
	homot\'{o}pica. En particular $A$ es un retracto d\'{e}bil de $X$.
	Pero $A$ no es un retracto de $X$, pues, por ejemplo, el punto
	$(0,1)$ posee puntos arbitrariamente cerca que debieran ser
	proyectados lejos.
\end{ejemploElPeineYElCuadrado}

\begin{ejemploPuntoDeformacionFuerteDeConvexos}%
	\label{ejemplo:puntodeformacionfuertedeconvexos}
	Si $C$ es un subespacio convexo en un espacio vectorial topol\'{o}gico
	y $x_{0}\in C$ es un punto arbitrario, entonces
	\begin{align*}
		F(x,t) & \,=\,t\,x_{0}+(1-t)\,x
	\end{align*}
	%
	es una homotop\'{\i}a de la identidad $\id[C]$ en la funci\'{o}n
	constante $x\mapsto x_{0}$ relatiova al punto $\{x_{0}\}$. Esto
	muestra que todo punto en un conjunto convexo es un retracto por
	deformaci\'{o}n fuerte del convexo.
\end{ejemploPuntoDeformacionFuerteDeConvexos}

\begin{ejemploEsferaDeformacionFuerteDelEspacioSinUnPunto}%
	\label{ejemplo:esferadeformacionfuertedelespaciosinunpunto}
	Sea $\esfera{n}$ la esera de dimenensi\'{o}n $n$ vista como
	subespacio de $\bb{R}^{n+1}\setmin\{0\}$ y sea
	\begin{math}
		F:\,\big(\bb{R}^{n+1}\setmin\{0\}\big)\times\intervalo
			\bb{R}^{n+1}\setmin\{0\}
	\end{math}
	la funci\'{o}n
	\begin{align*}
		F(x,t) & \,=\,(1-t)\,x +t\,\frac{x}{|x|}
	\end{align*}
	%
	entonces $F$ es una retracci\'{o}n por deformaci\'{o}n fuerte de
	$\bb{R}^{n+1}\setmin\{0\}$ en $\esfera{n}$, pues, si llamamos
	$r(x)=F(x,1)=\frac{x}{|x|}$, entonces $r$ es una retracci\'{o}n
	de la inclusi\'{o}n ($r\circ\inc[\esfera{n}]=\id[\esfera{n}]$) y
	$F$ es una homotop\'{\i}a de $\inc[\esfera{n}]\circ r$ en
	$\id[\bb{R}^{n+1}\setmin\{0\}]$ relativa a la esfera.
\end{ejemploEsferaDeformacionFuerteDelEspacioSinUnPunto}

\begin{ejemploElPeineYElCuadradoNoEsRetracto}%
	\label{ejemplo:elpeineyelcuadradonoesretracto}
	Sea $X=\intervalo^{2}$ y sea $A\subset X$ el subespacio del ejemplo
	\ref{ejemplo:elpeine}. Como $X$ y $A$ son contr\'{a}ctiles, la
	inclusi\'{o}n $A\subset X$ es una equivalencia homot\'{o}pica, con lo
	$A$ es un retracto d\'{e}bil de $X$ y una deformaci\'{o}n,
	un retracto por deformaci\'{o}n d\'{e}bil. Pero, seg\'{u}n el ejemplo
	\ref{ejemplo:elpeineyelcuadrado}, $A$ no es un retracto de $X$ y, por
	lo tanto, $A$ no es un retracto por deformaci\'{o}n de $X$.

	Tal vez $A$ sea un retracto d\'{e}bil por deformaci\'{o}n fuerte, es
	decir, la deformaci\'{o}n se pueda tomar relativa a $A$.
\end{ejemploElPeineYElCuadradoNoEsRetracto}

\begin{ejemploElPuntoYElPeine}\label{ejemplo:elpuntoyelpeine}
	Si ahora $X$ es el espacio del ejemplo \ref{ejemplo:elpeine} y
	$A=\{(0,1)\}$, entonces, como $X$ es contr\'{a}ctil,
	$\id[X]\simeq\inc[A]\circ c$, donde $c:\,X\rightarrow A$ es la
	funci\'{o}n constante $(x,y)\mapsto (0,1)$. Como
	$c\circ\inc[A]=\id[A]$, se deduce que $A$ es un retracto por
	deformaci\'{o}n de $X$. Pero, como se mencion\'{o} en el ejemplo
	\ref{ejemplo:elpeineyelcuadrado}, no existe una homotop\'{\i}a
	de $\id[X]$ en $\inc[A]\circ c$ relativa al punto $A$, es decir que
	$A$ no es un retracto por deformaci\'{o}n fuerte de $X$.
\end{ejemploElPuntoYElPeine}
