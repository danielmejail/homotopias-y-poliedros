\theoremstyle{plain}
\newtheorem{teoExtensionDeHomotopiasRetractoEquivaleARetractoDebil}{Teorema}%
	[section]

\theoremstyle{remark}
\newtheorem{obsExtensionDeHomotopias}%
	[teoExtensionDeHomotopiasRetractoEquivaleARetractoDebil]%
	{Observaci\'{o}n}

%-------------

Sea $X$ un espacio topol\'{o}gico y sea $A\subset X$ un subespacio. Se dice
que $A$ es un \emph{retracto de $X$}, si existe $r:\,X\rightarrow A$, continua,
tal que $r\circ\inc[A]=\id[A]$. Se dice que $A$ es un \emph{retracto %
d\'{e}bil de $X$}, si existe $r:\,X\rightarrow A$, continua, tal que
$r\circ\inc[A]\simeq\id[A]$. En el primer caso, decimos que $r$ es una
\emph{retracci\'{o}n} y, en el segundo, que es una \emph{retracci\'{o}n %
d\'{e}bil}. En este segundo caso, no se requiere que exista una homotop\'{\i}a
reltiva a $A$ de $r\circ\inc[A]$ en $\id[A]$, s\'{o}lo que exista alguna
homotop\'{\i}a. En otras palabras, $A\subset X$ es un retracto de $X$, si la
inclusi\'{o}n $\inc[A]:\,A\rightarrow X$ admite una inversa a izquierda en la
categor\'{\i}a de espacios topol\'{o}gicos y es un retracto d\'{e}bil de $X$,
si la inclusi\'{o}n admite una inversa a izquierda en la categor\'{\i}a
homot\'{o}pica.

Un poco m\'{a}s en general, decimos que una funci\'{o}n $r:\,X\rightarrow Y$
es una retracci\'{o}n, si admite una inversa a derecha, una funci\'{o}n
$j:\,Y\rightarrow X$ tal que $r\circ j=\id[Y]$ (y decimos que $Y$ es un
retracto de $X$, si $r$ es la inversa a izquierda de alguna funci\'{o}n
$Y\rightarrow X$). An\'{a}logamente, decimos que la funci\'{o}n $r$ es una
retracci\'{o}n d\'{e}bil, si admite una inversa a derecha en la categor\'{\i}a
homot\'{o}pica, es decir, existe $j:\,Y\rightarrow X$ y una homotop\'{\i}a
$r\circ j\simeq\id[Y]$.

Sean $X$ e $Y$ espacios topol\'{o}gicos y sea $A\subset X$ un subespacio.
Decimos que el par $(X,A)$ \emph{tiene la propiedad de extensi\'{o}n de
homotop\'{\i}as con respecto a $Y$}, si, dadas $g:\,X\rightarrow Y$ y
$G:\,A\times\intervalo\rightarrow Y$ tal que
\begin{align*}
	G(x,0) & \,=\,g(x)\quad\text{para todo }x\in A
	\text{ ,}
\end{align*}
%
existe $F:\,X\times\intervalo\rightarrow Y$ tal que
\begin{align*}
	F(x,0) & \,=\,g(x) \quad\text{para todo }x\in X\quad\text{y} \\
	F|_{A\times\intervalo} & \,=\,G
	\text{ .}
\end{align*}
%
En t\'{e}rminos de diagramas, $(X,A)$ tiene la propiedad de extensi\'{o}n
de homotop\'{\i}as con respecto a $Y$, si para todo diagrama
\begin{center}
\begin{tikzcd}[column sep=small]%,row sep=small]
	A\times\{0\} \arrow[rr,hook] \arrow[dd,hook] & &
		A\times\intervalo \arrow[dl,"G"'] \arrow[dd,hook] \\
	& Y & \\
	X\times\{0\} \arrow[ur,"g"] \arrow[rr,hook] & &
		X\times\intervalo \arrow[ul,dotted, "F"']
\end{tikzcd}
\end{center}
con el tri\'{a}ngulo por encima de la diagonal conmutativo, existe
$F:\,X\times\intervalo\rightarrow Y$ que hace conmutar a los tri\'{a}ngulos
por debajo de la diagonal.

\begin{obsExtensionDeHomotopias}\label{obs:extensiondehomotopias}
	Si $(X,A)$ tiene la propiedad de extensi\'{o}n de homotop\'{\i}as
	con respecto a un espacio $Y$ y $f_{0},f_{1}:\,A\rightarrow Y$
	son homot\'{o}picas (en $A$), entonces $f_{0}$ tiene una extensi\'{o}n
	continua a $X$, si y s\'{o}lo si existe una extensi\'{o}n continua de
	$f_{1}$. Si $g:\,X\rightarrow Y$ es una extensi\'{o}n de $f_{0}$ y
	$G:\,A\times\intervalo\rightarrow Y$ es una homotop\'{\i}a en $A$ de
	$f_{0}=g|_{A}$ en $f_{1}$, entonces existe una ``extensi\'{o}n'' de
	$G$, $F;\,X\times\intervalo\rightarrow Y$, tal que
	$F(x,0)=g(x)$ para $x\in X$ y $F(x,t)=G(x,t)$ para todo $x\in A$.
	En particular, si $x\in A$, $F(x,0)=f_{0}(x)$ y $F(x,1)=f_{1}(x)$,
	con lo que $x\mapsto F(x,0)$ es la extensi\'{o}n $g$ de $f_{0}$ y
	$x\mapsto F(x,1)$ es una extensi\'{o}n de $f_{1}$ a todo el espacio
	$X$.

	De esto se deduce que, si $(X,A)$ tiene la propiedad de extensi\'{o}n
	de homotop\'{\i}as con respecto a $Y$, entonces el problema de
	determinar si una funci\'{o}n $A\rightarrow Y$ se puede extender a $X$
	es un problema en la categor\'{\i}a homot\'{o}pica.
\end{obsExtensionDeHomotopias}

En general, un morfismo $f:\,X'\rightarrow X$ se dice que es una
\emph{cofibraci\'{o}n}, si dados un objeto arbitrario $Y$ y morfismos
$g:\,X\rightarrow Y$ y $G:\,X'\times\intervalo\rightarrow Y$ tales que
\begin{align*}
	g\circ f(x') & \,=\,G(x',0)
	\text{ ,}
\end{align*}
%
es decir, $g\circ (f\times\id[\{0\}])=G\circ\inc[X'\times\{0\}]$, existe
$F:\,X\times\intervalo\rightarrow Y$ tal que $F(x,0)=g(x)$ para todo
$x\in X$ y $F(f(x'),t)=G(x',t)$ para todo $x'\in X'$ y $t\in\intervalo$.
Dicho de otra manera, para \emph{todo} diagrama
\begin{center}
\begin{tikzcd}[column sep=small]
	X'\times\{0\} \arrow[rr,hook] \arrow[dd,"f\times{\id[\{0\}]}"'] & &
		X'\times\intervalo \arrow[dl,"G"']
			\arrow[dd,"f\times{\id[\intervalo]}"] \\
	& Y & \\
	X\times\{0\} \arrow[rr,hook] \arrow[ur,"g"] & &
		X\times\intervalo \arrow[ul,dotted,"F"']
\end{tikzcd}
\end{center}
cuyo tri\'{a}ngulo encima de la diagonal sea conmutativo, existe una
$F$ tal que los tri\'{a}ngulos inferiores tambi\'{e}n conmuten. En estos
t\'{e}rminos, la inclusi\'{o}n $\inc[A]:\,A\rightarrow X$ es una
cofibraci\'{o}n, si y s\'{o}lo si $(X,A)$ tiene la propiedad de extensi\'{o}n
de homotop\'{\i}as con respecto a cualquier espacio.

\begin{teoExtensionDeHomotopiasRetractoEquivaleARetractoDebil}%
	\label{thm:extensiondehomotopiasretractoequivalearetractodebil}
	Sea $A\subset X$ un subespacio. Si $(X,A)$ tiene la propiedad de
	extensi\'{o}n de homotop\'{\i}as con respecto a $A$, entonces
	$A$ es un retracto de $X$, si s\'{o}lo si es un retracto d\'{e}bil.
\end{teoExtensionDeHomotopiasRetractoEquivaleARetractoDebil}

\begin{proof}
	Sea $r:\,X\rightarrow A$ tal que $r\circ\inc[A]\simeq\id[A]$ y sea
	$G:\,A\times\intervalo\rightarrow A$ una homotop\'{\i}a de
	$r\circ\inc[A]$ en $\id[A]$. Si $(X,A)$ tiene la propiedad de
	extensi\'{o}n de homotop\'{\i}as con respecto a $A$, entonces
	existe $F:\,X\times\intervalo\rightarrow A$ tal que
	\begin{align*}
		F(\inc[A](x'),t) & \,=\,G(x',t) \quad\text{y} \\
		F(x,0) & \,=\,r(x)
	\end{align*}
	%
	para todo $x'\in A$ y todo $x\in X$. Sea $r':\,X\rightarrow A$
	la funci\'{o}n continua $r'(x)=F(x,1)$. Entonces, por un lado, $F$ es
	una homotop\'{\i}a de $r$ en $r'$ definida en $X$ y, por otro,
	\begin{align*}
		r'\circ\inc[A](x') & \,=\, F(\inc[A](x'),1)
			\,=\, G(x',1) \,=\,\id[A](x')
		\text{ ,}
	\end{align*}
	%
	con lo que $r'$ es una retracci\'{o}n de $X$ en $A$.
\end{proof}

Notemos que la retracci\'{o}n $r'$ obtenida en la demostraci\'{o}n de
\ref{thm:extensiondehomotopiasretractoequivalearetractodebil} es homot\'{o}pica
a la retracci\'{o}n d\'{e}bil $r$.
