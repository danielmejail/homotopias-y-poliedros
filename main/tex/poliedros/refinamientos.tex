\theoremstyle{plain}
\newtheorem{lemaAcotarLaMetricaEnSimplices}{Lema}[section]
\newtheorem{lemaAcotarLaMetricaEnComplejos}[lemaAcotarLaMetricaEnSimplices]%
	{Lema}
\newtheorem{teoRefinarPorTriangulaciones}[lemaAcotarLaMetricaEnSimplices]%
	{Teorema}

\theoremstyle{remark}

%-------------

Todo complejo simplcial $K$ admite una m\'{e}trica, pero, en general,
la topolog\'{\i}a inducida no coincide con la topolog\'{\i}a coherente
en $\geom{K}$. De hecho esto es as\'{\i} si y s\'{o}lo si $K$ es localmente
finito. Diremos que una m\'{e}trica en $\geom{K}$ es \emph{lineal}, si
coincide con la m\'{e}trica inducida por una realizaci\'{o}n de $K$ dentro de
un espacio vectorial topol\'{o}gico $E$ cuya topolog\'{\i}a sea compatible con
una m\'{e}trica. M\'{a}s espec\'{\i}ficamente, diremos que una m\'{e}trica en
$\geom{K}$ es lineal, si es la m\'{e}trica inducida por una norma
(cualquiera) en alg\'{u}n espacio $\bb{R}^{n}$ v\'{\i}a una realizaci\'{o}n
de $K$ en $\bb{R}^{n}$. Notemos que todo complejo finito admite una
m\'{e}trica lineal. Tambi\'{e}n es cierto que, dada una m\'{e}trica lineal
en $\geom{K}$ y una subdivisi\'{o}n $K'$ de $K$, entonces la misma m\'{e}trica
es lineal en $\geom{K'}$. Dada una m\'{e}trica (arbitraria) en $\geom{K}$,
se define la \emph{densidad} (o \emph{apertura}, \emph{fineza}) de $K$ como
el supremo de los di\'{a}metros de sus s\'{\i}mplices con respecto a
esta m\'{e}trica:
\begin{align*}
	\mesh K & \,=\,\sup\,\{\diam\geom{s}\,:\,s\in K\}
	\text{ .}
\end{align*}
%

\begin{lemaAcotarLaMetricaEnSimplices}\label{thm:acotarlametricaensimplices}
	Dada una m\'{e}trica lineal en un $m$ s\'{\i}mplice $s$ y dado un
	s\'{\i}mplice $s'\in\subdiv{\caras s}$ de la subdivisi\'{o}n
	baric\'{e}ntrica, se cumple que
	\begin{align*}
		\diam\geom{s'} & \,\leq\,\frac{m}{m+1}\,
			\diam\geom{s}
		\text{ .}
	\end{align*}
	%
\end{lemaAcotarLaMetricaEnSimplices}

\begin{lemaAcotarLaMetricaEnComplejos}\label{thm:acotarlametricaencomplejos}
	Si $K$ es un complejo de dimensi\'{o}n $m$, dada una m\'{e}trica
	lineal en $\geom{K}$, se cumple que
	\begin{align*}
		\mesh{\big(\subdiv K\big)} & \,\leq\,\frac{m}{m+1}\,
			\mesh K
		\text{ .}
	\end{align*}
	%
\end{lemaAcotarLaMetricaEnComplejos}

Sea $X$ un espacio topol\'{o}gico y sea $(K,f)$, $f:\,\geom{K}\rightarrow X$,
una triangulaci\'{o}n de $X$ por un complejo simplicial $K$. Sea $\cal{U}$ un
cubrimiento de $X$ por abierto. Se dice que la triangulaci\'{o}n
\emph{es m\'{a}s fina que $\cal{U}$} o que \emph{refina a $\cal{U}$}, si,
para todo v\'{e}rtice $v$ de $K$, existe un abierto $U\in\cal{U}$ tal que
$f\big(\estrella v\big)\subset U$. Si $X=\geom{K}$ y
$f:\,\geom{K}\rightarrow\geom{K}$ es la identidad de $\geom K$, decimos
tambi\'{e}n que el complejo $K$ refina el cubrimiento $\cal U$. Si $K'$ es
una subdivisi\'{o}n de $K$ y $f:\,\geom{K'}\rightarrow\geom{K}$ es el
homeomorfismo dado por $\alpha_{v'}\mapsto v'$ en los v\'{e}rtices, tambi\'{e}n
decimos que $K'$ refina el cubrimiento $\cal U$.

\begin{teoRefinarPorTriangulaciones}\label{thm:refinarportriangulaciones}
	Sea $\cal U$ un cubrimiento por abiertos de un espacio triangulable
	compacto $X$. Entonces existe una triangulaci\'{o}n de $X$ m\'{a}s
	fina que $\cal U$.
\end{teoRefinarPorTriangulaciones}

La siguiente definici\'{o}n aprecer\'{a} en la demostraci\'{o}n del teorema.
Denominamos \emph{subdivisiones baric\'{e}ntricas iteradas} de un complejo
simplicial $K$ a las subdivisiones definidas recursivamente de la siguiente
manera:
\begin{align*}
	\subdiv[0]{K} & \,=\, K\text{ ,}\\
	\subdiv[n]{K} & \,=\,\subdiv{\big(\subdiv[n-1]{K}\big)}
		\quad\text{si } n\geq 1
	\text{ .}
\end{align*}
%

\begin{proof}
	Sea $(K,f)$ una triangulaci\'{o}n arbitraria de $X$. Entonces,
	como $f:\,\geom{K}\rightarrow X$ es un homeomorfismo y $X$ es
	compacto, por el corolario \ref{thm:complejofinitoespaciocompacto},
	$K$ es un complejo finito y, por el teorema
	\ref{thm:inmersiondeuncomplejo}, admite una realizaci\'{o}n en
	un $\bb{R}$-espacio de dimensi\'{o}n finita. La m\'{e}trica inducida
	por la norma euclidea de este espacio, por ejemplo, es lineal en
	$\geom{K}$.

	Supongamos que el espacio $\geom{K}$ viene dado con una m\'{e}trica
	lineal. Sea $\epsilon >0$ un n\'{u}mero de Lebesque para el
	cubrimiento $\big\{f^{-1}(U)\,:\,U\in\cal U\big\}$ de $\geom{K}$
	respecto de esta m\'{e}trica. Esto quiere decir que, si
	$A\subset\geom{K}$ tiene di\'{a}metro menor que $\epsilon$, entonces
	$f(A)$ est\'{a} contenido en $U$ para alg\'{u}n abieto $U$ del
	cubrimiento $\cal U$. Sea $m=\dim\,K$ y sea $N\geq 1$ tal que
	\begin{align*}
		\Big(\frac{m}{m+1}\Big)^{N}\cdot\mesh K & \,\leq\,
			\epsilon / 2
		\text{ .}
	\end{align*}
	%
	Para $n\geq N$, $\mesh{\subdiv[n]{K}}\leq\epsilon/2$. Si
	$v'$ es un v\'{e}rtice de $\subdiv[n]{K}$, entonces el conjunto
	$\estrella{v'}$ tiene di\'{a}metro acotado por
	\begin{align*}
		\diam{\estrella{v'}} & \,\leq\,2\cdot
			\mesh{\subdiv[n]{K}} \,\leq\,\epsilon
		\text{ .}
	\end{align*}
	%
	Entonces, para $n\geq N$, vale que
	\begin{align*}
		f\big(\estrella{v'}\big)\,\subset\,U
	\end{align*}
	%
	para alg\'{u}n abierto $U\in\cal U$, con lo que
	$(\subdiv[n]{K},f)$ es refina el cubrimiento $\cal U$.
\end{proof}

% ?`Es cierto este resultado para pares triangulables? ?`Es cierto para espacios
% triangulables no compactos?
