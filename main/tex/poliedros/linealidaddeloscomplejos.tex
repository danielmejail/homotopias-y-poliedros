\theoremstyle{plain}
\newtheorem{lemaCombinacionConvexaEnComplejos}{Lema}[section]

\theoremstyle{remark}
\newtheorem{obsSimplicialEsLineal}[lemaCombinacionConvexaEnComplejos]%
	{Observaci\'{o}n}

%-------------

\begin{lemaCombinacionConvexaEnComplejos}%
	\label{thm:combinacionconvexaencomplejos}
	Sea $K$ un complejo simplicial. Una combinaci\'{o}n convexa
	de puntos de $\geom{K}$ pertenece a $\geom{K}$, si y s\'{o}lo si
	dichos puntos pertenecen a un mismo s\'{\i}mplice en $K$.
\end{lemaCombinacionConvexaEnComplejos}

\begin{proof}
	Sean $\lista{\alpha}{r}\in\geom{K}$ puntos pertenecientes a un
	mismo s\'{\i}mplice $\geom{s}$, $s\in K$. Dados n\'{u}meros reales
	$\lista*{t}{r}\in [0,1]$ tales que $\sum_{i=1}^{r}\,t^{i}=1$,
	la funci\'{o}n $\alpha:\,V\rightarrow [0,1]$ dada por
	$\alpha=\sum_{i=1}^{r}\,t^{i}\alpha_{i}$ pertenece a $\geom{s}$,
	pues $\{\alpha\not=0\}\subset s$ y $s\in K$ es un s\'{\i}mplice.
	Rec\'{\i}procamente, si $\lista{\alpha}{r}\in\geom{K}$ son puntos que
	verifican $\alpha=\sum_{i=1}^{r}\,t^{i}\alpha_{i}\in\geom{K}$ para
	ciertos n\'{u}meros reales \emph{positivos} $t^{i}$ que suman $1$
	(ninguno de los puntos $\alpha_{i}$ aparece trivialmente), entonces
	existe un s\'{\i}mplice $s\in K$ m\'{a}s chico que contiene a
	$\alpha$. Como $\alpha\in\simpinterior{s}$, si $v\not\in s$ es un
	v\'{e}rtice de $K$ que no pertenece a $s$, entonces
	$\alpha(v)=0$ y, para cada $i=1,\,\dots,\,r$, $\alpha_{i}(v)=0$.
	Por lo tanto, $\alpha_{i}\in\geom{s}$.
\end{proof}

Dado un v\'{e}rtice $v\in K$, llamamos $\alpha_{v}$ a la funci\'{o}n
$\alpha_{v}\in\geom{K}$ tal que $\alpha_{v}(v)=1$ (\textit{a fortiori},
$\alpha_{v}(v')=0$, si $v'\not = v$). Todo elemento $\alpha\in\geom{K}$
se puede escribir (de manera \'{u}nica) como una combinaci\'{o}n lineal
convexa de estas funciones:
\begin{align*}
	\alpha & \,=\,\sum_{v\in V}\,\alpha(v)\,\alpha_{v}
	\text{ .}
\end{align*}
%

Sea $E$ un $\bb{R}$-espacio vectorial y sea $X\subset E$ un espacio
topol\'{o}gico cuya topolog\'{\i}a es coherente con su intersecci\'{o}n
con los subespacios de dimensi\'{o}n finita de $E$ (d\'{a}ndole a \'{e}stos
la \'{u}nica topolog\'{\i}a con respecto a la cual son espacios vectoriales
topol\'{o}gicos).
% (Por ejemplo, $E$ tiene la topolog\'{\i}a d\'{e}bil respecto
% de una familia de funcionales lineales que separa puntos)
Entonces, si $F\subset E$ es un subespacio vectorial de dimensi\'{o}n finita,
$X\cap F$ tiene la topolog\'{\i}a de subespacio de $F$, donde $F$ tiene
la etructura de e.v.t. de dimnsi\'{o}n finita.

Una funci\'{o}n $f:\,\geom{K}\rightarrow E$ se dice \emph{lineal}, si
\begin{align*}
	f(\alpha) & \,=\,f\Big(\sum_{v\in V}\,\alpha(v)\alpha_{v}\Big) \,=\,
		\sum_{v\in V}\,\alpha(v)\,f(\alpha_{v})
\end{align*}
%
para todo $\alpha\in\geom{K}$. Una funci\'{o}n $f:\,\geom{K}\rightarrow X$ se
dice lineal, si $\inc[X\hookrightarrow E]\circ f:\,\geom{K}\rightarrow E$ es
lineal (en particular, la combinaci\'{o}n lineal
$\sum_{v}\,\alpha(v)\,f(\alpha_{v})$ tiene que ser un punto de $X$).

Toda funci\'{o}n lineal en $\geom{K}$ est\'{a} determinada por su valor en
los v\'{e}rtices $\alpha_{v}$. Rec\'{\i}procamente, toda funci\'{o}n
$f_{0}:\,V\rightarrow E$ determina una funci\'{o}n lineal
$f:\,\geom{K}\rightarrow E$. Si $X\subset E$ y $f_{0}:\,V\rightarrow X$,
entonces la funci\'{o}n lineal correspondiente, $f$, es una funci\'{o}n
lineal en $X$, si y s\'{o}lo si, para todo s\'{\i}mplice $s\in K$, las
combinaciones convexas $\sum_{v\in s}\,t^{v}f_{0}(\alpha_{v})$ tambi\'{e}n
pertenecen a $X$. En particular, si $X$ es un subconjunto convexo de $E$,
entonces toda funci\'{o}n $V\rightarrow X$ se extiende a una funci\'{o}n
lineal $\geom{K}\rightarrow X$.

En cuanto a la continuidad de una funci\'{o}n lineal
$f:\,\geom{K}\rightarrow X$, si $s\in K$ es un $q$-s\'{\i}mplice, entonces la
restricci\'{o}n $f|_{\geom{s}}$ coincide con la restricci\'{o}n de una
funci\'{o}n lineal definida en el $\bb{R}$-espacio vectorial generado
por los v\'{e}rtices de $s$. A este espacio lo denotamos
$F\big(\{\alpha_{v}\}_{v\in s}\big)$. Todo punto de este espacio vectorial se
escribe de manera \'{u}nica como combinaci\'{o}n lineal de los v\'{e}rtices
$\{\alpha_{v}\}_{v\in s}$. A su vez, los puntos de $\geom{s}$ se escriben
de manera \'{u}nica como combinaciones lineales convexas de estos mismos
elementos. As\'{\i}, podemos identificar can\'{o}nicamente $\geom{s}$
con un subconjunto de $F\big(\{\alpha_{v}\}_{v\in s}\big)$. En cuanto a la
funci\'{o}n $f$, si $\alpha\in\geom{s}$ y
$\alpha=\sum_{v\in s}\,\alpha(v)\,\alpha_{v}$, entonces
$f(\alpha)=\sum_{v\in s}\,\alpha(v)\,f(\alpha_{v})$. Pero, por otro lado,
por propiedad universal de $F\big(\{\alpha_{v}\}_{v\in s}\big)$, las
asignaciones $\alpha_{v}\mapsto f(\alpha_{v})$ determinan un\'{\i}vocamente
una funci\'{o}n lineal
$f_{s}:\,F\big(\{\alpha_{v}\}_{v\in s}\big)\rightarrow E$. Esta extensi\'{o}n
cumple que
\begin{align*}
	f_{s}|_{\geom{s}} & \,=\,f|_{\geom{s}}
	\text{ ,}
\end{align*}
%
pues ambas est\'{a}n dadas por la misma f\'{o}rmula en t\'{e}rminos de los
elementos de la base $\{\alpha_{v}\}_{v\in s}$. En particular,
$f_{s}(\geom{s})\subset X$. Notemos tambi\'{e}n que la topolog\'{\i}a en
$\geom{s}$ es exactamente la topolog\'{\i}a inducida como subespacio de
$F\big(\{\alpha_{v}\}_{v\in s}\big)$ y que $X$ es un subespacio topol\'{o}gico
de $E$, con lo cual, para ver $f:\,\geom{K}\rightarrow X$ es continua,
alcanza con ver que las funciones $f_{s}$ son continuas. Pero $f_{s}$, por
ser lineal, tiene imagen en un subespacio $W\subset E$ de dimensi\'{o}n
finita. Por lo tanto,
$f_{s}:\,F\big(\{\alpha_{v}\}_{v\in s}\big)\rightarrow W$ es continua.
Como la inclusi\'{o}n $W\rightarrow E$ es continua, $f_{s}$ es continua.
En definitiva, toda funci\'{o}n lineal $f:\,\geom{K}\rightarrow X$ es
autom\'{a}ticamente continua\dots

\begin{obsSimplicialEsLineal}\label{obs:simplicialeslineal}
	Dado un complejo simplicial $K$ consideramos el $\bb{R}$-espacio
	vectorial generado por todos los v\'{e}rtices de $K$:
	$E=F\big(\{\alpha_{v}\}_{v\in K}\big)$. El conjunto $\geom{K}$ es
	un subconjunto de este espacio: todo elemento $\alpha\in\geom{K}$
	es una combinaci\'{o}n lineal convexa de los elementos de la base
	(pero no todas tales combinaciones pertenecen a $\geom{K}$).
	La topolog\'{\i}a coherente en $\geom{K}$ est\'{a} determinada
	por la topolog\'{\i}a en cada uno de los subconjuntos $\geom{s}$,
	$s\in K$. Pero la topolog\'{\i}a en $\geom{s}$ es la topolog\'{\i}a
	de subespacio del $\bb{R}$-espacio de dimensi\'{o}n finita
	generado por aquellos v\'{e}rtices de $K$ que pertenecen a $s$.

	Dada una transformaci\'{o}n simplicial $\varphi:\,K\rightarrow K'$,
	aplicando el funtor $\geom{\cdot}$, se obtiene una funci\'{o}n
	lineal: $\geom{\varphi}:\,\geom{K}\rightarrow\geom{K'}$ est\'{a}
	dada por la f\'{o}rmula
	\begin{align*}
		\geom{\varphi}(\alpha) & \,=\,
			\sum_{v\in V}\,\alpha(v)\,\alpha_{\varphi(v)} \,=\,
			\sum_{v\in V}\,\alpha(v)\,\geom{\varphi}(\alpha_{v})
		\text{ .}
	\end{align*}
	%
\end{obsSimplicialEsLineal}
