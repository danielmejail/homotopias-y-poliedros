\theoremstyle{plain}
\newtheorem{lemaSubcomplejoPleno}{Lema}[section]

\theoremstyle{remark}
\newtheorem{obsGeneradoPorVertices}{Observaci\'{o}n}[section]
\newtheorem{obsSubcomplejoPleno}[obsGeneradoPorVertices]{Observaci\'{o}n}
\newtheorem{obsParesSimpliciales}[obsGeneradoPorVertices]{Observaci\'{o}n}

%-------------

\subsection{Objetos}
Un \emph{complejo simplicial} es un par $(K,V)$, compuesto por un conjunto
$V$ de \emph{v\'{e}rtices} y un conjunto $K$ de subconjuntos finitos y no
vac\'{\i}os de $V$, denominados \emph{s\'{\i}mplices} del complejo. Se
requiere, adem\'{a}s, que \emph{(i)} todo subconjunto $\{v\}\subset V$
compuesto por un \'{u}nico elemento de $V$ pertenezca a $K$, es decir,
sea un s\'{\i}mplice; y \emph{(ii)} todo subconjunto no vac\'{\i}o de un
s\'{\i}mplice sea, tambi\'{e}n, un s\'{\i}mplice. El complejo se suele
denotar simplemente por $K$.
% En s\'{\i}mbolos, \emph{(i)} $\{v\}\in K$
% para todo $v\in V$; y \emph{(ii)} si $\varnothing\not= s'\subset s\in K$,
% entonces $s'\in K$, tambi\'{e}n.
Un $q$-s\'{\i}mplice en un complejo simplicial $K$ es un s\'{\i}mplice
compuesto por exactamente $q+1$ v\'{e}rtices distintos; se dice que $q$ es
la \emph{dimensi\'{o}n} de $s$. Si $s$ es un s\'{\i}mplice y $s'\subset s$,
entonces se dice que $s'$ es una \emph{cara} de $s$; si, adem\'{a}s, $s'$ es
un $p$-s\'{\i}mplice, se dice que es una $p$-cara de $s$. Una cara
$s'\subset s$ se dice \emph{propia}, si $s'\not= s$. Las caras de un
complejo simplicial $K$ est\'{a}n parcialmente ordenadas por inclusi\'{o}n.
Dado un s\'{\i}mplice $s\in K$, llamamos \emph{v\'{e}rtices de $s$} a los
elementos de $s$, es decir, aquellos v\'{e}rtices de $K$ que pertenecen
a $s$.

\begin{obsGeneradoPorVertices}\label{obs:generadoporvertices}
	Sea $K$ un complejo simplicial. Los $0$-s\'{\i}mplices de $K$ se
	corresponden exactamente con los v\'{e}rtices de $K$ (es decir, $V$).
	Podemos pensar, entonces, a $K$ simplemente como el conjunto de
	s\'{\i}mplices, identificando los v\'{e}rtices $V$ con los
	$0$-s\'{\i}mplices del complejo. Adem\'{a}s, todo s\'{\i}mplice de
	$K$ est\'{a} \emph{generado}, determinado, por los v\'{e}rtices que
	contiene, es decir, por sus $0$-caras: todo s\'{\i}mplice es un
	subconjunto de los v\'{e}rtices de $K$, es decir, es de la forma
	$s=\{\lista[0]{v}{q}\}\subset V$ y, dado un suconjunto
	$\{\lista[0]{v}{q}\}\subset V$ de v\'{e}rtices, existe a lo sumo un
	s\'{\i}mplice cuyos v\'{e}rtices sean exactamente $\lista[0]{v}{q}$.

	Otra manera de pensar en un complejo simplicial es en un conjunto
	$K$ de objetos denominados \emph{s\'{\i}mplices}, parcialmente
	ordenado y que cumple ciertas propiedades de ``finitud'':
	existen elementos $v$ minimales en $K$, es decir, tales que,
	si $s\in K$, entonces $v\leq s$, o bien $v$ y $s$ no son comparables;
	si $s\in K$, entonces existe al menos un elemento minimal
	$v\in K$ tal que $v\leq s$; dado $s\in K$, la cantidad de elementos
	minimales por debajo de $s$ es igual a la cantidad de elementos
	intermedios (incluyendo los extremos) entre $s$ y cualquiera
	de los elementos minimales debajo de $s$; si $s,s'\in K$ y $s'\leq s$,
	entonces existen a lo sumo finitos elementos intermedios
	$s'\leq s''\leq s$. No estoy seguro de que estas propiedades sean
	suficientes para caracterizar a los complejos simpliciales como
	conjuntos parcialmente ordenados.
\end{obsGeneradoPorVertices}

La \emph{dimensi\'{o}n} de un complejo simplicial $K$ se define como
\begin{align*}
	\dim\,K & \,=\,\sup\,\{\dim\,s\,:\,s\in K\}\cup\{-1\}
	\text{ ,}
\end{align*}
%
con lo cual $\dim\,K=n$, si $K$ contiene un $n$-s\'{\i}mplice pero no
contiene $(n+1)$-s\'{\i}mplices, $\dim\,K=\infty$ si $K$ contiene un
$n$-s\'{\i}mplice para $n$ arbitrariamente grande y $\dim\,K=-1$, si
$K=\varnothing$ es el complejo vac\'{\i}o. Un complejo se dice
\emph{finito}, si contiene una cantidad finita de s\'{\i}mplices
(no confundir ser finito con ser de dimensi\'{o}n finita). Tambi\'{e}n se
dice que $K$ es \emph{localmente finito}, si cada v\'{e}rtice de $K$
es parte de a lo sumo finitos s\'{\i}mplices de $K$.

Un \emph{subcomplejo} de un complejo simplicial $K$ es un subconjunto de
s\'{\i}mplices de $K$ que constituye, a su vez, un complejo simplicial.

\subsection{Morfismos}
Una \emph{transformaci\'{o}n simplicial}
$(K_{1},V_{1})\rightarrow (K_{2},V_{2})$ (o aplicaci\'{o}n simplicial, o un
mapa simplicial o morfismo simplicial) es un par $(\varphi,\varphi_{0})$,
donde $\varphi_{0}:\,V_{1}\rightarrow V_{2}$ y
$\varphi:\,K_{1}\rightarrow K_{2}$ son funciones que verifican que, si
$s=\{\lista[0]{v}{q}\}\in K_{1}$, entonces
\begin{equation}
	\label{eq:transformacionsimplicial}
	\varphi(s) \,=\, \varphi\big(\{\lista[0]{v}{q}\}\big) \,=\,
		\{\varphi_{0}(v_{0}),\,\dots,\,\varphi_{0}(v_{q})\}
	\text{ .}
\end{equation}
%
En particular, $\varphi\big(\{v\}\big)=\varphi_{0}(v)$, para todo v\'{e}rtice
$v\in V_{1}$. Adem\'{a}s, la funci\'{o}n en s\'{\i}mplices $\varphi$ est\'{a}
determinada por la funci\'{o}n en v\'{e}rtices $\varphi_{0}$, pues, dada
una funci\'{o}n $\varphi_{0}:\,V_{1}\rightarrow V_{2}$ arbitraria, existe
una \'{u}nica funci\'{o}n $\varphi:\,\partes(V_{1})\rightarrow\partes(V_{2})$
en subconjuntos de v\'{e}rtices que verifica
\eqref{eq:transformacionsimplicial}. Pero no toda funci\'{o}n en los
v\'{e}rtices define una transformaci\'{o}n simplicial. Para que un
par $(\varphi,\varphi_{0})$ que cumple \eqref{eq:transformacionsimplicial}
sea una transformaci\'{o}n simplicial es necesario y suficiente que
$\varphi_{0}$ cumpla
\begin{equation}
	\label{eq:transformacionsimplicialvertices}
	\{\varphi_{0}(v_{0}),\,\dots,\,\varphi_{0}(v_{q})\}\in K_{2}
		\quad\text{para todo s\'{\i}mplice }
		\{\lista[0]{v}{q}\}\in K_{1}
	\text{ .}
\end{equation}
%
En definitiva, las transformaciones simpliciales se pueden ver
equivalentemente como pares
\begin{math}
	(\varphi:\,K_{1}\rightarrow K_{2},\,
		\varphi_{0}:\,V_{1}\rightarrow V_{2})
\end{math}
que verifican \eqref{eq:transformacionsimplicial}, o bien como una
funci\'{o}n $\varphi_{0}:\,V_{1}\rightarrow V_{2}$ que cumple
\eqref{eq:transformacionsimplicialvertices}. Por \'{u}ltimo, notemos que,
identificando los v\'{e}rtices $V_{i}$ con los $0$-s\'{\i}mplices
$\qesq{0}{K_{i}}$ de los complejos, se cumple que $\varphi(V_{1})\subset V_{2}$
y que $\varphi|_{V_{1}}=\varphi_{0}$. As\'{\i}, podemos pensar en una
transformaci\'{o}n simpicial $(\varphi,\varphi_{0})$ como la funci\'{o}n en
s\'{\i}mplices $\varphi:\,K_{1}\rightarrow K_{2}$, identificando $\varphi_{0}$
con la restricci\'{o}n de $\varphi$ al $0$-esqueleto. De esta manera, nos
podemos referir a $(\varphi,\varphi_{0})$ simplemente por $\varphi$.

La composici\'{o}n $\varphi\circ\psi$ de transformaciones simpliciales
$\varphi$ y $\psi$ se define como la transformaci\'{o}n determinada por la
composici\'{o}n $\varphi_{0}\circ\psi_{0}$ de las funciones en los
v\'{e}rtices. Adem\'{a}s, dado un complejo $K$ la funci\'{o}n identidad en
v\'{e}rtices determina una transformaci\'{o}n simplicial $\id[K]$ que es la
identidad en el conjunto de s\'{\i}mplices. As\'{\i}, queda determinada
una categor\'{\i}a cuyos objetos son los complejos simpliciales y cuyos
morfismos son las transformaciones simpliciales entre ellos.

Sea $K$ un complejo simplicial. Los subcomplejos de $K$ se corresponden con
transformaciones simpliciales $\inc:\,L\rightarrow K$ que son inclusiones
$\inc:\,L\subset K$ como funciones entre los conjuntos de s\'{\i}mplices.
Llamamos a estas transformaciones, \emph{inclusiones simpliciales}.
Toda transformaci\'{o}n simplicial $L\rightarrow K$ que es inyectiva como
funci\'{o}n entre los conjuntos de s\'{\i}mplices se puede pensar como una
inclusi\'{o}n simplicial v\'{\i}a alguna biyecci\'{o}n con un subcomplejo
de $K$ en el sentido estricto de la definici\'{o}n. Tambi\'{e}n llamaremos
inclusiones simpliciales a estas transformaciones y subcomplejos a los
complejos de partida de las mismas.

Dado un subcomplejo $L\subset K$, el subconjunto $N\subset K$ de
s\'{\i}mplices sin v\'{e}rtices en $L$ es un subcomplejo de $K$, al que
podemos llamar \emph{complemento de $L$}; es el subcomplejo m\'{a}s grande
de $K$ disjunto de $L$, es decir, que no comparte v\'{e}rtices con $L$.
Dado $s=\{\lista[0]{v}{q}\}\in K$, entonces hay tres posibilidades:
una primera posibilidad es $v_{i}\not\in L$ para todo $i$, con lo cual
$s\in N$; una segunda posibilidad es que, renombrando los v\'{e}rtices,
exista $p\geq 0,p<q$ tal que $v_{i}\in L$, si $i\leq p$ y $v_{i}\not\in L$,
si $i>p$, en tal caso, quedan determinados dos s\'{\i}mplices
$s'=\{\lista[0]{v}{p}\}$ y $s''=\{\lista[p+1]{v}{q}\}$ y vale que
$s''\in N$; la tercera posibilidad es que todos los v\'{e}rtices pertenezcan
a $L$. Un subcomplejo $L\subset K$ se dice \emph{pleno}, si se cumple que todo
s\'{\i}mplice de $K$ cuyos v\'{e}rtices pertenecen al conjunto de
v\'{e}rtices de $L$ pertenece al complejo $L$.

\begin{lemaSubcomplejoPleno}\label{thm:subcomplejopleno}
	Si $L\subset K$ es un subcomplejo pleno de $K$ y $N$ es
	el complemento de $L$ en $K$, entonces, dado $s\in K$, vale que:
	$s\in N$, o bien $s=s'\cup s''$, con $s'\in L$ y $s''\in N$, o bien
	$s\in L$.
\end{lemaSubcomplejoPleno}

\begin{obsSubcomplejoPleno}\label{obs:subcomplejopleno}
	La descomposici\'{o}n $s=s'\cup s''$ no tiene en cuenta las caras
	del s\'{\i}mplice $s$, es simplemente una descomposici\'{o}n del
	conjunto de v\'{e}rtices de $s$.
\end{obsSubcomplejoPleno}

\subsection{Subcategor\'{\i}as}
Un \emph{par simplicial} es un par $(K,L)$, donde $K$ es un complejo
simplicial y $L\subset K$ es un subcomplejo ($(K,\varnothing)$ es una
posibilidad). Un morfismo de pares simpliciales
$(K_{1},L_{1})\rightarrow (K_{2},L_{2})$ es una transformaci\'{o}n
simplicial $\varphi:\,K_{1}\rightarrow K_{2}$ que verifica
$\varphi(L_{1})\subset L_{2}$.

\begin{obsParesSimpliciales}
	\label{obs:paressimpliciales}
	Un poco m\'{a}s en general, podemos llamar par simplicial a
	un par de complejos $K,K'$ y una transformaci\'{o}n simplicial
	$\varphi:\,K\rightarrow K'$ --es decir, un par simplicial se
	corresponde con una transformaci\'{o}n simplicial, con esta
	definici\'{o}n. Dadas transformaciones simpliciales
	$\varphi_{1}:\,K_{1}\rightarrow K'_{1}$ y
	$\varphi_{2}:\,K_{2}\rightarrow K'_{2}$, un morfismo de pares
	simpliciales $\varphi_{1}\rightarrow\varphi_{2}$ es un
	par de transformaciones simpliciales $(\psi',\psi)$,
	$\psi:\,K_{1}\rightarrow K_{2}$ y $\psi':\,K'_{1}\rightarrow K'_{2}$
	tales que
	\begin{center}
	\begin{tikzcd}
		K_{1}\arrow[r,"\varphi_{1}"]\arrow[d,"\psi"'] &
			K'_{1}\arrow[d,"\psi'"] \\
		K_{2}\arrow[r,"\varphi_{2}"'] & K'_{2}
	\end{tikzcd}
	\end{center}
	conmuta.
\end{obsParesSimpliciales}

Los pares simpliciales junto con los morfismos de pares constituyen una
categor\'{\i}a, la \emph{categor\'{\i}a de pares simpliciales}.
La categor\'{\i}a de complejos simpliciales se puede ver como una
subcategor\'{\i}a plena de la categor\'{\i}a de pares simpliciales
v\'{\i}a $K\mapsto (K,\varnothing)$. Otra subcategor\'{\i}a plena de la
categor\'{\i}a de pares simpliciales es la categor\'{\i}a de \emph{complejos %
simpliciales con un v\'{e}rtice distinguido}, cuyos objetos
son los pares $(K,v)$, $v\in V$ y cuyos morfismos son los morfismos de
complejos que preservan el v\'{e}rtice distinguido. La inclusi\'{o}n
en la categor\'{\i}a de pares est\'{a} dada por $(K,v)\mapsto (K,\{v\})$.

