\theoremstyle{plain}
\newtheorem{teoHomotopicasAdmitenAproximacionesContiguas}{Teorema}[section]
\newtheorem{lemaAproximacionesSonContiguas}%
	[teoHomotopicasAdmitenAproximacionesContiguas]{Lema}
\newtheorem{teoHomotopicasPorContiguas}%
	[teoHomotopicasAdmitenAproximacionesContiguas]{Teorema}
\newtheorem{coroCodominioNumerableHomotopicasNumerables}%
	[teoHomotopicasAdmitenAproximacionesContiguas]{Corolario}

\theoremstyle{remark}
\newtheorem{obsComposicionesDeContiguasSonContiguas}%
	[teoHomotopicasAdmitenAproximacionesContiguas]{Observaci\'{o}n}
\newtheorem{obsContiguasSonHomotopicas}%
	[teoHomotopicasAdmitenAproximacionesContiguas]{Observaci\'{o}n}
\newtheorem{obsLimiteDirectoDeSubdivisiones}%
	[teoHomotopicasAdmitenAproximacionesContiguas]{Observaci\'{o}n}

%-------------

En esta secci\'{o}n definimos una relaci\'{o}n an\'{a}loga a la relaci\'{o}n
de homotop\'{\i}a entre pares de espacios topol\'{o}gicos. Nos concentraremos
en pares $(K,L)$ de complejos simpliciales donde $L\subset K$ es un
subcomplejo. El objetivo es entender la relaci\'{o}n, valaga la redundancia,
entre estas dos nociones de equivalencia.

Sean $K_{1},K_{2}$ complejos simpliciales y sea $L_{1}\subset K_{1}$ y
$L_{2}\subset K_{2}$ subcomplejos. Dos transformaciones simpliciales
$\varphi,\varphi':\,(K_{1},L_{1})\rightarrow (K_{2},L_{2})$ se dicen
\emph{contiguas}, si,
\begin{itemize}
	\item[(i)] dado un s\'{\i}mplice $s\in K_{1}$, la uni\'{o}n
		$\varphi(s)\cup\varphi'(s)$ es un s\'{\i}mplice de $K_{2}$ y
	\item[(ii)] si $s\in L_{1}$, entonces $\varphi(s)\cup\varphi'(s)$
		es un s\'{\i}mplice de $L_{2}$.
\end{itemize}
%
Esto no define una relaci\'{o}n de equivalencia, pero s\'{\i} sim\'{e}trica y
reflexiva en el conjunto de transformaciones simpliciales
$(K_{1},L_{1})\rightarrow (K_{2},L_{2})$. Para definir una relaci\'{o}n de
equivalencia, transitivizamos la relaci\'{o}n: decimos que dos transformaciones
simpliciales $\varphi,\varphi'$ est\'{a}n relacionadas, o que
\emph{pertenecen a la misma clase de contig\"{u}idad}, si existe una
cantidad finita de transformaciones simpliciales $\lista{\varphi}{r}$
tales que $\varphi_{1}=\varphi$, $\varphi_{r}=\varphi'$ y
$\varphi_{i-1}$ y $\varphi_{i}$ son contiguas. Escribimos
$\varphi\sim\varphi'$ para denotar que $\varphi$ y $\varphi'$ pertenecen a la
misma clase de contig\"{u}idad. Denotamos el conjunto de estas clases por
$\contiguas{K_{1},L_{1}}{K_{2},L_{2}}$ y la clase de una transformaci\'{o}n
simplicial $\varphi$ por $\clase{\varphi}$.

\begin{obsComposicionesDeContiguasSonContiguas}%
	\label{obs:composicionesdecontiguassoncontiguas}
	Si $\varphi,\varphi':\,(K_{1},L_{1})\rightarrow (K_{2},L_{2})$
	son contiguas y
	$\psi,\psi':\,(K_{2},L_{2})\rightarrow (K_{3},L_{3})$ son contiguas,
	entonces $\psi\varphi$ y $\psi'\varphi'$ son contiguas: dado un
	s\'{\i}mplice $s\in K_{1}$,
	\begin{align*}
		\psi\varphi(s)\,\cup\,\psi'\varphi'(s) & \,\subset\,
			\psi\big(\varphi(s)\cup\varphi'(s)\big)\,\cup\,
			\psi'\big(\varphi(s)\cup\varphi'(s)\big)
		\text{ .}
	\end{align*}
	%
	Entonces, como la uni\'{o}n de la derecha es, por hip\'{o}tesis, un
	s\'{\i}mplice de $K_{3}$, el subconjunto de la izquierda tambi\'{e}n
	lo es; si $s\in L_{1}$, entonces lo mismo es cierto reemplazando los
	s\'{\i}mplices $K_{i}$ por los $L_{i}$.

	De esto se deduce que, si, m\'{a}s en general, $\varphi\sim\varphi'$
	y $\psi\sim\psi'$, entonces $\psi\varphi\sim\psi'\varphi'$ y la
	composici\'{o}n de clases de contig\"{u}idad est\'{a} bien
	definida como la clase de las composiciones:
	\begin{align*}
		\clase{\psi}\circ\clase{\varphi} & \,=\,
			\clase{\psi\circ\varphi}
		\text{ .}
	\end{align*}
	%
\end{obsComposicionesDeContiguasSonContiguas}

\begin{obsContiguasSonHomotopicas}\label{obs:contiguassonhomotopicas}
	Sean
	\begin{math}
		\varphi,\varphi':\,(K_{1},L_{1})\rightarrow (K_{2},L_{2})
	\end{math}
	transformaciones simpliciales contiguas que coinciden en alg\'{u}n
	subcomplejo $L\subset K_{1}$. Sea
	\begin{align*}
		F & \,:\,\big(\geom{K_{1}}\times\intervalo,
			\geom{L_{1}}\times\intervalo\big)\,\rightarrow\,
				(\geom{K_{2}},\geom{L_{2}})
	\end{align*}
	%
	la homotop\'{\i}a de $\geom{\varphi}$ en $\geom{\varphi'}$ dada por
	\begin{align*}
		F(\alpha,t) & \,=\,(1-t),\geom{\varphi}\alpha + 
					t\,\geom{\varphi'}\alpha
		\text{ .}
	\end{align*}
	%
	Para $\alpha\in\geom{L_{1}}$, como $s\in L_{1}$ implica
	$\varphi(s)\cup\varphi'(s)\in L_{2}$, $F(\alpha,t)\in\geom{L_{2}}$.
	Para $\alpha\in\geom{L}$, el resultado no depende de $t$. En
	definitiva, si $\varphi$ y $\varphi'$ son contiguas y coinciden en
	alg\'{u}n subcomplejo del dominio (independiente de $L_{1}$ y de
	$L_{2}$), entocnes
	\begin{math}
		\geom{\varphi}\simeq\geom{\varphi'}\big(\rel{\geom{L}}\big)
	\end{math}~.

	En particular, $\varphi\sim\varphi'$ implica
	$\geom{\varphi}\simeq\geom{\varphi'}$.
\end{obsContiguasSonHomotopicas}

De las \'{u}ltimas dos observaciones se deduce que, en primer lugar,
las clases de contig\"{u}idad de transformaciones simpliciales determinan
una categor\'{\i}a cuyos objetos son los pares simpliciales, cuyos morfismos
son las clases de contig\"{u}idad de transformaciones simpliciales con
la composici\'{o}n dada por la clase de las composiciones. Los objetos son
los mismos que en la categor\'{\i}a de pares simpliciales. En segundo lugar,
las aplicaciones $(K,L)\mapsto (\geom{K},\geom{L})$ y
$\clase{\varphi}\mapsto\clase{\geom{\varphi}}$ determinan un funtor de la
categor\'{\i}a \emph{de contig\"{u}idad} de pares de complejos simpliciales
en la categor\'{\i}a de homotop\'{\i}a de pares de espacios topol\'{o}gicos.

\begin{lemaAproximacionesSonContiguas}\label{thm:aproximacionessoncontiguas}
	Sean
	\begin{math}
		\varphi,\varphi':\,(K_{1},L_{1})\rightarrow (K_{2},L_{2})
	\end{math}
	dos aproximaciones simpliciales de una misma funci\'{o}n continua
	\begin{math}
		f:\,(\geom{K_{1}},\geom{L_{1}})\rightarrow
			(\geom{K_{2}},\geom{L_{2}})
	\end{math}~.
	Entonces $\varphi$ y $\varphi'$ son contiguas.
\end{lemaAproximacionesSonContiguas}

\begin{proof}
	Sea $s=\{\lista[0]{v}{q}\}\in K_{1}$ un s\'{\i}mplice. Por
	el corolario \ref{thm:simplicedesubcomplejo},
	$\bigcap_{i=0}^{q}\,\estrella v_{i}\not=\varnothing$. Por el teorema
	\ref{thm:caracterizacionaproximacionessimpliciales},
	\begin{align*}
		\bigcap_{i=0}^{q}\,\big(
			\estrella{\varphi(v_{i})}\cap\estrella{\varphi'(v_{i})}
			\big) & \,\supset\,
			\bigcap_{i=0}^{q}\,f\big(\estrella{v_{i}}\big)
				\,\supset\,
			f\Big(\bigcap_{i=0}^{q}\,\estrella{v_{i}}\Big)
				\,\not=\,\varnothing
		\text{ .}
	\end{align*}
	%
	Por el corolario \ref{thm:simplicedesubcomplejo}, la uni\'{o}n
	\begin{math}
		\varphi(s)\cup\varphi'(s)=
			\{\varphi(v_{i})\}_{i}\cup\{\varphi'(v_{i})\}_{i}
	\end{math}
	es el conjunto de v\'{e}rtices de un s\'{\i}mplice en $K_{2}$. Si
	asumimos que $s\in L_{1}$, entonces esta uni\'{o}n es un subconjunto
	de v\'{e}rtices de $L_{2}$ tales que la intersecci\'{o}n de las
	estrellas correspondientes es no vac\'{\i}a y, por lo tanto,
	constituyen el conjunto de v\'{e}rtices de un s\'{\i}mplice en $L_{2}$.
	En definitiva, $\varphi$ y $\varphi'$ son contiguas.
\end{proof}

Transformaciones simpliciales que definen funciones continuas homot\'{o}picas
en los espacios de los complejos pueden no pertenecer a la misma clase de
contig\"{u}idad. Aun as\'{\i}, en el caso de que el dominio sea un cimplejo
finito, es posible subdividir este complejo de forma tal que transformaciones
que inducen funciones homot\'{o}picas son aproximables en la subdivisi\'{o}n
por transformaciones en la misma clase de contig\"{u}idad.

\begin{teoHomotopicasAdmitenAproximacionesContiguas}%
	\label{thm:homotopicasadmitenaproximacionescontiguas}
	Sea $K_{1}$ un complejo simplicial finito. Sean
	\begin{align*}
		f,f' & \,:\,(\geom{K_{1}},\geom{L_{1}})\,\rightarrow\,
			(\geom{K_{2}},\geom{L_{2}})
	\end{align*}
	funciones homot\'{o}picas. Entonces existe $N\geq 1$ y aproximaciones
	simpliciales $\varphi$ de $f$ y $\varphi'$ de $f'$
	\begin{align*}
		\varphi,\varphi' & \,:\,(\subdiv[N]{K_{1}},\subdiv[N]{L_{1}})
			\,\rightarrow\,(K_{2},L_{2})
	\end{align*}
	%
	en la misma clase de contig\"{u}idad.
\end{teoHomotopicasAdmitenAproximacionesContiguas}

\begin{proof}
	Sea
	\begin{align*}
		F & \,:\,\big(\geom{K_{1}}\times\intervalo,
				\geom{L_{1}}\times\intervalo\big)
			\,\rightarrow\, (\geom{K_{2}},\geom{L_{2}})
	\end{align*}
	%
	una homotop\'{\i}a de $f$ en $f'$. Al ser $\geom{K_{1}}$ compacto y
	\begin{math}
		\big\{F^{-1}\big(\estrella{v'}\big)\,:\,
			v'\text{ v\'{e}rtice en }K_{2}\big\}
	\end{math}
	un cubrimiento por abiertos de $K_{1}$, existen
	$0=t_{0}<t_{1}<\cdots<t_{r}=1$ tales que, para todo $\alpha\in K_{1}$,
	$F(\alpha,t_{i-1})$ y $F(\alpha,t_{i})$ pertenezcan a un mismo
	abierto $\estrella{v'}$ de $K_{2}$.

	Para cada $i=0,\,\dots,\,r$, sea
	\begin{align*}
		f_{i} & \,:\,(\geom{K_{1}},\geom{L_{1}})\,\rightarrow\,
			(\geom{K_{2}},\geom{L_{2}})
	\end{align*}
	%
	la funci\'{o}n dada por $f_{i}(\alpha)=F(\alpha,t_{i})$ y sea,
	para $i\geq 1$, $\cal{U}_{i}$ el cubrimiento
	\begin{align*}
		\cal{U}_{i} & \,=\,\big\{
			f_{i}^{-1}\big(\estrella{v'}\big)\cap
				f_{i-1}^{-1}\big(\estrella{v'}\big)\,:\,
			v'\in K_{2}\big\}
	\end{align*}
	%
	por abiertos de $K_{1}$. Por el teorema
	\ref{thm:refinarportriangulaciones}, existe $N\geq 1$ tal que la
	subdivisi\'{o}n $\subdiv[N]{K_{1}}$ sea m\'{a}s fina que todos los
	cubrimientos $\lista{\cal{U}}{r}$. Esto significa que, para cada
	$i=1,\,\dots,\,r$, si $v$ es un v\'{e}rtice de $\subdiv[N]{K_{1}}$,
	entonces existe $U\in\cal{U}_{i}$ tal que $\estrella v\subset U$.
	Dicho de otra manera, para cada v\'{e}rtice $v$ de la subdivisi\'{o}n,
	existe al menos un v\'{e}rtice $v'$ de $K_{2}$ tal que
	\begin{align*}
		\estrella v & \,\subset\,
			f_{i}^{-1}\big(\estrella{v'}\big)\,\cap\,
			f_{i-1}^{-1}\big(\estrella{v'}\big)
		\text{ .}
	\end{align*}
	%
	Eligiendo, debe existir una funci\'{o}n $\varphi_{i}$ definida en
	los v\'{e}rtices de $\subdiv[N]{K_{1}}$ con imagen en los
	v\'{e}rtices de $K_{2}$ tal que
	\begin{align*}
		f_{i}\big(\estrella v\big),\cup\,
			f_{i-1}\big(\estrella v\big) & \,\subset\,
			\estrella{\varphi_{i}(v)}
	\end{align*}
	%
	para todo v\'{e}rtice $v$ de la subdivisi\'{o}n. En particular, por
	el teorema \ref{thm:caracterizacionaproximacionessimpliciales},
	cada una de las funciones $\varphi_{i}$ determina una
	transformaci\'{o}n simplicial
	\begin{align*}
		\varphi_{i} & \,:\,(\subdiv[N]{K_{1}},\subdiv[N]{L_{1}})
			\,\rightarrow\,(K_{2},L_{2})
	\end{align*}
	%
	que es, a la vez, aproximaci\'{o}n de
	$f_{i}$ y de $f_{i-1}$. Del lema \ref{thm:aproximacionessoncontiguas},
	se deduce que $\varphi_{i}$ y $\varphi_{i+1}$ son contiguas.
	En particular, $\varphi_{1}\sim\varphi_{r}$. Pero $\varphi_{1}$ es
	una aproximaci\'{o}n de $f_{0}=f$ y $\varphi_{r}$ es una
	aproximaci\'{o}n de $f_{r}=f'$.
\end{proof}

Este resultado no es cierto en general, si $K_{1}$ no es finito (ver el
ejemplo \ref{ejemplo:homotopicasnocontiguas}).

\begin{obsLimiteDirectoDeSubdivisiones}\label{obs:limitedirectodesubdivisiones}
	Sea $K_{1}$ un complejo simplicial y sea $L_{1}$ un subcomplejo. Por
	el corolario \ref{thm:caracterizaciondeaproximacionesporsubdivision},
	existen aproximaciones simpliciales
	\begin{align*}
		\varphi & \,:\,(\subdiv{K_{1}},\subdiv{L_{1}})\,\rightarrow\,
			(K_{1},L_{1})
	\end{align*}
	%
	de la identidad
	\begin{align*}
		& (\geom{\subdiv{K_{1}}},\geom{\subdiv{L_{1}}})\,\rightarrow\,
			(\geom{K_{1}},\geom{L_{1}})
		\text{ .}
	\end{align*}
	%
	Dos aproximaciones de la identidad son, por
	\ref{thm:aproximacionessoncontiguas}, contiguas. Teniendo en cuenta
	esto, si
	\begin{math}
		\lambda:\,(\subdiv{K_{1}},\subdiv{L_{1}})\rightarrow
			(K_{1},L_{1})
	\end{math}
	es una aproximaci\'{o}n simplicial de la identidad y
	$\varphi:\,(K_{1},L_{1})\rightarrow (K_{2},L_{2})$ es una
	transformaci\'{o}n simplicial, la aplicaci\'{o}n
	\begin{align*}
		\subdiv{\clase{\varphi}} & \,=\,\clase{\varphi\circ\lambda}
			\,=\,\clase{\varphi}\circ\clase{\lambda}
	\end{align*}
	%
	est\'{a} bien definida y define una funci\'{o}n
	\begin{align*}
		\subdiv{\null} & \,:\,
			\contiguas{K_{1},L_{1}}{K_{2},L_{2}}\,\rightarrow\,
			\contiguas{\subdiv{K_{1}},\subdiv{L_{1}}}{K_{2},L_{2}}
		\text{ .}
	\end{align*}
	%
	Iterando este procedimiento, se obtiene una sucesi\'{o}n
	\begin{align*}
		\subdiv[n,n+1]{\null} & \,:\,
			\contiguas{\subdiv[n]{K_{1}},\subdiv[n]{L_{1}}}%
				{K_{2},L_{2}}\,\rightarrow\,
			\contiguas{\subdiv[n+1]{K_{1}},\subdiv[n+1]{L_{1}}}%
				{K_{2},L_{2}}
	\end{align*}
	%
	para $n\geq 0$. Para cada entero no negativo, elegimos alguna
	aproximaci\'{o}n
	\begin{align*}
		\lambda_{n+1,n} & \,:\,
			(\subdiv[n+1]{K_{1}},\subdiv[n+1]{L_{1}})
			\,\rightarrow\,
			(\subdiv[n]{K_{1}},\subdiv[n]{L_{1}})
	\end{align*}
	%
	de la identidad
	\begin{align*}
		\id & \,:\,
			(\geom{\subdiv[n+1]{K_{1}}},\geom{\subdiv[n+1]{L_{1}}})
			\,\rightarrow\,
			(\geom{\subdiv[n]{K_{1}}},\geom{\subdiv[n]{L_{1}}})
		\text{ .}
	\end{align*}
	%
	Entonces
	\begin{align*}
		\subdiv[n,n+1]{\clase{\varphi}} & \,=\,
			\clase{\varphi\circ\lambda_{n+1,n}}\,=\,
			\clase{\varphi}\circ\clase{\lambda_{n+1,n}} \,=\,
			\clase{\lambda_{n+1,n}}^{*}\big(\clase{\varphi}\big)
		\text{ .}
	\end{align*}
	%
	Notemos que, como $\lambda_{n+1,n}$ es una aproximaci\'{o}n
	simplicial de la identidad, la realizaci\'{o}n es homot\'{o}pica a
	la identidad: $\geom{\lambda_{n+1,n}}\simeq\id$. En particular, las
	clases de homotop\'{\i}a
	\begin{align*}
		\clase{\geom{\varphi\circ\lambda_{n+1,n}}} & \,=\,
			\clase{\geom{\varphi}}
	\end{align*}
	%
	coinciden.

	Dados $m\geq n$, sea
	\begin{align*}
		\lambda_{m,n} & \,:\,(\subdiv[m]{K_{1}},\subdiv[m]{L_{1}})
			\,\rightarrow\,(\subdiv[n]{K_{1}},\subdiv[n]{L_{1}})
	\end{align*}
	%
	la composici\'{o}n
	\begin{align*}
		\lambda_{m,n} & \,=\,\lambda_{n+1,n}\circ\lambda_{n+2,n+1}
			\circ\cdots\circ\lambda_{m,m-1}
	\end{align*}
	%
	y sea
	\begin{align*}
		\subdiv[n,m]{\null} & \,:\,
			\contiguas{\subdiv[m]{K_{1}},\subdiv[m]{L_{1}}}%
				{K_{2},L_{2}} \,\rightarrow\,
			\contiguas{\subdiv[n]{K_{1}},\subdiv[n]{L_{1}}}%
				{K_{2},L_{2}}
	\end{align*}
	%
	la composici\'{o}n
	\begin{align*}
		\subdiv[n,m]{\null} & \,=\,\subdiv[n,n+1]{\null}\circ
			\subdiv[n+1,n+2]{\null}\circ\cdots\circ
			\subdiv[m-1,m]{\null}
		\text{ .}
	\end{align*}
	%
	Entonces, por un lado,
	\begin{align*}
		\subdiv[n,m]{\clase{\varphi}} & \,=\,
			\clase{\varphi\circ\lambda_{m,n}}
		\text{ .}
	\end{align*}
	%
	Por otro lado, como $\lambda_{m,n}$ es una aproximaci\'{o}n de la
	identidad, $\geom{\lambda_{m,n}}\simeq\id$. Si
	\begin{math}
		\varphi:\,(\subdiv[n]{K_{1}},\subdiv[n]{L_{1}})\rightarrow
			(K_{2},L_{2})
	\end{math}~,
	entonces
	\begin{align*}
		\clase{\geom{\varphi\circ\lambda_{m,n}}} & \,=\,
			\clase{\geom{\varphi}}
		\text{ .}
	\end{align*}
	%
	
	Tomando el l\'{\i}mite directo, se obtiene un funtor
	\begin{align*}
		& \lim_{\to}\,\contiguas{\subdiv[n]{K_{1}},\subdiv[n]{L_{1}}}%
				{K_{2},L_{2}}
		\text{ ,}
	\end{align*}
	%
	contravariante en $(K_{1},L_{1})$ y covariante en $(K_{2},L_{2})$.
\end{obsLimiteDirectoDeSubdivisiones}

\begin{teoHomotopicasPorContiguas}\label{thm:homotopicasporcontiguas}
	Sea $K_{1}$ un complejo simplicial finito. Entonces existe una
	isomorfismo natural
	\begin{align*}
		\lim_{\to}\,\contiguas{\subdiv[n]{K_{1}},\subdiv[n]{L_{1}}}%
				{K_{2},L_{2}} & \,\simeq\,
			\homotopicas{\geom{K_{1}},\geom{L_{1}}}%
				{\geom{K_{2}},\geom{L_{2}}}
		\text{ .}
	\end{align*}
	%
\end{teoHomotopicasPorContiguas}

\begin{proof}
	Para definir una transformaci\'{o}n natural, es necesario definir,
	para cada par de subcomplejos $(K_{1},L_{1})$ y $(K_{2},L_{2})$,
	una funci\'{o}n del l\'{\i}mite directo en el conjunto de clases
	de homotop\'{\i}a de funciones de
	$(\geom{K_{1}},\geom{L_{1}})$ en $(\geom{K_{2}},\geom{L_{2}})$. Una
	funci\'{o}n definida en el l\'{\i}mite directo, equivale a una
	sucesi\'{o}n de funciones
	\begin{align*}
		f_{n} & \,:\,\contiguas{\subdiv[n]{K_{1}},\subdiv[n]{L_{1}}}%
				{K_{2},L_{2}}\,\rightarrow\,
			\homotopicas{\geom{K_{1}},\geom{L_{1}}}%
				{\geom{K_{2}},\geom{L_{2}}}
		\text{ ,}
	\end{align*}
	%
	para $n\geq 0$, tales que
	\begin{align*}
		f_{n} & \,=\,f_{n+1}\circ\subdiv[n,n+1]{\null}
		\text{ .}
	\end{align*}
	%
	Definimos $f_{n}$ como la funci\'{o}n
	\begin{align*}
		f_{n}\big(\clase{\varphi}\big) & \,=\,\clase{\geom{\varphi}}
		\text{ ,}
	\end{align*}
	%
	si
	\begin{math}
		\varphi:\,(\subdiv[n]{K_{1}},\subdiv[n]{L_{1}})\rightarrow
			(K_{2},L_{2})
	\end{math}~.
	Esta funci\'{o}n est\'{a} bien definida, por la observaci\'{o}n
	\ref{obs:contiguassonhomotopicas}. Entonces, por la observaci\'{o}n
	\ref{obs:limitedirectodesubdivisiones},
	\begin{align*}
		f_{n+1}\circ\subdiv[n,n+1]{\null}\big(\clase{\varphi}\big)
			& \,=\,\clase{\geom{\varphi\circ\lambda_{n+1,n}}}
			\,=\,\clase{\geom{\varphi}}\,=\,
			f_{n}\big(\clase{\varphi}\big)
		\text{ .}
	\end{align*}
	%
	Las funciones $f_{n}$ son naturales en el par $(K_{1},L_{1})$:
	dada $\psi_{1}:\,(K_{1},L_{1})\rightarrow (K'_{1},L'_{1})$,
	el diagrama
	\begin{center}
		\begin{tikzcd}
			\contiguas{\subdiv[n]{K'_{1}},\subdiv[n]{L'_{1}}}%
				{K_{2},L_{2}} \arrow[r,"f_{n}"]
					\arrow[d,"\clase{\psi}^{*}"'] &
			\homotopicas{\geom{K'_{1}},\geom{L'_{1}}}%
				{\geom{K_{2}},\geom{L_{2}}}
					\arrow[d,"\clase{\geom{\psi}}^{*}"] \\
			\contiguas{\subdiv[n]{K_{1}},\subdiv[n]{L_{1}}}%
				{K_{2},L_{2}} \arrow[r,"f_{n}"'] &
			\homotopicas{\geom{K_{1}},\geom{L_{1}}}%
				{\geom{K_{2}},\geom{L_{2}}}
		\end{tikzcd}
	\end{center}
	conmuta: dada $\varphi'$ una representante de una clase de
	contig\"{u}idad en el extremo superior izquierdo,
	\begin{align*}
		f_{n}\circ\clase{\psi}^{*}\big(\clase{\varphi'}\big) & \,=\,
			f_{n}\big(\clase{\varphi'\circ\psi}\big) \,=\,
			\clase{\geom{\varphi'\circ\psi}}
		\quad\text{y} \\
		\clase{\geom{\psi}}^{*}\circ f_{n}\big(\clase{\varphi'}\big)
			& \,=\,\clase{\geom{\psi}}^{*}
				\big(\clase{\geom{\varphi'}}\big) \,=\,
			\clase{\geom{\varphi'}}\circ\clase{\geom{\psi}}
		\text{ .}
	\end{align*}
	%
	De manera similar, se puede ver que la definici\'{o}n de $f_{n}$ es
	natural en $(K_{2},L_{2})$. Queda determinada, entonces, una
	transformaci\'{o}n natural
	\begin{align*}
		f & \,=\, \{f_{n}\}_{n\geq 0} \,:\,
			\lim_{\to}\,
			\contiguas{\subdiv[n]{K_{1}},\subdiv[n]{L_{1}}}%
				{K_{2},L_{2}} \,\rightarrow\,
			\homotopicas{\geom{K_{1}},\geom{L_{1}}}%
				{\geom{K_{2}},\geom{L_{2}}}
		\text{ .}
	\end{align*}
	%

	Veamos que $f$ es una biyecci\'{o}n natural. Sea
	\begin{math}
		g:\,(\geom{K_{1}},\geom{L_{1}})\rightarrow
			(\geom{K_{2}},\geom{L_{2}})
	\end{math}
	una funci\'{o}n continua y sea
	\begin{math}
		\varphi:\,(\subdiv[n]{K_{1}},\subdiv[n]{L_{1}})\rightarrow
			(K_{2},L_{2})
	\end{math}
	una aproximaci\'{o}n simplicial de $g$, cuya existencia est\'{a}
	garantizada por el teorema \ref{thm:existenciadeaproximaciones}.
	Entonces
	\begin{align*}
		f_{n}\big(\clase{\varphi}\big) & \,=\,\clase{\geom{\varphi}}	
			\,=\,\clase{g}
		\text{ .}
	\end{align*}
	%
	En particular, $f$ es sobreyectiva. Sean ahora
	\begin{align*}
		\varphi & \,:\,(\subdiv[n]{K_{1}},\subdiv[n]{L_{1}})
			\,\rightarrow\,(K_{2},L_{2}) \\
		\varphi' & \,:\,(\subdiv[n']{K_{1}},\subdiv[n']{L_{1}})
			\,\rightarrow\,(K_{2},L_{2})
	\end{align*}
	%
	transformaciones simpliciales tales que
	$\geom{\varphi}\simeq\geom{\varphi'}$, es decir,
	\begin{align*}
		f_{n}\big(\clase{\varphi}\big) & \,=\,
			\clase{\geom{\varphi}} \,=\,
			\clase{\geom{\varphi'}} \,=\,
			f_{n'}\big(\clase{\varphi'}\big)
	\end{align*}
	%
	Por el teorema \ref{thm:homotopicasadmitenaproximacionescontiguas},
	existe $m\geq n,n'$ y aproximaciones
	\begin{align*}
		\psi,\psi' & \,:\,(\subdiv[m]{K_{1}},\subdiv[m]{L_{1}})
			\,\rightarrow\,(K_{2},L_{2})
	\end{align*}
	%
	de $\varphi$ y, respectivamente, de $\varphi'$ que pertenecen a la
	misma clase de contig\"{u}idad \textit{\{revisar la demo del teo\}}.
	Ahora bien, como $\varphi$ es una aproximaci\'{o}n de $\geom{\varphi}$
	y $\lambda_{m,n}$ es una aproximaci\'{o}n de la identidad, la
	composici\'{o}n $\varphi\circ\lambda_{m,n}$ es, tambi\'{e}n, seg\'{u}n
	la observaci\'{o}n \ref{obs:composiciondeaproximacionesesaproximacion},
	una aproximaci\'{o}n simplicial de $\geom{\varphi}$. En particular,
	por el lema \ref{thm:aproximacionessoncontiguas}, las transformaciones
	$\varphi\circ\lambda_{m,n}$ y $\psi$ son contiguas. An\'{a}logamente,
	$\varphi'\circ\lambda_{m,n'}$ y $\psi'$ tambi\'{e}n son contiguas.
	Pero entonces
	\begin{align*}
		\subdiv[m-n]{\clase{\varphi}} & \,=\,
			\clase{\varphi\circ\lambda_{m,n}} \,=\,
			\clase{\varphi'\circ\lambda_{m,n'}} \,=\,
			\subdiv[m-n']{\clase{\varphi'}}
	\end{align*}
	%
	en $\contiguas{\subdiv[m]{K_{1}},\subdiv[m]{L_{1}}}{K_{2},L_{2}}$.
	Concluimos, as\'{\i}, que $f$ es inyectiva.
\end{proof}

\begin{coroCodominioNumerableHomotopicasNumerables}%
	\label{thm:codominionumerablehomotpicasnumerables}
	Sea $X$ un espacio topol\'{o}gico compacto y sea $Y$ el espacio de
	un complejo simplicial a lo sumo numerable. Sean $A\subset X$ y
	$B\subset Y$ subespacios. Entonces el conjunto de clases de
	homotop\'{\i}a de pares $\homotopicas{X,A}{Y,B}$ es a lo sumo
	numerable.
\end{coroCodominioNumerableHomotopicasNumerables}
