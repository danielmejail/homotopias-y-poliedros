\theoremstyle{plain}
\newtheorem{teoCompactoHausdorff}{Teorema}[section]
\newtheorem{teoTopologiaCoherente}[teoCompactoHausdorff]{Teorema}
\newtheorem{coroTopologiaCoherenteComplejos}[teoCompactoHausdorff]{Corolario}
\newtheorem{coroTopologiaCoherenteIdentidad}[teoCompactoHausdorff]{Corolario}
\newtheorem{propoRealizacionMorfismos}[teoCompactoHausdorff]{Proposici\'{o}n}
\newtheorem{coroTopologiaCoherenteSubcomplejos}[teoCompactoHausdorff]%
	{Corolario}

\theoremstyle{remark}
\newtheorem{obsTopologiaEnSimplices}[teoCompactoHausdorff]{Observaci\'{o}n}
\newtheorem{obsRealizacionMorfismos}[teoCompactoHausdorff]{Observaci\'{o}n}
\newtheorem{obsRealizacionUnionInterseccion}[teoCompactoHausdorff]%
	{Observaci\'{o}n}

%-------------

\subsection{El complejo geom\'{e}trico}
Sea $K=(K,V)$ un complejo simplicial no va\'{\i}o. Denotamos $\geom{K}$ al
conjunto de funciones $\alpha:\,V\rightarrow [0,1]$ que verifican:
\begin{itemize}
	\item[(i)] el conjunto $\{v\in V\,:\,\alpha(v)\not =0\}$ es un
		s\'{\i}mplice en $K$; y
	\item[(ii)] $\sum_{v\in V}\,\alpha(v)=1$.
\end{itemize}
%
Dicho de otra manera, $\geom{K}$ es el conjunto de combinaciones lineales
convexas de v\'{e}rtices de $K$. Si $K=\varnothing$, se define
$\geom{K}=\varnothing$. En el conjunto $\geom{K}$ definimos una m\'{e}trica:
sea $d:\,\geom{K}\times\geom{K}\rightarrow\bb{R}$ la funci\'{o}n dada por
\begin{align*}
	d(\alpha,\beta) & \,=\,\Big(\sum_{v\in V}\,
				|\alpha(v)-\beta(v)|^{2}\Big)^{1/2}
	\text{ .}
\end{align*}
%
Entonces la funci\'{o}n $d$ es sim\'{e}trica, $d(\alpha,\beta)\geq 0$ y es
igual a cero, si y s\'{o}lo si $\alpha(v)=\beta(v)$ para todo v\'{e}rtice $v$.
En cuanto a la desigualdad triangular, como el conjunto de v\'{e}rtices
$v$ tales que $\alpha(v)$ y $\beta(v)$ son no nulos es finito, la
desigualdad triangular se deduce del caso usual en un $\bb{R}$-espacio
vectorial de dimensi\'{o}n finita. Denotamos $\geom{K}_{d}$ al espacio
m\'{e}trico $(\geom{K},d)$.

Si $s\in K$ es un $q$-s\'{\i}mplice, definimos
\begin{align*}
	\geom{s} & \,=\,\Big\{\alpha\in\geom{K}\,:\,
		\alpha(v)=0\text{ , si }v\not\in s\Big\}
	\text{ ,}
\end{align*}
%
es decir, $\geom{s}$ es el subconjunto de funciones $\alpha\in\geom{K}$
que son nulas en todos los v\'{e}rtices que no pertenecen a $s$. En
t\'{e}rminos de la definici\'{o}n anterior, $\geom{s}=\geom{\caras{s}}$. Si
$s=\{\lista[0]{v}{q}\}$, sea $E$ el $\bb{R}$-espacio vectorial generado
por los $q+1$ v\'{e}rtices de $s$. Sea $\simp{\lista[0]{v}{q}}$ el conjunto
de puntos $x\in E$ de la forma
\begin{align*}
	x & \,=\,\sum_{i=0}^{q}\,t^{i}v_{i}
\end{align*}
%
tales que $t^{i}\geq 0$ y $\sum_{i=0}^{q}\,t^{i}=1$. Es decir,
$\simp{\lista[0]{v}{q}}$ es el conjunto de combinaciones lineales
convexas de los puntos correspondientes a los v\'{e}rtices $v_{i}$ en $E$,
o, lo que es lo mismo, el conjunto convexo m\'{a}s chico que los contiene,
la \emph{c\'{a}scara convexa}. Se ve entonces que el subconjunto
$\geom{s}\subset\geom{K}$ est\'{a} en correspondencia con el subconjunto
$\simp{\lista[0]{v}{q}}$ de $E$ v\'{\i}a
$\Phi:\,\alpha\mapsto (\alpha(v_{0}),\,\dots,\,\alpha(v_{q}))$.

En tanto $\bb{R}$-espacio vectorial de dimensi\'{o}n finita, $E$ admite una
\'{u}nica topolog\'{\i}a con respecto a la cual las operaciones de suma y
producto por escalares son continuas y la m\'{e}trica euclidea es compatible
con esta topolog\'{\i}a. Por otro lado, $\geom{s}\subset\geom{K}$ hereda la
m\'{e}trica $d$ definida anteriormente y
$\Phi:\,\geom{s}\rightarrow\simp{\lista[0]{v}{q}}$ es una isometr\'{\i}a.
En particular, $\geom{s}$ con la topolog\'{\i}a dada por la m\'{e}trica $d$
es homeomorfo al conjunto convexo y compacto $\simp{\lista[0]{v}{q}}$.
Notemos que este espacio es tambi\'{e}n Hausdorff.

\subsection{Repaso de Topolog\'{\i}a}
Con el prop\'{o}sito de darle una topolog\'{\i}a m\'{a}s natural al
conjunto $\geom{K}$, recordamos dos teoremas b\'{a}sicos:

\begin{teoCompactoHausdorff}[de rigidez]\label{thm:compactohausdorff}
	Sea $f:\,X\rightarrow Y$ una funci\'{o}n continua e inyectiva
	definida en un espacio topol\'{o}gico compacto $X$ y codominio
	un espacio topol\'{o}gico Hausdorff. Entonces $f$ determina un
	homeomorfismo entre $X$ y el subespacio $f(X)\subset Y$.
\end{teoCompactoHausdorff}

De este teorema se deduce que, si $\tau$ es una topolog\'{\i}a en un conjunto
$X$ con respecto a la cual $X$ resulta compacto y Hausdorff, entonces
no existe una topolog\'{\i}a m\'{a}s d\'{e}bil en $X$ con respecto a la
cual $X$ sea Hausdorff, ni existe una topolog\'{\i}a m\'{a}s fuerte con
respecto a la cual $X$ sea compacto. Dicho de otra manera, dos topolog\'{\i}as
comparables, compactas y Hausdorff deben ser iguales.

Dado un conjunto $X$ y una familia indexada de espacios topol\'{o}gicos
$\{X_{j}\}_{j}$ y funciones $\{g_{j}:\,X_{j}\rightarrow X\}_{j}$, la
\emph{topolog\'{\i}a coinducida por las funciones $\{g_{j}\}_{j}$}
es la topolog\'{\i}a m\'{a}s fina/fuerte en $X$ tal que las funciones $g_{j}$
resulten ser continuas. Esta topolog\'{\i}a est\'{a} caracterizada por la
propiedad de que, si $Y$ es un espacio topol\'{o}gico, una funci\'{o}n
$f:\,X\rightarrow Y$ es continua, si y s\'{o}lo si las composiciones
$f\circ g_{j}:\,X_{j}\rightarrow Y$ lo son para todo $j$.

\begin{teoTopologiaCoherente}\label{thm:topologiacoherente}
	Sea $X$ un conjunto y sea $\{A_{j}\}_{j}$ una familia indexada de
	espacios topol\'{o}gicos contenidos en $X$. Si
	\begin{itemize}
		\item[(i)] la intersecci\'{o}n $A_{j}\cap A_{j'}$ es cerrada
			en $A_{j}$ y en $A_{j'}$ para todo par $j,j'$ y
		\item[(ii)] la topolog\'{\i}a inducida en $A_{j}\cap A_{j'}$
			como subconjunto de $A_{j}$ coincide con la inducida
			en tanto subconjunto de $A_{j'}$,
	\end{itemize}
	%
	entonces la topolog\'{\i}a determinada en $X$ por las inclusiones
	$\inc:\,A_{j}\hookrightarrow X$ se caracteriza por ser la \'{u}nica
	topolog\'{\i}a en $X$ tal que cada subconjunto $A_{j}$ sea
	cerrado en $X$ y que sea coherente con las topolog\'{\i}as de los
	espacios $A_{j}$. Lo mismo es cierto si se requiere que las
	intersecciones $A_{j}\cap A_{j'}$ sean abiertas en $A_{j}$ y en
	$A_{j'}$, en lugar de cerradas, y los subconjuntos $A_{j}$ sean
	abiertos en $X$, en lugar de cerrados. La propiedad de coherencia
	de la topolog\'{\i}a de $X$ con respecto a la de los $A_{j}$
	significa que un subconjunto $B\subset X$ es cerrado (o abierto), si 
	$B\cap A_{j}$ es cerrado para todo $j$ (respectivamente, abierto).
\end{teoTopologiaCoherente}

\subsection{Una topolog\'{\i}a coherente}
Sea $K$ un complejo simplicial. En lugar de darle a $\geom{K}$ la
topolog\'{\i}a inducida por la m\'{e}trica $d$, definiremos una
topolog\'{\i}a coherente con la topolog\'{\i}a de los s\'{\i}mplices. Lo que
buscamos es que la noci\'{o}n de continuidad de una funci\'{o}n definida
en $\geom{K}$ est\'{e} determinada por lo que ocurre con la restricci\'{o}n
de dicha funci\'{o}n a los subconjuntos $\geom{s}$.

Dados s\'{\i}mplices $s,s'\in K$, entonces, o bien $s\cap s'=\varnothing$,
o bien $s\cap s'$ es un subconjunto no vac\'{\i}o tanto de $s$ como de $s'$.
En el primer caso, $\geom{s}\cap\geom{s'}=\varnothing$, pues la \'{u}nica
funci\'{o}n en los v\'{e}rtices de $K$ que se anula fuera de $s$ y fuera
de $s'$ es la funci\'{o}n cero. En el segundo caso, $s\cap s'$ es una cara
de $s$ y una cara de $s'$, tambi\'{e}n. En particular, si
$s\cap s'\not=\varnothing$, vale que $\geom{s\cap s'}=\geom{s}\cap\geom{s'}$,
pues las funciones que se anulan fuera de $s$ y fuera de $s'$ son las
funciones que se anulan fuera de $s\cap s'$. En cualquiera de los dos casos,
$\geom{s}\cap\geom{s'}$ tiene una toplog\'{\i}a comacta y Hausdorff
(proveniente de la m\'{e}trica $d$, si la intersecci\'{o}n es no vac\'{\i}a).
Adem\'{a}s, las inclusiones en $\geom{s}_{d}$ y en $\geom{s'}_{d}$ son
continuas, con lo que la topolog\'{\i}a en $\geom{s}\cap\geom{s'}$ es
comparable a las heredadas de $\geom{s}_{d}$ y de $\geom{s'}_{d}$. Pero
la intersecci\'{o}n $\geom{s}\cap\geom{s'}$ es cerrada tanto como subespacio
de $\geom{s}$ como en tanto subespacio de $\geom{s'}$ y, por en consecuencia,
un espacio compacto y Hausdorff con cualquiera de estas dos topolog\'{\i}as.
En definitiva, por \ref{thm:compactohausdorff}, la topolog\'{\i}a inducida
en $\geom{s}\cap\geom{s'}$ como subespacio de $\geom{s}_{d}$ coincide con
la inducida como subespacio de $\geom{s'}_{d}$. Notemos que el conjunto
$\geom{K}$ y la colecci\'{o}n de espacios topol\'{o}gicos
$\big\{\geom{s}_{d}\,:\,s\in K\big\}$ verifican las condiciones
\emph{(i)} y \emph{(ii)} del teorema \ref{thm:topologiacoherente}.
De esta manera, queda determinada una topolog\'{\i}a en $\geom{K}$ que
denominaremos \emph{coherente}.

En el contexto de espacios topol\'{o}gicos, denotaremos $\geom{K}$ al espacio
topol\'{o}gico cuyo conjunto subyacente es $\geom{K}$ y cuya topolog\'{\i}a
es la topolog\'{\i}a coherente reci\'{e}n definida.

\begin{obsTopologiaEnSimplices}\label{obs:topologiaensimplices}
	La igualdad de conjuntos $\geom{s}=\geom{\caras{s}}$ da lugar a
	una identificaci\'{o}n entre los espacios topol\'{o}gicos
	$\geom{\caras{s}}$ y $\geom{s}_{d}$. Denotaremos todos estos
	espacios por $\geom{s}$. Adem\'{a}s, los espacios
	$\geom{\carasp{s}}$ y $\geom{\carasp{s}}_{d}$ obtenidos a partir
	del complejo $\carasp{s}$ de paras propias de un s\'{\i}mplice
	$s$ tambi\'{e}n son homeomorfos. En particular, $\geom{s}$ y
	$\geom{\carasp{s}}$ son subespacios del espacio vectorial
	topol\'{o}gico generado por los v\'{e}rtices del s\'{\i}mplice $s$ y
	\begin{align*}
		\geom{\carasp{s}} & \,=\,\borde\big(\geom{s}\big)
		\text{ .}
	\end{align*}
	%
\end{obsTopologiaEnSimplices}

De la definici\'{o}n de la topolog\'{\i}a coherente de un complejo
simplicial podemos deducir los siguientes corolarios.

\begin{coroTopologiaCoherenteComplejos}\label{thm:topologiacoherentecomplejos}
	Sea $K$ un complejo simplicial. Entonces una funci\'{o}n
	$f:\,\geom{K}\rightarrow X$ es continua, si y s\'{o}lo si las
	restricciones $f|_{\geom{s}}$ son continuas. Equivalentemente,
	$f:\,\geom{K}\rightarrow X$ es continua, si y s\'{o}lo si
	$f|_{\geom{\qesq{q}{K}}}$ es continua.
\end{coroTopologiaCoherenteComplejos}

\begin{coroTopologiaCoherenteIdentidad}\label{thm:topologiacoherenteidentidad}
	La identidad $\geom{K}\rightarrow\geom{K}_{d}$ es continua
\end{coroTopologiaCoherenteIdentidad}

Si $L\subset K$ es un subcomplejo, entonces, conjust\'{\i}sticamente,
$\geom{L}\subset\geom{K}$: si $\alpha\in\geom{L}$, entonces
$\{\alpha\not=0\}\in L$ y, como $L\hookrightarrow K$ es simplicial,
$\{\alpha\not=0\}$ es un s\'{\i}mplice de $K$.

\begin{obsRealizacionMorfismos}\label{obs:realizacionmorfismos}
	Un poco m\'{a}s en general, si $\varphi:\,K\rightarrow K'$ es una
	transformaci\'{o}n simplicial y $\alpha\in\geom{K}$, entonces
	$\{v\in V\,:\,\alpha(v)\not =0\}$ es un s\'{\i}mplice en $K$ y,
	aplicando $\varphi$, el conjunto
	$\{\varphi(v)\in V'\,:\,\alpha(v)\not =0\}$ es un s\'{\i}mplice de
	$K'$. Definimos $\alpha':\,V'\rightarrow[0,1]$ por
	\begin{align*}
		\alpha'(v') & \,=\,
			\begin{cases}
				\sum_{\varphi(v)=v'}\,\alpha(v) &
					\quad\text{si } v'\in\img\,\varphi \\
				0 & \quad\text{si } v'\not\in\img\,\varphi
			\end{cases}
		\text{ .}
	\end{align*}
	%
	Como los valores de $\alpha$ son no negativos, se cumple que
	\begin{align*}
		\{\alpha'\not =0\} & \,=\,\varphi\big(\{\alpha\not =0\}\big)
		\text{ .}
	\end{align*}
	%
	Definimos $\geom{\varphi}(\alpha)=\alpha'$.
\end{obsRealizacionMorfismos}

\begin{propoRealizacionMorfismos}\label{thm:realizacionmorfismos}
	Sea $\varphi:\,K\rightarrow K'$ una transformaci\'{o}n simplicial.
	La funci\'{o}n $\geom{\varphi}:\,\geom{K}\rightarrow\geom{K'}$ es
	continua, tanto con respecto a la topolog\'{\i}a inducida por la
	m\'{e}trica euclidea en $\geom{K}$ y en $\geom{K'}$, como con
	respecto a la topolog\'{\i}a coherente. Usaremos $\geom{\varphi}_{d}$
	para referirnos a la funci\'{o}n entre los espacios m\'{e}tricos y
	$\geom{\varphi}$ para referirnos a la funci\'{o}n entre los
	espacios con la topolog\'{\i}a determinada por sus s\'{\i}mplices.
\end{propoRealizacionMorfismos}

\begin{proof}
	Veamos que $\geom{\varphi}:\,\geom{K}_{d}\rightarrow\geom{K'}_{d}$
	es continua con respecto a las m\'{e}tricas en los complejos
	$K$ y $K'$. Sea entonces $\alpha\in\geom{K}$, sea $\epsilon>0$ y
	sean $\lista{v}{r}$ los v\'{e}rtices del s\'{\i}mplice
	$\{\alpha\not=0\}\in K$. Si $\beta\in\geom{K}$ y $v'\in V'$, entonces
	\begin{align*}
		\Big|\sum_{v\in V|\varphi(v)=v'}\,\alpha(v)-\beta(v)\Big|^{2}
			& \,=\,\big|(\geom{\varphi}\alpha)(v)-
				(\geom{\varphi}\beta)(v)\big|^{2}
			\,\leq\,\sum_{\varphi(v)=v'}\,
				\big|\alpha(v)-\beta(v)\big|
		\text{ ,}
	\end{align*}
	%
	pues
	\begin{align*}
		-1 & \,\leq\,
			(\geom{\varphi}\alpha)(v)-(\geom{\varphi}\beta)(v)
			\,\leq\, 1
		\text{ .}
	\end{align*}
	%
	Pero entonces
	\begin{align*}
		d(\geom{\varphi}\alpha,\geom{\varphi}\beta)^{2} & \,=\,
			\sum_{v'\in V'}\,\Big|\sum_{\varphi(v)=v'}\,
				\alpha(v)-\beta(v)\Big|^{2} \\
		& \,\leq\,\sum_{v'}\,\sum_{\varphi(v)=v'}\,
				\big|\alpha(v)-\beta(v)\big|
			\,=\,\sum_{v}\,\big|\alpha(v)-\beta(v)\big|
		\text{ .}
	\end{align*}
	%
	Si $d(\alpha,\beta)<\delta$ para cierto $\delta>0$, entonces
	$\big|\alpha(v_{i})-\beta(v_{i})\big|<\delta$ para $i=1,\,\dots,\,r$.
	En particular,
	\begin{align*}
		d(\geom{\varphi}\alpha,\geom{\varphi}\beta)^{2} & \,\leq\,
			\sum_{i=1}^{r}\,\big|\alpha(v_{i})-\beta(v_{i})\big|
			\,+\, \sum_{v\not=v_{i}}\,\beta(v) \\
		& \,=\,\sum_{i=1}^{r}\,\big|\alpha(v_{i})-\beta(v_{i})\big|
			\,+\,\Big(1 - \sum_{i=1}^{r}\,\beta(v_{i})\Big) \\
		& \,=\,\sum_{i=1}^{r}\,\big|\alpha(v_{i})-\beta(v_{i})\big|
			\,+\,\sum_{i=1}^{r}\,
				\big(\alpha(v_{i})-\beta(v_{i})\big) \\
		& \,\leq\,\sum_{i=1}^{r}\,\big|\alpha(v_{i})-\beta(v_{i})\big|
			\,+\,\sqrt{r}\cdot\Big(\sum_{i=1}^{r}\,
				\big|\alpha(v_{i})-\beta(v_{i})\big|^{2}
				\Big)^{1/2} \\
		& \,<\, r\cdot\delta\,+\,\sqrt{r}\cdot\delta
		\text{ .}
	\end{align*}
	%
	De esto se deduce que $\geom{\varphi}$ es continua con respecto a
	las m\'{e}tricas euclideas en los complejos. En particular,
	si $L\subset K$ es un subcomplejo, la inclusi\'{o}n
	$\geom{L}_{d}\hookrightarrow\geom{K}_{d}$ es continua.

	Veamos ahora que la misma funci\'{o}n
	$\geom{\varphi}:\,\geom{K}\rightarrow\geom{K'}$ es continua con
	respecto a la topolog\'{\i}a coherente. Dado un s\'{\i}mplice
	$s\in K$, hay que ver que
	$\geom{\varphi}|_{\geom{s}}:\,\geom{s}\rightarrow\geom{K'}$ sea
	continua. Determinemos primero cu\'{a}l es la imagen de esta
	funci\'{o}n: por un lado, $\varphi(s)\in K'$ es un s\'{\i}mplice,
	por definici\'{o}n de $\varphi$, y, por otro,
	\begin{align*}
		\geom{\varphi}(\geom{s}) & \,=\,
			\big\{\geom{\varphi}\alpha\,:\,
				\alpha(v)=0\text{ , si }v\not\in s\big\}
		\text{ .}
	\end{align*}
	%
	Ahora, si $\alpha\in\geom{\varphi}(\geom{s})$ y $v'\not\in\varphi(s)$,
	entonces
	\begin{align*}
		(\geom{\varphi}\alpha)(v') & \,=\,
			\sum_{\varphi(v)=v'}\,\alpha(v) \,=\,
			\sum_{v\in s|\varphi(v)=v'}\,\alpha(v)\,=\,0
		\text{ .}
	\end{align*}
	%
	De esto se deduce que $\geom{\varphi}\alpha\in\geom{\varphi(s)}$ y
	$\geom{\varphi}(\geom{s})\subset\geom{\varphi(s)}$.
	(Esto ya es suficiente para concluir que $\geom{\varphi}|_{\geom{s}}$
	es continua). Rec\'{\i}procamente, si $\varphi(s)=\{\lista[0]{v'}{q}\}$
	sean $\lista[0]{v}{q}\in s$ tales que $\varphi(v_{i})=v'_{i}$. Si
	$\alpha'\in\geom{\varphi(s)}$, entonces $\alpha'(v'_{i})\geq 0$ y
	$\alpha'(v')=0$, si $v'\not=v'_{i}$ para todo $i$. Sea
	$\alpha:\,V\rightarrow [0,1]$ la funci\'{o}n
	\begin{align*}
		\alpha(v) & \,=\,
			\begin{cases}
				\alpha'(v'_{i}) &\quad\text{ si } v=v_{i} \\
				0 & \quad\text{ si } v\not =v_{i}
					\text{ para todo } i
			\end{cases}
		\text{ .}
	\end{align*}
	%
	Entonces $\{\alpha\not=0\}\subset\{\lista[0]{v}{q}\}\subset s$ y
	$\sum_{v}\,\alpha(v)=1$. En particular, $\{\alpha\not =0\}\in K$
	y $\alpha\in\geom{K}$. Pero, adem\'{a}s, $\alpha(v)=0$, si
	$v\not\in s$, con lo que $\alpha\in\geom{s}$ y, por definici\'{o}n,
	$\geom{\varphi}\alpha=\alpha'$. En definitiva,
	$\geom{\varphi(s)}\subset\geom{\varphi}(\geom{s})$ y
	\begin{align*}
		\geom{\varphi}(\geom{s}) & \,=\,\geom{\varphi(s)}
		\text{ .}
	\end{align*}
	%

	Dado que $\geom{\varphi}(\geom{s})\subset\geom{\varphi(s)}$ y
	que $\geom{\varphi(s)}\subset\geom{K'}$ es subespacio (cerrado)
	porque $\varphi(s)\in K'$ es un s\'{\i}mplice, la restricci\'{o}n
	$\geom{\varphi}:\,\geom{s}\rightarrow\geom{K'}$ es continua, si y
	s\'{o}lo si la correstricci\'{o}n
	\begin{align*}
		\geom{\varphi} \,:\,\geom{s}\,\rightarrow\,
			\geom{\varphi(s)}
	\end{align*}
	%
	es continua. Pero $\geom{s}$ y $\geom{\varphi(s)}$ tienen
	la topolog\'{\i}a dada por la m\'{e}trica euclidea y ya vimos
	que $\geom{\varphi}_{d}:\,\geom{s}_{d}\rightarrow\geom{\varphi(s)}_{d}$
	es continua. En definitiva,
	$\geom{\varphi}:\,\geom{s}\rightarrow\geom{\varphi(s)}$ es continua
	y $\geom{\varphi}:\,\geom{s}\rightarrow\geom{K'}$. Como
	$s\in K$ era arbitrario, por \ref{thm:topologiacoherentecomplejos},
	deducimos que $\geom{\varphi}:\,\geom{K}\rightarrow\geom{K'}$ es
	continua.
\end{proof}

\begin{coroTopologiaCoherenteSubcomplejos}%
	\label{thm:topologiacoherentesubcomplejos}
	Si $L\subset K$ es un subcomplejo, $\geom{L}_{d}$ es un subespacio
	cerrado de $\geom{K}_{d}$. En particular, $\geom{L}$ es un
	subespacio cerrado de $\geom{K}$.
\end{coroTopologiaCoherenteSubcomplejos}

\begin{proof}
	En primer lugar, la m\'{e}trica en $\geom{L}_{d}$ es la
	restricci\'{o}n de la m\'{e}trica en $\geom{K}_{d}$, con lo que
	$\geom{L}_{d}$ es un subespaico m\'{e}trico de $\geom{K}_{d}$
	y, en particular, un subespacio topol\'{o}gico.

	Sea $\alpha\in\geom{K}_{d}$ un elemento en la clausura de
	$\geom{L}_{d}$. Por definici\'{o}n, $\alpha(v)\geq 0$ para todo
	$v\in\qesq{0}{K}$ y $\alpha(v)>0$ s\'{o}lo para finitos
	v\'{e}rtices de $K$. Sean $\lista{v}{r}$ los v\'{e}rtices tales
	que $\alpha(v_{i})>0$, es decir,
	\begin{align*}
		\sum_{j=1}^{r}\,\alpha(v_{j}) & \,=\,1
	\end{align*}
	%
	Sea $a=\min\,\{\alpha(v_{i})\}_{i}>0$ y sea $\epsilon<a/2$ un
	n\'{u}mero positivo. Por hip\'{o}tesis, existe $\beta\in\geom{L}$
	tal que $d(\alpha,\beta)<\epsilon$. En particular,
	\begin{align*}
		\beta(v_{i}) & \,>\,\alpha(v_{i})-\epsilon\,>\,a/2\,>\,0
		\text{ .}
	\end{align*}
	%
	Como $\beta\in\geom{L}$, esto implica, por un lado, que los
	$v_{i}$ son v\'{e}rtices en el subcomplejo $L$ y, por otro, que
	\begin{align*}
		\{\beta\not =0\} & \,=\,\{\lista{v}{r}\}\,\cup\,B
		\text{ ,}
	\end{align*}
	%
	donde $B\subset\qesq{0}{L}$ es alg\'{u}n subconjunto (posiblemente
	vac\'{\i}o) de v\'{e}rtices de $L$. Como $\{\beta\not =0\}$ es
	un s\'{\i}mplice en $L$, se deduce que $\{\lista{v}{r}\}$
	tambi\'{e}n lo es. En conclusi\'{o}n, $\alpha\in\geom{L}$ y
	$\geom{L}_{d}$ es cerrado en $\geom{K}_{d}$.

	La \'{u}ltima afirmaci\'{o}n se deduce de
	\ref{thm:topologiacoherenteidentidad} y de que
	$\geom{L}\subset\geom{K}$ es un subespacio topol\'{o}gico (los
	s\'{\i}mplices que determinan la topolog\'{\i}a en $\geom{L}$ son
	los s\'{\i}mplices que determinan la topolog\'{\i}a en $\geom{K}$
	que pertenecen al subcomplejo $L$).
\end{proof}

\begin{obsRealizacionUnionInterseccion}\label{obs:realizacionunioninterseccion}
	Sea $K$ un complejo simplicial y sea $\{L_{i}\}_{i}$ una familia
	de subcomplejos de $K$. Entnonces
	\begin{math}
		\geom{\bigcup_{i}\,L_{i}} = \bigcup_{i}\,\geom{L_{i}}
	\end{math}
	y
	\begin{math}
		\geom{\bigcap_{i}\,L_{i}} = \bigcap_{i}\,\geom{L_{i}}
	\end{math}~.
\end{obsRealizacionUnionInterseccion}

Las aplicaciones $K\mapsto\geom{K}$ en complejos simpliciales y
$\varphi\mapsto\geom{\varphi}$ en transformaciones simpliciales
define un funtor de la categor\'{\i}a de complejos simpliciales en la
categor\'{\i}a de espacios topol\'{o}gicos. Lo mismo es cierto para las
aplicaciones $K\mapsto\geom{K}_{d}$ y $\varphi\mapsto\geom{\varphi}_{d}$.
M\'{a}s aun, como la identidad $\geom{K}\rightarrow\geom{K}_{d}$ es continua
y los diagramas
\begin{center}
	\begin{tikzcd}
		\geom{K} \arrow[r] \arrow[d,"\geom{\varphi}"'] &
			\geom{K}_{d} \arrow[d,"\geom{\varphi}_{d}"] \\
		\geom{K'} \arrow[r] & \geom{K'}_{d}
	\end{tikzcd}
\end{center}
son diagramas conmutativos en la categor\'{\i}a de espacios topol\'{o}gicos,
la identidad $\geom{K}\rightarrow\geom{K}_{d}$ determina una transformaci\'{o}n
natural $\geom{\cdot}\rightarrow\geom{\cdot}_{d}$ entre estos funtores.
Finalmente, podemos extender estos funtores a la categor\'{\i}a de pares
simpliciales: a un par simplicial $\varphi:\,K\rightarrow K'$ le
asociamos el par de espacios topol\'{o}gicos
$\geom{\varphi}:\,\geom{K}\rightarrow\geom{K'}$ (o bien $\geom{\varphi}_{d}$);
a un morfismo de pares simpliciales $(\psi',\psi)$ le asociamos el
morfismo de pares de espacios topol\'{o}gicos $(\geom{\psi'},\geom{\psi})$
(o, respectivamente, $(\geom{\psi'}_{d},\geom{\psi}_{d})$):
\begin{center}
\begin{tikzcd}
	K_{1} \arrow[r,"\varphi_{1}"] \arrow[d,"\psi"'] &
		K_{1}' \arrow[d,"\psi'"] \\
	K_{2} \arrow[r,"\varphi_{2}"'] & K'_{2}
\end{tikzcd}
	\qquad\begin{math} \longmapsto\end{math}\qquad
\begin{tikzcd}
	\geom{K_{1}} \arrow[r,"\geom{\varphi_{1}}"] \arrow[d,"\geom{\psi}"'] &
		\geom{K'_{1}} \arrow[d,"\geom{\psi'}"] \\
	\geom{K_{2}} \arrow[r,"\geom{\varphi_{2}}"'] & \geom{K'_{2}}
\end{tikzcd}
\end{center}

Una \emph{triangulaci\'{o}n} de un espacio topol\'{o}gico $X$ consiste en un
par $(K,f)$, donde $K$ es un complejo simplicial y $f:\,\geom{K}\rightarrow X$
es un homeomorfismo. An\'{a}logamente, definimos una traingulaci\'{o}n
de un par de espacios $g:\,X\rightarrow X'$ como un par simplicial
$\varphi:\,K\rightarrow K'$ junto con un homeomorfismo de pares
$(f',f):\,\geom{\varphi}\rightarrow g$, es decir, un par de homeomorfismos
$f:\,\geom{K}\rightarrow X$ y $f':\,K'\rightarrow X'$ tales que el diagrama
\begin{center}
\begin{tikzcd}
	\geom{K}\arrow[r,"\geom{\varphi}"] \arrow[d,"f"'] &
		\geom{K'} \arrow[d,"f'"] \\
	X \arrow[r,"g"'] & X'
\end{tikzcd}
\end{center}
conmuta. Notemos que, si el par $(X,A)$ consiste est\'{a} compuesto por
un espacio topol\'{o}gico $X$ y un subespacio $A\subset X$, entonces,
de existir una triangulaci\'{o}n $(K,L)$ la transformaci\'{o}n simplicial
correspondiente $L\rightarrow K$ debe ser (isomorfa a) un subcomplejo.
En particular, podemos ver el par $(\geom{K},\geom{L})$ como un par
topol\'{o}gico compuesto por el espacio $\geom{K}$ y un subespacio
$\geom{L}\subset\geom{K}$. En tal caso, los homeomorfismos
$f:\,\geom{L}\rightarrow A$ y $f':\,\geom{K}\rightarrow X$ est\'{a}n
forzados a cumplir $f=f'|_{\geom{L}}$.
