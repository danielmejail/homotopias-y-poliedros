\theoremstyle{plain}
\newtheorem{lemaInmersionDeUnSimplice}{Lema}[section]
\newtheorem{lemaLocalmenteFinitoLocalmenteFinito}[lemaInmersionDeUnSimplice]%
	{Lema}
\newtheorem{teoEquivalenciasLocalmenteFinito}[lemaInmersionDeUnSimplice]%
	{Teorema}
\newtheorem{teoInmersionDeUnComplejo}[lemaInmersionDeUnSimplice]{Teorema}

\theoremstyle{remark}

%-------------

\begin{lemaInmersionDeUnSimplice}\label{thm:inmersiondeunsimplice}
	Una funci\'{o}n lineal $f:\,\geom{s}\rightarrow\bb{R}^{n}$ es una
	inmersi\'{o}n si y s\'{o}lo si el conjunto $\{f(v)\}_{v\in s}$
	es un conjunto en posici\'{o}n general, af\'{\i}nmente independiente.
\end{lemaInmersionDeUnSimplice}

\begin{proof}
	Sea $s=\{\lista[0]{v}{q}$ y sea $p_{i}=f(\alpha_{v_{i}})$ el punto
	correspondiente al v\'{e}rtice $v_{i}$. Supongamos que existen
	n\'{u}meros reales $\lista*[0]{t}{q}$ no todos nulos tales que
	$\sum_{i=0}^{q}\,t^{i}p_{i}=0$ y $\sum_{i=0}^{q}\,t^{i}=0$ y
	supongamos, sin p\'{e}rdida de generalidad, que $t^{i}\geq 0$ para
	$i\geq i_{0}$ y $t^{i}<0$, si $i<i_{0}$. Entonces
	\begin{align*}
		p & \,:=\,\sum_{i<i_{0}}\,(-t^{i})\,p_{i} \,=\,
			\sum_{i\geq i_{0}}\,t^{i}\,p_{i} \quad\text{y} \\
		a & \,:=\,\sum_{i<i_{0}}\,(-t^{i}) \,=\,
			\sum_{i\geq i_{0}}\,t^{i}
		\text{ .}
	\end{align*}
	%
	Sean $\alpha,\beta\in\geom{s}$ los puntos dados por
	\begin{align*}
		\alpha & \,=\,\sum_{i<i_{0}}\,\frac{-t^{i}}{a}\,
				\alpha_{v_{i}} \quad\text{y} \\
		\beta & \,=\,\sum_{i\geq i_{0}}\,\frac{t^{i}}{a}\,
				\alpha_{v_{i}}
		\text{ .}
	\end{align*}
	%
	Entonces $f(\alpha)=f(\beta)$, por linealidad, y $f$ no es inyectiva.
	Rec\'{\i}procamoente, si $\alpha,\beta\in\geom{s}$,
	$\alpha\not=\beta$ y $f(\alpha)=f(\beta)$, supongamos que
	$\alpha^{i_{0}}\not=\beta^{i_{0}}$. Entonces
	\begin{align*}
		\sum_{i=0}^{q}\,\alpha^{i}\,p_{i} & \,=\,f(\alpha) \,=\,
			f(\beta) \,=\,\sum_{i=0}^{q}\,\beta^{i}\,p_{i}
			\quad\text{y} \\
		\sum_{i=0}^{q}\,\alpha^{i} & \,=\,1\,=\,
			\sum_{i=0}^{q}\,\beta^{i}
		\text{ .}
	\end{align*}
	%
	Tomando la diferncia, los coeficientes $\alpha^{i}-\beta^{i}$ no son
	todos nulos y
	\begin{align*}
		\sum_{i=0}^{q}\,(\alpha^{i}-\beta^{i})\,p_{i} & \,=\,0
			\quad\text{y} \\
		\sum_{i=0}^{q}\,(\alpha^{i}-\beta^{i}) & \,=\,0
		\text{ ,}
	\end{align*}
	%
	con lo cual, los puntos $\{\lista[0]{p}{q}\}$ no est\'{a}n en
	posici\'{o}n general.
\end{proof}

Sea $\alpha\in\geom{K}$. Entonces $\alpha\in\simpinterior{s}$ para
alg\'{u}n s\'{\i}mplice $s\in K$ y $\alpha\estrella{v}$ para alg\'{u}n
v\'{e}rtice $v$ de $K$. Este v\'{e}rtice, si $K$ es un complejo localmente
finito, pertenece a finitos s\'{\i}mplices. Sea $C=\bigcup_{v\in s}\,\geom{s}$.
Entonces $\alpha\in C$ y $C$ es una uni\'{o}n finita de subconjuntos
compactos de $\geom{K}_{d}$ y, por lo tanto, compacto. Si
$L=\bigcup_{v\in s}\,\caras{s}$, entonces $L$ es un subcomplejo finito de $K$
y $\alpha\in\estrella{v}\subset\geom{L}_{d}$. Esto demuestra el siguiente lema.
(Notemos que tambi\'{e}n podemos incluir $C$ en la uni\'{o}n de una cantidad
finita de s\'{\i}mplices ``abiertos'' $\simpinterior{s}$).

\begin{lemaLocalmenteFinitoLocalmenteFinito}
	\label{thm:localmentefinitolocalmentefinito}
	Sea $K$ un complejo simplicial localmente finito. Todo punto
	$\alpha\in \geom{K}_{d}$ tiene un entorno de la forma
	$\geom{L}_{d}$ para alg\'{u}n subcomplejo finito $L\subset K$.
	Es decir, $\alpha\in U$ para cierto abierto $U\subset\geom{L}_{d}$.
\end{lemaLocalmenteFinitoLocalmenteFinito}

\begin{teoEquivalenciasLocalmenteFinito}%
	\label{thm:equivalenciaslocalmentefinito}
	Sea $K$ un complejo simplicial. Las siguientes afirmaciones son
	equivalentes:
	\begin{itemize}
		\item[(a)] $K$ es localmente finito;
		\item[(b)] $\geom{K}$ es localmente compacto;
		\item[(c)] la identidad $\geom{K}\rightarrow\geom{K}_{d}$ es
			un homeomorfismo;
		\item[(d)] $\geom{K}$ es metrizable;
		\item[(e)] $\geom{K}$ verifica el primer axioma de
			numerabilidad: todo punto posee una base numerable de
			entornos.
	\end{itemize}
	%
\end{teoEquivalenciasLocalmenteFinito}

\begin{proof}
	\emph{(b) implica (c):} Sea $U\subset\geom{K}$ un abierto con clausura
	$\clos{U}$ compacta ($\geom{K}$ es Hausdorff). Como $\clos{U}$ es
	compacta, existe una cantidad finita de s\'{\i}mplices $s$
	tales que $\clos{U}$ est\'{a} contenida en la uni\'{o}n de los finitos
	s\'{\i}mplices abiertos $\simpinterior{s}$, por el corolario
	\ref{thm:compactocontenidoenfinitossimplices}. La uni\'{o}n de estos
	s\'{\i}mplices forma un subcomplejo finito $L\subset K$ y
	$\clos{U}\subset\geom{L}$. Como $L$ es finito, la identidad
	$\geom{L}\rightarrow\geom{L}_{d}$ es un homeomorfismo. Por lo tanto,
	$U$ es abierto en $\geom{L}_{d}$. Sea $K_{1}\subset K$ el subcomplejo
	\begin{align*}
		K_{1} &\,=\,\big\{ s\in K\,:\,U\cap \geom{s}=\varnothing\big\}
		\text{ .}
	\end{align*}
	%
	Como $U\subset\geom{K}$ es abierto (con la topolog\'{\i}a coherente),
	\begin{align*}
		U\,\cap\,\geom{s}\,=\,\varnothing & \quad\Leftrightarrow\quad
			U\,\cap\,\simpinterior{s}\,=\,\varnothing
		\text{ ,}
	\end{align*}
	%
	con lo cual, si $s\in K\setmin K_{1}$, entonces
	$\simpinterior{s}\cap\geom{L}\not=\varnothing$. En particular,
	si $s\in K\setmin K_{1}$, existe $\alpha\in\geom{K}$ tal que
	\begin{align*}
		\alpha\,\in\,\simpinterior{s}\,\cap\,\geom{L}
		\text{ .}
	\end{align*}
	%
	Pero $\alpha\in\geom{L}$ quiere decir que $\{\alpha\not=0\}\in L$
	y $\alpha\in\simpinterior{s}$ quiere decir que
	$\{\alpha\not=0\}=s$. En definitiva, tenemos una descomposici\'{o}n
	del complejo $K$ como uni\'{o}n de los dos subcomplejos:
	\begin{align*}
		K & \,=\, K_{1}\,\cup\,L
		\text{ .}
	\end{align*}
	%
	De esta descomposici\'{o}n se deduce la descomposici\'{o}n
	\begin{align*}
		\geom{K}_{d} & \,=\,\geom{K_{1}}_{d}\,\cup\,
					\geom{L}_{d}
		\text{ .}
	\end{align*}
	%
	En particular,
	\begin{math}
		\geom{K}_{d}\setmin\geom{K_{1}}_{d}=
			\geom{L}_{d}\setmin\geom{K_{1}}_{d}
	\end{math}~.
	Como $U\subset\geom{L}_{d}$ es abierto y $U$ est\'{a} contenido en
	$\geom{L}_{d}\setmin\geom{K_{1}}_{d}$, se deduce que $U$ es abierto
	en $\geom{K}_{d}\setmin\geom{K_{1}}_{d}$. Pero
	$\geom{K_{1}}_{d}$ es cerrado en $\geom{K}_{d}$ y, en consecuencia,
	$U$ es abierto en $\geom{K}_{d}$.

	\emph{(e) implica (a): } si $K$ no es localmente finito, existe
	alg\'{u}n v\'{e}rtice $v\in K$ que pertenece a todos los
	s\'{\i}mplices de una familia infinita numerable de s\'{\i}mplices de
	$K$, $\{s_{i}\}_{i\geq 1}$. Supongamos que dicho v\'{e}rtice $v$
	admite una base numerable de entornos (del punto correspondiente
	$\alpha_{v}\in\geom{K}$), $\{U_{i}\}_{i\geq 1}$ en $\geom{K}$
	(topolog\'{\i}a coherente), y supongamos, sin p\'{e}rdida de
	generalidad, que $U_{i}\supset U_{i+1}$ para todo $i\geq 1$.
	Para cada $i\geq 1$, como $\geom{s_{i}}\subset\geom{K}$ es cerrado,
	$\clos{\simpinterior{s_{i}}}=\geom{s_{i}}$. Como
	$\alpha_{v}\in\geom{s_{i}}$ y $U_{j}$ es un abierto que contiene al
	punto $\alpha_{v}$, se deduce que existe
	\begin{align*}
		& \alpha_{i}\,\in\,\simpinterior{s_{i}}\,\cap\,U_{i}
		\text{ .}
	\end{align*}
	%
	Como los abiertos $U_{i}$ constituyen una base de entornos de
	$\alpha_{v}$ y $U_{i}\supset U_{i+1}$ para todo $i$, la sucesi\'{o}n
	$\{\alpha_{i}\}_{i\geq 1}$ tiende a $\alpha_{v}$ en la topolog\'{\i}a
	de $\geom{K}$. Como $\alpha_{i}\in\simpinterior{s_{i}}$, en
	particular $\alpha_{i}\not=\alpha_{v}$ para todo $i$, pero
	tambi\'{e}n, dado un s\'{\i}mplice arbitrario $s\in K$, la
	intersecci\'{o}n $\geom{s}\cap\{\alpha_{i}\}_{i\geq 1}$ es a lo sumo
	un conjunto finito. Como en el comentario previo al lema
	\ref{thm:compactocontenidoenfinitossimplices}, esto implica que
	$\{\alpha_{i}\}_{\i\geq 1}$ es discreto en $\geom{K}$, lo que
	contradice el hecho de que tenga un punto de acumulaci\'{o}n.
\end{proof}

Una \emph{realizaci\'{o}n} de un complejo simplicial $K$ en un
$\bb{R}$-espacio vectorial $E$ (cuya topolog\'{\i}a se caracteriza por
ser coherente con los subespacios de dimensi\'{o}n finita) es una
inmersi\'{o}n lineal $\geom{K}\rightarrow\bb{R}^{n}$.

\begin{teoInmersionDeUnComplejo}\label{thm:inmersiondeuncomplejo}
	Sea $K$ un complejo simplicial. Si $K$ admite una realizaci\'{o}n
	$\geom{K}\rightarrow\bb{R}^{n}$ en un espacio de dimensi\'{o}n
	finita, entonces $K$ es a lo sumo numerable, localmente finito y
	$\dim\,K\leq n$. Rec\'{\i}procamente, si $K$ es numerable, localmente
	finito y $\dim\,K\leq n$, entonces $K$ admite una realizaci\'{o}n
	en $\bb{R}^{2n+1}$.
\end{teoInmersionDeUnComplejo}
