\theoremstyle{plain}
\newtheorem{lemaConjuntoDirigidoSubdivisiones}{Lema}[section]
\newtheorem{lemaSubdivisionEquivaleAParticion}%
	[lemaConjuntoDirigidoSubdivisiones]{Lema}

\theoremstyle{remark}
\newtheorem{obsDefinicionSubdivisiones}[lemaConjuntoDirigidoSubdivisiones]%
	{Observaci\'{o}n}
\newtheorem{obsSubdividirUnSubcomplejo}[lemaConjuntoDirigidoSubdivisiones]%
	{Observaci\'{o}n}
\newtheorem{obsSubdividirUnParSimplicial}[lemaConjuntoDirigidoSubdivisiones]%
	{Observaci\'{o}n}
\newtheorem{obsTriangularPorSubdivisiones}[lemaConjuntoDirigidoSubdivisiones]%
	{Observaci\'{o}n}
\newtheorem{obsLinealEnLaSubdivision}[lemaConjuntoDirigidoSubdivisiones]%
	{Observaci\'{o}n}
\newtheorem{obsSubdivisionesYEstrellas}[lemaConjuntoDirigidoSubdivisiones]%
	{Observaci\'{o}n}

%-------------

Una \emph{subdivisi\'{o}n} de un complejo simplicial $K$ es un complejo
simplicial $K'$ con las siguientes tres propiedades:
\begin{itemize}
	\item[(i)] los v\'{e}rtices de $K'$ son puntos de $\geom{K}$,
	\item[(ii)] si $s'\in K'$ es un s\'{\i}mplice, entonces existe un
		s\'{\i}mplice $s\in K$ tal que $s'\subset\geom{s}$ y
	\item[(iii)] la funci\'{o}n lineal $\geom{K'}\rightarrow\geom{K}$
		que se obtiene a partir de la aplicaci\'{o}n
		$\alpha_{v'}\mapsto v'$	definida en los puntos
		correspondientes a los v\'{e}rtices de $K'$ es un
		homeomorfismo.
\end{itemize}
%
\begin{obsLinealEnLaSubdivision}\label{obs:linealenlasubdivision}
	Sea $K$ un complejo simplicial y sea $\Phi:\,\geom{K}\rightarrow X$
	una funci\'{o}n lineal. Si $K'$ es una subdivisi\'{o}n de $K$,
	entonces $\Phi$ tambi\'{e}n es lineal en $K'$
	($\Phi\circ f:\,\geom{K'}\rightarrow X$ es lineal).
\end{obsLinealEnLaSubdivision}

\begin{lemaConjuntoDirigidoSubdivisiones}%
	\label{thm:conjuntodirigidosubdivisiones}
	Sea $K$ un complejo simplicial. Toda subdivisi\'{o}n de una
	subdivisi\'{o}n de $K$ es una subdvisi\'{o}n de $K$ y, dadas dos
	subdivisiones $K'$ y $K''$ de $K$ existe una subdivisi\'{o}n de $K$
	que es subdivisi\'{o}n de $K'$ y de $K''$.
\end{lemaConjuntoDirigidoSubdivisiones}

\begin{obsDefinicionSubdivisiones}\label{obs:definicionsubdivisiones}
	Sean $K,K'$ complejos simpliciales que verifican las condiciones
	\emph{(i)} y \emph{(ii)} de la definici\'{o}n de subdivisi\'{o}n.
	Entonces la aplicaci\'{o}n
	$\alpha_{v'}\in\geom{K'}\mapsto v'\in\geom{K}$ determina una
	funci\'{o}n lineal $f:\,\geom{K'}\rightarrow\geom{K}$: si $s'\in K'$ es
	un s\'{\i}mplice y $s\in K$ es tal que $s'\subset\geom{s}$, entonces
	definimos $f_{s',s}:\,\geom{s'}\rightarrow\geom{s}$ por
	\begin{align*}
		f_{s'}\Big(\sum_{v'\in s'}\,t^{v'}\,\alpha_{v'}\Big) & \,=\,
			\sum_{v'\in s'}\,t^{v'}\,v'
		\text{ .}
	\end{align*}
	%
	Como $s'\subset\geom{s}$, la funci\'{o}n $f_{s',s}$ est\'{a} bien
	definida. La funci\'{o}n $f_{s',s}$ proyecta el s\'{\i}mplice
	$\geom{s'}$ sobre la c\'{a}scara convexa generada por las im\'{a}genes
	de sus v\'{e}rtices en $\geom{s}$. Si el s\'{\i}mplice $s$ est\'{a}
	contenido en alg\'{u}n s\'{\i}mplice $s_{1}\in K$, entonces la
	inclusi\'{o}n $s\subset s_{1}$ determina una inclusi\'{o}n
	$\inc[s,s_{1}]:\,\geom{s}\subset\geom{s_{1}}$ y vale que
	\begin{align*}
		\inc[s,s_{1}]\circ f_{s',s}(\alpha) & \,=\,
			f_{s',s_{1}}(\alpha)
	\end{align*}
	%
	para todo $\alpha\in\geom{s'}$, por unicidad. En particular,
	queda determinada una funci\'{o}n lineal y continua
	$f_{s'}:\,\geom{s'}\rightarrow\geom{K}$. Si $s''\in K'$ es otro
	s\'{\i}mplice y $s'\cap s''\not=\varnothing$, entonces, nuevamente
	porque las funciones lineales est\'{a}n determinadas por su valor
	en los v\'{e}rtices,
	\begin{align*}
		f_{s'}|_{\geom{s'\cap s''}} & \,=\,f_{s''}|_{\geom{s'\cap s''}}
		\text{ .}
	\end{align*}
	%
	En definitiva, queda determinada una funci\'{o}n lineal y continua en
	todo el espacio $\geom{K'}$ del complejo.

	Esta funci\'{o}n no es, en general, inyectiva y, por lo tanto, no es
	posible identificar $\geom{K'}$ con un subespacio de $\geom{K}$ (no
	estamos asumiendo la propiedad \emph{(iii)}, aun). La inyectividad
	puede fallar por distintas razones: por ejemplo, si
	$s',s''\in K'$ y $s\in K$ es tal que $s',s''\subset\geom{s}$
	las c\'{a}scaras convexas de $s'$ y de $s''$ en $\geom{s}$ pueden
	solaparse de manera tal que su intersecci\'{o}n no sea la
	c\'{a}scara convexa generada por ning\'{u}n s\'{\i}mplice de $K'$;
	otra posibilidad es que las funciones parciales
	$f_{s'}:\,\geom{s'}\rightarrow\geom{K}$ reduzcan la dimensi\'{o}n de
	los s\'{\i}mpices en donde est\'{a}n definidas, es decir,
	la imagen de los v\'{e}rtices de $\geom{s'}$ puede ser un conjunto
	af\'{\i}nmente dependiente de $\geom{s}$, los v\'{e}rtices de
	$s'\subset\geom{s}$ pueden no estar en posici\'{o}n general.

	A pesar de la posible falla de la inyetividad de
	$f:\,\geom{K'}\rightarrow\geom{K}$, todo s\'{\i}mplice $s'\in K'$
	est\'{a} incluido en alguno de los espacios $\geom{s}$ y, por
	finitud de $s$, existe un s\'{\i}mplice $s\in K$ m\'{a}s chico
	tal que $\geom{s}\supset s'$ (y dicho s\'{\i}mplice es \'{u}nico).
	La c\'{a}scara convexa $f(\geom{s'})$ de $s'$ en $\geom{s}$ es
	cerrada. Por otro lado, como $s'$ es un subconjunto finito arbitrario
	de $\geom{s}$, la c\'{a}scara convexa puede estar contenida en el
	interior $\simpinterior{s}$, o bien intersecar alguna cara propia de
	$s$ y, en tal caso, la intersecci\'{o}n puede ser total o parcial. En
	todo caso, si $\beta\in\simpinterior{s'}$ entonces se cumple que
	$f(\beta)\in\simpinterior{s}$.

	Demostremos esta \'{u}ltima afirmaci\'{o}n. Si $v'\in s'$, como
	$s'\subset\geom{s}$, entonces
	\begin{align*}
		v' & \,=\,\sum_{v\in s}\,t_{v'}^{v}\,\alpha_{v}
	\end{align*}
	%
	y, como $s$ es el s\'{\i}mplice m\'{a}s chico tal que
	$s'\subset\geom{s}$, entonces, para todo $v\in s$, existe alg\'{u}n
	$v'\in s'$ tal que $t_{v'}^{v}\not=0$ (si no el v\'{e}rtice $v\in s$
	se podr\'{\i}a omitir). Si $\beta\in\simpinterior{s'}$, entonces
	\begin{align*}
		\beta & \,=\,\sum_{v'\in s'}\,\beta^{v'}\,\beta_{v'}
			\quad\text{y} \\
		f(\beta) & \,=\,\sum_{v'\in s'}\,\beta^{v'}\,f(\beta_{v'})
			\,=\,\sum_{v'\in s'}\,\beta^{v'}\,
				\sum_{v\in s}\,t_{v'}^{v}\,\alpha_{v} \\
		& \,=\,\sum_{v\in s}\,
			\Big(\sum_{v'\in s'}\,\beta^{v'}t_{v'}^{v}\Big)\,
				\alpha_{v}
		\text{ .}
	\end{align*}
	%
	Como $\beta^{v'}>0$ para todo $v'$ y, para cada $v$ existe
	$t_{v'}^{v}>0$, vale que los coeficientes
	$\sum_{v'\in s'}\,\beta^{v'}t_{v'}^{v}$ son todos positivos.
	Por lo tanto, $f(\beta)\in\simpinterior{s}$.
\end{obsDefinicionSubdivisiones}

\begin{lemaSubdivisionEquivaleAParticion}%
	\label{thm:subdivisionequivaleaparticion}
	Sean $K',K$ complejos simpliciales que verifican las condiciones
	\emph{(i)} y \emph{(ii)} de la definici\'{o}n de subdivisi\'{o}n.
	Entonces $K'$ es una subdivisi\'{o}n de $K$, si y s\'{o}lo si,
	\emph{(iii')} para todo s\'{\i}mplice $s\in K$, el conjunto
	\begin{align*}
		& \big\{f\big(\simpinterior{s'}\big)\,:\,
			s'\in K',\,f\big(\simpinterior{s'}\big)\subset
				\simpinterior{s}\big\}
	\end{align*}
	%
	es una partici\'{o}n finita de $\simpinterior{s}$.
	% partici\'{o}n en el sentido de que la uni\'{o}n es todo y la
	% intersecci\'{o}n de dos elementos es vac\'{\i}a
\end{lemaSubdivisionEquivaleAParticion}

\begin{proof}
	Si $s\in K$, sea $K'(s)\subset K'$ el subconjunto
	\begin{align*}
		K'(s) & \,=\,\bigcup_{s_{1}\in\caras{s}}\,
			\big\{s'\in K'\,:\,f\big(\simpinterior{s'}\big)\subset
				\simpinterior{s_{1}}\big\}
		\text{ .}
	\end{align*}
	%
	Asumamos que se verifica \emph{(i)} y definamos el subconjunto
	$V'(s) =\big\{v'\in K'\,:\,v'\in\geom{s}\big\}$ de los v\'{e}rtices de
	$K'$. Entonces $\{v'\}\in K'(s)$, para todo $v'\in V'$.
	Sean $s'\in K'(s)$ y $s''\subset s'$.  Como $K'$ es un complejo,
	$s''\in K'$. Como el conjunto $f\big(\simpinterior{s'}\big)$ est\'{a}
	contenido en $\simpinterior{s_{1}}$ para alguna cara $s_{1}\subset s$,
	los v\'{e}rtices de $s'$ est\'{a}n contenidos en $\geom{s_{1}}$ y,
	en particular, en $s$. Como $s''$ es un subconjunto de estos
	v\'{e}rtices, $s''\subset\geom{s}$, tambi\'{e}n. En particular,
	$s''\subset\geom{s_{2}}$ para alguna cara $s_{2}$ de $s$. Tomando
	$s_{2}$ como la cara m\'{a}s chica con esta propiedad, se deduce,
	por lo visto en la observaci\'{o}n \ref{obs:definicionsubdivisiones},
	que $f\big(\simpinterior{s''}\big)\subset\simpinterior{s_{2}}$ y, por
	lo tanto, $s''\in K'(s)$. En definitiva, $K'(s)$ es un subcomplejo
	de $K'$ cuyo conjunto de v\'{e}rtices es $V'(s)$.

	Supongamos que se cumplen \emph{(i)} y que el conjunto de la
	condici\'{o}n \emph{(iii')} es finito. Entonces dado $s\in K$, el
	subcomplejo $K'(s)\subset K'$ es un complejo finito. Si, adem\'{a}s,
	suponemos que el conjunto de \emph{(iii')} es una partici\'{o}n,
	entonces, la funci\'{o}n lineal
	$h_{s}:\,\geom{K'(s)}\rightarrow\geom{s}$ dada en v\'{e}rtices
	por $h_{s}(\alpha_{v'})=v'\in\geom{s}$ es continua y biyectiva entre
	espacios compactos Hausdorff. En particular, las condiciones
	\emph{(i)} y \emph{(iii')} implican que $h_{s}$ es un homeomorfismo.
	La condici\'{o}n \emph{(ii)} implica que todo s\'{\i}mplice de $K'$
	pertence a alguno de los subcomplejos $K'(s)$. Notemos que
	\begin{align*}
		h_{s} & \,=\,f|_{\geom{K'(s)}}
		\text{ .}
	\end{align*}
	%
	Por lo tanto, \emph{(i)}, \emph{(ii)} y \emph{(iii')} implican que
	$f:\,\geom{K'}\rightarrow\geom{K}$ tiene una inversa continua dada
	por $f^{-1}|_{\geom{s}}=h_{s}^{-1}$. En definitiva, $K'$ es una
	subdivisi\'{o}n de $K$.

	Rec\'{\i}procamente, la familia
	$\big\{\simpinterior{s'}\,:\,s'\in K'\big\}$ particiona al espacio
	$\geom{K'}$. Sea $s\in K$ y sea $s'\in K'$. Entonces, por la
	observaci\'{o}n \ref{obs:definicionsubdivisiones}, o bien
	$f\big(\simpinterior{s'}\big)\cap\simpinterior{s}=\varnothing$, o bien
	$f\big(\simpinterior{s'}\big)\subset\simpinterior{s}$. Si
	asumimos \emph{(iii)}, que $f:\,\geom{K'}\rightarrow\geom{K}$ es un
	homeomorfismo, entonces el conjunto
	\begin{align*}
		& \big\{f\big(\simpinterior{s'}\big)\,:\,
			s'\in K',\,f\big(\simpinterior{s'}\big)\subset
				\simpinterior{s}\big\}
	\end{align*}
	%
	es una partici\'{o}n de $\simpinterior{s}$. Pero, como $\geom{s}$
	es compacto en $\geom{K'}$ (v\'{\i}a el homeomorfismo $f$), el
	lema \ref{thm:compactocontenidoenfinitossimplices} implica que
	$\geom{s}$ est\'{a} contenido en una uni\'{o}n finita de conjuntos de
	la forma $\simpinterior{s'}$ con $s'\in K'$ y, por lo tanto, la
	partici\'{o}n de $\simpinterior{s}$ debe ser finita.
\end{proof}

\begin{obsSubdividirUnSubcomplejo}\label{obs:subdividirunsubcomplejo}
	Si $L\subset K$ es un subcomplejo y $K'$ es una subdivisi\'{o}n de $K$,
	entonces el subconjunto
	\begin{align*}
		L' & \,=\,\big\{s'\in K'\,:\,f\big(\simpinterior{s'}\big)
			\subset\geom{L}\big\}
	\end{align*}
	%
	es un subcomplejo de $K'$ y, si $t\in L$, entonces
	\begin{align*}
		\big\{f\big(\simpinterior{s'}\big)\,:\,s'\in K',\,
			f\big(\simpinterior{s'}\big)\subset\simpinterior{t}
				\big\} & \,=\,
		\big\{f\big(\simpinterior{s'}\big)\,:\,s'\in L',\,
			f\big(\simpinterior{s'}\big)\subset\simpinterior{t}
				\big\}
		\text{ .}
	\end{align*}
	%
	En particular, por el lema \ref{thm:subdivisionequivaleaparticion},
	$L'$ es una subdivisi\'{o}n de $L$. Dicho de otra manera, toda
	subdivisi\'{o}n de un complejo subdivide a todo subcomplejo
	tambi\'{e}n. Notemos que $L'$ es el \'{u}nico subcomplejo de $K'$ con
	esta propiedad. Lo llamamos \emph{la subdivisi\'{o}n inducida por %
	$K'$}.
\end{obsSubdividirUnSubcomplejo}

\begin{obsSubdividirUnParSimplicial}\label{obs:subdividirunparsimplicial}
	?`Vale m\'{a}s en general? Dada una transformaci\'{o}n simplicial
	$\varphi:\,L\rightarrow K$ y una subdivisi\'{o}n $K'$ de $K$,
	?`existe una subdivisi\'{o}n $L'$ de $L$ y una transformaci\'{o}n
	simplicial $\varphi':\,L'\rightarrow K'$ compatibles con la
	subdivisi\'{o}n $K'$ y la transformaci\'{o}n $\varphi$?
	La compatibilidad podr\'{\i}a estar dada ``naturalmente'' en
	t\'{e}rminos de las realizaciones geom\'{e}tricas de los complejos
	y de los morfismos, es decir, en t\'{e}rminos del funtor
	$\geom{\cdot}$. La pregunta es la siguiente: dados un par simplicial
	$\varphi:\,L\rightarrow K$ y una subdivisi\'{o}n $K'$ de $K$, ?`existe
	un par simplicial de la forma $\varphi':\,L'\rightarrow K'$ tal que
	$L'$ sea una subdivisi\'{o}n de $L$ y tal que
	\begin{center}
		\begin{tikzcd}
			\geom{L'} \arrow[r,"\geom{\varphi'}"] \arrow[d,"g"'] &
				\geom{K'} \arrow[d,"f"] \\
			\geom{L} \arrow[r,"\geom{\varphi}"'] & \geom{K}
		\end{tikzcd}
	\end{center}
	sea un diagrama conmutativo de espacios topol\'{o}gicos y funciones
	continuas? Las funciones $f:\,\geom{K'}\rightarrow\geom{K}$ y
	$g:\,\geom{L'}\rightarrow\geom{L}$ son los homeomorfismos asociados
	a las subdivisiones: est\'{a}n dadas en los v\'{e}rtices por
	$f(\alpha_{v'})=v'$ para todo v\'{e}rtice $v'$ de $K'$ y por
	$g(\beta_{w'})=w'$ para todo v\'{e}rtice $w'$ de $L'$.

	Supongamos que existe un par simplicial con estas caracter\'{\i}sticas
	y sean
	\begin{align*}
		W\,=\,\vertices{L} & \quad\text{,}\quad W'\,=\,\vertices{L'}
			\text{ ,} \\
		V\,=\,\vertices{K} & \quad\text{,}\quad V'\,=\,\vertices{K'}
		\text{ .}
	\end{align*}
	%
	Entonces $W'\subset\geom{L}$ y
	$\varphi\big(W'\big)\subset V'\subset\geom{K}$. Si $w'\in W'$,
	entonces
	\begin{align*}
		f(\geom{\varphi'}\beta_{w'}) & \,=\,f(\alpha_{\varphi'w'})
			\,=\,\varphi'w'\quad\text{y} \\
		\geom{\varphi}\big(g(\beta_{w'})\big) & \,=\,
			\geom{\varphi}w'
		\text{ .}
	\end{align*}
	%
	Es decir, la transformaci\'{o}n simplicial $\varphi'$ verifica,
	en los v\'{e}rtices de $L'$, la igualdad
	\begin{align*}
		\varphi'w' & \,=\,\geom{\varphi}w'
		\text{ .}
	\end{align*}
	%
	Recordando que toda transformaci\'{o}n simplicial est\'{a} determinada
	por su valor en los v\'{e}rtices, se deduce que, fijada la
	subdivisi\'{o}n $L'$ (que, por definici\'{o}n, se construye con puntos
	de la realizaci\'{o}n $\geom{L}$), existe a lo sumo una
	transformaci\'{o}n simplicial $\varphi':\,L'\rightarrow K'$ tal que
	$f\circ\geom{\varphi'}=\geom{\varphi}\circ g$. Notemos que el conjunto
	$W'$ de v\'{e}rtices de $L'$ est\'{a} (casi) determinado por $\varphi$
	y por la subdivisi\'{o}n $K'$: si $w'\in W'$ entonces
	$\varphi(w')\in V'$ debe ser un v\'{e}rtice de la subdivisi\'{o}n $K'$.

	Otra observaci\'{o}n que puede llegar a ser importante es que la
	subdivisi\'{o}n $L'$ se puede definir en cada uno de los
	s\'{\i}mplices $t\in L$ y cada uno de los espacios $\geom{t}$ es
	compacto y homeomorfo a un s\'{\i}mplice de $\bb{R}^{q+1}$ para
	alg\'{u}n $q\geq 0$. En particular, si $v'\in V'$ es un v\'{e}rtice
	de $K'$, entonces
	\begin{math}
		\geom{t}\cap\big(\geom{\varphi}^{-1}\big(\{v'\}\big)\big)
	\end{math}
	es igual a un subespacio af\'{\i}n intersecado con $\geom{t}$ y,
	por lo tanto, algo que tiene ``forma de s\'{\i}mplice'': existe un
	conjunto finito $t'\subset\geom{t}$ tal que
	\begin{align*}
		\convexa{t'} & \,=\,\geom{t}\,\cap\,
			\big(\geom{\varphi}^{-1}\big(\{v'\}\big)\big)
	\end{align*}
	%
	y minimal con esta propiedad. Este conjunto $t'$ tiene necesariamente
	la propiedad de que, si llamamos $T'$ al complejo cuyos v\'{e}rtices
	son los puntos de $t'$ y cuyos s\'{\i}mplices son todos los
	subconjuntos no vac\'{\i}os de $t'$
	(es decir, $T'=\partes(t')\setmin\{\varnothing\}$), entonces $T'$,
	junto con la funci\'{o}n lineal que en los v\'{e}rtices (los puntos
	de $t'$) es la identidad, es una triangulaci\'{o}n de $\convexa{t'}$.
	Es de esperar que $L'$ tenga como v\'{e}rtices a la uni\'{o}n
	$\bigcup_{t\in L,v'\in V'}\,t'$. En caso de existir otros v\'{e}rtices
	en $L'$, \'{e}stos deber\'{a}n pertenecer a alguno de los conjuntos
	$\convexa{t'}$. Podr\'{\i}an existir distintas subdivisiones de $L$
	que cumplan lo pedido.

	Diremos que un par simplicial $\varphi':\,L'\rightarrow K'$
	subdivide un par simplicial $\varphi:\,L\rightarrow K$, si
	$L'$ es una subdivisi\'{o}n de $L$, $K'$ es una subdivisi\'{o}n de $K$
	y $f\circ\geom{\varphi'}=\geom{\varphi}\circ g$, donde $g$ y $f$
	son los hoemomorfismos asociados a las subdivisiones.

	Usando el lema \ref{thm:subdividirunsimplice}, se puede ver
	inductivamente que existe una subdivisi\'{o}n $L'$ de $L$, una
	subdivisi\'{o}n $K''$ de $K'$ y una transformaci\'{o}n simplicial
	$\varphi'':\,L'\rightarrow K''$ tales que $\varphi''$ subdivide a
	$\varphi$ (tal vez, eligiendo el punto $w_{0}$ en alguna de las
	``caras'' colapsadas de $L$ se pueda demostrar que
	\ref{thm:subdividirunsimplice} sigue valiendo y que no resulta
	necesario subdividir $K'$).
\end{obsSubdividirUnParSimplicial}

\begin{obsTriangularPorSubdivisiones}\label{obs:triangularporsubdivisiones}
	Sea $\varphi$ ($\varphi:\,L\rightarrow K$) una triangulaci\'{o}n de
	$h$ ($h:\,A\rightarrow X$) y sea
	$(\psi',\psi):\geom{\varphi}\rightarrow h$ el homeomorfismo de pares
	correspondiente. Si $\varphi'$ ($\varphi':\,L'\rightarrow K'$) es una
	subdivisi\'{o}n de $\varphi$, entonces $\varphi'$ es una
	triangulaci\'{o}n $h$, v\'{\i}a el homeomorfismo de pares
	$(\psi'\circ f,\psi\circ g)$.
	\begin{center}
		\begin{tikzcd}
			\geom{L'} \arrow[r,"\geom{\varphi'}"] \arrow[d,"g"'] &
				\geom{K'} \arrow[d, "f"] \\
			\geom{L} \arrow[r,"\geom{\varphi}"] \arrow[d,"\psi"'] &
				\geom{K} \arrow[d,"\psi'"] \\
			A \arrow[r,"h"'] & X
		\end{tikzcd}
	\end{center}
\end{obsTriangularPorSubdivisiones}

\begin{obsSubdivisionesYEstrellas}\label{obs:subdivisionesyestrellas}
	Sea $K'$ una subdivisi\'{o}n de un complejo simplicial $K$.
	Sean $v$ y $v'$ v\'{e}rtices de $K$ y de $K'$, respectivamente.
	Entonces $v'\in\estrella[K]{v}$ (estrella en $K$, pues $v$ podr\'{\i}a
	(deber\'{\i}a) ser un v\'{e}rtice de $K'$, tambi\'{e}n), si y s\'{o}lo
	si $\estrella[K']{v'}\subset\estrella[K]{v}$.
\end{obsSubdivisionesYEstrellas}
