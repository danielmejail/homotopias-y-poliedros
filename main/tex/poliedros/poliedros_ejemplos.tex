\theoremstyle{definition}
\newtheorem{ejemploVacio}{Ejemplo}[section]
\newtheorem{ejemploSubconjuntosFinitos}[ejemploVacio]{Ejemplo}
\newtheorem{ejemploCarasDeUnSimplice}[ejemploVacio]{Ejemplo}
\newtheorem{ejemploCarasPropias}[ejemploVacio]{Ejemplo}
\newtheorem{ejemploQEsqueleto}[ejemploVacio]{Ejemplo}
\newtheorem{ejemploNervioDeUnaFamilia}[ejemploVacio]{Ejemplo}
\newtheorem{ejemploSumaDeComplejos}[ejemploVacio]{Ejemplo}
\newtheorem{ejemploSegmentosEnteros}[ejemploVacio]{Ejemplo}
\newtheorem{ejemploReticulado}[ejemploVacio]{Ejemplo}
\newtheorem{ejemploSubcomplejoDeCaras}[ejemploVacio]{Ejemplo}
\newtheorem{ejemploSubcomplejosUnionInterseccion}[ejemploVacio]{Ejemplo}
\newtheorem{ejemploSubcomplejoDelNervio}[ejemploVacio]{Ejemplo}
\newtheorem{ejemploBolaYEsfera}[ejemploVacio]{Ejemplo}
\newtheorem{ejemploAproximacionesEnElCirculo}[ejemploVacio]{Ejemplo}
\newtheorem{ejemploAproximacionesEnElCirculoCont}[ejemploVacio]{Ejemplo}
\newtheorem{ejemploFuncionesNulhomotopicasEntreEsferas}[ejemploVacio]{Ejemplo}
\newtheorem{ejemploHomotopicasNoContiguas}[ejemploVacio]{Ejemplo}

%-------------

\begin{ejemploVacio}[El complejo vac\'{\i}o]%
	\label{ejemplo:vacio}
	El conjunto vac\'{\i}o visto como un conjunto vac\'{\i}o de
	s\'{\i}mplices es un complejo. Lo denotamos $\varnothing$ y lo
	denominamos \emph{complejo vac\'{\i}o}.
\end{ejemploVacio}

\begin{ejemploSubconjuntosFinitos}[El complejo de subconjuntos finitos]%
	\label{ejemplo:subconjuntosfinitos}
	Dado un conjunto $A$, el conjunto de subconjuntos finitos
	no vac\'{\i}os de $A$ constituye un complejo.
\end{ejemploSubconjuntosFinitos}

\begin{ejemploCarasDeUnSimplice}[Las caras de un s\'{\i}mplice]%
	\label{ejemplo:carasdeunsimplice}
	Sea $K$ un complejo simplicial y sea $s\in K$ un s\'{\i}mplice.
	El conjunto de caras de $s$ constituye un complejo, \emph{el %
	complejo de caras de $s$}, denotado $\caras{s}$.
\end{ejemploCarasDeUnSimplice}

\begin{ejemploCarasPropias}[Las caras propias]%
	\label{ejemplo:caraspropias}
	El conjunto de caras propias de un s\'{\i}mplice $s$ tambi\'{e}n
	constituye un complejo, lo denotamos $\carasp{s}$.
\end{ejemploCarasPropias}

\begin{ejemploQEsqueleto}[El $q$-esqueleto de un complejo]%
	\label{ejemplo:qesqueleto}
	Sea $K$ un complejo simplicial. El \emph{$q$-esqueleto de $K$}
	es el complejo, denotado $\qesq{q}{K}$, cuyos s\'{\i}mplices son todos
	los $p$-s\'{\i}mplices de $K$ con $p\leq q$. El $q$-esqueleto de
	un complejo $K$ es un subcomplejo de $K$.
\end{ejemploQEsqueleto}

\begin{ejemploNervioDeUnaFamilia}[El nervio de una familia de subconjuntos]%
	\label{ejemplo:nerviodeunafamilia}
	Sea $X$ un conjunto y sea $\cal{W}$ una familia de subconjuntos
	de $X$. El \emph{nervio de $\cal{W}$}, denotado $\nerv{\cal{W}}$,
	es el complejo simplicial cuyos s\'{\i}mplices son los subconjuntos
	finitos de $\cal{W}$ cuya intersecci\'{o}n sea no vac\'{\i}a.
	En particular, los v\'{e}rtices de $\nerv{\cal{W}}$ son exactamente
	los elementos de $\cal{W}$.
\end{ejemploNervioDeUnaFamilia}

\begin{ejemploSumaDeComplejos}[La suma de dos complejos]%
	\label{ejemplo:sumadecomplejos}
	Sean $K_{1},K_{2}$ dos complejos simpliciales. La \emph{suma de %
	$K_{1}$ con $K_{2}$}, denotada $K_{1}*K_{2}$, es el complejo
	simplicial cuyos s\'{\i}mplices son los s\'{\i}mplices del
	complejo $K_{1}$, los de $K_{2}$ y las uniones disjuntas de
	un s\'{\i}mplice de $K_{1}$ con uno de $K_{2}$. De esta manera,
	el conjunto de v\'{e}rtices de $K_{1}*K_{2}$ es igual a la uni\'{o}n
	disjunta del conjunto de v\'{e}rtices de $K_{1}$ con el de los de
	$K_{2}$. En s\'{\i}mbolos,
	\begin{align*}
		K_{1}*K_{2} & \,=\,K_{1}\,\sqcup\,K_{2}\,\cup\,
			\big\{s_{1}\sqcup s_{2}\,:\,s_{1}\in K_{1},\,
						s_{2}\in K_{2}\big\}
		\text{ .}
	\end{align*}
	%
\end{ejemploSumaDeComplejos}

\begin{ejemploSegmentosEnteros}[Segmentos enteros]%
	\label{ejemplo:segmentosenteros}
	Sea $V=\bb{Z}$ y sea $K$ la familia
	\begin{align*}
		K & \,=\,\big\{\{n\}\,:\,n\in\bb{Z}\big\}\,\cup\,
			\big\{\{n,n+1\}\,:\,n\in\bb{Z}\big\}
		\text{ .}
	\end{align*}
	%
	Entonces $K$ es un complejo simplicial cuyos v\'{e}rtices son
	los n\'{u}meros enteros y cuyos $1$-s\'{\i}mplices son los
	intervalos enteros $\{n,n+1\}$. El complejo $K$ no posee
	s\'{\i}mplices de mayor dimensi\'{o}n.
\end{ejemploSegmentosEnteros}

\begin{ejemploReticulado}[Puntos en un reticulado]%
	\label{ejemplo:reticulado}
	Sea $n\geq 1$ un entero fijo. Consideramos el conjunto de $n$-tuplas
	de enteros $\bb{Z}^{n}$ con el orden parcial dado por comparar
	las coordenadas: $(\lista*{x}{n})\leq (\lista*{y}{n})$, si
	$x^{i}\leq y^{i}$ para todo $i=1,\,\dots,\,n$. Sea $K$ la familia de
	conjuntos finitos no vac\'{\i}os y totalmente ordenados de
	$\bb{Z}^{n}$, $s=\{x_{0}\leq x_{1}\leq\dots\leq x_{q}\}$, que
	verifican $x_{q}^{i}-x_{0}^{i}=0\text{ o }1$ para todo
	$i=1,\,\dots,\,n$. Entonces $K$ es un complejo simplicial cuyo
	conjunto de v\'{e}rtices es $V=\bb{Z}^{n}$.
\end{ejemploReticulado}

\begin{ejemploSubcomplejoDeCaras}\label{ejemplo:subcomplejodecaras}
	Dado un complejo $K$ y un s\'{\i}mplice $s\in K$,
	el complejo de caras $\caras{s}\subset K$ es un subcomplejo de $K$
	y el complejo de caras propias $\carasp{s}\subset K$, tambi\'{e}n lo
	es. Tambi\'{e}n vale que $\carasp{s}\subset\caras{s}$ es un
	subcomplejo.
\end{ejemploSubcomplejoDeCaras}

\begin{ejemploSubcomplejosUnionInterseccion}
	\label{ejemplo:subcomplejosunioninterseccion}
	Dada una familia de subcomplejos $\{L_{i}\}_{i}$ de un complejo
	$K$, la uni\'{o}n (conjunt\'{\i}stica de los conjnuntos de
	s\'{\i}mplices) $\bigcup_{i}\,L_{i}$ y la intersecci\'{o}n (lo
	mismo) $\bigcap_{i}\,L_{i}$ son subcomplejos de $K$.
\end{ejemploSubcomplejosUnionInterseccion}

\begin{ejemploSubcomplejoDelNervio}\label{ejemplo:subcomplejodelnervio}
	Sea $X$ un conjunto, $A\subset X$ un subconjunto y sea
	$\cal{W}$ una familia de subconjuntos de $X$. Sea $\nerv{\cal{W}}$
	el nervio de la familia $\cal{W}$. Entonces, seg\'{u}n el ejemplo
	\ref{ejemplo:nerviodeunafamilia},
	\begin{align*}
		\nerv{\cal{W}} & \,=\,
			\Big\{ \{\lista{W}{r}\}\,:\,
				W_{i}\in\cal{W},\,
				W_{1}\cap\cdots\cap W_{r}\not=\varnothing
			\Big\}
		\text{ .}
	\end{align*}
	%
	Sea, entonces, $\nerv[A]{\cal{W}}$ el subconjunto de $\nerv{\cal{W}}$
	dado por
	\begin{align*}
		\nerv[A]{\cal{W}} & \,=\,
			\Big\{ \{\lista{W}{r}\}\,:\,
				W_{i}\in\cal{W},\,
				A\cap\big(W_{1}\cap\cdots\cap W_{r}\big)
					\not=\varnothing
			\Big\}
		\text{ .}
	\end{align*}
	%
	Entonces $\nerv[A]{\cal{W}}$ es un subcomplejo de $\nerv{\cal{W}}$.
\end{ejemploSubcomplejoDelNervio}

\begin{ejemploBolaYEsfera}[La bola y la esfera]\label{ejemplo:bolayesfera}
	Sea $n\geq 1$, sea $\mathrm{B}^{n+1}$ la bola unitaria en
	$\bb{R}^{n+1}$ y sea $\esfera{n}$ la esfera unitaria. Si $s$ es un
	$n+1$-s\'{\i}mplice (en alg\'{u}n complejo), existe un homeomorfismo
	entre el par $(\mathrm{B}^{n+1},\esfera{n})$ y el par
	$(\geom{\caras{s}},\geom{\carasp{s}})$.
\end{ejemploBolaYEsfera}

\begin{ejemploSegmentosEnteros}\label{ejemplo:segmentosenterostriangulacion}
	Sea $K$ el complejo del ejemplo \ref{ejemplo:segmentosenteros} y sea
	$f:\,\geom{K}\rightarrow\bb{R}$ una funci\'{o}n tal que
	$f|_{\geom{\{n\}}}=n$ y $f|_{\geom{\{n,n+1\}}}$ sea un homeo con el
	intervalo $[n,n+1]\subset\bb{R}$. Entonces $f$ es una triangulaci\'{o}n
	de $\bb{R}$.
\end{ejemploSegmentosEnteros}

\begin{ejemploReticulado}\label{ejemplo:reticuladotriangulacion}
	Sea $K$ el complejo definido en el ejemplo \ref{ejemplo:reticulado}.
	Sea $f:\,\geom{K}\rightarrow\bb{R}^{n}$ la funci\'{o}n
	$f(\alpha)^{i}=\sum_{x\in\bb{Z}^{n}}\,\alpha(x)\,x^{i}$.
	Entonces $f$ es una traingulaci\'{o}n\dots
\end{ejemploReticulado}

\begin{ejemploAproximacionesEnElCirculo}%
	\label{ejemplo:aproximacionesenelcirculo}
	Sea $s$ un $2$-s\'{\i}mplice y sea $\carasp{s}$ el complejo de sus
	caras propias. Entonces $\geom{\carasp{s}}$ es homeomorfo a
	$\esfera{1}$. En particular, las clases homotop\'{\i}a de funciones
	$\geom{\carasp{s}}\rightarrow\geom{\carasp{s}}$ son infinitas.
	Pero, para cada $n\geq 0$ entero no negativo, existen a lo sumo
	finitas transformaciones simpliciales
	$\subdiv[n]{\carasp{s}}\rightarrow\carasp{s}$. En
	consecuencia, fijado $n$, existen funciones
	$\geom{\carasp{s}}\rightarrow\geom{\carasp{s}}$ que no admiten
	aproximaciones definidas en $\subdiv[n]{\carasp{s}}$.
\end{ejemploAproximacionesEnElCirculo}

\begin{ejemploAproximacionesEnElCirculoCont}%
	\label{ejemplo:aproximacionesenelcirculocont}
	Sean $s$ y $\carasp{s}$ como en el ejemplo
	\ref{ejemplo:aproximacionesenelcirculo}. Sean $v_{0},\,v_{1},\,v_{2}$
	los v\'{e}rtices de $\carasp{s}$ (de $s$). Sean
	\begin{align*}
		v_{0} \,=\,\bari{v_{0}} & \quad\text{,}\quad
		v_{1} \,=\,\bari{v_{1}} \quad\text{y}\quad
		v_{2} \,=\,\bari{v_{2}} \text{ ,} \\
		w_{0} \,=\,\bari{\{v_{2},\,v_{0}\}} & \quad\text{,}\quad
		w_{1} \,=\,\bari{\{v_{0},\,v_{1}\}} \quad\text{y}\quad
		w_{2} \,=\,\bari{\{v_{1},\,v_{2}\}}
	\end{align*}
	%
	los baricentros del complejo y sea
	$f:\,\geom{\carasp{s}}\rightarrow\geom{\carasp{s}}$ la funci\'{o}n
	lineal dada por
	\begin{align*}
		v_{0}\,\mapsto\,w_{1}\,\mapsto\,v_{1}\,\mapsto\,w_{2}
			\,\mapsto\,v_{2}\,\mapsto\,w_{0}\,\mapsto\,v_{0}
	\end{align*}
	%
	en los v\'{e}rtices de $\subdiv{\carasp{s}}$. Entonces $f$ es
	homot\'{o}pica a la identidad de $\geom{\carasp{s}}$, pero
	no admite una aproximaci\'{o}n simplicial de la forma
	$\carasp{s}\rightarrow\carasp{s}$. Sin embargo, existen exactamente
	ocho aproximaciones de la forma
	$\subdiv{\carasp{s}}\rightarrow\carasp{s}$ determinadas por lo que
	valen en los v\'{e}rtices $v_{0},\,v_{1},\,v_{2}$.

	En cuanto a la existencia de la homotop\'{\i}a, una posibilidad es
	desandar gradualmente la misma funci\'{o}n $f$. En cuanto a la
	segunda afirmaci\'{o}n, si $\varphi:\,\carasp{s}\rightarrow\carasp{s}$
	es una transformaci\'{o}n simplicial, entonces
	$f(v_{0})=w_{1}$ implica que $\geom{\varphi}v_{0}$ pertenece
	a la cara generada por $v_{0}$ y $v_{1}$. Como $\varphi(v_{0})$ es
	un v\'{e}rtice,
	\begin{align*}
		\varphi(v_{0}) & \,\in\,\{v_{0},v_{1}\}
		\text{ .}
	\end{align*}
	%
	An\'{a}logamente,
	\begin{align*}
		\varphi(v_{1}) & \,\in\,\{v_{1},v_{2}\}\quad\text{y} \\
		\varphi(v_{2}) & \,\in\,\{v_{2},v_{0}\}
		\text{ .}
	\end{align*}
	%
	Notemos que esta observaci\'{o}n tambi\'{e}n es v\'{a}lida si el
	dominio de $\varphi$ es cualquier subdivisi\'{o}n de $\carasp{s}$.
	Por otro lado, $w_{1}=\frac{1}{2}\,v_{0}+\frac{1}{2}\,v_{1}$ implica
	\begin{align*}
		\geom{\varphi}w_{1}& \,=\,
			\frac{1}{2}\,\varphi(v_{0})
			+ \frac{1}{2}\,\varphi(v_{1})
	\end{align*}
	%
	y $f(w_{1})=v_{1}$ implica que $\geom{\varphi}w_{1}=v_{1}$,
	tambi\'{e}n. Esto fuerza
	\begin{align*}
		\varphi(v_{0}) & \,=\,\varphi(v_{1})=v_{1}
		\text{ .}
	\end{align*}
	%
	Pero, de manera similar, se deduce que debe cumplirse
	$\varphi(v_{2})=\varphi(v_{0})=v_{0}$. Lo que es absurdo.

	En cuanto a la existencia de las aproximaciones en
	$\subdiv{\carasp{s}}$, sabemos, por lo visto en el p\'{a}rrafo
	anterior, que de existir una aproximaci\'{o}n
	$\varphi:\,\subdiv{\carasp{s}}\rightarrow\carasp{s}$ para $f$,
	$\varphi(v_{i})\in\{v_{i},v_{(i+1\mod 3)}\}$, hay dos opciones para
	$\varphi(v_{i})$, para cada $i=0,1,2$. La diferencia con el caso
	anterior es que los baricentros no se escriben como
	combinaciones propias de v\'{e}rtices de $\subdiv{\carasp{s}}$,
	\emph{son} v\'{e}rtices del complejo subdividido. Al igual que antes,
	$f(w_{i})=v_{i}$, con lo que $\varphi(w_{i})$ est\'{a} forzada a
	tomar el valor $v_{i}$, para cada $i=0,1,2$. Esto no impone condiciones
	sobre los valores de $\varphi$ en los $v_{i}$ y, cualquiera sea la
	elecci\'{o}n de dichos valores queda determinada una transformaci\'{o}n
	simplicial $\varphi:\,\subdiv{\carasp{s}}\rightarrow\carasp{s}$ que
	es una aproximaci\'{o}n simplicial de $f$.
\end{ejemploAproximacionesEnElCirculoCont}

\begin{ejemploFuncionesNulhomotopicasEntreEsferas}%
	\label{ejemplo:funcionesnulhomotopicasentreesferas}
	Sea $n\geq 1$ y sea $m<n$. Toda funci\'{o}n
	$\esfera{m}\rightarrow\esfera{n}$ es homot\'{o}pica a una constante.
	Sea $s_{1}$ un $(m+1)$-s\'{\i}mplice y sea $s_{2}$ un
	$(n+1)$-s\'{\i}mplice. Entonces $\esfera{m}$ es homeomorfa a
	$\geom{\carasp{s_{1}}}$ y $\esfera{n}$ es homeomorfa a
	$\geom{\carasp{s_{2}}}$. Sea
	$f:\,\geom{\carasp{s_{1}}}\rightarrow\geom{\carasp{s_{1}}}$ una
	funci\'{o}n continua. Porque $\esfera{m}$ es compacta, el teorema
	de existencia de aproximaciones, \ref{thm:existenciadeaproximaciones},
	implica que para $i$ suficientemente grande, existe una
	aproximaci\'{o}n simplicial
	$\varphi:\,\subdiv[i]{\carasp{s_{1}}}\rightarrow\carasp{s_{2}}$ de
	$f$ y, por el lema \ref{thm:aproximacioneshomotopica},
	$\geom{\varphi}\simeq f$. Entonces, para demostrar que toda $f$ es
	homot\'{o}pica a una constante, ser\'{a} suficiente probar que, para
	toda aproximaci\'{o}n $\varphi$, la funci\'{o}n $\geom{\varphi}$ lo es.

	Como la dimensi\'{o}n del complejo $\subdiv[i]{\carasp{s_{1}}}$ es
	$m$, la imagen por una transformaci\'{o}n simplicial $\varphi$ en
	$\carasp{s_{2}}$ est\'{a} incluida en el $m$-esqueleto de
	$\carasp{s_{2}}$. En particular, como $m<n$, existe alg\'{u}n punto
	$\alpha\in\geom{\carasp{s_{2}}}$ que no pertenece a la imagen
	$\geom{\varphi}\big(\geom{\subdiv[i]{\carasp{s_{1}}}}\big)$. Pero
	esto implica que $\geom{\varphi}$ tiene imagen en el espacio
	$\geom{\carasp{s_{2}}}\setmin\{\alpha\}$, que es homeomorfo a la
	esfera $\esfera{n}$ sin un punto, que, a su vez, es homeomorfa a
	$\bb{R}^{n}$, que es contr\'{a}ctil. En definitiva, $\geom{\varphi}$
	es homot\'{o}pica a una constante.
\end{ejemploFuncionesNulhomotopicasEntreEsferas}

Un espacio topol\'{o}gico $X$ se dice \emph{$n$-conexo} ($n\geq 0$), si toda
funci\'{o}n continua $f:\,\esfera{k}\rightarrow X$ ($k\leq n$) se puede
extender de manera continua a una funci\'{o}n definida en la bola
$\disco{k+1}$. El ejemplo anterior muestra que la esfera $\esfera{n}$ es
$(n-1)$-conexa. En particular, si $n>1$, entonces $\esfera{n}$ es simplemente
conexa. Se deduce entonces que toda funci\'{o}n continua
$f:\,\esfera{n}\rightarrow\esfera{1}$ ($n>1$) se factoriza por el revestimiento
$\exp{\null}:\,\bb{R}\rightarrow\esfera{1}$. Como $\bb{R}$ es contr\'{a}ctil,
concluimos que toda funci\'{o}n continua de $f$ debe ser homot\'{o}pica a una
constante.

\begin{ejemploHomotopicasNoContiguas}\label{ejemplo:homotopicasnocontiguas}
	Sea $K_{1}=K_{2}$ el complejo del ejemplo
	\ref{ejemplo:segmentosenteros}. Entonces
	$\geom{K_{1}}=\geom{K_{2}}=\bb{R}$. Sea
	$\varphi:\,K_{1}\rightarrow K_{2}$ la identidad de complejos
	simpliciales y sea $\varphi':\,K_{1}\rightarrow K_{2}$ la
	transformaci\'{o}n constante dada por $\varphi'(n)=0$ en todo
	v\'{e}rtice $n\in\bb{Z}$ de $K_{1}$. Como el espacio del complejo
	$K_{2}$, $\bb{R}$, es contr\'{a}ctil, vale que
	$\geom{\varphi}\simeq\geom{\varphi'}$. Ahora bien, si $K_{1}'$ es una
	subdivisi\'{o}n de $K_{1}$ y $\psi,\psi':\,K_{1}'\rightarrow K_{2}$
	son transformaciones simpliciales tales que $\psi$ es una
	aproximaci\'{o}n de $\geom{\varphi}$ (la identidad) y $\psi'$ es una
	aproximaci\'{o}n de $\geom{\varphi'}$ (la funci\'{o}n constante), por
	un lado, en v\'{e}rtices, $\psi:\,\bb{Z}\rightarrow\bb{Z}$ es
	suryectiva y $\psi':\,\bb{Z}\rightarrow\bb{Z}$ es constante
	($\psi'(n)=0$ para todo $n\in\bb{Z}$). Pero, por otro lado, si
	$\psi$ y $\psi'$ fuesen contiguas, los conjuntos
	$\{\psi(n),\psi'(n)\}$ deber\'{\i}an ser s\'{\i}mplices de $K_{2}$.
	En particular --y esto es cierto en general, no s\'{o}lo en este
	ejemplo--, la imagen del conjunto de v\'{e}rtices por $\psi$
	deber\'{\i}a ser finita, si y\'{o}lo si la imagen por $\psi'$ lo
	fuese. En este caso, sin embargo, una es infinita numerable (es
	sobreyectiva) y la otra es finita (consiste en un \'{u}nico punto).
	Por lo tanto, $\psi$ y $\psi'$ no pueden pertenecer a la misma
	clase de contig\"{u}idad.
\end{ejemploHomotopicasNoContiguas}
