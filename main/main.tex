\documentclass[11pt]{report}
\usepackage[utf8]{inputenc}

%Page dimensions
\usepackage{geometry}
\geometry{a4paper}

\usepackage{graphicx}

%More packages
\usepackage{formato_general}
\usepackage{abreviaciones}
\usepackage{nombres}
\usepackage{abreviacionesPoliedros}

\title{Homotop\'{\i}as y poliedros}
\author{}
\date{24-XII-2020} % Activate to display a given date or no date (if empty),
	% otherwise the current date is printed


\begin{document}
\maketitle
\tableofcontents

%\pagebreak

%--------

\chapter{Homotop\'{\i}as}
\section{Definiciones y preliminares}
\theoremstyle{plain}
\newtheorem{teoDeLaEsferaAlDisco}{Teorema}[section]

\theoremstyle{remark}
\newtheorem{obsHomotopiaEsEquivalencia}[teoDeLaEsferaAlDisco]{Observaci\'{o}n}
\newtheorem{obsComposicionesDeHomotopicasSonHomotopicas}[teoDeLaEsferaAlDisco]%
	{Observaci\'{o}n}
\newtheorem{obsEnContractilSonHomotopicos}[teoDeLaEsferaAlDisco]%
	{Observaci\'{o}n}
\newtheorem{obsDeLaEsferaAlDisco}[teoDeLaEsferaAlDisco]{Observaci\'{o}n}

%-------------

\begin{quote}
	``Many of the computable functors, because they are computable, are
	invariant under continuous deformation. Therefore they cannot
	distiguish between spaces (or maps) that can be continuously deformed
	from one to the other; the most that can be hoped for from such
	functors is that they characterize the space (or map) up to
	continuous deformation.''
\end{quote}

Un \emph{par topol\'{o}gico} es un par de espacios $(X',X)$ junto con
una funci\'{o}n continua $f:\,X\rightarrow X'$ entre ellos (un par
topol\'{o}gico es una funci\'{o}n continua). Un morfismo de pares de
$f:\,X\rightarrow X'$ en $g:\,Y\rightarrow Y'$ consiste en un par de funciones
continuas $h:\,X\rightarrow Y$ y $h':\,X'\rightarrow Y'$ tales que
$h'\circ f= g\circ h$. Nos concentraremos en pares $(X,A)$, donde $A\subset X$
es un subespacio (y $f:\,A\rightarrow X$ es la inclusi\'{o}n).

Sean $(X,A)$ e $(Y,B)$ dos pares topol\'{o}gicos ($A\subset X$ y $B\subset Y$)
y sean $f_{0},f_{1}:\,(X,A)\rightarrow (Y,B)$ morfismos de pares que coinciden
en alg\'{u}n subespacio $X'\subset X$ (independiente de $A$ y de $B$).
Decimos que $f_{0}$ y $f_{1}$ son \emph{homot\'{o}picas relativas a $X'$},
si existe un morfismo de pares
\begin{align*}
	F & \,:\,(X\times\intervalo,A\times\intervalo)\,\rightarrow\,
		(Y,B)
\end{align*}
%
(es decir, una homotop\'{\i}a $X\times\intervalo\rightarrow Y$ tal que
$F(A\times\intervalo)\subset B$ (los puntos de $A$, que van a parar a $B$,
permanecen en $B$ durante la deformaci\'{o}n)) tal que
\begin{align*}
	F(x,0) & \,=\,f_{0}(x) \text{ ,} \\
	F(x,1) & \,=\,f_{1}(x) \quad\text{y} \\
	F(x,t) & \,=\,f_{0}(x)\,=\,f_{1}(x) \quad\text{si }x\in X'
	\text{ .}
\end{align*}
%
Denotamos esta relaci\'{o}n entre $f_{0}$ y $f_{1}$ por
$f_{0}\simeq f_{1}\,\rel{X'}$ y decimos que $F$ es una homotop\'{\i}a (de pares)
relativa a $X'$ de $f_{0}$ en $f_{1}$.

Dado $t\in\intervalo$, sea $h_{t}:\,(X,A)\rightarrow (X,A)\times\intervalo$ la
funci\'{o}n
\begin{align*}
	h_{t}(x) & \,=\,(x,t)
	\text{ .}
\end{align*}
%
Si $F$ es una homotop\'{\i}a relativa a $X'$ de $f_{0}$ en $f_{1}$, entonces
\begin{align*}
	F\circ h_{0} & \,=\,f_{0} \text{ ,} \\
	F\circ h_{1} & \,=\,f_{1} \quad\text{y} \\
	F\circ h_{t}|_{X'} & \,=\,f_{0}|_{X'} \,=\,f_{1}|_{X'}
	\text{ ,}
\end{align*}
%
para todo $t\in\intervalo$. Una \emph{familia a un par\'{a}metro} es una
familia de funciones $\{f_{t}\}_{t\in\intervalo}$ parametrizada por el
intervalo unitario (o $\bb{R}$, o\dots). En el contexto de pares
topol\'{o}gicos, se requiere que, para todo $t$, la funci\'{o}n $f_{t}$
sea un morfismo (continuo) de pares $f_{t}:\,(X,A)\rightarrow (Y,B)$ (dominio
y codominio independientes de $t$). Una familia a un par\'{a}metro
$\{f_{t}:\,(X,A)\rightarrow (Y,B)\}_{t\in\intervalo}$ se dice \emph{continua},
si el morfismo de pares
\begin{align*}
	\big((x,t)\mapsto f_{t}(x)\big) & \,:\,
		(X\times\intervalo,A\times\intervalo)\,\rightarrow (Y,B)
\end{align*}
%
es continuo. En tal caso, la aplicaci\'{o}n $F(x,t)=f_{t}(x)$ define una
homotop\'{\i}a de $f_{0}$ en $f_{1}$. Toda familia a un par\'{a}metro
tiene asociada una funci\'{o}n
\begin{align*}
	(t\mapsto f_{t}) & \,:\,\intervalo\,\rightarrow\,
		\continuas{(X,A)}{(Y,B)}
	\text{ .}
\end{align*}
%
Si $\intervalo$ tiene la topolog\'{\i}a usual y a $\continuas{(X,A)}{(Y,B)}$
se le da la topolog\'{\i}a compacto abierta, entonces la funci\'{o}n
asociada a una familia a un par\'{a}metro continua es continua.
Rec\'{\i}procamente, si $X$ es localmente compacto Hausdorff, entonces
dada una funci\'{o}n continua
\begin{math}
	\phi:\,\intervalo\rightarrow\continuas{(X,A)}{(Y,B)}
\end{math}~,
la familia a un par\'{a}metro $\{\phi(t)\}_{t}$ es continua y define una
homotop\'{\i}a de $\varphi(0)$ en $\varphi(1)$.

\begin{obsHomotopiaEsEquivalencia}\label{obs:homotopiaesequivalencia}
	La relaci\'{o}n de homotop\'{\i}a relativa a un subespacio es de una
	relaci\'{o}n de equivalencia. Si $f:\,(X,A)\rightarrow (Y,B)$ es
	continua y $X'\subset X$ es un subespacio, entonces la homotop\'{\i}a
	constante $F(x,t)=f(x)$ realiza $f\simeq f\,\rel{X'}$; si
	$f_{0},f_{1}:\,(X,A)\rightarrow (Y,B)$ y $f_{0}\simeq f_{1}\,\rel{X'}$
	y $F$ es una homotop\'{\i}a de $f_{0}$ en $f_{1}$ relativa a $X'$,
	entonces la homotop\'{\i}a inversa $(x,t)\mapsto F(x,1-t)$ realiza
	$f_{1}\simeq f_{0}\,\rel{X'}$; finalmente, si
	$f_{2}:\,(X,A)\rightarrow (Y,B)$ es un tercer morfismo de pares y
	$G$ es una homotop\'{\i}a de $f_{1}$ en $f_{2}$ relativa a $X'$,
	entonces la funci\'{o}n
	\begin{align*}
		H(x,t) & \,=\,
			\begin{cases}
				F(x,2t) & \quad\text{si }t\leq 1/2 \\
				G(x,2t-1) & \quad\text{si }t\geq 1/2
			\end{cases}
	\end{align*}
	%
	es continua, por ser continua en cada uno de los cerrados en donde
	es definida, verifica $F(A\times\intervalo)\subset B$ y es una
	homotop\'{\i}a de $f_{0}$ en $f_{2}$ relativa a $X'$.

	Dados pares  $(X,A)$ e $(Y,B)$ definimos el \emph{conjunto de clases %
	de homotop\'{\i}a de $(X,A)$ en $(Y,B)$ relativas a $X'$} como el
	conjunto de clases de equivalencia respecto de esta relaci\'{o}n.
	Denotamos este conjunto por $\homotopicas[X']{X,A}{Y,B}$ y la clase
	de un morfismo $f:\,(X,A)\rightarrow (Y,B)$ por $\clase[X']{f}$.
\end{obsHomotopiaEsEquivalencia}

\begin{obsComposicionesDeHomotopicasSonHomotopicas}%
	\label{obs:composicionesdehomotopicassonhomotopicas}
	Sean $f_{0},f_{1}:\,(X,A)\rightarrow (Y,B)$ y sean
	$g_{0},g_{1}:\,(Y,B)\rightarrow (Z,C)$. Sea $X'\subset X$ y sea
	$Y'\subset Y$. Supongamos que $f_{0}|_{X'}=f_{1}|_{X'}$, que
	$g_{0}|_{Y'}=g_{1}|_{Y'}$ y que $f_{0}(X')=f_{1}(X')\subset Y'$.
	Si $f_{0}\simeq f_{1}\,\rel{X'}$ y $g_{0}\simeq g_{1}\,\rel{Y'}$
	entonces
	\begin{align*}
		g_{0}\circ f_{0} & \,\simeq\, g_{1}\circ f_{1}\quad\rel{X'}
		\text{ .}
	\end{align*}
	%
	Si $F$ realiza la homotop\'{\i}a entre $f_{0}$ y $f_{1}$ y $G$ realiza
	la homotop\'{\i}a entre $g_{0}$ y $g_{1}$, entonces
	\begin{align*}
		g_{0}\circ F & \,:\,(X\times\intervalo,A\times\intervalo)
			\,\rightarrow\,(Y,B)\,\rightarrow\,(Z,C)
	\end{align*}
	%
	es una homotop\'{\i}a de $g_{0}\circ f_{0}$ en $g_{0}\circ f_{1}$
	relativa a $X'$ y
	\begin{align*}
		G\circ (f_{1}\times\id[\intervalo]) & \,:\,
			(X\times\intervalo,A\times\intervalo)\,\rightarrow\,
			(Y\times\intervalo,B\times\intervalo)\,\rightarrow\,
			(Z,C)
	\end{align*}
	%
	es una homotop\'{\i}a de $g_{0}\circ f_{1}$ en $g_{1}\circ f_{1}$
	relativa a $f_{1}^{-1}(Y')$. Pero $f_{1}(X')\subset Y'$ implica que
	esta homotop\'{\i}a tambi\'{e}n es relativa a $X'$. Por la
	observaci\'{o}n \ref{obs:HomotopiaEsEquivalencia}, la relaci\'{o}n de
	homotop\'{\i}a es transitiva y se deduce que $g_{0}\circ f_{0}$ es
	homot\'{o}pica a $g_{1}\circ f_{1}$ relativa a $X'$.
\end{obsComposicionesDeHomotopicasSonHomotopicas}

Teniendo en cuenta estas observaciones, podemos definir la
\emph{categor\'{\i}a homot\'{o}pica de pares} como la categor\'{\i}a cuyos
objetos son pares de espacios topol\'{o}gicos, al igual que en la
categor\'{\i}a de pares, $(X,A)$, $(Y,B)$, pero cuyos morfismos son las clases
de homotop\'{\i}a de morfismos de pares relativas al conjunto vac\'{\i}o,
$\homotopicas{X,A}{Y,B}=\homotopicas[\varnothing]{X,A}{Y,B}$. La clase de un
morfismo de pares relativa al conjunto vac\'{\i}o la denotamos simplemente
$\clase{f}$. Diremos que un diagrama
\begin{center}
	\begin{tikzcd}
		(X,A) \arrow[r] \arrow[d] & (Y,B) \arrow[d] \\
		(X',A') \arrow[r] & (Y'B')
	\end{tikzcd}
\end{center}
\emph{conmuta m\'{o}dulo homotop\'{\i}as}, si el diagrama correspondiente en
la categor\'{\i}a homoto\'{o}pica de pares es conmutativo. Una
\emph{equivalencia homot\'{o}pica} es un morfismo de pares
$f:\,(X,A)\rightarrow (Y,B)$ tal que su clase de homotop\'{\i}a
$\clase{f}\in\homotopicas{X,A}{Y,B}$ sea un isomorfismo en la categor\'{\i}a
homot\'{o}pica, es decir, existiese $g:\,(Y,B)\rightarrow (X,B)$ tal que
\begin{align*}
	\clase{g\circ f} & \,=\,\clase{g}\circ\clase{f} \,=\,
		\clase{\id[(X,A)]} \quad\text{y} \\
	\clase{f\circ g} & \,=\,\clase{f}\circ\clase{g} \,=\,
		\clase{\id[(Y,B)]}
	\text{ .}
\end{align*}
%
Tambi\'{e}n diremos que $\clase{f}$ es una equivalencia homot\'{o}pica y que
$g$, o que $\clase{g}$ es su \emph{inversa homot\'{o}pica}. Decimos que dos
pares $(X,A)$ e $(Y,B)$ \emph{tienen el mismo tipo homot\'{o}pico}, si
existe una equivalencia homot\'{o}pica entre ellos, es decir, si son isomorfos
en la categor\'{\i}a homot\'{o}pica.

Un espacio topol\'{o}gico $X$ se dice \emph{contr\'{a}ctil}, si la identidad
$\id[X]$ es homot\'{o}pica a una funci\'{o}n $X\rightarrow X$ constante, es
decir, si existe $x_{0}\in X$ tal que $\id[X]\simeq (x\mapsto x_{0})$. Una
\emph{contracci\'{o}n} de $X$ en $x_{0}\in X$ es una homotop\'{\i}a
de $F$ de $\id[X]$ en a funci\'{o}n constante $x\mapsto x_{0}$.

\begin{obsEnContractilSonHomotopicos}\label{obs:encontractilsonhomotopicos}
	Sea $Y$ un espacio topol\'{o}gico contr\'{a}ctil. Sea $y_{0}\in Y$
	y sea $c:\,Y\rightarrow Y$ la funci\'{o}n constante $c(y)=y_{0}$.
	Supongamos que $\id[Y]\simeq c$. Si $f_{0},f_{1}:\,X\rightarrow Y$
	son funciones continuas, entonces
	\begin{align*}
		f_{0} & \,=\,\id[Y]\circ f_{0} \,\simeq\,c\circ f_{0}
			\quad\text{y} \\
		f_{1} & \,=\,\id[Y]\circ f_{1} \,\simeq\,c\circ f_{1}
		\text{ .}
	\end{align*}
	%
	Pero $c\circ f_{0}=c\circ f_{1}$. Por transitividad,
	$f_{0}\simeq f_{1}$. En definitiva, el conjunto $\homotopicas{X}{Y}$
	consiste en una \'{u}nica clase cualquiera sea el espacio $X$, dos
	funciones continuas de $X$ en $Y$ son homot\'{o}picas.

	Si definimos $\tilde{c}:\,Y\rightarrow \{y_{0}\}$ (si identificamos $c$
	con la funci\'{o}n cuyo codominio es el conjunto puntual $\{y_{0}\}$),
	entonces $\tilde{c}$ es una equivalencia homot\'{o}pica: su inversa
	homot\'{o}pica est\'{a} dada por la inclusi\'{o}n
	$\inc:\,\{y_{0}\}\rightarrow Y$. Pero entonces $Y$ tiene el tipo
	homot\'{o}pico del punto $\{y_{0}\}$. En particular,
	\begin{align*}
		\homotopicas{X}{Y} & \,\simeq\,\homotopicas{X}{\{y_{0}\}}
	\end{align*}
	%
	v\'{\i}a
	\begin{math}
		\clase{f}\mapsto
			\clase{\tilde{c}}\circ\clase{f}=
			\clase{\tilde{c}\circ f}
	\end{math}~.
	Pero el conjunto $\homotopicas{X}{\{y_{0}\}}$ posee una \'{u}nica
	clase, la clase de la \'{u}nica funci\'{o}n $X\rightarrow \{y_{0}\}$.
	Concluimos, de esta manera tambi\'{e}n, que todo par de funciones
	$f_{0},f_{1}:\,X\rightarrow Y$ cuyo codominio es contr\'{a}ctil son
	homot\'{o}picas.

	Notemos que la noci\'{o}n de ser contr\'{a}ctil equivale a tener el
	tipo homot\'{o}pico del punto: ya vimos que
	$\id[\{y_{0}\}]=\tilde{c}\circ\inc$ y que
	$\id[Y]\simeq\inc\circ\tilde{c}$, con lo que $Y$ y el conjunto puntual
	$\{y_{0}\}$ tienen el mismo tipo homot\'{o}pico. Rec\'{\i}procamente,
	si $Y$ tiene el tipo homot\'{o}pico de un conjunto puntual $\{y_{0}\}$,
	entonces existen $f:\,Y\rightarrow \{y_{0}\}$ y
	$g:\,\{y_{0}\}\rightarrow Y$ tales que $f\circ g=\id[\{y_{0}\}]$
	(pues no hay otra funci\'{o}n $\{y_{0}\}\rightarrow\{y_{0}\}$) y
	$g\circ f\simeq\id[Y]$. En particular, la segunda equivalencia
	implica que $Y$ es contr\'{a}ctil.
\end{obsEnContractilSonHomotopicos}

La existencia de homotop\'{\i}as est\'{a} relacionada con la posibilidad de
extender funciones.

\begin{teoDeLaEsferaAlDisco}\label{thm:delaesferaaldisco}
	Sea $p_{0}\in\esfera{n}$ un punto arbitrario de la esfera de
	dimensi\'{o}n $n$ ($n\geq 1$, $n\geq 0$) y sea
	$f:\,\esfera{n}\rightarrow Y$ una funci\'{o}n continua. Las
	siguientes afirmaciones son equivalentes:
	\begin{itemize}
		\item[(i)] $f$ es homot\'{o}pica a una constante;
		\item[(ii)] $f$ se puede extender de manera continua al
			disco $\disco{n+1}$;
		\item[(iii)] $f$ es homot\'{o}pica a una constante relativa
			al punto $\{p_{0}\}$.
	\end{itemize}
	%
\end{teoDeLaEsferaAlDisco}

La \'{u}ltima afirmaci\'{o}n es equivalente a que $f\simeq (x\mapsto p_{0})$
v\'{\i}a una homotop\'{\i}a que deje quieto el punto $p_{0}$.

\begin{proof}
	\emph{(i) implica (ii):} sea $F$ una homotop\'{\i}a de $f$ en una
	funci\'{o}n constante $c:\,\esfera{n}\rightarrow Y$ y sea $y_{0}\in Y$
	tal que $c(x)=y_{0}$ para todo $x\in\esfera{n}$. Sea
	$f':\,\disco{n+1}\rightarrow y$ la funci\'{o}n
	\begin{align*}
		f'(x) & \,=\,
			\begin{cases}
				y_{0} & \quad\text{si }
					0\leq |x|\leq 1/2 \\[10pt]
				F\Big(\dfrac{x}{|x|},2-2|x|\Big)
					& \quad\text{si }1/2\leq |x|\leq 1
			\end{cases}
		\text{ .}
	\end{align*}
	%
	Entonces $f'$ est\'{a} bien definida y es continua en $\disco{n+1}$.
	Adem\'{a}s, como $F(x,0)=f(x)$ para $x\in\esfera{n}$, concluimos que
	$f'$ es una extensi\'{o}n continua de $f$ al disco.

	\emph{(ii) implica (iii):} Si $f':\,\disco{n+1}\rightarrow Y$ es una
	extensi\'{o}n continua de $f$, la expresi\'{o}n
	\begin{align*}
		F(x,t) & \,=\,f'((1-t)\,x)
	\end{align*}
	%
	define una homotop\'{\i}a de $f$ en la funci\'{o}n constante
	$x\mapsto f'(0)$. Pero esta homotop\'{\i}a no es necesariamente
	relativa a un punto de la esfera. Entonces, dado $p_{0}\in\esfera{n}$,
	sea $F:\,\esfera{n}\times\intervalo\rightarrow Y$ la funci\'{o}n
	\begin{align*}
		F(x,t) & \,=\,f'((1-t)\,x+t\,p_{0})
		\text{ .}
	\end{align*}
	%
	Esta funci\'{o}n es continua, porque $f'$ lo es, y verifica:
	\begin{align*}
		F(x,0) & \,=\,f'(x) \text{ ,} \\
		F(x,1) & \,=\,f'(p_{0}) \quad\text{y} \\
		F(p_{0},t) & \,=\,f'(p_{0})
		\text{ .}
	\end{align*}
	%
	Entonces, como $f'|_{\esfera{n}}=f$, $F$ es una homotop\'{\i}a de
	$f$ en la funci\'{o}n constante $x\mapsto f(p_{0})$ relativa a
	$\{p_{0}\}$.
\end{proof}

\begin{obsDeLaEsferaAlDisco}\label{obs:delaesferaaldisco}
	De la observaci\'{o}n \ref{obs:encontractilsonhomotopicos} y del
	teorema \ref{thm:delaesferaaldisco}, se deduce que toda funci\'{o}n de
	$\esfera{n}$ en un espacio contr\'{a}ctil admite una extensi\'{o}n
	continua al disco $\disco{n+1}$.
\end{obsDeLaEsferaAlDisco}

%
\section{Retracciones}
\theoremstyle{plain}
\newtheorem{teoExtensionDeHomotopiasRetractoEquivaleARetractoDebil}{Teorema}%
	[section]

\theoremstyle{remark}
\newtheorem{obsExtensionDeHomotopias}%
	[teoExtensionDeHomotopiasRetractoEquivaleARetractoDebil]%
	{Observaci\'{o}n}

%-------------

Sea $X$ un espacio topol\'{o}gico y sea $A\subset X$ un subespacio. Se dice
que $A$ es un \emph{retracto de $X$}, si existe $r:\,X\rightarrow A$, continua,
tal que $r\circ\inc[A]=\id[A]$. Se dice que $A$ es un \emph{retracto %
d\'{e}bil de $X$}, si existe $r:\,X\rightarrow A$, continua, tal que
$r\circ\inc[A]\simeq\id[A]$. En el primer caso, decimos que $r$ es una
\emph{retracci\'{o}n} y, en el segundo, que es una \emph{retracci\'{o}n %
d\'{e}bil}. En este segundo caso, no se requiere que exista una homotop\'{\i}a
reltiva a $A$ de $r\circ\inc[A]$ en $\id[A]$, s\'{o}lo que exista alguna
homotop\'{\i}a. En otras palabras, $A\subset X$ es un retracto de $X$, si la
inclusi\'{o}n $\inc[A]:\,A\rightarrow X$ admite una inversa a izquierda en la
categor\'{\i}a de espacios topol\'{o}gicos y es un retracto d\'{e}bil de $X$,
si la inclusi\'{o}n admite una inversa a izquierda en la categor\'{\i}a
homot\'{o}pica.

Un poco m\'{a}s en general, decimos que una funci\'{o}n $r:\,X\rightarrow Y$
es una retracci\'{o}n, si admite una inversa a derecha, una funci\'{o}n
$j:\,Y\rightarrow X$ tal que $r\circ j=\id[Y]$ (y decimos que $Y$ es un
retracto de $X$, si $r$ es la inversa a izquierda de alguna funci\'{o}n
$Y\rightarrow X$). An\'{a}logamente, decimos que la funci\'{o}n $r$ es una
retracci\'{o}n d\'{e}bil, si admite una inversa a derecha en la categor\'{\i}a
homot\'{o}pica, es decir, existe $j:\,Y\rightarrow X$ y una homotop\'{\i}a
$r\circ j\simeq\id[Y]$.

Sean $X$ e $Y$ espacios topol\'{o}gicos y sea $A\subset X$ un subespacio.
Decimos que el par $(X,A)$ \emph{tiene la propiedad de extensi\'{o}n de
homotop\'{\i}as con respecto a $Y$}, si, dadas $g:\,X\rightarrow Y$ y
$G:\,A\times\intervalo\rightarrow Y$ tal que
\begin{align*}
	G(x,0) & \,=\,g(x)\quad\text{para todo }x\in A
	\text{ ,}
\end{align*}
%
existe $F:\,X\times\intervalo\rightarrow Y$ tal que
\begin{align*}
	F(x,0) & \,=\,g(x) \quad\text{para todo }x\in X\quad\text{y} \\
	F|_{A\times\intervalo} & \,=\,G
	\text{ .}
\end{align*}
%
En t\'{e}rminos de diagramas, $(X,A)$ tiene la propiedad de extensi\'{o}n
de homotop\'{\i}as con respecto a $Y$, si para todo diagrama
\begin{center}
\begin{tikzcd}[column sep=small]%,row sep=small]
	A\times\{0\} \arrow[rr,hook] \arrow[dd,hook] & &
		A\times\intervalo \arrow[dl,"G"'] \arrow[dd,hook] \\
	& Y & \\
	X\times\{0\} \arrow[ur,"g"] \arrow[rr,hook] & &
		X\times\intervalo \arrow[ul,dotted, "F"']
\end{tikzcd}
\end{center}
con el tri\'{a}ngulo por encima de la diagonal conmutativo, existe
$F:\,X\times\intervalo\rightarrow Y$ que hace conmutar a los tri\'{a}ngulos
por debajo de la diagonal.

\begin{obsExtensionDeHomotopias}\label{obs:extensiondehomotopias}
	Si $(X,A)$ tiene la propiedad de extensi\'{o}n de homotop\'{\i}as
	con respecto a un espacio $Y$ y $f_{0},f_{1}:\,A\rightarrow Y$
	son homot\'{o}picas (en $A$), entonces $f_{0}$ tiene una extensi\'{o}n
	continua a $X$, si y s\'{o}lo si existe una extensi\'{o}n continua de
	$f_{1}$. Si $g:\,X\rightarrow Y$ es una extensi\'{o}n de $f_{0}$ y
	$G:\,A\times\intervalo\rightarrow Y$ es una homotop\'{\i}a en $A$ de
	$f_{0}=g|_{A}$ en $f_{1}$, entonces existe una ``extensi\'{o}n'' de
	$G$, $F;\,X\times\intervalo\rightarrow Y$, tal que
	$F(x,0)=g(x)$ para $x\in X$ y $F(x,t)=G(x,t)$ para todo $x\in A$.
	En particular, si $x\in A$, $F(x,0)=f_{0}(x)$ y $F(x,1)=f_{1}(x)$,
	con lo que $x\mapsto F(x,0)$ es la extensi\'{o}n $g$ de $f_{0}$ y
	$x\mapsto F(x,1)$ es una extensi\'{o}n de $f_{1}$ a todo el espacio
	$X$.

	De esto se deduce que, si $(X,A)$ tiene la propiedad de extensi\'{o}n
	de homotop\'{\i}as con respecto a $Y$, entonces el problema de
	determinar si una funci\'{o}n $A\rightarrow Y$ se puede extender a $X$
	es un problema en la categor\'{\i}a homot\'{o}pica.
\end{obsExtensionDeHomotopias}

En general, un morfismo $f:\,X'\rightarrow X$ se dice que es una
\emph{cofibraci\'{o}n}, si dados un objeto arbitrario $Y$ y morfismos
$g:\,X\rightarrow Y$ y $G:\,X'\times\intervalo\rightarrow Y$ tales que
\begin{align*}
	g\circ f(x') & \,=\,G(x',0)
	\text{ ,}
\end{align*}
%
es decir, $g\circ (f\times\id[\{0\}])=G\circ\inc[X'\times\{0\}]$, existe
$F:\,X\times\intervalo\rightarrow Y$ tal que $F(x,0)=g(x)$ para todo
$x\in X$ y $F(f(x'),t)=G(x',t)$ para todo $x'\in X'$ y $t\in\intervalo$.
Dicho de otra manera, para \emph{todo} diagrama
\begin{center}
\begin{tikzcd}[column sep=small]
	X'\times\{0\} \arrow[rr,hook] \arrow[dd,"f\times{\id[\{0\}]}"'] & &
		X'\times\intervalo \arrow[dl,"G"']
			\arrow[dd,"f\times{\id[\intervalo]}"] \\
	& Y & \\
	X\times\{0\} \arrow[rr,hook] \arrow[ur,"g"] & &
		X\times\intervalo \arrow[ul,dotted,"F"']
\end{tikzcd}
\end{center}
cuyo tri\'{a}ngulo encima de la diagonal sea conmutativo, existe una
$F$ tal que los tri\'{a}ngulos inferiores tambi\'{e}n conmuten. En estos
t\'{e}rminos, la inclusi\'{o}n $\inc[A]:\,A\rightarrow X$ es una
cofibraci\'{o}n, si y s\'{o}lo si $(X,A)$ tiene la propiedad de extensi\'{o}n
de homotop\'{\i}as con respecto a cualquier espacio.

\begin{teoExtensionDeHomotopiasRetractoEquivaleARetractoDebil}%
	\label{thm:extensiondehomotopiasretractoequivalearetractodebil}
	Sea $A\subset X$ un subespacio. Si $(X,A)$ tiene la propiedad de
	extensi\'{o}n de homotop\'{\i}as con respecto a $A$, entonces
	$A$ es un retracto de $X$, si s\'{o}lo si es un retracto d\'{e}bil.
\end{teoExtensionDeHomotopiasRetractoEquivaleARetractoDebil}

\begin{proof}
	Sea $r:\,X\rightarrow A$ tal que $r\circ\inc[A]\simeq\id[A]$ y sea
	$G:\,A\times\intervalo\rightarrow A$ una homotop\'{\i}a de
	$r\circ\inc[A]$ en $\id[A]$. Si $(X,A)$ tiene la propiedad de
	extensi\'{o}n de homotop\'{\i}as con respecto a $A$, entonces
	existe $F:\,X\times\intervalo\rightarrow A$ tal que
	\begin{align*}
		F(\inc[A](x'),t) & \,=\,G(x',t) \quad\text{y} \\
		F(x,0) & \,=\,r(x)
	\end{align*}
	%
	para todo $x'\in A$ y todo $x\in X$. Sea $r':\,X\rightarrow A$
	la funci\'{o}n continua $r'(x)=F(x,1)$. Entonces, por un lado, $F$ es
	una homotop\'{\i}a de $r$ en $r'$ definida en $X$ y, por otro,
	\begin{align*}
		r'\circ\inc[A](x') & \,=\, F(\inc[A](x'),1)
			\,=\, G(x',1) \,=\,\id[A](x')
		\text{ ,}
	\end{align*}
	%
	con lo que $r'$ es una retracci\'{o}n de $X$ en $A$.
\end{proof}

Notemos que la retracci\'{o}n $r'$ obtenida en la demostraci\'{o}n de
\ref{thm:extensiondehomotopiasretractoequivalearetractodebil} es homot\'{o}pica
a la retracci\'{o}n d\'{e}bil $r$.

%
\section{Deformaciones}
\theoremstyle{plain}
\newtheorem{teoExtensionDeHomotopiasYDeformaciones}{Teorema}[section]
\newtheorem{coroExtensionDeHomotopiasRetractoEquivaleARetractoDebil}%
	[teoExtensionDeHomotopiasYDeformaciones]{Corolario}

\theoremstyle{remark}
\newtheorem{obsDeformableEquivaleATieneInversaHomotopicaADerecha}%
	{Observaci\'{o}n}[section]

%-------------

Sea $X$ un espacio topol\'{o}gico. En la secci\'{o}n anterior consideramos
subespacios $A\subset X$ tales que la inclusi\'{o}n $\inc[A]:\,A\rightarrow X$
admite una inversa a izquierda en la categor\'{\i}a de espacios topol\'{o}gicos
y funciones continuas, o bien en la categor\'{\i}a homot\'{o}pica. Dado un
subespacio $X'\subset X$, una \emph{deformaci\'{o}n} de $X'$ en (dentro de)
$X$ es una homotop\'{\i}a $D:\,X'\times\intervalo\rightarrow X$ tal que
$D(x',0)=x'$ para todo $x'\in X'$. Si $D\big(X'\times\{1\}\big)\subset A$
para cierto subespacio $A$ de $X$, se dice que $X'$ es \emph{deformable en %
$X$ en $A$}. Si $X'=X$ se dice, simplemente, que $X$ es deformable en $A$.

\begin{obsDeformableEquivaleATieneInversaHomotopicaADerecha}%
	\label{obs:deformableequivaleatieneinversahomotopicaaderecha}
	Sea $X$ un espacio topol\'{o}gico y sea $A\subset X$ un subespacio.
	Entonces $X$ es deformable en $A$, si y s\'{o}lo si la inclusi\'{o}n
	$\inc[A]:\,A\rightarrow X$ admite una inversa homot\'{o}pica
	\emph{a derecha}. Veamos que esto es as\'{\i}. Si existe
	$f:\,X\rightarrow A$ y una homotop\'{\i}a
	$F:\,X\times\intervalo\rightarrow X$ tal que
	\begin{align*}
		F(x,0) & \,=\,x \quad\text{y} \\
		F(x,1) & \,=\,\inc[A]\circ f(x)
	\end{align*}
	%
	para todo $x\in X$ (existe una inversa a derecha de $\inc[A]$ en la
	categor\'{\i}a homot\'{o}pica), entonces $F$ es una deformaci\'{o}n
	de $X$ (dentro de $X$) y
	\begin{align*}
		F\big(X\times\{1\}\big) & \,=\,\inc[A]\circ f(X) \,\subset\,A
	\end{align*}
	%
	con lo que, por definici\'{o}n, $X$ es deformable en $A$.
	Rec\'{\i}procamente, si $X$ es deformable en $A$ y
	$D:\,X\times\intervalo\rightarrow X$ es una deformaci\'{o}n de $X$
	(una homotop\'{\i}a que cumple $D(x,0)=x$ en $X$) tal que
	$D\big(X\times\{1\}\big)\subset A$, entonces existe una \'{u}nica
	funci\'{o}n (de conjuntos) $f:\,X\rightarrow A$ tal que
	\begin{align*}
		\inc[A]\circ f(x) & \,=\,D(x,1)
	\end{align*}
	%
	para todo $x\in X$. Como $A$ es un subespacio, $f$ es continua.
	Como $D$ es una deformaci\'{o}n de $X$, $D$ es una homotop\'{\i}a de
	la identidad $\id[X]$ en $\inc[A]\circ f$, es decir, $\inc[A]$
	admite una inversa a derecha en la categr\'{\i}a homot\'{o}pica.
\end{obsDeformableEquivaleATieneInversaHomotopicaADerecha}

En la secci\'{o}n anterior consideramos dos tipos de retractos: aquellos
subespacios $A\subset X$ para los cuales existe una funci\'{o}n
$r:\,X\rightarrow A$ tal que $r\circ\inc[A]=\id[A]$ y aquellos subespacios
para los cuales existe una funci\'{o}n $r:\,X\rightarrow A$ tal que
$r\circ\inc[A]\simeq\id[A]$. En esta secci\'{o}n, dado un subespacio
$A\subset X$, nos preguntamos si existe una funci\'{o}n $f:\,X\rightarrow A$
tal que $\inc[A]\circ f\simeq\id[X]$, es decir, si la inclusi\'{o}n
$\inc[A]:\,A\rightarrow X$ admite una inversa homot\'{o}pica a derecha.
Notemos que el problema de la existencia de inversas a derecha de $\inc[A]$ en
la categor\'{\i}a de espacios topol\'{o}gicos y funciones continuas es trivial,
ya que, si $\inc[A]\circ f=\id[X]$, entonces $\inc[A]$ es sobre y $A=X$.

Dado un subespacio $A\subset X$, se dice que $A$ es un \emph{retracto por %
deformaci\'{o}n d\'{e}bil de $X$}, si $\inc[A]$ es una equivalencia
homot\'{o}pica: esto quiere decir que existen $f,r:\,X\rightarrow A$ tales que
$\inc[A]\circ f\simeq\id[X]$ y $r\circ\inc[A]\simeq\id[A]$ (necesariamente,
$r\simeq f$). Seg\'{u}n lo mencionado en la observaci\'{o}n
\ref{obs:deformableequivaleatieneinversahomotopicaaderecha},
$\inc[A]\circ f\simeq\id[X]$ equivale a que $X$ sea deformable en $A$ y
$r\circ\inc[A]\simeq\id[A]$ significa que $A$ es un retracto d\'{e}bil de $X$.
(Un retracto por deformaci\'{o}n d\'{e}bil es un retracto d\'{e}bil que es,
adem\'{a}s, deformaci\'{o}n).

Sea $A\subset X$ un subespacio. Si $A$ es un retracto de $X$ (existe
$r:\,X\rightarrow A$ tal que $r\circ\inc[A]=\id[A]$) y la inclusi\'{o}n
admite una inversa homot\'{o}pica a derecha, $f:\,X\rightarrow A$, entonces
decimos que $A$ es un \emph{retracto por deformaci\'{o}n de $X$}. Notemos que
$\inc[A]$ es una equivalencia homot\'{o}pica (y, necesariamente, $r\simeq f$),
con lo cual todo retracto por deformaci\'{o}n es un retracto por
deformaci\'{o}n d\'{e}bil. (Un retracto por deformaci\'{o}n es un retracto que
es deformaci\'{o}n).

Un \emph{retracto por deformaci\'{o}n fuerte} de un espacio $X$ es un
subespacio $A\subset X$ que es retracto y existe una funci\'{o}n
$f:\,X\rightarrow A$ tal que $\inc[A]\circ f\simeq\id[X]\,\rel{A}$.

para el cual existe una retracci\'{o}n
$r:\,X\rightarrow A$ tal que $\inc[A]\circ r \simeq\,\rel{A}$. En particular,
tomando $f=r$, se ve que todo retracto por deformaci\'{o}n fuerte es un
retracto por deformaci\'{o}n.


En estas definiciones, podemos asumir que $r=f$. En el caso de la primera
definici\'{o}n, si $\inc[A]\circ f\simeq\id[X]$ y $r\circ\inc[A]\simeq\id[A]$,
entonces
\begin{align*}
	\clase{\inc[A]}\,\clase{f} & \,=\,\clase{\id[X]}\quad\text{y} \\
	\clase{r}\,\clase{\inc[A]} & \,=\,\clase{\id[A]}
	\text{ .}
\end{align*}
%
Componiendo y cancelando, se deduce que $\clase{f}=\clase{r}$, es decir,
$f\simeq r$. Por lo tanto, $r$ verifica $\inc[A]\circ r\simeq\id[X]$,
tambi\'{e}n. Con respecto a la segunda definici\'{o}n el mismo argumento
muestra que la retracci\'{o}n $r$ verifica la condici\'{o}n sobre $f$.
En la definici\'{o}n de retracto por deformaci\'{o}n fuerte, el argumento debe
ser distinto, pues la relaci\'{o}n de homotop\'{\i}a se asume relativa al
subespacio $A$. Aun as\'{\i}, si
\begin{align*}
	\inc[A]\circ f & \,\simeq\,\id[X]\,\rel{A}\quad\text{y} \\
	r\circ\inc[A] & \,=\,\id[A]
	\text{ ,}
\end{align*}
%
entonces $r$ verifica
\begin{align*}
	\inc[A]\circ r & \,=\,\inc[A]\circ r\circ\id[X]
		\,\simeq_{(\rel{A})}\,\inc[A]\circ r\circ\inc[A]\circ f
		\,=\,\inc[A]\circ f
	\text{ .}
\end{align*}
%
Pero entonces $r$ verifica la condici\'{o}n sobre $f$ y podemos asumir que
$f=r$ en este caso tambi\'{e}n.


%----------
Si en lugar de concentrarnos en las propiedades de un espacio miramos
lo que pasa con las funciones con propiedades como las de la retracci\'{o}n
de la inclusi\'{o}n,\dots

En la secci\'{o}n anterior consideramos dos tipos de retracciones:
aquellas que admiten una inversa continua a derecha y aquellas que admiten
una inversa homot\'{o}pica a derecha, aquellas que son, respectivamente,
inversas a izquierda de la inclusi\'{o}n de un subespacio en la categor\'{\i}a
de espacios topol\'{o}gicos y funciones continuas y aquellas que son
inversas a izquierda de la inclusi\'{o}n en la categor\'{\i}a homot\'{o}pica.

%----------

\begin{coroExtensionDeHomotopiasRetractoEquivaleARetractoDebil}
	Si $(X,A)$ tiene la propiedad de extensi\'{o}n de homotop\'{\i}as
	con erspecto a $A$, entonces $A$ es un retracto por deformaci\'{o}n
	d\'{e}bil de $X$, si y s\'{o}lo si es un retracto por deformaci\'{o}n.
\end{coroExtensionDeHomotopiasRetractoEquivaleARetractoDebil}

\begin{proof}
	Este corolario es consecuencia del teorema
	\ref{thm:extensiondehomotopiasretractoequivalearetractodebil}.
\end{proof}

\begin{teoExtensionDeHomotopiasYDeformaciones}%
	\label{thm:extensiondehomotopiasydeformaciones}
	Sea $X$ un espacio topol\'{o}gico y sea $A\subset X$ un subespacio
	cerrado tal que el par $(X\times\intervalo,L)$, donde
	\begin{align*}
		L & \,=\,(X\times\{0\})\,\cup\,(A\times\intervalo)\,\cup\,
			(X\times\{1\})
		\text{ ,}
	\end{align*}
	%
	tiene la propiedad de extensi\'{o}n de homotop\'{\i}as con respecto a
	$X$. Entonces $A$ es un retracto por deformaci\'{o}n de $X$, si y
	s\'{o}lo si es un retracto por deformaci\'{o}n fuerte.
\end{teoExtensionDeHomotopiasYDeformaciones}

\begin{proof}
	Supongamos que $A$ es un retracto por deformaci\'{o}n de $X$. Sea
	$r:\,X\rightarrow A$ una retracci\'{o}n de $X$ en $A$ y sea
	$F:\,X\times\intervalo\rightarrow X$ una homotop\'{\i}a de $\id[X]$
	en $\inc[A]\circ r$. Lo que hay que ver es que existe una
	homotop\'{\i}a relativa a $A$. Sea $G:\,L\times\intervalo\rightarrow X$
	la funci\'{o}n definida por
	\begin{equation}
		\label{eq:extensiondehomotopiasydeformaciones}
	\begin{aligned}
		G((x,0),t') & \,=\, x \quad\text{en }
			(X\times\{0\})\times\intervalo \text{ ,} \\
		G((x,t),t') & \,=\, F(x,(1-t')\,t) \quad\text{en }
			(A\times\intervalo)\times\intervalo \quad\text{y} \\
		G((x,1),t') & \,=\,F(\inc[A]\circ r(x),1-t') \quad\text{en }
			(X\times\{1\})\times\intervalo
		\text{ .}
	\end{aligned}
	\end{equation}
	%
	Como $A$ es cerrado en $X$, $G$ es continua. Ahora bien, si
	$(x,t)\in L$, a tiempo $t'=0$,
	\begin{align*}
		G((x,t),0) & \,=\,F(x,t)
		\text{ ,}
	\end{align*}
	%
	pues
	\begin{equation}
		\label{eq:extensiondehomotopiasydeformacionesatiempocero}
	\begin{aligned}
		G((x,0),0) & \,=\,x\,=\,F(x,0) \text{ ,} \\
		G((x,t),0) & \,=\,F(x,t) \quad\text{y} \\
		G((x,1),0) & \,=\,F(\inc[A]\circ r(x),1) \,=\,
			(\inc[A]\circ r)\circ (\inc[A]\circ r) (x) \\
		& \,=\,\inc[A]\circ r(x) \,=\, F(x,1)
		\text{ .}
	\end{aligned}
	\end{equation}
	%
	Sea $f_{0}=G|_{L\times\{0\}}$ y sea $f_{1}=G|_{L\times\{1\}}$.
	Entonces las igualdades
	\eqref{eq:extensiondehomotopiasydeformacionesatiempocero},
	implican que $g:\,(X\times\intervalo)\times\{0\}$ definida por
	\begin{align*}
		g((x,t),0) & \,=\, F(x,t)	
	\end{align*}
	%
	es una extensi\'{o}n de $f_{0}$. Por la propiedad de extensi\'{o}n de
	homotop\'{\i}as (ver la observaci\'{o}n
	\ref{obs:extensiondehomotopias}), $f_{1}$ admite una extensi\'{o}n a
	$(X\times\intervalo)\times\{1\}$. Sea
	\begin{align*}
		G' & \,:\,(X\times\intervalo)\times\{1\}\,\rightarrow\,X
	\end{align*}
	%
	una extensi\'{o}n de $f_{1}=G|_{L\times\{1\}}$ y sea
	\begin{align*}
		H & \,:\,X\times\intervalo\,\rightarrow\,X
	\end{align*}
	%
	La funci\'{o}n
	\begin{align*}
		H(x,t) & \,=\,G'((x,t),1)
		\text{ .}
	\end{align*}
	%
	Entonces $H$ verifica
	\begin{equation}
		\label{eq:extensiondehomotopiasydeformacionesfuerte}
	\begin{aligned}
		H(x,0) & \,=\,G'((x,0),1)\,=\,G((x,0),1) \,=\,x
			\quad\text{si }x\in X \text{ ,} \\
		H(x,1) & \,=\,G'((x,1),1)\,=\,G((x,1),1) \\
		& \,=\,F(\inc[A]\circ r(x),0) \,=\,\inc[A]\circ r(x)
			\quad\text{si }x\in X\quad\text{y} \\
		H(x,t) & \,=\,G'((x,t),1)\,=\,G((x,t),1) \\
		& \,=\,F(x,0)\,=\,x
			\quad\text{si }x\in A,\,t\in\intervalo
		\text{ .}
	\end{aligned}
	\end{equation}
	%
	Entonces $H$ es una homotop\'{\i}a de $\id[X]$ en $\inc[A]\circ r$.
	relativa a $A$.
\end{proof}

%
\section{El cilindro de un morfismo}
\theoremstyle{plain}
\newtheorem{teoElCilindro}{Teorema}[section]

\theoremstyle{remark}

%-------------

Sea $f:\,X\rightarrow Y$ una funci\'{o}n continua y sea $\cilindro{f}$ el
espacio que se obtiene como cociente de $(X\times\intervalo)\sqcup Y$
por la relaci\'{o}n $(x,1)\sim f(x)$. Este espacio se denomina
\emph{cilindro de $f$}. Si $[x,t]$ denota la clase de un punto
$(x,t)\in X\times\intervalo$ e $[y]$ denota la clase de un punto $y\in Y$,
entonces las funciones $i:\,X\times\intervalo\cilindro{f}$ y
$j:\,Y\rightarrow\cilindro{f}$ dadas por
\begin{align*}
	i(x) & \,=\,[x,0]\quad\text{y} \\
	j(y) & \,=\,[y]
\end{align*}
%
realizan $X$ e $Y$ como subespacios de $\cilindro{f}$, es decir, podemos
identificar los puntos de $X$ con las clases $[x,0]$ y los puntos de $Y$ con
las clases $[y]$ en el cilindro. La funci\'{o}n $r:\,\cilindro{f}\rightarrow Y$
dada por
\begin{align*}
	& \begin{cases}
		r[x,t] \,=\, f(x) \quad\big(\equiv\,[f(x)]\big)
			& \quad\text{si }x\in X,\,t\in\intervalo
				\quad\text{y} \\[10pt]
		r[y] \,=\, y \quad\big(\equiv\,[y]\big)
			& \quad\text{si }y\in Y
	\end{cases}
	\text{ ,}
\end{align*}
%
es una retracci\'{o}n del subespacio $j:\,Y\rightarrow\cilindro{f}$, pues
\begin{align*}
	r\circ j & \,=\, \id[Y]
	\text{ .}
\end{align*}
%
Notemos que $r$ act\'{u}a como una proyecci\'{o}n, aplastando el
``cuadrado'' $X\times\intervalo$ en el subespacio $f(X)\subset Y$, siguiendo
las rectas $\{(x,t)\,t\in\intervalo\}$ hasta $t=1$.

\begin{teoElCilindro}\label{thm:elcilindro}
	Sea $f:\,X\rightarrow Y$ una funci\'{o}n continua. Sea $\cilindro{f}$
	el cilindro de $f$ y sean $i:\,X\rightarrow\cilindro{f}$ y
	$j:\,Y\rightarrow\cilindro{f}$ los embeddings $i(x)=[x,0]$ y
	$j(y)=[y]$. Existe un diagrama conmutativo
	\begin{center}
	\begin{tikzcd}[column sep=tiny]
		X \arrow[rr,"i"] \arrow[dr,"f"'] & &
			\cilindro{f} \arrow[dl,"r"] \\
		& Y &
	\end{tikzcd}
	\end{center}
	tal que \emph{(i)} $r\circ j =\id[Y]$, \emph{(ii)}
	$j\circ r\simeq\id[\cilindro{f}]\,\rel{Y}$ y \emph{(iii)} el embedding
	$i:\,X\rightarrow\cilindro{f}$ es una cofibraci\'{o}n.
\end{teoElCilindro}

\begin{proof}
	En cuanto a la existencia del diagrama conmutativo y a la
	afirmaci\'{o}n \emph{(i)}, la demostraci\'{o}n est\'{a} contenida en
	los comentarios previos al enunciado.

	Sea $F:\,\cilindro{f}\times\intervalo\rightarrow\cilindro{f}$ la
	homotop\'{\i}a
	\begin{align*}
		& \begin{cases}
			F([x,t],t') \,=\,[x,(1-t')\,t+t'] \\
			F([y],t') \,=\,[y]
		\end{cases}
		\text{ .}
	\end{align*}
	%
	Entonces $F$ es una homotop\'{\i}a de $\id[\cilindro{f}]$ en
	$j\circ r\,\rel{Y}$. Notemos que $F$ es continua porque est\'{a}
	inducida por la homotop\'{\i}a
	\begin{align*}
		& \begin{cases}
			\tilde{F}((x,t),t') \,=\,(x,(1-t')\,t+t') \\
			\tilde{F}(y,t') \,=\,y
		\end{cases}
		\text{ .}
	\end{align*}
	%
	El diagrama
	\begin{center}
	\begin{tikzcd}
		\big((X\times\intervalo)\,\sqcup\,Y\big)\times\intervalo
			\arrow[r,"\tilde{F}"] \arrow[d] &
		(X\times\intervalo)\,\sqcup\,Y \arrow[d] \\
		\cilindro{f}\times\intervalo \arrow[r,"F"'] & \cilindro{f}
	\end{tikzcd}~,
	\end{center}
	cuyas flechas verticales son las proyecciones can\'{o}nicas en los
	cocientes, conmuta y, como $\tilde{F}$ es continua, tambi\'{e}n lo es
	$F$.

	En cuanto a \emph{(iii)}, sean $g:\,\cilindro{f}\rightarrow W$ y
	$G:\,X\times\intervalo\rightarrow W$ tales que el tri\'{a}ngulo
	superior en el siguiente diagrama conmute:
	\begin{center}
	\begin{tikzcd}[column sep=small]
		X\times\{0\}\arrow[rr] \arrow[dd,"i\times\{0\}"'] & &
			X\times\intervalo\arrow[dl,"G"']
				\arrow[dd,"i\times\intervalo"] \\
		& W & \\
		\cilindro{f}\times\{0\}\arrow[rr] \arrow[ur,"g"] & &
			\cilindro{f}\times\intervalo
				\arrow[ul,"H"',dotted]
	\end{tikzcd}
	\end{center}
	Como no hay informaci\'{o}n acerca de la funci\'{o}n $f$, para
	definir $H:\,\cilindro{f}\times\intervalo\rightarrow W$, la \'{u}nica
	posibilidad parece ser $H([y],t')=g([y])$ para todo $t'\in\intervalo$,
	si $y\in Y$. La existencia de $H$ depender\'{a} de poder deformar
	$G$, definida en $X\times\intervalo$ (es decir, en
	$i(X)\times\intervalo$), en la funci\'{o}n $([y],t')\mapsto g(y)$,
	definida en $Y\times\intervalo$ (en $j(Y)\times\intervalo$).
	Definimos entonces
	\begin{align*}
		& \begin{cases}
			H([y],t') & \,=\, g([y]) \\[10pt]
			H([x,t],t') & \,=\,
				\begin{cases}
					g\big([x,\frac{2t-t'}{2-t'}]\big) &
						\quad\text{si }t'\leq 2t \\[5pt]
					G\big(x,\frac{t'-2t}{1-t}\big) &
						\quad\text{si }t'\geq 2t
				\end{cases}
		\end{cases}
		\text{ .}
	\end{align*}
	%
	Entonces $H$ es continua,
	\begin{align*}
		H([x,t],0) & \,=\, g([x,t]) \text{ ,} \\
		H([y],0) & \,=\, g([y]) \quad\text{y} \\
		H|_{X\times\intervalo} & \,=\,G
		\text{ .}
	\end{align*}
	%
\end{proof}

%
\section{Ejemplos}
\theoremstyle{definition}
\newtheorem{ejemploIdentidadDeEspacioEuclideo}{Ejemplo}[section]
\newtheorem{ejemploIdentidadDelIntervalo}[ejemploIdentidadDeEspacioEuclideo]%
	{Ejemplo}
\newtheorem{ejemploReflexionEnElDisco}[ejemploIdentidadDeEspacioEuclideo]%
	{Ejemplo}
\newtheorem{ejemploHomotopiaLinealEnUnConvexo}%
	[ejemploIdentidadDeEspacioEuclideo]{Ejemplo}
\newtheorem{ejemploElPeine}[ejemploIdentidadDeEspacioEuclideo]{Ejemplo}
\newtheorem{ejemploElPeineYElCuadrado}[ejemploIdentidadDeEspacioEuclideo]%
	{Ejemplo}
\newtheorem{ejemploPuntoDeformacionFuerteDeConvexos}%
	[ejemploIdentidadDeEspacioEuclideo]{Ejemplo}
\newtheorem{ejemploEsferaDeformacionFuerteDelEspacioSinUnPunto}%
	[ejemploIdentidadDeEspacioEuclideo]{Ejemplo}
\newtheorem{ejemploElPeineYElCuadradoNoEsRetracto}%
	[ejemploIdentidadDeEspacioEuclideo]{Ejemplo}
\newtheorem{ejemploElPuntoYElPeine}%
	[ejemploIdentidadDeEspacioEuclideo]{Ejemplo}

%-------------

\begin{ejemploIdentidadDeEspacioEuclideo}%
	\label{ejemplo:identidaddeespacioeuclideo}
	Sean $X=Y=\bb{R}^{n}$ y sean $f_{0}(x)=x$ y $f_{1}(x)=0$ para todo
	punto $x\in\bb{R}^{n}$. Sea
	$F:\,\bb{R}^{n}\times\intervalo\rightarrow\bb{R}^{n}$ la funci\'{o}n
	\begin{align*}
		F(x,t) & \,=\,(1-t)\,x
		\text{ .}
	\end{align*}
	%
	Entonces $F$ es una homotop\'{\i}a de $f_{0}=\id[\bb{R}^{n}]$ en la
	funci\'{o}n constante $f_{1}=0$ relativa al subespacio $\{0\}$.
\end{ejemploIdentidadDeEspacioEuclideo}

\begin{ejemploIdentidadDelIntervalo}\label{ejemplo:identidaddelintervalo}
	Si ahora $X=Y=\intervalo$, $f_{0}(t)=t$ y $f_{1}(t)=0$, para todo
	instante $t\in\intervalo$, y
	$F:\,\intervalo\times\intervalo\rightarrow\intervalo$ es la funci\'{o}n
	\begin{align*}
		F(t,t') & \,=\,(1-t')\,t
		\text{ ,}
	\end{align*}
	%
	entonces $F$ es una homotop\'{\i}a de $f_{0}=\id[\intervalo]$ en
	la funci\'{o}n constante $f_{1}=0$ relativa al subespacio $\{0\}$.
\end{ejemploIdentidadDelIntervalo}

\begin{ejemploReflexionEnElDisco}\label{ejemplo:reflexioneneldisco}
	Sean $X=Y=\disco{2}$ y sean $A=B=\esfera{1}$. Sea
	\begin{math}
		f_{0},f_{1}:\,(\disco{2},\esfera{1})\rightarrow
			(\disco{2},\esfera{1})
	\end{math}
	las funciones dadas por
	\begin{align*}
		f_{0}(re^{i\theta}) & \,=\,re^{i\theta} \\
		f_{1}(re^{i\theta}) & \,=\,re^{i(\theta+\pi)}
		\text{ ,}
	\end{align*}
	%
	es decir, $f_{0}=\id[\disco{2}]$ y $f_{1}$ es hacer medio giro, o bien
	reflejar un punto en el origen. Sean $F,F'$ las funciones definidas por
	\begin{align*}
		F(re^{i\theta},t) & \,=\,re^{i(\theta+t\pi)} \\
		F'(re^{i\theta},t) & \,=\,re^{i(\theta-t\pi)}
		\text{ .}
	\end{align*}
	%
	Entonces $f_{0}\simeq f_{1}\,\rel{\{0\}}$, tanto v\'{\i}a $F$ como
	v\'{\i}a $F'$.
\end{ejemploReflexionEnElDisco}

\begin{ejemploHomotopiaLinealEnUnConvexo}\label{ejemplo:contraccionenunconvexo}
	Sea $X$ un espacio topol\'{o}gico arbitrario y sea $Y\subset\bb{R}^{n}$
	un subespacio convexo. Sean $f_{0},f_{1}:\,X\rightarrow Y$ funciones
	continuas y sea $X'\subset X$ un subespacio en donde $f_{0}$ y $f_{1}$
	coinciden. Si $F:\,X\times\intervalo\rightarrow Y$ es la funci\'{o}n
	definida por
	\begin{align*}
		F(x,t) & \,=\,t\,f_{1}(x) + (1-t)\,f_{0}(x)
		\text{ ,}
	\end{align*}
	%
	entonces $F$ es una homotop\'{\i}a de $f_{0}$ en $f_{1}$ relativa a
	$X'$.
\end{ejemploHomotopiaLinealEnUnConvexo}

\begin{ejemploElPeine}\label{ejemplo:elpeine}
	Sea $Y\subset\bb{R}^{2}$ el subespacio del plano dado por
	\begin{align*}
		Y & \,=\,\big\{(x,y)\in\bb{R}^{n}\,:\,
			(0\leq y\leq 1 \wedge x=1/n)\vee
			(y=0\wedge 0\leq x\leq 1)\big\}
		\text{ .}
	\end{align*}
	%
	Si $F:\,Y\times\intervalo\rightarrow Y$ es la funci\'{o}n
	\begin{align*}
		F((x,y),t) & \,=\,(x,(1-t)\,y)
		\text{ ,}
	\end{align*}
	%
	entonces $F$ es una homotop\'{\i}a de $\id[Y]$ en la proyecci\'{o}n
	$\pi:\,(x,y)\mapsto (x,0)$ (relativa al subespacio
	$\{(x,0)\,:\,0\leq x\leq 1\}$). Como este subespacio es contr\'{a}ctil,
	por transitividad, se deduce que $Y$ es contr\'{a}ctil. En particular,
	si $c:\,Y\rightarrow Y$ es la funci\'{o}n constante
	$c(x,y)=(1,0)$, entonces $\id[Y]\simeq c$ y coinciden en $(1,0)$.
	Pero, como la topolog\'{\i}a de $Y$ es la de subespacio del plano,
	no existe una homotop\'{\i}a de $\id[Y]$ en $c$ relativa al
	punto $\{(1,0)\}$, es decir, que deje fijo el punto.
\end{ejemploElPeine}

\begin{ejemploElPeineYElCuadrado}\label{ejemplo:elpeineyelcuadrado}
	Sea $X=\intervalo^{2}\subset\bb{R}^{2}$ y sea $A$ el espacio del
	ejemplo \ref{ejemplo:elpeine}. Entonces $A$ y $X$ son contr\'{a}ctiles.
	De acuerdo con la observaci\'{o}n \ref{obs:encontractilsonhomotopicos},
	la inclusi\'{o}n $\inc:\,A\rightarrow X$ es una equivalencia
	homot\'{o}pica. En particular $A$ es un retracto d\'{e}bil de $X$.
	Pero $A$ no es un retracto de $X$, pues, por ejemplo, el punto
	$(0,1)$ posee puntos arbitrariamente cerca que debieran ser
	proyectados lejos.
\end{ejemploElPeineYElCuadrado}

\begin{ejemploPuntoDeformacionFuerteDeConvexos}%
	\label{ejemplo:puntodeformacionfuertedeconvexos}
	Si $C$ es un subespacio convexo en un espacio vectorial topol\'{o}gico
	y $x_{0}\in C$ es un punto arbitrario, entonces
	\begin{align*}
		F(x,t) & \,=\,t\,x_{0}+(1-t)\,x
	\end{align*}
	%
	es una homotop\'{\i}a de la identidad $\id[C]$ en la funci\'{o}n
	constante $x\mapsto x_{0}$ relatiova al punto $\{x_{0}\}$. Esto
	muestra que todo punto en un conjunto convexo es un retracto por
	deformaci\'{o}n fuerte del convexo.
\end{ejemploPuntoDeformacionFuerteDeConvexos}

\begin{ejemploEsferaDeformacionFuerteDelEspacioSinUnPunto}%
	\label{ejemplo:esferadeformacionfuertedelespaciosinunpunto}
	Sea $\esfera{n}$ la esera de dimenensi\'{o}n $n$ vista como
	subespacio de $\bb{R}^{n+1}\setmin\{0\}$ y sea
	\begin{math}
		F:\,\big(\bb{R}^{n+1}\setmin\{0\}\big)\times\intervalo
			\bb{R}^{n+1}\setmin\{0\}
	\end{math}
	la funci\'{o}n
	\begin{align*}
		F(x,t) & \,=\,(1-t)\,x +t\,\frac{x}{|x|}
	\end{align*}
	%
	entonces $F$ es una retracci\'{o}n por deformaci\'{o}n fuerte de
	$\bb{R}^{n+1}\setmin\{0\}$ en $\esfera{n}$, pues, si llamamos
	$r(x)=F(x,1)=\frac{x}{|x|}$, entonces $r$ es una retracci\'{o}n
	de la inclusi\'{o}n ($r\circ\inc[\esfera{n}]=\id[\esfera{n}]$) y
	$F$ es una homotop\'{\i}a de $\inc[\esfera{n}]\circ r$ en
	$\id[\bb{R}^{n+1}\setmin\{0\}]$ relativa a la esfera.
\end{ejemploEsferaDeformacionFuerteDelEspacioSinUnPunto}

\begin{ejemploElPeineYElCuadradoNoEsRetracto}%
	\label{ejemplo:elpeineyelcuadradonoesretracto}
	Sea $X=\intervalo^{2}$ y sea $A\subset X$ el subespacio del ejemplo
	\ref{ejemplo:elpeine}. Como $X$ y $A$ son contr\'{a}ctiles, la
	inclusi\'{o}n $A\subset X$ es una equivalencia homot\'{o}pica, con lo
	$A$ es un retracto d\'{e}bil de $X$ y una deformaci\'{o}n,
	un retracto por deformaci\'{o}n d\'{e}bil. Pero, seg\'{u}n el ejemplo
	\ref{ejemplo:elpeineyelcuadrado}, $A$ no es un retracto de $X$ y, por
	lo tanto, $A$ no es un retracto por deformaci\'{o}n de $X$.

	Tal vez $A$ sea un retracto d\'{e}bil por deformaci\'{o}n fuerte, es
	decir, la deformaci\'{o}n se pueda tomar relativa a $A$.
\end{ejemploElPeineYElCuadradoNoEsRetracto}

\begin{ejemploElPuntoYElPeine}\label{ejemplo:elpuntoyelpeine}
	Si ahora $X$ es el espacio del ejemplo \ref{ejemplo:elpeine} y
	$A=\{(0,1)\}$, entonces, como $X$ es contr\'{a}ctil,
	$\id[X]\simeq\inc[A]\circ c$, donde $c:\,X\rightarrow A$ es la
	funci\'{o}n constante $(x,y)\mapsto (0,1)$. Como
	$c\circ\inc[A]=\id[A]$, se deduce que $A$ es un retracto por
	deformaci\'{o}n de $X$. Pero, como se mencion\'{o} en el ejemplo
	\ref{ejemplo:elpeineyelcuadrado}, no existe una homotop\'{\i}a
	de $\id[X]$ en $\inc[A]\circ c$ relativa al punto $A$, es decir que
	$A$ no es un retracto por deformaci\'{o}n fuerte de $X$.
\end{ejemploElPuntoYElPeine}

%
%

\chapter{Poliedros}
\section{Complejos simpliciales}
\theoremstyle{plain}
\newtheorem{lemaSubcomplejoPleno}{Lema}[section]

\theoremstyle{remark}
\newtheorem{obsGeneradoPorVertices}{Observaci\'{o}n}[section]
\newtheorem{obsSubcomplejoPleno}[obsGeneradoPorVertices]{Observaci\'{o}n}
\newtheorem{obsParesSimpliciales}[obsGeneradoPorVertices]{Observaci\'{o}n}

%-------------

\subsection{Objetos}
Un \emph{complejo simplicial} es un par $(K,V)$, compuesto por un conjunto
$V$ de \emph{v\'{e}rtices} y un conjunto $K$ de subconjuntos finitos y no
vac\'{\i}os de $V$, denominados \emph{s\'{\i}mplices} del complejo. Se
requiere, adem\'{a}s, que \emph{(i)} todo subconjunto $\{v\}\subset V$
compuesto por un \'{u}nico elemento de $V$ pertenezca a $K$, es decir,
sea un s\'{\i}mplice; y \emph{(ii)} todo subconjunto no vac\'{\i}o de un
s\'{\i}mplice sea, tambi\'{e}n, un s\'{\i}mplice. El complejo se suele
denotar simplemente por $K$.
% En s\'{\i}mbolos, \emph{(i)} $\{v\}\in K$
% para todo $v\in V$; y \emph{(ii)} si $\varnothing\not= s'\subset s\in K$,
% entonces $s'\in K$, tambi\'{e}n.
Un $q$-s\'{\i}mplice en un complejo simplicial $K$ es un s\'{\i}mplice
compuesto por exactamente $q+1$ v\'{e}rtices distintos; se dice que $q$ es
la \emph{dimensi\'{o}n} de $s$. Si $s$ es un s\'{\i}mplice y $s'\subset s$,
entonces se dice que $s'$ es una \emph{cara} de $s$; si, adem\'{a}s, $s'$ es
un $p$-s\'{\i}mplice, se dice que es una $p$-cara de $s$. Una cara
$s'\subset s$ se dice \emph{propia}, si $s'\not= s$. Las caras de un
complejo simplicial $K$ est\'{a}n parcialmente ordenadas por inclusi\'{o}n.
Dado un s\'{\i}mplice $s\in K$, llamamos \emph{v\'{e}rtices de $s$} a los
elementos de $s$, es decir, aquellos v\'{e}rtices de $K$ que pertenecen
a $s$.

\begin{obsGeneradoPorVertices}\label{obs:generadoporvertices}
	Sea $K$ un complejo simplicial. Los $0$-s\'{\i}mplices de $K$ se
	corresponden exactamente con los v\'{e}rtices de $K$ (es decir, $V$).
	Podemos pensar, entonces, a $K$ simplemente como el conjunto de
	s\'{\i}mplices, identificando los v\'{e}rtices $V$ con los
	$0$-s\'{\i}mplices del complejo. Adem\'{a}s, todo s\'{\i}mplice de
	$K$ est\'{a} \emph{generado}, determinado, por los v\'{e}rtices que
	contiene, es decir, por sus $0$-caras: todo s\'{\i}mplice es un
	subconjunto de los v\'{e}rtices de $K$, es decir, es de la forma
	$s=\{\lista[0]{v}{q}\}\subset V$ y, dado un suconjunto
	$\{\lista[0]{v}{q}\}\subset V$ de v\'{e}rtices, existe a lo sumo un
	s\'{\i}mplice cuyos v\'{e}rtices sean exactamente $\lista[0]{v}{q}$.

	Otra manera de pensar en un complejo simplicial es en un conjunto
	$K$ de objetos denominados \emph{s\'{\i}mplices}, parcialmente
	ordenado y que cumple ciertas propiedades de ``finitud'':
	existen elementos $v$ minimales en $K$, es decir, tales que,
	si $s\in K$, entonces $v\leq s$, o bien $v$ y $s$ no son comparables;
	si $s\in K$, entonces existe al menos un elemento minimal
	$v\in K$ tal que $v\leq s$; dado $s\in K$, la cantidad de elementos
	minimales por debajo de $s$ es igual a la cantidad de elementos
	intermedios (incluyendo los extremos) entre $s$ y cualquiera
	de los elementos minimales debajo de $s$; si $s,s'\in K$ y $s'\leq s$,
	entonces existen a lo sumo finitos elementos intermedios
	$s'\leq s''\leq s$. No estoy seguro de que estas propiedades sean
	suficientes para caracterizar a los complejos simpliciales como
	conjuntos parcialmente ordenados.
\end{obsGeneradoPorVertices}

La \emph{dimensi\'{o}n} de un complejo simplicial $K$ se define como
\begin{align*}
	\dim\,K & \,=\,\sup\,\{\dim\,s\,:\,s\in K\}\cup\{-1\}
	\text{ ,}
\end{align*}
%
con lo cual $\dim\,K=n$, si $K$ contiene un $n$-s\'{\i}mplice pero no
contiene $(n+1)$-s\'{\i}mplices, $\dim\,K=\infty$ si $K$ contiene un
$n$-s\'{\i}mplice para $n$ arbitrariamente grande y $\dim\,K=-1$, si
$K=\varnothing$ es el complejo vac\'{\i}o. Un complejo se dice
\emph{finito}, si contiene una cantidad finita de s\'{\i}mplices
(no confundir ser finito con ser de dimensi\'{o}n finita). Tambi\'{e}n se
dice que $K$ es \emph{localmente finito}, si cada v\'{e}rtice de $K$
es parte de a lo sumo finitos s\'{\i}mplices de $K$.

Un \emph{subcomplejo} de un complejo simplicial $K$ es un subconjunto de
s\'{\i}mplices de $K$ que constituye, a su vez, un complejo simplicial.

\subsection{Morfismos}
Una \emph{transformaci\'{o}n simplicial}
$(K_{1},V_{1})\rightarrow (K_{2},V_{2})$ (o aplicaci\'{o}n simplicial, o un
mapa simplicial o morfismo simplicial) es un par $(\varphi,\varphi_{0})$,
donde $\varphi_{0}:\,V_{1}\rightarrow V_{2}$ y
$\varphi:\,K_{1}\rightarrow K_{2}$ son funciones que verifican que, si
$s=\{\lista[0]{v}{q}\}\in K_{1}$, entonces
\begin{equation}
	\label{eq:transformacionsimplicial}
	\varphi(s) \,=\, \varphi\big(\{\lista[0]{v}{q}\}\big) \,=\,
		\{\varphi_{0}(v_{0}),\,\dots,\,\varphi_{0}(v_{q})\}
	\text{ .}
\end{equation}
%
En particular, $\varphi\big(\{v\}\big)=\varphi_{0}(v)$, para todo v\'{e}rtice
$v\in V_{1}$. Adem\'{a}s, la funci\'{o}n en s\'{\i}mplices $\varphi$ est\'{a}
determinada por la funci\'{o}n en v\'{e}rtices $\varphi_{0}$, pues, dada
una funci\'{o}n $\varphi_{0}:\,V_{1}\rightarrow V_{2}$ arbitraria, existe
una \'{u}nica funci\'{o}n $\varphi:\,\partes(V_{1})\rightarrow\partes(V_{2})$
en subconjuntos de v\'{e}rtices que verifica
\eqref{eq:transformacionsimplicial}. Pero no toda funci\'{o}n en los
v\'{e}rtices define una transformaci\'{o}n simplicial. Para que un
par $(\varphi,\varphi_{0})$ que cumple \eqref{eq:transformacionsimplicial}
sea una transformaci\'{o}n simplicial es necesario y suficiente que
$\varphi_{0}$ cumpla
\begin{equation}
	\label{eq:transformacionsimplicialvertices}
	\{\varphi_{0}(v_{0}),\,\dots,\,\varphi_{0}(v_{q})\}\in K_{2}
		\quad\text{para todo s\'{\i}mplice }
		\{\lista[0]{v}{q}\}\in K_{1}
	\text{ .}
\end{equation}
%
En definitiva, las transformaciones simpliciales se pueden ver
equivalentemente como pares
\begin{math}
	(\varphi:\,K_{1}\rightarrow K_{2},\,
		\varphi_{0}:\,V_{1}\rightarrow V_{2})
\end{math}
que verifican \eqref{eq:transformacionsimplicial}, o bien como una
funci\'{o}n $\varphi_{0}:\,V_{1}\rightarrow V_{2}$ que cumple
\eqref{eq:transformacionsimplicialvertices}. Por \'{u}ltimo, notemos que,
identificando los v\'{e}rtices $V_{i}$ con los $0$-s\'{\i}mplices
$\qesq{0}{K_{i}}$ de los complejos, se cumple que $\varphi(V_{1})\subset V_{2}$
y que $\varphi|_{V_{1}}=\varphi_{0}$. As\'{\i}, podemos pensar en una
transformaci\'{o}n simpicial $(\varphi,\varphi_{0})$ como la funci\'{o}n en
s\'{\i}mplices $\varphi:\,K_{1}\rightarrow K_{2}$, identificando $\varphi_{0}$
con la restricci\'{o}n de $\varphi$ al $0$-esqueleto. De esta manera, nos
podemos referir a $(\varphi,\varphi_{0})$ simplemente por $\varphi$.

La composici\'{o}n $\varphi\circ\psi$ de transformaciones simpliciales
$\varphi$ y $\psi$ se define como la transformaci\'{o}n determinada por la
composici\'{o}n $\varphi_{0}\circ\psi_{0}$ de las funciones en los
v\'{e}rtices. Adem\'{a}s, dado un complejo $K$ la funci\'{o}n identidad en
v\'{e}rtices determina una transformaci\'{o}n simplicial $\id[K]$ que es la
identidad en el conjunto de s\'{\i}mplices. As\'{\i}, queda determinada
una categor\'{\i}a cuyos objetos son los complejos simpliciales y cuyos
morfismos son las transformaciones simpliciales entre ellos.

Sea $K$ un complejo simplicial. Los subcomplejos de $K$ se corresponden con
transformaciones simpliciales $\inc:\,L\rightarrow K$ que son inclusiones
$\inc:\,L\subset K$ como funciones entre los conjuntos de s\'{\i}mplices.
Llamamos a estas transformaciones, \emph{inclusiones simpliciales}.
Toda transformaci\'{o}n simplicial $L\rightarrow K$ que es inyectiva como
funci\'{o}n entre los conjuntos de s\'{\i}mplices se puede pensar como una
inclusi\'{o}n simplicial v\'{\i}a alguna biyecci\'{o}n con un subcomplejo
de $K$ en el sentido estricto de la definici\'{o}n. Tambi\'{e}n llamaremos
inclusiones simpliciales a estas transformaciones y subcomplejos a los
complejos de partida de las mismas.

Dado un subcomplejo $L\subset K$, el subconjunto $N\subset K$ de
s\'{\i}mplices sin v\'{e}rtices en $L$ es un subcomplejo de $K$, al que
podemos llamar \emph{complemento de $L$}; es el subcomplejo m\'{a}s grande
de $K$ disjunto de $L$, es decir, que no comparte v\'{e}rtices con $L$.
Dado $s=\{\lista[0]{v}{q}\}\in K$, entonces hay tres posibilidades:
una primera posibilidad es $v_{i}\not\in L$ para todo $i$, con lo cual
$s\in N$; una segunda posibilidad es que, renombrando los v\'{e}rtices,
exista $p\geq 0,p<q$ tal que $v_{i}\in L$, si $i\leq p$ y $v_{i}\not\in L$,
si $i>p$, en tal caso, quedan determinados dos s\'{\i}mplices
$s'=\{\lista[0]{v}{p}\}$ y $s''=\{\lista[p+1]{v}{q}\}$ y vale que
$s''\in N$; la tercera posibilidad es que todos los v\'{e}rtices pertenezcan
a $L$. Un subcomplejo $L\subset K$ se dice \emph{pleno}, si se cumple que todo
s\'{\i}mplice de $K$ cuyos v\'{e}rtices pertenecen al conjunto de
v\'{e}rtices de $L$ pertenece al complejo $L$.

\begin{lemaSubcomplejoPleno}\label{thm:subcomplejopleno}
	Si $L\subset K$ es un subcomplejo pleno de $K$ y $N$ es
	el complemento de $L$ en $K$, entonces, dado $s\in K$, vale que:
	$s\in N$, o bien $s=s'\cup s''$, con $s'\in L$ y $s''\in N$, o bien
	$s\in L$.
\end{lemaSubcomplejoPleno}

\begin{obsSubcomplejoPleno}\label{obs:subcomplejopleno}
	La descomposici\'{o}n $s=s'\cup s''$ no tiene en cuenta las caras
	del s\'{\i}mplice $s$, es simplemente una descomposici\'{o}n del
	conjunto de v\'{e}rtices de $s$.
\end{obsSubcomplejoPleno}

\subsection{Subcategor\'{\i}as}
Un \emph{par simplicial} es un par $(K,L)$, donde $K$ es un complejo
simplicial y $L\subset K$ es un subcomplejo ($(K,\varnothing)$ es una
posibilidad). Un morfismo de pares simpliciales
$(K_{1},L_{1})\rightarrow (K_{2},L_{2})$ es una transformaci\'{o}n
simplicial $\varphi:\,K_{1}\rightarrow K_{2}$ que verifica
$\varphi(L_{1})\subset L_{2}$.

\begin{obsParesSimpliciales}
	\label{obs:paressimpliciales}
	Un poco m\'{a}s en general, podemos llamar par simplicial a
	un par de complejos $K,K'$ y una transformaci\'{o}n simplicial
	$\varphi:\,K\rightarrow K'$ --es decir, un par simplicial se
	corresponde con una transformaci\'{o}n simplicial, con esta
	definici\'{o}n. Dadas transformaciones simpliciales
	$\varphi_{1}:\,K_{1}\rightarrow K'_{1}$ y
	$\varphi_{2}:\,K_{2}\rightarrow K'_{2}$, un morfismo de pares
	simpliciales $\varphi_{1}\rightarrow\varphi_{2}$ es un
	par de transformaciones simpliciales $(\psi',\psi)$,
	$\psi:\,K_{1}\rightarrow K_{2}$ y $\psi':\,K'_{1}\rightarrow K'_{2}$
	tales que
	\begin{center}
	\begin{tikzcd}
		K_{1}\arrow[r,"\varphi_{1}"]\arrow[d,"\psi"'] &
			K'_{1}\arrow[d,"\psi'"] \\
		K_{2}\arrow[r,"\varphi_{2}"'] & K'_{2}
	\end{tikzcd}
	\end{center}
	conmuta.
\end{obsParesSimpliciales}

Los pares simpliciales junto con los morfismos de pares constituyen una
categor\'{\i}a, la \emph{categor\'{\i}a de pares simpliciales}.
La categor\'{\i}a de complejos simpliciales se puede ver como una
subcategor\'{\i}a plena de la categor\'{\i}a de pares simpliciales
v\'{\i}a $K\mapsto (K,\varnothing)$. Otra subcategor\'{\i}a plena de la
categor\'{\i}a de pares simpliciales es la categor\'{\i}a de \emph{complejos %
simpliciales con un v\'{e}rtice distinguido}, cuyos objetos
son los pares $(K,v)$, $v\in V$ y cuyos morfismos son los morfismos de
complejos que preservan el v\'{e}rtice distinguido. La inclusi\'{o}n
en la categor\'{\i}a de pares est\'{a} dada por $(K,v)\mapsto (K,\{v\})$.


%
\section{El espacio topol\'{o}gico de un complejo simplicial}
\theoremstyle{plain}
\newtheorem{teoCompactoHausdorff}{Teorema}[section]
\newtheorem{teoTopologiaCoherente}[teoCompactoHausdorff]{Teorema}
\newtheorem{coroTopologiaCoherenteComplejos}[teoCompactoHausdorff]{Corolario}
\newtheorem{coroTopologiaCoherenteIdentidad}[teoCompactoHausdorff]{Corolario}
\newtheorem{propoRealizacionMorfismos}[teoCompactoHausdorff]{Proposici\'{o}n}
\newtheorem{coroTopologiaCoherenteSubcomplejos}[teoCompactoHausdorff]%
	{Corolario}

\theoremstyle{remark}
\newtheorem{obsTopologiaEnSimplices}[teoCompactoHausdorff]{Observaci\'{o}n}
\newtheorem{obsRealizacionMorfismos}[teoCompactoHausdorff]{Observaci\'{o}n}
\newtheorem{obsRealizacionUnionInterseccion}[teoCompactoHausdorff]%
	{Observaci\'{o}n}

%-------------

\subsection{El complejo geom\'{e}trico}
Sea $K=(K,V)$ un complejo simplicial no va\'{\i}o. Denotamos $\geom{K}$ al
conjunto de funciones $\alpha:\,V\rightarrow [0,1]$ que verifican:
\begin{itemize}
	\item[(i)] el conjunto $\{v\in V\,:\,\alpha(v)\not =0\}$ es un
		s\'{\i}mplice en $K$; y
	\item[(ii)] $\sum_{v\in V}\,\alpha(v)=1$.
\end{itemize}
%
Dicho de otra manera, $\geom{K}$ es el conjunto de combinaciones lineales
convexas de v\'{e}rtices de $K$. Si $K=\varnothing$, se define
$\geom{K}=\varnothing$. En el conjunto $\geom{K}$ definimos una m\'{e}trica:
sea $d:\,\geom{K}\times\geom{K}\rightarrow\bb{R}$ la funci\'{o}n dada por
\begin{align*}
	d(\alpha,\beta) & \,=\,\Big(\sum_{v\in V}\,
				|\alpha(v)-\beta(v)|^{2}\Big)^{1/2}
	\text{ .}
\end{align*}
%
Entonces la funci\'{o}n $d$ es sim\'{e}trica, $d(\alpha,\beta)\geq 0$ y es
igual a cero, si y s\'{o}lo si $\alpha(v)=\beta(v)$ para todo v\'{e}rtice $v$.
En cuanto a la desigualdad triangular, como el conjunto de v\'{e}rtices
$v$ tales que $\alpha(v)$ y $\beta(v)$ son no nulos es finito, la
desigualdad triangular se deduce del caso usual en un $\bb{R}$-espacio
vectorial de dimensi\'{o}n finita. Denotamos $\geom{K}_{d}$ al espacio
m\'{e}trico $(\geom{K},d)$.

Si $s\in K$ es un $q$-s\'{\i}mplice, definimos
\begin{align*}
	\geom{s} & \,=\,\Big\{\alpha\in\geom{K}\,:\,
		\alpha(v)=0\text{ , si }v\not\in s\Big\}
	\text{ ,}
\end{align*}
%
es decir, $\geom{s}$ es el subconjunto de funciones $\alpha\in\geom{K}$
que son nulas en todos los v\'{e}rtices que no pertenecen a $s$. En
t\'{e}rminos de la definici\'{o}n anterior, $\geom{s}=\geom{\caras{s}}$. Si
$s=\{\lista[0]{v}{q}\}$, sea $E$ el $\bb{R}$-espacio vectorial generado
por los $q+1$ v\'{e}rtices de $s$. Sea $\simp{\lista[0]{v}{q}}$ el conjunto
de puntos $x\in E$ de la forma
\begin{align*}
	x & \,=\,\sum_{i=0}^{q}\,t^{i}v_{i}
\end{align*}
%
tales que $t^{i}\geq 0$ y $\sum_{i=0}^{q}\,t^{i}=1$. Es decir,
$\simp{\lista[0]{v}{q}}$ es el conjunto de combinaciones lineales
convexas de los puntos correspondientes a los v\'{e}rtices $v_{i}$ en $E$,
o, lo que es lo mismo, el conjunto convexo m\'{a}s chico que los contiene,
la \emph{c\'{a}scara convexa}. Se ve entonces que el subconjunto
$\geom{s}\subset\geom{K}$ est\'{a} en correspondencia con el subconjunto
$\simp{\lista[0]{v}{q}}$ de $E$ v\'{\i}a
$\Phi:\,\alpha\mapsto (\alpha(v_{0}),\,\dots,\,\alpha(v_{q}))$.

En tanto $\bb{R}$-espacio vectorial de dimensi\'{o}n finita, $E$ admite una
\'{u}nica topolog\'{\i}a con respecto a la cual las operaciones de suma y
producto por escalares son continuas y la m\'{e}trica euclidea es compatible
con esta topolog\'{\i}a. Por otro lado, $\geom{s}\subset\geom{K}$ hereda la
m\'{e}trica $d$ definida anteriormente y
$\Phi:\,\geom{s}\rightarrow\simp{\lista[0]{v}{q}}$ es una isometr\'{\i}a.
En particular, $\geom{s}$ con la topolog\'{\i}a dada por la m\'{e}trica $d$
es homeomorfo al conjunto convexo y compacto $\simp{\lista[0]{v}{q}}$.
Notemos que este espacio es tambi\'{e}n Hausdorff.

\subsection{Repaso de Topolog\'{\i}a}
Con el prop\'{o}sito de darle una topolog\'{\i}a m\'{a}s natural al
conjunto $\geom{K}$, recordamos dos teoremas b\'{a}sicos:

\begin{teoCompactoHausdorff}[de rigidez]\label{thm:compactohausdorff}
	Sea $f:\,X\rightarrow Y$ una funci\'{o}n continua e inyectiva
	definida en un espacio topol\'{o}gico compacto $X$ y codominio
	un espacio topol\'{o}gico Hausdorff. Entonces $f$ determina un
	homeomorfismo entre $X$ y el subespacio $f(X)\subset Y$.
\end{teoCompactoHausdorff}

De este teorema se deduce que, si $\tau$ es una topolog\'{\i}a en un conjunto
$X$ con respecto a la cual $X$ resulta compacto y Hausdorff, entonces
no existe una topolog\'{\i}a m\'{a}s d\'{e}bil en $X$ con respecto a la
cual $X$ sea Hausdorff, ni existe una topolog\'{\i}a m\'{a}s fuerte con
respecto a la cual $X$ sea compacto. Dicho de otra manera, dos topolog\'{\i}as
comparables, compactas y Hausdorff deben ser iguales.

Dado un conjunto $X$ y una familia indexada de espacios topol\'{o}gicos
$\{X_{j}\}_{j}$ y funciones $\{g_{j}:\,X_{j}\rightarrow X\}_{j}$, la
\emph{topolog\'{\i}a coinducida por las funciones $\{g_{j}\}_{j}$}
es la topolog\'{\i}a m\'{a}s fina/fuerte en $X$ tal que las funciones $g_{j}$
resulten ser continuas. Esta topolog\'{\i}a est\'{a} caracterizada por la
propiedad de que, si $Y$ es un espacio topol\'{o}gico, una funci\'{o}n
$f:\,X\rightarrow Y$ es continua, si y s\'{o}lo si las composiciones
$f\circ g_{j}:\,X_{j}\rightarrow Y$ lo son para todo $j$.

\begin{teoTopologiaCoherente}\label{thm:topologiacoherente}
	Sea $X$ un conjunto y sea $\{A_{j}\}_{j}$ una familia indexada de
	espacios topol\'{o}gicos contenidos en $X$. Si
	\begin{itemize}
		\item[(i)] la intersecci\'{o}n $A_{j}\cap A_{j'}$ es cerrada
			en $A_{j}$ y en $A_{j'}$ para todo par $j,j'$ y
		\item[(ii)] la topolog\'{\i}a inducida en $A_{j}\cap A_{j'}$
			como subconjunto de $A_{j}$ coincide con la inducida
			en tanto subconjunto de $A_{j'}$,
	\end{itemize}
	%
	entonces la topolog\'{\i}a determinada en $X$ por las inclusiones
	$\inc:\,A_{j}\hookrightarrow X$ se caracteriza por ser la \'{u}nica
	topolog\'{\i}a en $X$ tal que cada subconjunto $A_{j}$ sea
	cerrado en $X$ y que sea coherente con las topolog\'{\i}as de los
	espacios $A_{j}$. Lo mismo es cierto si se requiere que las
	intersecciones $A_{j}\cap A_{j'}$ sean abiertas en $A_{j}$ y en
	$A_{j'}$, en lugar de cerradas, y los subconjuntos $A_{j}$ sean
	abiertos en $X$, en lugar de cerrados. La propiedad de coherencia
	de la topolog\'{\i}a de $X$ con respecto a la de los $A_{j}$
	significa que un subconjunto $B\subset X$ es cerrado (o abierto), si 
	$B\cap A_{j}$ es cerrado para todo $j$ (respectivamente, abierto).
\end{teoTopologiaCoherente}

\subsection{Una topolog\'{\i}a coherente}
Sea $K$ un complejo simplicial. En lugar de darle a $\geom{K}$ la
topolog\'{\i}a inducida por la m\'{e}trica $d$, definiremos una
topolog\'{\i}a coherente con la topolog\'{\i}a de los s\'{\i}mplices. Lo que
buscamos es que la noci\'{o}n de continuidad de una funci\'{o}n definida
en $\geom{K}$ est\'{e} determinada por lo que ocurre con la restricci\'{o}n
de dicha funci\'{o}n a los subconjuntos $\geom{s}$.

Dados s\'{\i}mplices $s,s'\in K$, entonces, o bien $s\cap s'=\varnothing$,
o bien $s\cap s'$ es un subconjunto no vac\'{\i}o tanto de $s$ como de $s'$.
En el primer caso, $\geom{s}\cap\geom{s'}=\varnothing$, pues la \'{u}nica
funci\'{o}n en los v\'{e}rtices de $K$ que se anula fuera de $s$ y fuera
de $s'$ es la funci\'{o}n cero. En el segundo caso, $s\cap s'$ es una cara
de $s$ y una cara de $s'$, tambi\'{e}n. En particular, si
$s\cap s'\not=\varnothing$, vale que $\geom{s\cap s'}=\geom{s}\cap\geom{s'}$,
pues las funciones que se anulan fuera de $s$ y fuera de $s'$ son las
funciones que se anulan fuera de $s\cap s'$. En cualquiera de los dos casos,
$\geom{s}\cap\geom{s'}$ tiene una toplog\'{\i}a comacta y Hausdorff
(proveniente de la m\'{e}trica $d$, si la intersecci\'{o}n es no vac\'{\i}a).
Adem\'{a}s, las inclusiones en $\geom{s}_{d}$ y en $\geom{s'}_{d}$ son
continuas, con lo que la topolog\'{\i}a en $\geom{s}\cap\geom{s'}$ es
comparable a las heredadas de $\geom{s}_{d}$ y de $\geom{s'}_{d}$. Pero
la intersecci\'{o}n $\geom{s}\cap\geom{s'}$ es cerrada tanto como subespacio
de $\geom{s}$ como en tanto subespacio de $\geom{s'}$ y, por en consecuencia,
un espacio compacto y Hausdorff con cualquiera de estas dos topolog\'{\i}as.
En definitiva, por \ref{thm:compactohausdorff}, la topolog\'{\i}a inducida
en $\geom{s}\cap\geom{s'}$ como subespacio de $\geom{s}_{d}$ coincide con
la inducida como subespacio de $\geom{s'}_{d}$. Notemos que el conjunto
$\geom{K}$ y la colecci\'{o}n de espacios topol\'{o}gicos
$\big\{\geom{s}_{d}\,:\,s\in K\big\}$ verifican las condiciones
\emph{(i)} y \emph{(ii)} del teorema \ref{thm:topologiacoherente}.
De esta manera, queda determinada una topolog\'{\i}a en $\geom{K}$ que
denominaremos \emph{coherente}.

En el contexto de espacios topol\'{o}gicos, denotaremos $\geom{K}$ al espacio
topol\'{o}gico cuyo conjunto subyacente es $\geom{K}$ y cuya topolog\'{\i}a
es la topolog\'{\i}a coherente reci\'{e}n definida.

\begin{obsTopologiaEnSimplices}\label{obs:topologiaensimplices}
	La igualdad de conjuntos $\geom{s}=\geom{\caras{s}}$ da lugar a
	una identificaci\'{o}n entre los espacios topol\'{o}gicos
	$\geom{\caras{s}}$ y $\geom{s}_{d}$. Denotaremos todos estos
	espacios por $\geom{s}$. Adem\'{a}s, los espacios
	$\geom{\carasp{s}}$ y $\geom{\carasp{s}}_{d}$ obtenidos a partir
	del complejo $\carasp{s}$ de paras propias de un s\'{\i}mplice
	$s$ tambi\'{e}n son homeomorfos. En particular, $\geom{s}$ y
	$\geom{\carasp{s}}$ son subespacios del espacio vectorial
	topol\'{o}gico generado por los v\'{e}rtices del s\'{\i}mplice $s$ y
	\begin{align*}
		\geom{\carasp{s}} & \,=\,\borde\big(\geom{s}\big)
		\text{ .}
	\end{align*}
	%
\end{obsTopologiaEnSimplices}

De la definici\'{o}n de la topolog\'{\i}a coherente de un complejo
simplicial podemos deducir los siguientes corolarios.

\begin{coroTopologiaCoherenteComplejos}\label{thm:topologiacoherentecomplejos}
	Sea $K$ un complejo simplicial. Entonces una funci\'{o}n
	$f:\,\geom{K}\rightarrow X$ es continua, si y s\'{o}lo si las
	restricciones $f|_{\geom{s}}$ son continuas. Equivalentemente,
	$f:\,\geom{K}\rightarrow X$ es continua, si y s\'{o}lo si
	$f|_{\geom{\qesq{q}{K}}}$ es continua.
\end{coroTopologiaCoherenteComplejos}

\begin{coroTopologiaCoherenteIdentidad}\label{thm:topologiacoherenteidentidad}
	La identidad $\geom{K}\rightarrow\geom{K}_{d}$ es continua
\end{coroTopologiaCoherenteIdentidad}

Si $L\subset K$ es un subcomplejo, entonces, conjust\'{\i}sticamente,
$\geom{L}\subset\geom{K}$: si $\alpha\in\geom{L}$, entonces
$\{\alpha\not=0\}\in L$ y, como $L\hookrightarrow K$ es simplicial,
$\{\alpha\not=0\}$ es un s\'{\i}mplice de $K$.

\begin{obsRealizacionMorfismos}\label{obs:realizacionmorfismos}
	Un poco m\'{a}s en general, si $\varphi:\,K\rightarrow K'$ es una
	transformaci\'{o}n simplicial y $\alpha\in\geom{K}$, entonces
	$\{v\in V\,:\,\alpha(v)\not =0\}$ es un s\'{\i}mplice en $K$ y,
	aplicando $\varphi$, el conjunto
	$\{\varphi(v)\in V'\,:\,\alpha(v)\not =0\}$ es un s\'{\i}mplice de
	$K'$. Definimos $\alpha':\,V'\rightarrow[0,1]$ por
	\begin{align*}
		\alpha'(v') & \,=\,
			\begin{cases}
				\sum_{\varphi(v)=v'}\,\alpha(v) &
					\quad\text{si } v'\in\img\,\varphi \\
				0 & \quad\text{si } v'\not\in\img\,\varphi
			\end{cases}
		\text{ .}
	\end{align*}
	%
	Como los valores de $\alpha$ son no negativos, se cumple que
	\begin{align*}
		\{\alpha'\not =0\} & \,=\,\varphi\big(\{\alpha\not =0\}\big)
		\text{ .}
	\end{align*}
	%
	Definimos $\geom{\varphi}(\alpha)=\alpha'$.
\end{obsRealizacionMorfismos}

\begin{propoRealizacionMorfismos}\label{thm:realizacionmorfismos}
	Sea $\varphi:\,K\rightarrow K'$ una transformaci\'{o}n simplicial.
	La funci\'{o}n $\geom{\varphi}:\,\geom{K}\rightarrow\geom{K'}$ es
	continua, tanto con respecto a la topolog\'{\i}a inducida por la
	m\'{e}trica euclidea en $\geom{K}$ y en $\geom{K'}$, como con
	respecto a la topolog\'{\i}a coherente. Usaremos $\geom{\varphi}_{d}$
	para referirnos a la funci\'{o}n entre los espacios m\'{e}tricos y
	$\geom{\varphi}$ para referirnos a la funci\'{o}n entre los
	espacios con la topolog\'{\i}a determinada por sus s\'{\i}mplices.
\end{propoRealizacionMorfismos}

\begin{proof}
	Veamos que $\geom{\varphi}:\,\geom{K}_{d}\rightarrow\geom{K'}_{d}$
	es continua con respecto a las m\'{e}tricas en los complejos
	$K$ y $K'$. Sea entonces $\alpha\in\geom{K}$, sea $\epsilon>0$ y
	sean $\lista{v}{r}$ los v\'{e}rtices del s\'{\i}mplice
	$\{\alpha\not=0\}\in K$. Si $\beta\in\geom{K}$ y $v'\in V'$, entonces
	\begin{align*}
		\Big|\sum_{v\in V|\varphi(v)=v'}\,\alpha(v)-\beta(v)\Big|^{2}
			& \,=\,\big|(\geom{\varphi}\alpha)(v)-
				(\geom{\varphi}\beta)(v)\big|^{2}
			\,\leq\,\sum_{\varphi(v)=v'}\,
				\big|\alpha(v)-\beta(v)\big|
		\text{ ,}
	\end{align*}
	%
	pues
	\begin{align*}
		-1 & \,\leq\,
			(\geom{\varphi}\alpha)(v)-(\geom{\varphi}\beta)(v)
			\,\leq\, 1
		\text{ .}
	\end{align*}
	%
	Pero entonces
	\begin{align*}
		d(\geom{\varphi}\alpha,\geom{\varphi}\beta)^{2} & \,=\,
			\sum_{v'\in V'}\,\Big|\sum_{\varphi(v)=v'}\,
				\alpha(v)-\beta(v)\Big|^{2} \\
		& \,\leq\,\sum_{v'}\,\sum_{\varphi(v)=v'}\,
				\big|\alpha(v)-\beta(v)\big|
			\,=\,\sum_{v}\,\big|\alpha(v)-\beta(v)\big|
		\text{ .}
	\end{align*}
	%
	Si $d(\alpha,\beta)<\delta$ para cierto $\delta>0$, entonces
	$\big|\alpha(v_{i})-\beta(v_{i})\big|<\delta$ para $i=1,\,\dots,\,r$.
	En particular,
	\begin{align*}
		d(\geom{\varphi}\alpha,\geom{\varphi}\beta)^{2} & \,\leq\,
			\sum_{i=1}^{r}\,\big|\alpha(v_{i})-\beta(v_{i})\big|
			\,+\, \sum_{v\not=v_{i}}\,\beta(v) \\
		& \,=\,\sum_{i=1}^{r}\,\big|\alpha(v_{i})-\beta(v_{i})\big|
			\,+\,\Big(1 - \sum_{i=1}^{r}\,\beta(v_{i})\Big) \\
		& \,=\,\sum_{i=1}^{r}\,\big|\alpha(v_{i})-\beta(v_{i})\big|
			\,+\,\sum_{i=1}^{r}\,
				\big(\alpha(v_{i})-\beta(v_{i})\big) \\
		& \,\leq\,\sum_{i=1}^{r}\,\big|\alpha(v_{i})-\beta(v_{i})\big|
			\,+\,\sqrt{r}\cdot\Big(\sum_{i=1}^{r}\,
				\big|\alpha(v_{i})-\beta(v_{i})\big|^{2}
				\Big)^{1/2} \\
		& \,<\, r\cdot\delta\,+\,\sqrt{r}\cdot\delta
		\text{ .}
	\end{align*}
	%
	De esto se deduce que $\geom{\varphi}$ es continua con respecto a
	las m\'{e}tricas euclideas en los complejos. En particular,
	si $L\subset K$ es un subcomplejo, la inclusi\'{o}n
	$\geom{L}_{d}\hookrightarrow\geom{K}_{d}$ es continua.

	Veamos ahora que la misma funci\'{o}n
	$\geom{\varphi}:\,\geom{K}\rightarrow\geom{K'}$ es continua con
	respecto a la topolog\'{\i}a coherente. Dado un s\'{\i}mplice
	$s\in K$, hay que ver que
	$\geom{\varphi}|_{\geom{s}}:\,\geom{s}\rightarrow\geom{K'}$ sea
	continua. Determinemos primero cu\'{a}l es la imagen de esta
	funci\'{o}n: por un lado, $\varphi(s)\in K'$ es un s\'{\i}mplice,
	por definici\'{o}n de $\varphi$, y, por otro,
	\begin{align*}
		\geom{\varphi}(\geom{s}) & \,=\,
			\big\{\geom{\varphi}\alpha\,:\,
				\alpha(v)=0\text{ , si }v\not\in s\big\}
		\text{ .}
	\end{align*}
	%
	Ahora, si $\alpha\in\geom{\varphi}(\geom{s})$ y $v'\not\in\varphi(s)$,
	entonces
	\begin{align*}
		(\geom{\varphi}\alpha)(v') & \,=\,
			\sum_{\varphi(v)=v'}\,\alpha(v) \,=\,
			\sum_{v\in s|\varphi(v)=v'}\,\alpha(v)\,=\,0
		\text{ .}
	\end{align*}
	%
	De esto se deduce que $\geom{\varphi}\alpha\in\geom{\varphi(s)}$ y
	$\geom{\varphi}(\geom{s})\subset\geom{\varphi(s)}$.
	(Esto ya es suficiente para concluir que $\geom{\varphi}|_{\geom{s}}$
	es continua). Rec\'{\i}procamente, si $\varphi(s)=\{\lista[0]{v'}{q}\}$
	sean $\lista[0]{v}{q}\in s$ tales que $\varphi(v_{i})=v'_{i}$. Si
	$\alpha'\in\geom{\varphi(s)}$, entonces $\alpha'(v'_{i})\geq 0$ y
	$\alpha'(v')=0$, si $v'\not=v'_{i}$ para todo $i$. Sea
	$\alpha:\,V\rightarrow [0,1]$ la funci\'{o}n
	\begin{align*}
		\alpha(v) & \,=\,
			\begin{cases}
				\alpha'(v'_{i}) &\quad\text{ si } v=v_{i} \\
				0 & \quad\text{ si } v\not =v_{i}
					\text{ para todo } i
			\end{cases}
		\text{ .}
	\end{align*}
	%
	Entonces $\{\alpha\not=0\}\subset\{\lista[0]{v}{q}\}\subset s$ y
	$\sum_{v}\,\alpha(v)=1$. En particular, $\{\alpha\not =0\}\in K$
	y $\alpha\in\geom{K}$. Pero, adem\'{a}s, $\alpha(v)=0$, si
	$v\not\in s$, con lo que $\alpha\in\geom{s}$ y, por definici\'{o}n,
	$\geom{\varphi}\alpha=\alpha'$. En definitiva,
	$\geom{\varphi(s)}\subset\geom{\varphi}(\geom{s})$ y
	\begin{align*}
		\geom{\varphi}(\geom{s}) & \,=\,\geom{\varphi(s)}
		\text{ .}
	\end{align*}
	%

	Dado que $\geom{\varphi}(\geom{s})\subset\geom{\varphi(s)}$ y
	que $\geom{\varphi(s)}\subset\geom{K'}$ es subespacio (cerrado)
	porque $\varphi(s)\in K'$ es un s\'{\i}mplice, la restricci\'{o}n
	$\geom{\varphi}:\,\geom{s}\rightarrow\geom{K'}$ es continua, si y
	s\'{o}lo si la correstricci\'{o}n
	\begin{align*}
		\geom{\varphi} \,:\,\geom{s}\,\rightarrow\,
			\geom{\varphi(s)}
	\end{align*}
	%
	es continua. Pero $\geom{s}$ y $\geom{\varphi(s)}$ tienen
	la topolog\'{\i}a dada por la m\'{e}trica euclidea y ya vimos
	que $\geom{\varphi}_{d}:\,\geom{s}_{d}\rightarrow\geom{\varphi(s)}_{d}$
	es continua. En definitiva,
	$\geom{\varphi}:\,\geom{s}\rightarrow\geom{\varphi(s)}$ es continua
	y $\geom{\varphi}:\,\geom{s}\rightarrow\geom{K'}$. Como
	$s\in K$ era arbitrario, por \ref{thm:topologiacoherentecomplejos},
	deducimos que $\geom{\varphi}:\,\geom{K}\rightarrow\geom{K'}$ es
	continua.
\end{proof}

\begin{coroTopologiaCoherenteSubcomplejos}%
	\label{thm:topologiacoherentesubcomplejos}
	Si $L\subset K$ es un subcomplejo, $\geom{L}_{d}$ es un subespacio
	cerrado de $\geom{K}_{d}$. En particular, $\geom{L}$ es un
	subespacio cerrado de $\geom{K}$.
\end{coroTopologiaCoherenteSubcomplejos}

\begin{proof}
	En primer lugar, la m\'{e}trica en $\geom{L}_{d}$ es la
	restricci\'{o}n de la m\'{e}trica en $\geom{K}_{d}$, con lo que
	$\geom{L}_{d}$ es un subespaico m\'{e}trico de $\geom{K}_{d}$
	y, en particular, un subespacio topol\'{o}gico.

	Sea $\alpha\in\geom{K}_{d}$ un elemento en la clausura de
	$\geom{L}_{d}$. Por definici\'{o}n, $\alpha(v)\geq 0$ para todo
	$v\in\qesq{0}{K}$ y $\alpha(v)>0$ s\'{o}lo para finitos
	v\'{e}rtices de $K$. Sean $\lista{v}{r}$ los v\'{e}rtices tales
	que $\alpha(v_{i})>0$, es decir,
	\begin{align*}
		\sum_{j=1}^{r}\,\alpha(v_{j}) & \,=\,1
	\end{align*}
	%
	Sea $a=\min\,\{\alpha(v_{i})\}_{i}>0$ y sea $\epsilon<a/2$ un
	n\'{u}mero positivo. Por hip\'{o}tesis, existe $\beta\in\geom{L}$
	tal que $d(\alpha,\beta)<\epsilon$. En particular,
	\begin{align*}
		\beta(v_{i}) & \,>\,\alpha(v_{i})-\epsilon\,>\,a/2\,>\,0
		\text{ .}
	\end{align*}
	%
	Como $\beta\in\geom{L}$, esto implica, por un lado, que los
	$v_{i}$ son v\'{e}rtices en el subcomplejo $L$ y, por otro, que
	\begin{align*}
		\{\beta\not =0\} & \,=\,\{\lista{v}{r}\}\,\cup\,B
		\text{ ,}
	\end{align*}
	%
	donde $B\subset\qesq{0}{L}$ es alg\'{u}n subconjunto (posiblemente
	vac\'{\i}o) de v\'{e}rtices de $L$. Como $\{\beta\not =0\}$ es
	un s\'{\i}mplice en $L$, se deduce que $\{\lista{v}{r}\}$
	tambi\'{e}n lo es. En conclusi\'{o}n, $\alpha\in\geom{L}$ y
	$\geom{L}_{d}$ es cerrado en $\geom{K}_{d}$.

	La \'{u}ltima afirmaci\'{o}n se deduce de
	\ref{thm:topologiacoherenteidentidad} y de que
	$\geom{L}\subset\geom{K}$ es un subespacio topol\'{o}gico (los
	s\'{\i}mplices que determinan la topolog\'{\i}a en $\geom{L}$ son
	los s\'{\i}mplices que determinan la topolog\'{\i}a en $\geom{K}$
	que pertenecen al subcomplejo $L$).
\end{proof}

\begin{obsRealizacionUnionInterseccion}\label{obs:realizacionunioninterseccion}
	Sea $K$ un complejo simplicial y sea $\{L_{i}\}_{i}$ una familia
	de subcomplejos de $K$. Entnonces
	\begin{math}
		\geom{\bigcup_{i}\,L_{i}} = \bigcup_{i}\,\geom{L_{i}}
	\end{math}
	y
	\begin{math}
		\geom{\bigcap_{i}\,L_{i}} = \bigcap_{i}\,\geom{L_{i}}
	\end{math}~.
\end{obsRealizacionUnionInterseccion}

Las aplicaciones $K\mapsto\geom{K}$ en complejos simpliciales y
$\varphi\mapsto\geom{\varphi}$ en transformaciones simpliciales
define un funtor de la categor\'{\i}a de complejos simpliciales en la
categor\'{\i}a de espacios topol\'{o}gicos. Lo mismo es cierto para las
aplicaciones $K\mapsto\geom{K}_{d}$ y $\varphi\mapsto\geom{\varphi}_{d}$.
M\'{a}s aun, como la identidad $\geom{K}\rightarrow\geom{K}_{d}$ es continua
y los diagramas
\begin{center}
	\begin{tikzcd}
		\geom{K} \arrow[r] \arrow[d,"\geom{\varphi}"'] &
			\geom{K}_{d} \arrow[d,"\geom{\varphi}_{d}"] \\
		\geom{K'} \arrow[r] & \geom{K'}_{d}
	\end{tikzcd}
\end{center}
son diagramas conmutativos en la categor\'{\i}a de espacios topol\'{o}gicos,
la identidad $\geom{K}\rightarrow\geom{K}_{d}$ determina una transformaci\'{o}n
natural $\geom{\cdot}\rightarrow\geom{\cdot}_{d}$ entre estos funtores.
Finalmente, podemos extender estos funtores a la categor\'{\i}a de pares
simpliciales: a un par simplicial $\varphi:\,K\rightarrow K'$ le
asociamos el par de espacios topol\'{o}gicos
$\geom{\varphi}:\,\geom{K}\rightarrow\geom{K'}$ (o bien $\geom{\varphi}_{d}$);
a un morfismo de pares simpliciales $(\psi',\psi)$ le asociamos el
morfismo de pares de espacios topol\'{o}gicos $(\geom{\psi'},\geom{\psi})$
(o, respectivamente, $(\geom{\psi'}_{d},\geom{\psi}_{d})$):
\begin{center}
\begin{tikzcd}
	K_{1} \arrow[r,"\varphi_{1}"] \arrow[d,"\psi"'] &
		K_{1}' \arrow[d,"\psi'"] \\
	K_{2} \arrow[r,"\varphi_{2}"'] & K'_{2}
\end{tikzcd}
	\qquad\begin{math} \longmapsto\end{math}\qquad
\begin{tikzcd}
	\geom{K_{1}} \arrow[r,"\geom{\varphi_{1}}"] \arrow[d,"\geom{\psi}"'] &
		\geom{K'_{1}} \arrow[d,"\geom{\psi'}"] \\
	\geom{K_{2}} \arrow[r,"\geom{\varphi_{2}}"'] & \geom{K'_{2}}
\end{tikzcd}
\end{center}

Una \emph{triangulaci\'{o}n} de un espacio topol\'{o}gico $X$ consiste en un
par $(K,f)$, donde $K$ es un complejo simplicial y $f:\,\geom{K}\rightarrow X$
es un homeomorfismo. An\'{a}logamente, definimos una traingulaci\'{o}n
de un par de espacios $g:\,X\rightarrow X'$ como un par simplicial
$\varphi:\,K\rightarrow K'$ junto con un homeomorfismo de pares
$(f',f):\,\geom{\varphi}\rightarrow g$, es decir, un par de homeomorfismos
$f:\,\geom{K}\rightarrow X$ y $f':\,K'\rightarrow X'$ tales que el diagrama
\begin{center}
\begin{tikzcd}
	\geom{K}\arrow[r,"\geom{\varphi}"] \arrow[d,"f"'] &
		\geom{K'} \arrow[d,"f'"] \\
	X \arrow[r,"g"'] & X'
\end{tikzcd}
\end{center}
conmuta. Notemos que, si el par $(X,A)$ consiste est\'{a} compuesto por
un espacio topol\'{o}gico $X$ y un subespacio $A\subset X$, entonces,
de existir una triangulaci\'{o}n $(K,L)$ la transformaci\'{o}n simplicial
correspondiente $L\rightarrow K$ debe ser (isomorfa a) un subcomplejo.
En particular, podemos ver el par $(\geom{K},\geom{L})$ como un par
topol\'{o}gico compuesto por el espacio $\geom{K}$ y un subespacio
$\geom{L}\subset\geom{K}$. En tal caso, los homeomorfismos
$f:\,\geom{L}\rightarrow A$ y $f':\,\geom{K}\rightarrow X$ est\'{a}n
forzados a cumplir $f=f'|_{\geom{L}}$.

%
\section{Propiedades de la topolog\'{\i}a coherente}
\theoremstyle{plain}
\newtheorem{teoTodoComplejoEsNormal}{Teorema}[section]
\newtheorem{coroTodoComplejoEsPerfectamenteNormal}[teoTodoComplejoEsNormal]%
	{Corolario}
\newtheorem{coroCompactoContenidoEnFinitosSimplices}[teoTodoComplejoEsNormal]%
	{Corolario}
\newtheorem{coroComplejoFinitoEspacioCompacto}[teoTodoComplejoEsNormal]%
	{Corolario}
\newtheorem{coroTodoComplejoEsCompactamenteGenerado}[teoTodoComplejoEsNormal]%
	{Corolario}
\newtheorem{coroComplejoLocalmenteFinitoEspacioLocalmenteCompacto}%
	[teoTodoComplejoEsNormal]{Corolario}
\newtheorem{teoHomotopiasDeComplejos}[teoTodoComplejoEsNormal]{Teorema}
\newtheorem{coroSimpliceDeSubcomplejo}[teoTodoComplejoEsNormal]{Corolario}
\newtheorem{teoTodoComplejoEsNervio}[teoTodoComplejoEsNormal]{Teorema}

\theoremstyle{remark}
\newtheorem{obsFuncionesContinuasEnUnComplejo}[teoTodoComplejoEsNormal]%
	{Observaci\'{o}n}

%-------------

Dado un complejo simplicial $K$, el espacio topol\'{o}gico $\geom{K}_{d}$
es perfectamente normal, pues es metrizable. Veamos que propiedades de
separabilidad verifica $\geom{K}$. El espacio $\geom{K}$ es Hausdorff, pues
posee m\'{a}s abiertos que $\geom{K}_{d}$. Pero esto no es todo.

\begin{teoTodoComplejoEsNormal}\label{thm:todocomplejoesnormal}
	Sea $K$ un complejo simplicial. Entonces $\geom{K}$ es un
	espacio topol\'{o}gico $T_{4}$ (normal Hausdorff)
\end{teoTodoComplejoEsNormal}

\begin{obsFuncionesContinuasEnUnComplejo}%
	\label{obs:funcionescontinuasenuncomplejo}
	Una funci\'{o}n $F:\,\geom{K}\rightarrow [0,1]$ es continua, si y
	s\'{o}lo si las restricciones $F|_{\geom{s}}$ lo son. As\'{\i} la
	existencia de una funci\'{o}n continua $F:\,\geom{K}\rightarrow [0,1]$
	equivale a la existencia de una familia compatible de funciones
	continuas $\{f_{s}:\,\geom{s}\rightarrow [0,1]\}_{s\in K}$. La
	compatibilidad de esta familia significa que, dados s\'{\i}mplices
	$s$ y $s'$ y dadas sus funciones correspondientes $f_{s}$ y $f_{s'}$,
	o bien $\geom{s}\cap\geom{s'}=\varnothing$, o, si la intersecci\'{o}n
	es no vac\'{\i}a, se cumple
	\begin{align*}
		f_{s}|_{\geom{s}\cap\geom{s'}} & \,=\,
			f_{s'}|_{\geom{s}\cap\geom{s'}}
		\text{ .}
	\end{align*}
	%
	Como $\geom{s}\cap\geom{s'}=\varnothing$ equivale a
	$s\cap s'=\varnothing$ y, en caso contrario,
	$\geom{s}\cap\geom{s'}=\geom{s\cap s'}$ es una cara tanto de $s$ como
	de $s'$, la comatibilidad de las funciones $\{f_{s}\}_{s\in K}$ se
	traduce en la condici\'{o}n
	\begin{equation}
		\label{eq:funcionescontinuasenuncomplejo}
		f_{s}|_{\geom{s'}} \,=\,f_{s'}
		\text{ ,}
	\end{equation}
	%
	para todo s\'{\i}mplice $s$ y toda cara $s'\subset s$.
\end{obsFuncionesContinuasEnUnComplejo}

\begin{proof}%[Demostraci\'{o}n de \ref{thm:todocomplejoesnormal}]
	Demostraremos que $\geom{K}$ es normal probando que es
	posible extender toda funci\'{o}n continua definida en un subespacio
	cerrado. Sea $A\subset\geom{K}$ un subespacio cerrado y sea
	$f:\,A\rightarrow [0,1]$ una funci\'{o}n continua. Una funci\'{o}n
	$F:\,\geom{K}\rightarrow [0,1]$ es una extensi\'{o}n continua de $f$,
	si y s\'{o}lo si, adem\'{a}s de la condici\'{o}n de continuidad
	(compatibilidad) \eqref{eq:funcionescontinuasenuncomplejo}, se
	verifica
	\begin{equation}
		\label{eq:todocomplejoesnormal}
		F_{s}|_{A\cap\geom{s}} \,=\,f|_{A\cap\geom{s}}
	\end{equation}
	%
	para todo s\'{\i}mplice $s\in K$. Definimos una extensi\'{o}n de
	manera inductiva en la dimensi\'{o}n de los s\'{\i}mplices de $K$.
	Si $s\in K$ es un $0$-s\'{\i}mplice, un v\'{e}rtice, entonces
	$\geom{s}\in\geom{K}$ consiste en un \'{u}nico punto, al que llamamos
	$\alpha_{s}$ y, o bien $\alpha_{s}\in A$, o bien
	$\alpha_{s}\not\in A$. En el primer caso definimos
	$f_{s}=f(\alpha_{s})$; en el segundo caso definimos $f_{s}$ de manera
	arbitraria, por ejemplo, $f_{s}=1$. As\'{\i},
	\begin{align*}
		f_{s} & \,=\,
			\begin{cases}
				f(\alpha_{s}) & \quad\text{ si }
					\alpha_{s}\in A \\
				1 & \quad\text{ si no}
			\end{cases}
		\text{ .}
	\end{align*}
	%
	Supongamos que $q>0$ y que existen funciones $\{f_{s}\,:\,\dim\,s<q\}$
	que verifican \eqref{eq:funcionescontinuasenuncomplejo} y
	\eqref{eq:todocomplejoesnormal}. Dado un $q$-s\'{\i}mplice $s$ y un
	punto $\alpha\in\geom{s}$, o bien $\alpha\in\geom{carasp{s}}$, es
	decir $\alpha\in\geom{s'}$ para cierta cara $s'\subset s$ propia,
	o bien $\alpha$ pertenece al interior de $\geom{s}$ (ver
	\ref{obs:topologiaensimplices}, este conjunto no es el interior
	como subespacio de $\geom{K}$, son simplemente los puntos de
	$\geom{s}$ que no pertenecen a caras propias, pero s\'{\i} es abierto
	en $\geom{s}$). Definimos un funci\'{o}n intermedia
	$f'_{s}:\,\geom{\carasp{s}}\cup (A\cap\geom{s})\rightarrow [0,1]$
	de la siguiente manera:
	\begin{align*}
		f'_{s}|_{\geom{s'}} & \,=\,f_{s'}
			\quad\text{ si } s'\text{ es una cara propia,} \\
		f'_{s}|_{A\cap\geom{s}} & \,=\,f|_{A\cap\geom{s}}\text{ .}
	\end{align*}
	%
	(Si la intersecci\'{o}n $A\cap\geom{s}$ es vac\'{\i}a, $f'_{s}$
	queda sin definir all\'{\i}). Esto define una funci\'{o}n continua
	en el subespacio cerrado $\geom{\carasp{s}}\cup (A\cap\geom{s})$ de
	$\geom{s}$ con imagen en el intervalo $[0,1]$. Por el teorema de
	extensi\'{o}n de Tietze ($\geom{s}$ es, entre otras cosas, un
	espacio $T_{4}$), existe una funci\'{o}n continua
	$f_{s}:\,\geom{s}\rightarrow [0,1]$ que extiende a $f'_{s}$. La
	familia $\{f_{s}\}_{s\in K}$ definida inductivamente determina una
	extensi\'{o}n continua de la funci\'{o}n $f:\,A\rightarrow [0,1]$ a
	todo el espacio $\geom{K}$.
\end{proof}

!`Y aun hay m\'{a}s!

\begin{coroTodoComplejoEsPerfectamenteNormal}%
	\label{thm:todocomplejoesperfectamentenormal}
	El espacio $\geom{K}$ es $T_{6}$\dots (perfectamente normal Hausdorff).
\end{coroTodoComplejoEsPerfectamenteNormal}

\begin{proof}
	Sea $A\subset\geom{K}$ un subespacio cerrado. Veremos que $A$ es el
	conjunto de ceros de una funci\'{o}n continua definida en $\geom{K}$.
	Consideramos la funci\'{o}n $f:\,A\rightarrow [0,1]$, $f=0$.
	Procedemos inductivamente, como en la demostraci\'{o}n de
	\ref{thm:todocomplejoesnormal}, pero con algunas salvedades.
	Empezamos con los $0$-s\'{\i}mplices: si $s\in K$ es tal que
	$\dim\,s=0$ y llamamos $\alpha_{s}$ al \'{u}nico punto en $\geom{s}$,
	entonces definimos
	\begin{align*}
		f_{s} & \,=\,
			\begin{cases}
				0 & \quad\text{si } \alpha_{s}\in A \\
				1 & \quad\text{si } \alpha_{s}\not\in A
			\end{cases}
		\text{ .}
	\end{align*}
	%
	Supongamos que $q>0$ y que las funciones $f_{s}$ est\'{a}n definidas
	para todo s\'{\i}mplice de dimensi\'{o}n $\dim\,s<q$, de manera tal
	que
	\begin{equation}
		\label{eq:todocomplejoesperfectamentenormal}
		\begin{aligned}
			f_{s}|_{\geom{s'}} & \,=\,f_{s'}
				\quad\text{si } s'\subset s\text{ es una %
				cara propia} \\
			f_{s}|_{A\cap\geom{s}} & \,=\,0
			\text{ .}
		\end{aligned}
	\end{equation}
	%
	(Para ser un poco m\'{a}s precisos, podemos suponer, adem\'{a}s, que
	$f_{s}=1$, si $\geom{s}\cap A=\varnothing$ y que $f_{s}(\alpha)>0$,
	si $\alpha\not\in A$, aunque esto se podr\'{a} deducir de la forma
	en la que se realizar\'{a}n los pasos inductivos).
	Sea $s\in K$ un $q$-s\'{\i}mplice. Si $\geom{s}\subset A$,
	definimos $f_{s}=0$, si $\geom{s}\cap A=\varnothing$, definimos
	$f_{s}=1$. En otro caso, $A\cap\geom{s}\subset\geom{s}$ es un
	subespacio cerrado y tambi\'{e}n lo es el subespacio que se
	obtiene a partir de las caras propias de $s$, $\geom{\carasp{s}}$.
	Definimos dos funciones intermedias: sean $f'_{s}$ y $f''_{s}$ las
	funciones $\geom{\carasp{s}}\cup(A\cap\geom{s})\rightarrow [0,1]$
	dadas por:
	\begin{align*}
		f'_{s}|_{\geom{s'}} & \,=\,f_{s'}
			\quad\text{si } s'\subset s
			\text{ es una cara propia} \\
		f'_{s}|_{A\cap\geom{s}} & \,=\, 0 \text{ ;}
	\end{align*}
	%
	y por
	\begin{align*}
		f''_{s}|_{\geom{s'}} & \,=\,0 \\
		f''_{s}|_{A\cap\geom{s}} & \,=\,0 \text{ .}
	\end{align*}
	%
	Como $\geom{\carasp{s}}\cup(A\cap\geom{s})\subset\geom{s}$ es cerrado
	y $\geom{s}$ es $T_{4}$, existe, por el teorema de extensi\'{o}n
	de Tietze, una funci\'{o}n $g_{s}:\,\geom{s}\rightarrow [0,1]$
	continua que extiende a $f'_{s}$ (de nuevo remarcamos que, si
	$\geom{s}\cap A=\varnothing$, elegimos $g_{s}=1$). Por otro lado,
	como, adem\'{a}s, $\geom{s}$ es $T_{6}$, existe una extensi\'{o}n
	$h_{s}:\,\geom{s}\rightarrow [0,1]$ de $f''_{s}=0$ que es
	estrictamente positiva fuera de $\geom{\carasp{s}}\cup(A\cap\geom{s})$.
	Sea finalmente $f_{s}:\,\geom{s}\rightarrow [0,1]$ la funci\'{o}n
	dada por
	\begin{align*}
		f_{s} & \,=\,g_{s} + \frac{h_{s}}{h_{s}+1}\cdot(1-g_{s})
		\text{ .}
	\end{align*}
	%
	Entonces $f_{s}$ es continua, coincide con $f'_{s}$ en
	$\geom{\carasp{s}}\cup(A\cap\geom{s})$ y es estrictamente positiva
	$\geom{s}\setmin A$. Adem\'{a}s, $f_{s}=0$, si $\geom{s}\subset A$ y
	$f_{s}=1$, si $\geom{s}\cap A=\varnothing$. La familia de funciones
	$\{f_{s}\}_{s\in K}$ definida de manera inductiva determina una
	funci\'{o}n continua en $\geom{K}$ que es cero en $A$ y estrictamente
	positiva fuera de $A$. Esta funci\'{o}n verifica tambi\'{e}n que
	vale $1$ en todo s\'{\i}mplice disjunto de $A$.
\end{proof}

Sea $\simpinterior{s}=\geom{s}\setmin\geom{\carasp{s}}$. Este subconjunto
es abierto en $\geom{s}$, pero no necesariamente en $\geom{K}$ (nunca ser\'{a}
abierto si $s$ est\'{a} contenido en un s\'{\i}mplice de dimensi\'{o}n
estrictamente mayor). Los subconjuntos $\simpinterior{s}$ cubren $\geom{K}$.
Sea $A\subset\geom{K}$ un subconjunto arbitrario. Para cada s\'{\i}plice
$s\in K$, sea $\alpha_{s}\in A\cap\simpinterior{s}$ (siempre que la
intersecci\'{o}n sea no vac\'{\i}a). Como $\alpha_{s}$ no pertenece a ninguna
cara propia de $s$,
% (e, invirtiendo la perspectiva, no pertenece a $\simpinterior{s'}$ para
% ning\'{u}n s\'{\i}mplice del cual $s$ sea una cara propia),
$\alpha_{s}$ est\'{a}
un\'{\i}vocamente asociado a $s$, es decir, $\alpha_{s}=\alpha_{s'}$ implica
$s=s'$. En particular, $\alpha_{s}\in \geom{s'}$ implica que $s\subset s'$,
con lo cual, si $A'=\{\alpha_{s}\}_{A\cap\simpinterior{s}\not=\varnothing}$,
entonces $\geom{s}\cap A'$ es a lo sumo finita para todo s\'{\i}mplice
$s\in K$. Se deduce entonces que $A'\subset\geom{K}$ es discreto.

\begin{coroCompactoContenidoEnFinitosSimplices}%
	\label{thm:compactocontenidoenfinitossimplices}
	Todo subespacio compacto $A\subset\geom{K}$ est\'{a} contenido en
	la uni\'{o}n de una cantidad finita de s\'{\i}mplices
	abiertos $\simpinterior{s}$.
\end{coroCompactoContenidoEnFinitosSimplices}

\begin{coroTodoComplejoEsCompactamenteGenerado}%
	\label{thm:todocomplejoescompactamentegenerado}
	Dado un complejo simplicial $K$, el espacio $\geom{K}$ con
	la topolog\'{\i}a coherente es compactamente generado, es decir,
	la topolog\'{\i}a es coherente con la familia de subespacios
	compactos de $\geom{K}$.
\end{coroTodoComplejoEsCompactamenteGenerado}

\begin{proof}
	Si $A\subset\geom{K}$ es cerrado, como $\geom{K}$ es Hausdorff,
	$A\cap C$ es cerrado en $C$ para todo compacto $C\subset\geom{K}$.
	Rec\'{\i}procamente, si $A\cap C$ es cerrado en $C$ para todo
	compacto $C$, entonces $A\cap\geom{s}$ es cerrado para todo
	s\'{\i}mplice $s\in K$ y, en particular, $A$ es cerrado.
\end{proof}

\begin{coroComplejoFinitoEspacioCompacto}%
	\label{thm:complejofinitoespaciocompacto}
	Un complejo simplicial $K$ es finito, si y s\'{o}lo si $\geom{K}$
	es compacto.
\end{coroComplejoFinitoEspacioCompacto}

\begin{coroComplejoLocalmenteFinitoEspacioLocalmenteCompacto}%
	\label{thm:complejolocalmentefinitoespaciolocalmentecompacto}
	Un complejo simplicial $K$ es localmente finito, si y s\'{o}lo si
	$\geom{K}$ es localmente compacto. (???)
\end{coroComplejoLocalmenteFinitoEspacioLocalmenteCompacto}

\begin{teoHomotopiasDeComplejos}\label{thm:homotopiasdecomplejos}
	Sea $K$ un complejo simplicial. Una funci\'{o}n
	$F:\,\geom{K}\times [0,1]\rightarrow X$ es continua, si y s\'{o}lo si
	las restricciones $F|_{\geom{s}\times [0,1]}$ son continuas
	para todo s\'{\i}mplice $s\in K$.
\end{teoHomotopiasDeComplejos}

\begin{proof}
	Como $[0,1]$ es localmente compacto Hausdorff y $\geom{K}$ es
	compactamente generado, el producto $\geom{K}\times [0,1]$ es
	compactamente generado, tambi\'{e}n. Si $C\subset\geom{K}\times [0,1]$
	es compacto y $C_{1}\subset\geom{K}$ denota su proyecci\'{o}n en
	el primer factor, entonces $C_{1}$ es compacto y, por el
	corolario \ref{thm:compactocontenidoenfinitossimplices}, existen
	s\'{\i}mplices $\lista{s}{r}$ tales que
	\begin{align*}
		C_{1} & \,\subset\,\simpinterior{s_{1}}\,\cup\,\cdots\,\cup\,
					\simpinterior{s_{r}}
	\end{align*}
	%
	En particular, el subcomplejo finito
	$L=\caras{s_{1}}\cup\cdots\cup\caras{s_{r}}$ de $\geom{K}$ cumple que
	\begin{align*}
		C & \,\subset\,\geom{L}\times [0,1]
		\text{ .}
	\end{align*}
	%
	De esto y de que $\geom{K}\times [0,1]$ es compactamente generado,
	se deduce que la topolog\'{\i}a de $\geom{K}\times [0,1]$ coincide
	con la topolog\'{\i}a determinada por la familia de subconjuntos
	$\big\{\geom{L}\times [0,1]\big\}_{L}$, donde $L$ var\'{\i}a entre
	todos los subcomplejos finitos de $K$.
	% un subconjunto A\subset $\geom{K}\times [0,1]$ es cerrado, si
	% y s\'{o}lo si su intersecci\'{o}n con $\geom{L}\times [0,1]$
	% es cerrada.
	Esta topolog\'{\i}a es, a su vez, equivalente a la topolog\'{\i}a
	determinada por $\big\{\geom{s}\times [0,1]\,:\,s\in K\big\}$, pues,
	si $L$ es finito, la topolog\'{\i}a en $\geom{L}\times [0,1]$
	coincide con la topolog\'{\i}a determinada por la familia
	$\big\{\geom{s}\times [0,1]\,:\,s\in L\big\}$.
\end{proof}

Sea $K$ un complejo simplicial. Sea $v\in V$ un v\'{e}rtice. Definimos la
\emph{estrella centrada en $v$} como el subconjunto
\begin{align*}
	\estrella{v} & \,=\,\big\{\alpha\in\geom{K}\,:\,\alpha(v)\not=0\big\}
	\text{ .}
\end{align*}
%
La funci\'{o}n
\begin{math}
	\big(\alpha\mapsto\alpha(v)\big):\,\geom{K}_{d}\rightarrow [0,1]
\end{math}~,
dada por tomar ``coordenada en $v$'', es continua con respecto a la
m\'{e}trica en $\geom{K}_{d}$. En particular, $\estrella{v}$ es un
subconjunto abierto de $\geom{K}_{d}$ y, por lo tanto, tambi\'{e}n de
$\geom{K}$.

Sea $\alpha\in\geom{K}$ un punto arbitrario. Entonces, excepto que $\alpha$
sea un v\'{e}rtice de $\geom{K}$ ($\alpha(v)=1$ para alg\'{u}n v\'{e}rtice
en $K$), existe un s\'{\i}mplice $s$ tal que $\alpha\in\geom{s}$, pero
$\alpha\not\in\geom{s'}$, si $s'\subset s$ es una cara propia. Este
s\'{\i}mplice (o $\{v\}$, si $\alpha(v)=1$ para alg\'{u}n $v$), es el
s\'{\i}mplice m\'{a}s chico que contiene a $\alpha$. Dado que
\begin{align*}
	\simpinterior{s} & \,=\,\geom{s}\setmin\geom{\carasp{s}} \,=\,
		\big\{\alpha\in\geom{K}\,:\,
			\alpha(v)\not =0\Leftrightarrow v\in s\big\}
		\text{ ,}
\end{align*}
%
entonces $\alpha\in\estrella{v}$, si y s\'{o}lo si $\alpha\in\simpinterior{s}$
para alg\'{u}n s\'{\i}mplice $s\in K$ que tenga a $v$ como v\'{e}rtice. En
particular,
\begin{align*}
	\estrella{v} & \,=\,\bigcup\,
		\big\{\simpinterior{s}\,:\,v\in s, s\in K\big\}
	\text{ .}
\end{align*}
%

\begin{coroSimpliceDeSubcomplejo}\label{thm:simplicedesubcomplejo}
	Sea $L\subset K$ un subcomplejo y sean $\lista[0]{v}{q}$ v\'{e}rtices
	(distintos) de $K$. Entonces $\lista[0]{v}{q}$ son los v\'{e}rtices
	de un s\'{\i}mplice en $L$, si y s\'{o}lo si
	\begin{align*}
		\bigcap_{i=0}^{q}\,\estrella{v_{i}}\cap\geom{L} & \,\not=\,
			\varnothing
		\text{ .}
	\end{align*}
	%
\end{coroSimpliceDeSubcomplejo}

\begin{teoTodoComplejoEsNervio}\label{thm:todocomplejoesnervio}
	Sea $K$ un complejo simplicial y sea
	$\cal{U} =\big\{\estrella{v}\,:\,v\in V$ la familia de estrellas
	centradas en v\'{e}rtices de $K$. La aplicaci\'{o}n
	$v\mapsto\estrella{v}$ define una transformaci\'{o}n simplicial
	$\varphi:\,K\rightarrow\nerv{\cal{U}}$ de $K$ en el nervio de la
	colecci\'{o}n de abiertos $\cal{U}$ de $\geom{K}$. La
	transformaci\'{o}n $\varphi$ es un isomorfismo y, para todo
	subcomplejo $L\subset K$, la restricci\'{o}n
	$\varphi|_{L}:\,L\rightarrow\nerv[\geom{L}]{\cal{U}}$ tambi\'{e}n
	lo es.
\end{teoTodoComplejoEsNervio}

%
\section{La estructura lineal de los complejos simpliciales}
\theoremstyle{plain}
\newtheorem{lemaCombinacionConvexaEnComplejos}{Lema}[section]

\theoremstyle{remark}
\newtheorem{obsSimplicialEsLineal}[lemaCombinacionConvexaEnComplejos]%
	{Observaci\'{o}n}

%-------------

\begin{lemaCombinacionConvexaEnComplejos}%
	\label{thm:combinacionconvexaencomplejos}
	Sea $K$ un complejo simplicial. Una combinaci\'{o}n convexa
	de puntos de $\geom{K}$ pertenece a $\geom{K}$, si y s\'{o}lo si
	dichos puntos pertenecen a un mismo s\'{\i}mplice en $K$.
\end{lemaCombinacionConvexaEnComplejos}

\begin{proof}
	Sean $\lista{\alpha}{r}\in\geom{K}$ puntos pertenecientes a un
	mismo s\'{\i}mplice $\geom{s}$, $s\in K$. Dados n\'{u}meros reales
	$\lista*{t}{r}\in [0,1]$ tales que $\sum_{i=1}^{r}\,t^{i}=1$,
	la funci\'{o}n $\alpha:\,V\rightarrow [0,1]$ dada por
	$\alpha=\sum_{i=1}^{r}\,t^{i}\alpha_{i}$ pertenece a $\geom{s}$,
	pues $\{\alpha\not=0\}\subset s$ y $s\in K$ es un s\'{\i}mplice.
	Rec\'{\i}procamente, si $\lista{\alpha}{r}\in\geom{K}$ son puntos que
	verifican $\alpha=\sum_{i=1}^{r}\,t^{i}\alpha_{i}\in\geom{K}$ para
	ciertos n\'{u}meros reales \emph{positivos} $t^{i}$ que suman $1$
	(ninguno de los puntos $\alpha_{i}$ aparece trivialmente), entonces
	existe un s\'{\i}mplice $s\in K$ m\'{a}s chico que contiene a
	$\alpha$. Como $\alpha\in\simpinterior{s}$, si $v\not\in s$ es un
	v\'{e}rtice de $K$ que no pertenece a $s$, entonces
	$\alpha(v)=0$ y, para cada $i=1,\,\dots,\,r$, $\alpha_{i}(v)=0$.
	Por lo tanto, $\alpha_{i}\in\geom{s}$.
\end{proof}

Dado un v\'{e}rtice $v\in K$, llamamos $\alpha_{v}$ a la funci\'{o}n
$\alpha_{v}\in\geom{K}$ tal que $\alpha_{v}(v)=1$ (\textit{a fortiori},
$\alpha_{v}(v')=0$, si $v'\not = v$). Todo elemento $\alpha\in\geom{K}$
se puede escribir (de manera \'{u}nica) como una combinaci\'{o}n lineal
convexa de estas funciones:
\begin{align*}
	\alpha & \,=\,\sum_{v\in V}\,\alpha(v)\,\alpha_{v}
	\text{ .}
\end{align*}
%

Sea $E$ un $\bb{R}$-espacio vectorial y sea $X\subset E$ un espacio
topol\'{o}gico cuya topolog\'{\i}a es coherente con su intersecci\'{o}n
con los subespacios de dimensi\'{o}n finita de $E$ (d\'{a}ndole a \'{e}stos
la \'{u}nica topolog\'{\i}a con respecto a la cual son espacios vectoriales
topol\'{o}gicos).
% (Por ejemplo, $E$ tiene la topolog\'{\i}a d\'{e}bil respecto
% de una familia de funcionales lineales que separa puntos)
Entonces, si $F\subset E$ es un subespacio vectorial de dimensi\'{o}n finita,
$X\cap F$ tiene la topolog\'{\i}a de subespacio de $F$, donde $F$ tiene
la etructura de e.v.t. de dimnsi\'{o}n finita.

Una funci\'{o}n $f:\,\geom{K}\rightarrow E$ se dice \emph{lineal}, si
\begin{align*}
	f(\alpha) & \,=\,f\Big(\sum_{v\in V}\,\alpha(v)\alpha_{v}\Big) \,=\,
		\sum_{v\in V}\,\alpha(v)\,f(\alpha_{v})
\end{align*}
%
para todo $\alpha\in\geom{K}$. Una funci\'{o}n $f:\,\geom{K}\rightarrow X$ se
dice lineal, si $\inc[X\hookrightarrow E]\circ f:\,\geom{K}\rightarrow E$ es
lineal (en particular, la combinaci\'{o}n lineal
$\sum_{v}\,\alpha(v)\,f(\alpha_{v})$ tiene que ser un punto de $X$).

Toda funci\'{o}n lineal en $\geom{K}$ est\'{a} determinada por su valor en
los v\'{e}rtices $\alpha_{v}$. Rec\'{\i}procamente, toda funci\'{o}n
$f_{0}:\,V\rightarrow E$ determina una funci\'{o}n lineal
$f:\,\geom{K}\rightarrow E$. Si $X\subset E$ y $f_{0}:\,V\rightarrow X$,
entonces la funci\'{o}n lineal correspondiente, $f$, es una funci\'{o}n
lineal en $X$, si y s\'{o}lo si, para todo s\'{\i}mplice $s\in K$, las
combinaciones convexas $\sum_{v\in s}\,t^{v}f_{0}(\alpha_{v})$ tambi\'{e}n
pertenecen a $X$. En particular, si $X$ es un subconjunto convexo de $E$,
entonces toda funci\'{o}n $V\rightarrow X$ se extiende a una funci\'{o}n
lineal $\geom{K}\rightarrow X$.

En cuanto a la continuidad de una funci\'{o}n lineal
$f:\,\geom{K}\rightarrow X$, si $s\in K$ es un $q$-s\'{\i}mplice, entonces la
restricci\'{o}n $f|_{\geom{s}}$ coincide con la restricci\'{o}n de una
funci\'{o}n lineal definida en el $\bb{R}$-espacio vectorial generado
por los v\'{e}rtices de $s$. A este espacio lo denotamos
$F\big(\{\alpha_{v}\}_{v\in s}\big)$. Todo punto de este espacio vectorial se
escribe de manera \'{u}nica como combinaci\'{o}n lineal de los v\'{e}rtices
$\{\alpha_{v}\}_{v\in s}$. A su vez, los puntos de $\geom{s}$ se escriben
de manera \'{u}nica como combinaciones lineales convexas de estos mismos
elementos. As\'{\i}, podemos identificar can\'{o}nicamente $\geom{s}$
con un subconjunto de $F\big(\{\alpha_{v}\}_{v\in s}\big)$. En cuanto a la
funci\'{o}n $f$, si $\alpha\in\geom{s}$ y
$\alpha=\sum_{v\in s}\,\alpha(v)\,\alpha_{v}$, entonces
$f(\alpha)=\sum_{v\in s}\,\alpha(v)\,f(\alpha_{v})$. Pero, por otro lado,
por propiedad universal de $F\big(\{\alpha_{v}\}_{v\in s}\big)$, las
asignaciones $\alpha_{v}\mapsto f(\alpha_{v})$ determinan un\'{\i}vocamente
una funci\'{o}n lineal
$f_{s}:\,F\big(\{\alpha_{v}\}_{v\in s}\big)\rightarrow E$. Esta extensi\'{o}n
cumple que
\begin{align*}
	f_{s}|_{\geom{s}} & \,=\,f|_{\geom{s}}
	\text{ ,}
\end{align*}
%
pues ambas est\'{a}n dadas por la misma f\'{o}rmula en t\'{e}rminos de los
elementos de la base $\{\alpha_{v}\}_{v\in s}$. En particular,
$f_{s}(\geom{s})\subset X$. Notemos tambi\'{e}n que la topolog\'{\i}a en
$\geom{s}$ es exactamente la topolog\'{\i}a inducida como subespacio de
$F\big(\{\alpha_{v}\}_{v\in s}\big)$ y que $X$ es un subespacio topol\'{o}gico
de $E$, con lo cual, para ver $f:\,\geom{K}\rightarrow X$ es continua,
alcanza con ver que las funciones $f_{s}$ son continuas. Pero $f_{s}$, por
ser lineal, tiene imagen en un subespacio $W\subset E$ de dimensi\'{o}n
finita. Por lo tanto,
$f_{s}:\,F\big(\{\alpha_{v}\}_{v\in s}\big)\rightarrow W$ es continua.
Como la inclusi\'{o}n $W\rightarrow E$ es continua, $f_{s}$ es continua.
En definitiva, toda funci\'{o}n lineal $f:\,\geom{K}\rightarrow X$ es
autom\'{a}ticamente continua\dots

\begin{obsSimplicialEsLineal}\label{obs:simplicialeslineal}
	Dado un complejo simplicial $K$ consideramos el $\bb{R}$-espacio
	vectorial generado por todos los v\'{e}rtices de $K$:
	$E=F\big(\{\alpha_{v}\}_{v\in K}\big)$. El conjunto $\geom{K}$ es
	un subconjunto de este espacio: todo elemento $\alpha\in\geom{K}$
	es una combinaci\'{o}n lineal convexa de los elementos de la base
	(pero no todas tales combinaciones pertenecen a $\geom{K}$).
	La topolog\'{\i}a coherente en $\geom{K}$ est\'{a} determinada
	por la topolog\'{\i}a en cada uno de los subconjuntos $\geom{s}$,
	$s\in K$. Pero la topolog\'{\i}a en $\geom{s}$ es la topolog\'{\i}a
	de subespacio del $\bb{R}$-espacio de dimensi\'{o}n finita
	generado por aquellos v\'{e}rtices de $K$ que pertenecen a $s$.

	Dada una transformaci\'{o}n simplicial $\varphi:\,K\rightarrow K'$,
	aplicando el funtor $\geom{\cdot}$, se obtiene una funci\'{o}n
	lineal: $\geom{\varphi}:\,\geom{K}\rightarrow\geom{K'}$ est\'{a}
	dada por la f\'{o}rmula
	\begin{align*}
		\geom{\varphi}(\alpha) & \,=\,
			\sum_{v\in V}\,\alpha(v)\,\alpha_{\varphi(v)} \,=\,
			\sum_{v\in V}\,\alpha(v)\,\geom{\varphi}(\alpha_{v})
		\text{ .}
	\end{align*}
	%
\end{obsSimplicialEsLineal}

%
\section{Homotop\'{\i}as con complejos}
\theoremstyle{plain}
\newtheorem{lemaConoSobreLasCarasPropias}{Lema}[section]
\newtheorem{lemaRetractoPorDeformacionSimplices}%
	[lemaConoSobreLasCarasPropias]{Lema}
\newtheorem{coroRetractoPorDeformacionComplejos}%
	[lemaConoSobreLasCarasPropias]{Corolario}
\newtheorem{coroExtensionDeHomotopiasEnSubcomplejos}%
	[lemaConoSobreLasCarasPropias]{Corolario}

\theoremstyle{remark}

%-------------

Sea $f:\,X\rightarrow Y$ una funci\'{o}n continua. Definimos el
\emph{cilindro de $f$} como el cociente
\begin{align*}
	\cilindro{f} & \,=\,\big((X\times\intervalo)\sqcup Y\big)/\sim
	\text{ ,}
\end{align*}
%
identificando los puntos $(x,1)\sim f(x)\in Y$. Dado un espacio topol\'{o}gico
$X$ y un conjunto puntual $w$, el \emph{cono de $X$ con v\'{e}rtice $w$} es el
cilindro de la funci\'{o}n constante $X\rightarrow w$. Denotaremos este
espacio por $X*w$, o simplemente $\cono{X}$, omitiendo la referencia al
v\'{e}rtice $w$.

\begin{lemaConoSobreLasCarasPropias}\label{thm:conosobrelascaraspropias}
	Sea $s\in K$ un s\'{\i}mplice de un complejo $K$. El cono
	$\geom{\carasp{s}}*w$ es homeomorfo a $\geom{s}$.
\end{lemaConoSobreLasCarasPropias}

\begin{proof}
	Sea $w_{0}\in\simpinterior{s}$ un punto arbitrario (una funci\'{o}n)
	y sea $f:\,\geom{\carasp{s}}*w\rightarrow\geom{s}$ la funci\'{o}n
	\begin{align*}
		f[\alpha,t] & \,=\, tw_{0}+(1-t)\alpha
		\text{ .}
	\end{align*}
	%
	Esta funci\'{o}n es la factorizaci\'{o}n de la funci\'{o}n
	en $(\geom{\carasp{s}}\times\intervalo)\sqcup w$ dada por
	\begin{align*}
		(\alpha,t) & \,\mapsto\,tw_{0}+(1-t)\alpha\quad\text{y} \\
		w & \,\mapsto \,w_{0}\text{ .}
	\end{align*}
	%
	Notemos que esta funci\'{o}n (y, por lo tanto, su factorizaci\'{o}n)
	est\'{a} bien definida porque $\geom{s}$ es convexo. Ambas partes
	de la definici\'{o}n en la uni\'{o}n disjunta son continuas.
	La funci\'{o}n $f$ es, en consecuencia, continua.

	Para ver que $f$ es un homeomorfismo, es suficiente ver que es una
	biyecci\'{o}n. Supongamos que $[\alpha,t]$ y $[\beta,t']$ son
	tales que
	\begin{align*}
		tw_{0}+(1-t)\alpha & \,=\,t'w_{0}+(1-t')\beta
		\text{ .}
	\end{align*}
	%
	Como $\alpha\in\geom{\carasp{s}}$, $\alpha$ se anula en al menos uno
	de los v\'{e}rtices de $s$. Por otro lado, como $w_{0}$ est\'{a}
	en el s\'{\i}mplice interior $\simpinterior{s}$, todas las
	coordenadas de $w_{0}$ son positivas. Si $v\in s$ es tal que
	$\alpha(v)=0$, entonces
	\begin{align*}
		tw_{0}(v) & \,=\,t'w_{0}(v)+(1-t')\beta(v)
		\text{ .}
	\end{align*}
	%
	Como $w_{0}(v)>0$, se deduce que $t\geq t'$. An\'{a}logamente,
	intercambiando los roles de $\alpha$ y de $\beta$, se deduce que
	$t'\geq t$. En definitiva, $t=t'$. La igualdad
	\begin{align*}
		(1-t)\alpha & \,=\,(1-t)\beta
	\end{align*}
	%
	implica que, o bien $t=t'=1$, o bien $t=t'$ y $\alpha=\beta$. En todo
	caso $[\alpha,t]=[\beta,t']$.

	Para ver que $f$ es suryectiva, notamos, primero, que, si
	$\alpha\in\geom{\carasp{s}}$, entonces $f[\alpha,0]=\alpha$ y que
	$f[w]=w_{0}$. Resta ver que los puntos en
	$\simpinterior{s}\setmin\{w_{0}\}$ tambi\'{e}n est\'{a}n en la
	imagen de $f$. Para eso, veremos que por todo punto
	$\alpha\in\simpinterior{s}$, $\alpha\not=w_{0}$ pasa un segmento
	que va desde $w_{0}$ a alg\'{u}n punto de $\geom{\carasp{s}}$.
	Sea $\phi:\,\bb{R}\rightarrow F$ la funci\'{o}n
	\begin{align*}
		\phi(t) & \,=\, (1+t)\alpha -tw_{0}
		\text{ ,}
	\end{align*}
	%
	donde $F=F\big(\{\alpha_{v}\}_{v\in s}\big)$ (abusando un poco m\'{a}s
	de la notaci\'{o}n, podr\'{\i}amos escribir $F(s)$ para denotar este
	espacio). Esta funci\'{o}n es continua y
	$\phi(0)=\alpha\in\simpinterior{s}$ que es abierto en $F$. Como
	$\alpha\not =w_{0}$ y las coordenadas de ambos suman $1$, debe valer
	que $\alpha(v')<w_{0}(v')$ para alg\'{u}n v\'{e}rtice $v'\in s$.
	Mirando esta coordenada, se obtiene una funci\'{o}n mon\'{o}tona
	decresciente estrictamente en $t$:
	\begin{align*}
		\phi(t)(v') & \,=\,\alpha(v') - t(w_{0}(v')-\alpha(v'))
		\text{ .}
	\end{align*}
	%
	Por lo tanto, existe un \'{u}nico valor de $t$ tal que
	$\phi(t)(v')=0$, adem\'{a}s este valor de $t$ debe ser positivo.
	Como los v\'{e}rtices de $s$ son finitos, debe existir un menor
	valor positivo $t_{0}>0$ tal que $\phi(t_{0})(v)=0$ para alg\'{u}n
	v\'{e}rtice $v\in s$. En particular, el punto $\phi(t_{0})\in\geom{s}$
	pertenece, en realidad, a $\geom{\carasp{s}}$ y
	\begin{align*}
		\alpha & \,=\,\frac{t_{0}}{1+t_{0}}\,w_{0}+
			\frac{1}{1+t_{0}}\,\phi(t_{0})
		\text{ .}
	\end{align*}
	%
	En particular, $f$ es sobre.
\end{proof}

Contar con un punto en $\simpinterior{s}$ permite parametrizar los
puntos del espacio $\geom{s}$ mediante las coordenadas $[\alpha,t]$ del
cono $\cono{\geom{\carasp{s}}}$ y la funci\'{o}n correspondiente
definida como en la demostraci\'{o}n del lema
\ref{thm:conosobrelascaraspropias}. Para poder hacer esto de alguna manera
can\'{o}nica, definimos el \emph{baricentro} de un s\'{\i}mplice $s$ como
el punto $b(s)\in\simpinterior{s}$ dado por
\begin{align*}
	b(s) & \,=\,\sum_{v\in s}\,\frac{1}{1+\dim\,s}\,\alpha_{v}
	\text{ .}
\end{align*}
%

\begin{lemaRetractoPorDeformacionSimplices}%
	\label{thm:retractopordeformacionsimplices}
	Sea $s\in K$ un s\'{\i}mplice. El subespacio
	$(\geom{s}\times\{0\})\cup(\geom{\carasp{s}}\times\intervalo)$ es un
	retracto por deformaci\'{o}n fuerte de $\geom{s}\times\intervalo$.
\end{lemaRetractoPorDeformacionSimplices}

\begin{proof}
	Si $\dim\,s=0$, $\geom{s}$ consta de un \'{u}nico punto y
	$\geom{\carasp{s}}=\varnothing$, $\geom{s}\times\{0\}$ es un punto
	en el intervalo $\geom{s}\times\intervalo$. Si $\dim\,s>0$, se
	define una homotop\'{\i}a
	\begin{align*}
		& F\,:\,(\geom{s}\times\intervalo)\times\intervalo
			\,\rightarrow\,\geom{s}\times\intervalo
	\end{align*}
	%
	mediante una especie de proyecci\'{o}n estereogr\'{a}fica desde un
	punto imaginario por fuera del s\'{\i}mplice y por encima del
	baricentro, siguiendo los rayos que parten desde este punto hacia
	los lados de
	$(\geom{s}\times\{0\})\cup(\geom{\carasp{s}}\times\intervalo)$.
	La f\'{o}rmula es la siguiente:
	\begin{align*}
		F([\alpha,t],t',t'') & \,=\,
			\begin{cases}
				\Big(\left[\alpha,(1-t'')t+
					\frac{t''(2t-t')}{2-t'}\right],
				(1-t'')t'\Big) & t'\leq 2t \\[10pt]
				\Big(\left[\alpha,(1-t'')t\right],
				(1-t'')t'+\frac{t''(t'-2t)}{1-t}\Big) &
					t'\geq 2t
			\end{cases}
			\text{ .}
	\end{align*}
	%
\end{proof}

\begin{coroRetractoPorDeformacionComplejos}%
	\label{thm:retractopordeformacioncomplejos}
	Sea $K$ un complejo simplicial y sea $L\subset K$ un subcomplejo.
	Entonces el subespacio
	$(\geom{K}\times\{0\})\cup(\geom{L}\times\intervalo)$ es un
	retracto por deformaci\'{o}n fuerte de $\geom{K}\times\intervalo$.
\end{coroRetractoPorDeformacionComplejos}

\begin{proof}
	Para cada $n\geq -1$, definimos
	\begin{align*}
		X^{n} & \,=\,(\geom{K}\times\{0\})\,\cup\,
			(\geom{\qesq{n}{K}\cup L}\times\intervalo)
		\text{ .}
	\end{align*}
	%
	La demostraci\'{o}n se divide en dos partes: primero deformar
	$X^{n}$ en $X^{n-1}$ para $n\geq 0$ y luego pegar, concatenar
	adecuadamente las deformaciones. Notemos que
	\begin{align*}
		X^{-1} & \,=\,(\geom{K}\times\{0\})\,\cup\,
			(\geom{L}\times\intervalo)\quad\text{y} \\
		\geom{K}\times I & \,=\,\bigcup_{n\geq -1}\,X^{-1}
		\text{ .}
	\end{align*}
	%

	Sea $n\geq 0$ y sea $s\in\qesq{n}{K}\setmin L$. Por el lema
	\ref{thm:retractopordeformacionsimplices}, existe una retracci\'{o}n
	por deformaci\'{o}n fuerte
	\begin{align*}
		& F_{s}\,:\,\geom{s}\times\intervalo\times\intervalo
			\,\rightarrow\,
			\geom{s}\times\intervalo
	\end{align*}
	%
	de $\geom{s}\times\intervalo$ en
	\begin{math}
		(\geom{s}\times\{0\})\cup
			(\geom{\carasp{s}}\times\intervalo)
	\end{math}~.
	Definimos $F_{n}:\,X^{n}\times\intervalo\rightarrow X^{n}$ por
	\begin{align*}
		F_{n}|_{\geom{s}\times\intervalo\times\intervalo} & \,=\,F_{s}
			\qquad\text{si }s\in\qesq{K}\setmin L\quad\text{y} \\
		F_{n}(x,t) & \,=\,x\qquad\text{si }
			x\in X^{n-1},\,t\in\intervalo
		\text{ .}
	\end{align*}
	%
	Entonces $F_{n}$ es una retracci\'{o}n por deformaci\'{o}n fuerte
	de $X^{n}$ en $X^{n-1}$. Sea $f_{n}:\,X^{n}\rightarrow X^{n-1}$
	la retracci\'{o}n correspondiente $f_{n}(x)=F_{n}(x,1)$.

	Sea $a_{n}=\frac{1}{n}$ ($n\geq 1$). Para $n\geq 0$, sea
	$G_{n}:\,X^{n}\times\intervalo\rightarrow X^{n}$ la funci\'{o}n
	dada por
	\begin{align*}
		G_{0}(x,t) & \,=\,
			\begin{cases}
				x &\quad\text{si }t\leq a_{2} \\
				F_{0}\big(x,\frac{t-a_{2}}{1-a_{2}}\big) &
					\quad\text{si }t\geq a_{2}
			\end{cases}
		\text{ ,}
	\end{align*}
	%
	si $n=0$ y, para $n\geq 1$,
	\begin{align*}
		G_{n}(x,t) & \,=\,
			\begin{cases}
				x & \quad t\leq a_{n+2} \\[10pt]
				F_{n}\big(x,\frac{t-a_{n+2}}{a_{n+1}-a_{n+2}}
					\big) & \quad
				a_{n+2}\leq t\leq a_{n+1} \\[10pt]
				G_{n-1}(f_{n}(x),t) & \quad\geq a_{n+1}
			\end{cases}
		\text{ ,}
	\end{align*}
	%
	definida inductivamente. Para cada $n\geq 0$, la funci\'{o}n
	$G_{n}$ es una retracci\'{o}n por deformaci\'{o}n fuerte de $X^{n}$
	en $X^{n-1}$ y cumple que
	\begin{align*}
		G_{n}|_{X^{n-1}\times\intervalo} & \,=\,G_{n-1}
		\text{ .}
	\end{align*}
	%
	Entonces queda definida una funci\'{o}n
	\begin{align*}
		& G\,:\,\geom{K}\times\intervalo\times\intervalo
			\,\rightarrow\,\geom{K}\times\intervalo
	\end{align*}
	%
	tal que $G|_{X^{n}\times\intervalo}=G_{n}$. Esta funci\'{o}n es una
	retracci\'{o}n por deformaci\'{o}n fuerte de $\geom{K}\times\intervalo$
	en $(\geom{K}\times\{0\})\cup(\geom{L}\times\intervalo)$.
\end{proof}

\begin{coroExtensionDeHomotopiasEnSubcomplejos}%
	\label{thm:extensiondehomotopiasensubcomplejos}
	Sea $L\subset K$ un subcomplejo. El par $(\geom{K},\geom{L})$
	(con la inclusi\'{o}n) tiene la propiedad de extensi\'{o}n de
	homotop\'{\i}as respecto de cualquier espacio.
\end{coroExtensionDeHomotopiasEnSubcomplejos}

\begin{proof}
	Sea $g:\,\geom{K}\rightarrow Y$ una funci\'{o}n continua y sea
	$G:\,\geom{L}\times\intervalo\rightarrow Y$ tal que
	$G(\alpha,0)=g(\alpha)$, si $\alpha\in\geom{L}$. Sea
	$f:\,(\geom{K}\times\{0\})\cup(\geom{L}\times\intervalo)\rightarrow Y$
	la funci\'{o}n
	\begin{align*}
		f(\alpha,0) & \,=\,g(\alpha)
			\qquad\text{si }\alpha\in\geom{K} \\
		f(\alpha,t) & \,=\,G(\alpha,t)
			\qquad\text{si }\alpha\in\geom{L},\,t\in\intervalo
		\text{ .}
	\end{align*}
	%
	Esta funci\'{o}n es continua. Sea $H$ la funci\'{o}n del
	corolario \ref{thm:retractopordeformacioncomplejos} y sea
	\begin{align*}
		& r\,:\,\geom{K}\times\intervalo\,\rightarrow\,
			(\geom{K}\times\{0\})\,\cup\,(\geom{L}\times\intervalo)
	\end{align*}
	%
	la retracci\'{o}n fuerte $r(x,t)=H((x,t),1)$. Entonces
	$F=f\circ r$ es una extensi\'{o}n de $f$ y verifica
	\begin{align*}
		F(\alpha,0) & \,=\,g(\alpha)
			\qquad\text{si }\alpha\in\geom{K} \\
		F|_{\geom{L}\times\intervalo} & \,=\,G
		\text{ .}
	\end{align*}
	%
\end{proof}

%
\section{Realizaci\'{o}n geom\'{e}trica}
\theoremstyle{plain}
\newtheorem{lemaInmersionDeUnSimplice}{Lema}[section]
\newtheorem{lemaLocalmenteFinitoLocalmenteFinito}[lemaInmersionDeUnSimplice]%
	{Lema}
\newtheorem{teoEquivalenciasLocalmenteFinito}[lemaInmersionDeUnSimplice]%
	{Teorema}
\newtheorem{teoInmersionDeUnComplejo}[lemaInmersionDeUnSimplice]{Teorema}

\theoremstyle{remark}

%-------------

\begin{lemaInmersionDeUnSimplice}\label{thm:inmersiondeunsimplice}
	Una funci\'{o}n lineal $f:\,\geom{s}\rightarrow\bb{R}^{n}$ es una
	inmersi\'{o}n si y s\'{o}lo si el conjunto $\{f(v)\}_{v\in s}$
	es un conjunto en posici\'{o}n general, af\'{\i}nmente independiente.
\end{lemaInmersionDeUnSimplice}

\begin{proof}
	Sea $s=\{\lista[0]{v}{q}$ y sea $p_{i}=f(\alpha_{v_{i}})$ el punto
	correspondiente al v\'{e}rtice $v_{i}$. Supongamos que existen
	n\'{u}meros reales $\lista*[0]{t}{q}$ no todos nulos tales que
	$\sum_{i=0}^{q}\,t^{i}p_{i}=0$ y $\sum_{i=0}^{q}\,t^{i}=0$ y
	supongamos, sin p\'{e}rdida de generalidad, que $t^{i}\geq 0$ para
	$i\geq i_{0}$ y $t^{i}<0$, si $i<i_{0}$. Entonces
	\begin{align*}
		p & \,:=\,\sum_{i<i_{0}}\,(-t^{i})\,p_{i} \,=\,
			\sum_{i\geq i_{0}}\,t^{i}\,p_{i} \quad\text{y} \\
		a & \,:=\,\sum_{i<i_{0}}\,(-t^{i}) \,=\,
			\sum_{i\geq i_{0}}\,t^{i}
		\text{ .}
	\end{align*}
	%
	Sean $\alpha,\beta\in\geom{s}$ los puntos dados por
	\begin{align*}
		\alpha & \,=\,\sum_{i<i_{0}}\,\frac{-t^{i}}{a}\,
				\alpha_{v_{i}} \quad\text{y} \\
		\beta & \,=\,\sum_{i\geq i_{0}}\,\frac{t^{i}}{a}\,
				\alpha_{v_{i}}
		\text{ .}
	\end{align*}
	%
	Entonces $f(\alpha)=f(\beta)$, por linealidad, y $f$ no es inyectiva.
	Rec\'{\i}procamoente, si $\alpha,\beta\in\geom{s}$,
	$\alpha\not=\beta$ y $f(\alpha)=f(\beta)$, supongamos que
	$\alpha^{i_{0}}\not=\beta^{i_{0}}$. Entonces
	\begin{align*}
		\sum_{i=0}^{q}\,\alpha^{i}\,p_{i} & \,=\,f(\alpha) \,=\,
			f(\beta) \,=\,\sum_{i=0}^{q}\,\beta^{i}\,p_{i}
			\quad\text{y} \\
		\sum_{i=0}^{q}\,\alpha^{i} & \,=\,1\,=\,
			\sum_{i=0}^{q}\,\beta^{i}
		\text{ .}
	\end{align*}
	%
	Tomando la diferncia, los coeficientes $\alpha^{i}-\beta^{i}$ no son
	todos nulos y
	\begin{align*}
		\sum_{i=0}^{q}\,(\alpha^{i}-\beta^{i})\,p_{i} & \,=\,0
			\quad\text{y} \\
		\sum_{i=0}^{q}\,(\alpha^{i}-\beta^{i}) & \,=\,0
		\text{ ,}
	\end{align*}
	%
	con lo cual, los puntos $\{\lista[0]{p}{q}\}$ no est\'{a}n en
	posici\'{o}n general.
\end{proof}

Sea $\alpha\in\geom{K}$. Entonces $\alpha\in\simpinterior{s}$ para
alg\'{u}n s\'{\i}mplice $s\in K$ y $\alpha\estrella{v}$ para alg\'{u}n
v\'{e}rtice $v$ de $K$. Este v\'{e}rtice, si $K$ es un complejo localmente
finito, pertenece a finitos s\'{\i}mplices. Sea $C=\bigcup_{v\in s}\,\geom{s}$.
Entonces $\alpha\in C$ y $C$ es una uni\'{o}n finita de subconjuntos
compactos de $\geom{K}_{d}$ y, por lo tanto, compacto. Si
$L=\bigcup_{v\in s}\,\caras{s}$, entonces $L$ es un subcomplejo finito de $K$
y $\alpha\in\estrella{v}\subset\geom{L}_{d}$. Esto demuestra el siguiente lema.
(Notemos que tambi\'{e}n podemos incluir $C$ en la uni\'{o}n de una cantidad
finita de s\'{\i}mplices ``abiertos'' $\simpinterior{s}$).

\begin{lemaLocalmenteFinitoLocalmenteFinito}
	\label{thm:localmentefinitolocalmentefinito}
	Sea $K$ un complejo simplicial localmente finito. Todo punto
	$\alpha\in \geom{K}_{d}$ tiene un entorno de la forma
	$\geom{L}_{d}$ para alg\'{u}n subcomplejo finito $L\subset K$.
	Es decir, $\alpha\in U$ para cierto abierto $U\subset\geom{L}_{d}$.
\end{lemaLocalmenteFinitoLocalmenteFinito}

\begin{teoEquivalenciasLocalmenteFinito}%
	\label{thm:equivalenciaslocalmentefinito}
	Sea $K$ un complejo simplicial. Las siguientes afirmaciones son
	equivalentes:
	\begin{itemize}
		\item[(a)] $K$ es localmente finito;
		\item[(b)] $\geom{K}$ es localmente compacto;
		\item[(c)] la identidad $\geom{K}\rightarrow\geom{K}_{d}$ es
			un homeomorfismo;
		\item[(d)] $\geom{K}$ es metrizable;
		\item[(e)] $\geom{K}$ verifica el primer axioma de
			numerabilidad: todo punto posee una base numerable de
			entornos.
	\end{itemize}
	%
\end{teoEquivalenciasLocalmenteFinito}

\begin{proof}
	\emph{(b) implica (c):} Sea $U\subset\geom{K}$ un abierto con clausura
	$\clos{U}$ compacta ($\geom{K}$ es Hausdorff). Como $\clos{U}$ es
	compacta, existe una cantidad finita de s\'{\i}mplices $s$
	tales que $\clos{U}$ est\'{a} contenida en la uni\'{o}n de los finitos
	s\'{\i}mplices abiertos $\simpinterior{s}$, por el corolario
	\ref{thm:compactocontenidoenfinitossimplices}. La uni\'{o}n de estos
	s\'{\i}mplices forma un subcomplejo finito $L\subset K$ y
	$\clos{U}\subset\geom{L}$. Como $L$ es finito, la identidad
	$\geom{L}\rightarrow\geom{L}_{d}$ es un homeomorfismo. Por lo tanto,
	$U$ es abierto en $\geom{L}_{d}$. Sea $K_{1}\subset K$ el subcomplejo
	\begin{align*}
		K_{1} &\,=\,\big\{ s\in K\,:\,U\cap \geom{s}=\varnothing\big\}
		\text{ .}
	\end{align*}
	%
	Como $U\subset\geom{K}$ es abierto (con la topolog\'{\i}a coherente),
	\begin{align*}
		U\,\cap\,\geom{s}\,=\,\varnothing & \quad\Leftrightarrow\quad
			U\,\cap\,\simpinterior{s}\,=\,\varnothing
		\text{ ,}
	\end{align*}
	%
	con lo cual, si $s\in K\setmin K_{1}$, entonces
	$\simpinterior{s}\cap\geom{L}\not=\varnothing$. En particular,
	si $s\in K\setmin K_{1}$, existe $\alpha\in\geom{K}$ tal que
	\begin{align*}
		\alpha\,\in\,\simpinterior{s}\,\cap\,\geom{L}
		\text{ .}
	\end{align*}
	%
	Pero $\alpha\in\geom{L}$ quiere decir que $\{\alpha\not=0\}\in L$
	y $\alpha\in\simpinterior{s}$ quiere decir que
	$\{\alpha\not=0\}=s$. En definitiva, tenemos una descomposici\'{o}n
	del complejo $K$ como uni\'{o}n de los dos subcomplejos:
	\begin{align*}
		K & \,=\, K_{1}\,\cup\,L
		\text{ .}
	\end{align*}
	%
	De esta descomposici\'{o}n se deduce la descomposici\'{o}n
	\begin{align*}
		\geom{K}_{d} & \,=\,\geom{K_{1}}_{d}\,\cup\,
					\geom{L}_{d}
		\text{ .}
	\end{align*}
	%
	En particular,
	\begin{math}
		\geom{K}_{d}\setmin\geom{K_{1}}_{d}=
			\geom{L}_{d}\setmin\geom{K_{1}}_{d}
	\end{math}~.
	Como $U\subset\geom{L}_{d}$ es abierto y $U$ est\'{a} contenido en
	$\geom{L}_{d}\setmin\geom{K_{1}}_{d}$, se deduce que $U$ es abierto
	en $\geom{K}_{d}\setmin\geom{K_{1}}_{d}$. Pero
	$\geom{K_{1}}_{d}$ es cerrado en $\geom{K}_{d}$ y, en consecuencia,
	$U$ es abierto en $\geom{K}_{d}$.

	\emph{(e) implica (a): } si $K$ no es localmente finito, existe
	alg\'{u}n v\'{e}rtice $v\in K$ que pertenece a todos los
	s\'{\i}mplices de una familia infinita numerable de s\'{\i}mplices de
	$K$, $\{s_{i}\}_{i\geq 1}$. Supongamos que dicho v\'{e}rtice $v$
	admite una base numerable de entornos (del punto correspondiente
	$\alpha_{v}\in\geom{K}$), $\{U_{i}\}_{i\geq 1}$ en $\geom{K}$
	(topolog\'{\i}a coherente), y supongamos, sin p\'{e}rdida de
	generalidad, que $U_{i}\supset U_{i+1}$ para todo $i\geq 1$.
	Para cada $i\geq 1$, como $\geom{s_{i}}\subset\geom{K}$ es cerrado,
	$\clos{\simpinterior{s_{i}}}=\geom{s_{i}}$. Como
	$\alpha_{v}\in\geom{s_{i}}$ y $U_{j}$ es un abierto que contiene al
	punto $\alpha_{v}$, se deduce que existe
	\begin{align*}
		& \alpha_{i}\,\in\,\simpinterior{s_{i}}\,\cap\,U_{i}
		\text{ .}
	\end{align*}
	%
	Como los abiertos $U_{i}$ constituyen una base de entornos de
	$\alpha_{v}$ y $U_{i}\supset U_{i+1}$ para todo $i$, la sucesi\'{o}n
	$\{\alpha_{i}\}_{i\geq 1}$ tiende a $\alpha_{v}$ en la topolog\'{\i}a
	de $\geom{K}$. Como $\alpha_{i}\in\simpinterior{s_{i}}$, en
	particular $\alpha_{i}\not=\alpha_{v}$ para todo $i$, pero
	tambi\'{e}n, dado un s\'{\i}mplice arbitrario $s\in K$, la
	intersecci\'{o}n $\geom{s}\cap\{\alpha_{i}\}_{i\geq 1}$ es a lo sumo
	un conjunto finito. Como en el comentario previo al lema
	\ref{thm:compactocontenidoenfinitossimplices}, esto implica que
	$\{\alpha_{i}\}_{\i\geq 1}$ es discreto en $\geom{K}$, lo que
	contradice el hecho de que tenga un punto de acumulaci\'{o}n.
\end{proof}

Una \emph{realizaci\'{o}n} de un complejo simplicial $K$ en un
$\bb{R}$-espacio vectorial $E$ (cuya topolog\'{\i}a se caracteriza por
ser coherente con los subespacios de dimensi\'{o}n finita) es una
inmersi\'{o}n lineal $\geom{K}\rightarrow\bb{R}^{n}$.

\begin{teoInmersionDeUnComplejo}\label{thm:inmersiondeuncomplejo}
	Sea $K$ un complejo simplicial. Si $K$ admite una realizaci\'{o}n
	$\geom{K}\rightarrow\bb{R}^{n}$ en un espacio de dimensi\'{o}n
	finita, entonces $K$ es a lo sumo numerable, localmente finito y
	$\dim\,K\leq n$. Rec\'{\i}procamente, si $K$ es numerable, localmente
	finito y $\dim\,K\leq n$, entonces $K$ admite una realizaci\'{o}n
	en $\bb{R}^{2n+1}$.
\end{teoInmersionDeUnComplejo}

%
\section{Subdivisiones}
\theoremstyle{plain}
\newtheorem{lemaConjuntoDirigidoSubdivisiones}{Lema}[section]
\newtheorem{lemaSubdivisionEquivaleAParticion}%
	[lemaConjuntoDirigidoSubdivisiones]{Lema}

\theoremstyle{remark}
\newtheorem{obsDefinicionSubdivisiones}[lemaConjuntoDirigidoSubdivisiones]%
	{Observaci\'{o}n}
\newtheorem{obsSubdividirUnSubcomplejo}[lemaConjuntoDirigidoSubdivisiones]%
	{Observaci\'{o}n}
\newtheorem{obsSubdividirUnParSimplicial}[lemaConjuntoDirigidoSubdivisiones]%
	{Observaci\'{o}n}
\newtheorem{obsTriangularPorSubdivisiones}[lemaConjuntoDirigidoSubdivisiones]%
	{Observaci\'{o}n}
\newtheorem{obsLinealEnLaSubdivision}[lemaConjuntoDirigidoSubdivisiones]%
	{Observaci\'{o}n}
\newtheorem{obsSubdivisionesYEstrellas}[lemaConjuntoDirigidoSubdivisiones]%
	{Observaci\'{o}n}

%-------------

Una \emph{subdivisi\'{o}n} de un complejo simplicial $K$ es un complejo
simplicial $K'$ con las siguientes tres propiedades:
\begin{itemize}
	\item[(i)] los v\'{e}rtices de $K'$ son puntos de $\geom{K}$,
	\item[(ii)] si $s'\in K'$ es un s\'{\i}mplice, entonces existe un
		s\'{\i}mplice $s\in K$ tal que $s'\subset\geom{s}$ y
	\item[(iii)] la funci\'{o}n lineal $\geom{K'}\rightarrow\geom{K}$
		que se obtiene a partir de la aplicaci\'{o}n
		$\alpha_{v'}\mapsto v'$	definida en los puntos
		correspondientes a los v\'{e}rtices de $K'$ es un
		homeomorfismo.
\end{itemize}
%
\begin{obsLinealEnLaSubdivision}\label{obs:linealenlasubdivision}
	Sea $K$ un complejo simplicial y sea $\Phi:\,\geom{K}\rightarrow X$
	una funci\'{o}n lineal. Si $K'$ es una subdivisi\'{o}n de $K$,
	entonces $\Phi$ tambi\'{e}n es lineal en $K'$
	($\Phi\circ f:\,\geom{K'}\rightarrow X$ es lineal).
\end{obsLinealEnLaSubdivision}

\begin{lemaConjuntoDirigidoSubdivisiones}%
	\label{thm:conjuntodirigidosubdivisiones}
	Sea $K$ un complejo simplicial. Toda subdivisi\'{o}n de una
	subdivisi\'{o}n de $K$ es una subdvisi\'{o}n de $K$ y, dadas dos
	subdivisiones $K'$ y $K''$ de $K$ existe una subdivisi\'{o}n de $K$
	que es subdivisi\'{o}n de $K'$ y de $K''$.
\end{lemaConjuntoDirigidoSubdivisiones}

\begin{obsDefinicionSubdivisiones}\label{obs:definicionsubdivisiones}
	Sean $K,K'$ complejos simpliciales que verifican las condiciones
	\emph{(i)} y \emph{(ii)} de la definici\'{o}n de subdivisi\'{o}n.
	Entonces la aplicaci\'{o}n
	$\alpha_{v'}\in\geom{K'}\mapsto v'\in\geom{K}$ determina una
	funci\'{o}n lineal $f:\,\geom{K'}\rightarrow\geom{K}$: si $s'\in K'$ es
	un s\'{\i}mplice y $s\in K$ es tal que $s'\subset\geom{s}$, entonces
	definimos $f_{s',s}:\,\geom{s'}\rightarrow\geom{s}$ por
	\begin{align*}
		f_{s'}\Big(\sum_{v'\in s'}\,t^{v'}\,\alpha_{v'}\Big) & \,=\,
			\sum_{v'\in s'}\,t^{v'}\,v'
		\text{ .}
	\end{align*}
	%
	Como $s'\subset\geom{s}$, la funci\'{o}n $f_{s',s}$ est\'{a} bien
	definida. La funci\'{o}n $f_{s',s}$ proyecta el s\'{\i}mplice
	$\geom{s'}$ sobre la c\'{a}scara convexa generada por las im\'{a}genes
	de sus v\'{e}rtices en $\geom{s}$. Si el s\'{\i}mplice $s$ est\'{a}
	contenido en alg\'{u}n s\'{\i}mplice $s_{1}\in K$, entonces la
	inclusi\'{o}n $s\subset s_{1}$ determina una inclusi\'{o}n
	$\inc[s,s_{1}]:\,\geom{s}\subset\geom{s_{1}}$ y vale que
	\begin{align*}
		\inc[s,s_{1}]\circ f_{s',s}(\alpha) & \,=\,
			f_{s',s_{1}}(\alpha)
	\end{align*}
	%
	para todo $\alpha\in\geom{s'}$, por unicidad. En particular,
	queda determinada una funci\'{o}n lineal y continua
	$f_{s'}:\,\geom{s'}\rightarrow\geom{K}$. Si $s''\in K'$ es otro
	s\'{\i}mplice y $s'\cap s''\not=\varnothing$, entonces, nuevamente
	porque las funciones lineales est\'{a}n determinadas por su valor
	en los v\'{e}rtices,
	\begin{align*}
		f_{s'}|_{\geom{s'\cap s''}} & \,=\,f_{s''}|_{\geom{s'\cap s''}}
		\text{ .}
	\end{align*}
	%
	En definitiva, queda determinada una funci\'{o}n lineal y continua en
	todo el espacio $\geom{K'}$ del complejo.

	Esta funci\'{o}n no es, en general, inyectiva y, por lo tanto, no es
	posible identificar $\geom{K'}$ con un subespacio de $\geom{K}$ (no
	estamos asumiendo la propiedad \emph{(iii)}, aun). La inyectividad
	puede fallar por distintas razones: por ejemplo, si
	$s',s''\in K'$ y $s\in K$ es tal que $s',s''\subset\geom{s}$
	las c\'{a}scaras convexas de $s'$ y de $s''$ en $\geom{s}$ pueden
	solaparse de manera tal que su intersecci\'{o}n no sea la
	c\'{a}scara convexa generada por ning\'{u}n s\'{\i}mplice de $K'$;
	otra posibilidad es que las funciones parciales
	$f_{s'}:\,\geom{s'}\rightarrow\geom{K}$ reduzcan la dimensi\'{o}n de
	los s\'{\i}mpices en donde est\'{a}n definidas, es decir,
	la imagen de los v\'{e}rtices de $\geom{s'}$ puede ser un conjunto
	af\'{\i}nmente dependiente de $\geom{s}$, los v\'{e}rtices de
	$s'\subset\geom{s}$ pueden no estar en posici\'{o}n general.

	A pesar de la posible falla de la inyetividad de
	$f:\,\geom{K'}\rightarrow\geom{K}$, todo s\'{\i}mplice $s'\in K'$
	est\'{a} incluido en alguno de los espacios $\geom{s}$ y, por
	finitud de $s$, existe un s\'{\i}mplice $s\in K$ m\'{a}s chico
	tal que $\geom{s}\supset s'$ (y dicho s\'{\i}mplice es \'{u}nico).
	La c\'{a}scara convexa $f(\geom{s'})$ de $s'$ en $\geom{s}$ es
	cerrada. Por otro lado, como $s'$ es un subconjunto finito arbitrario
	de $\geom{s}$, la c\'{a}scara convexa puede estar contenida en el
	interior $\simpinterior{s}$, o bien intersecar alguna cara propia de
	$s$ y, en tal caso, la intersecci\'{o}n puede ser total o parcial. En
	todo caso, si $\beta\in\simpinterior{s'}$ entonces se cumple que
	$f(\beta)\in\simpinterior{s}$.

	Demostremos esta \'{u}ltima afirmaci\'{o}n. Si $v'\in s'$, como
	$s'\subset\geom{s}$, entonces
	\begin{align*}
		v' & \,=\,\sum_{v\in s}\,t_{v'}^{v}\,\alpha_{v}
	\end{align*}
	%
	y, como $s$ es el s\'{\i}mplice m\'{a}s chico tal que
	$s'\subset\geom{s}$, entonces, para todo $v\in s$, existe alg\'{u}n
	$v'\in s'$ tal que $t_{v'}^{v}\not=0$ (si no el v\'{e}rtice $v\in s$
	se podr\'{\i}a omitir). Si $\beta\in\simpinterior{s'}$, entonces
	\begin{align*}
		\beta & \,=\,\sum_{v'\in s'}\,\beta^{v'}\,\beta_{v'}
			\quad\text{y} \\
		f(\beta) & \,=\,\sum_{v'\in s'}\,\beta^{v'}\,f(\beta_{v'})
			\,=\,\sum_{v'\in s'}\,\beta^{v'}\,
				\sum_{v\in s}\,t_{v'}^{v}\,\alpha_{v} \\
		& \,=\,\sum_{v\in s}\,
			\Big(\sum_{v'\in s'}\,\beta^{v'}t_{v'}^{v}\Big)\,
				\alpha_{v}
		\text{ .}
	\end{align*}
	%
	Como $\beta^{v'}>0$ para todo $v'$ y, para cada $v$ existe
	$t_{v'}^{v}>0$, vale que los coeficientes
	$\sum_{v'\in s'}\,\beta^{v'}t_{v'}^{v}$ son todos positivos.
	Por lo tanto, $f(\beta)\in\simpinterior{s}$.
\end{obsDefinicionSubdivisiones}

\begin{lemaSubdivisionEquivaleAParticion}%
	\label{thm:subdivisionequivaleaparticion}
	Sean $K',K$ complejos simpliciales que verifican las condiciones
	\emph{(i)} y \emph{(ii)} de la definici\'{o}n de subdivisi\'{o}n.
	Entonces $K'$ es una subdivisi\'{o}n de $K$, si y s\'{o}lo si,
	\emph{(iii')} para todo s\'{\i}mplice $s\in K$, el conjunto
	\begin{align*}
		& \big\{f\big(\simpinterior{s'}\big)\,:\,
			s'\in K',\,f\big(\simpinterior{s'}\big)\subset
				\simpinterior{s}\big\}
	\end{align*}
	%
	es una partici\'{o}n finita de $\simpinterior{s}$.
	% partici\'{o}n en el sentido de que la uni\'{o}n es todo y la
	% intersecci\'{o}n de dos elementos es vac\'{\i}a
\end{lemaSubdivisionEquivaleAParticion}

\begin{proof}
	Si $s\in K$, sea $K'(s)\subset K'$ el subconjunto
	\begin{align*}
		K'(s) & \,=\,\bigcup_{s_{1}\in\caras{s}}\,
			\big\{s'\in K'\,:\,f\big(\simpinterior{s'}\big)\subset
				\simpinterior{s_{1}}\big\}
		\text{ .}
	\end{align*}
	%
	Asumamos que se verifica \emph{(i)} y definamos el subconjunto
	$V'(s) =\big\{v'\in K'\,:\,v'\in\geom{s}\big\}$ de los v\'{e}rtices de
	$K'$. Entonces $\{v'\}\in K'(s)$, para todo $v'\in V'$.
	Sean $s'\in K'(s)$ y $s''\subset s'$.  Como $K'$ es un complejo,
	$s''\in K'$. Como el conjunto $f\big(\simpinterior{s'}\big)$ est\'{a}
	contenido en $\simpinterior{s_{1}}$ para alguna cara $s_{1}\subset s$,
	los v\'{e}rtices de $s'$ est\'{a}n contenidos en $\geom{s_{1}}$ y,
	en particular, en $s$. Como $s''$ es un subconjunto de estos
	v\'{e}rtices, $s''\subset\geom{s}$, tambi\'{e}n. En particular,
	$s''\subset\geom{s_{2}}$ para alguna cara $s_{2}$ de $s$. Tomando
	$s_{2}$ como la cara m\'{a}s chica con esta propiedad, se deduce,
	por lo visto en la observaci\'{o}n \ref{obs:definicionsubdivisiones},
	que $f\big(\simpinterior{s''}\big)\subset\simpinterior{s_{2}}$ y, por
	lo tanto, $s''\in K'(s)$. En definitiva, $K'(s)$ es un subcomplejo
	de $K'$ cuyo conjunto de v\'{e}rtices es $V'(s)$.

	Supongamos que se cumplen \emph{(i)} y que el conjunto de la
	condici\'{o}n \emph{(iii')} es finito. Entonces dado $s\in K$, el
	subcomplejo $K'(s)\subset K'$ es un complejo finito. Si, adem\'{a}s,
	suponemos que el conjunto de \emph{(iii')} es una partici\'{o}n,
	entonces, la funci\'{o}n lineal
	$h_{s}:\,\geom{K'(s)}\rightarrow\geom{s}$ dada en v\'{e}rtices
	por $h_{s}(\alpha_{v'})=v'\in\geom{s}$ es continua y biyectiva entre
	espacios compactos Hausdorff. En particular, las condiciones
	\emph{(i)} y \emph{(iii')} implican que $h_{s}$ es un homeomorfismo.
	La condici\'{o}n \emph{(ii)} implica que todo s\'{\i}mplice de $K'$
	pertence a alguno de los subcomplejos $K'(s)$. Notemos que
	\begin{align*}
		h_{s} & \,=\,f|_{\geom{K'(s)}}
		\text{ .}
	\end{align*}
	%
	Por lo tanto, \emph{(i)}, \emph{(ii)} y \emph{(iii')} implican que
	$f:\,\geom{K'}\rightarrow\geom{K}$ tiene una inversa continua dada
	por $f^{-1}|_{\geom{s}}=h_{s}^{-1}$. En definitiva, $K'$ es una
	subdivisi\'{o}n de $K$.

	Rec\'{\i}procamente, la familia
	$\big\{\simpinterior{s'}\,:\,s'\in K'\big\}$ particiona al espacio
	$\geom{K'}$. Sea $s\in K$ y sea $s'\in K'$. Entonces, por la
	observaci\'{o}n \ref{obs:definicionsubdivisiones}, o bien
	$f\big(\simpinterior{s'}\big)\cap\simpinterior{s}=\varnothing$, o bien
	$f\big(\simpinterior{s'}\big)\subset\simpinterior{s}$. Si
	asumimos \emph{(iii)}, que $f:\,\geom{K'}\rightarrow\geom{K}$ es un
	homeomorfismo, entonces el conjunto
	\begin{align*}
		& \big\{f\big(\simpinterior{s'}\big)\,:\,
			s'\in K',\,f\big(\simpinterior{s'}\big)\subset
				\simpinterior{s}\big\}
	\end{align*}
	%
	es una partici\'{o}n de $\simpinterior{s}$. Pero, como $\geom{s}$
	es compacto en $\geom{K'}$ (v\'{\i}a el homeomorfismo $f$), el
	lema \ref{thm:compactocontenidoenfinitossimplices} implica que
	$\geom{s}$ est\'{a} contenido en una uni\'{o}n finita de conjuntos de
	la forma $\simpinterior{s'}$ con $s'\in K'$ y, por lo tanto, la
	partici\'{o}n de $\simpinterior{s}$ debe ser finita.
\end{proof}

\begin{obsSubdividirUnSubcomplejo}\label{obs:subdividirunsubcomplejo}
	Si $L\subset K$ es un subcomplejo y $K'$ es una subdivisi\'{o}n de $K$,
	entonces el subconjunto
	\begin{align*}
		L' & \,=\,\big\{s'\in K'\,:\,f\big(\simpinterior{s'}\big)
			\subset\geom{L}\big\}
	\end{align*}
	%
	es un subcomplejo de $K'$ y, si $t\in L$, entonces
	\begin{align*}
		\big\{f\big(\simpinterior{s'}\big)\,:\,s'\in K',\,
			f\big(\simpinterior{s'}\big)\subset\simpinterior{t}
				\big\} & \,=\,
		\big\{f\big(\simpinterior{s'}\big)\,:\,s'\in L',\,
			f\big(\simpinterior{s'}\big)\subset\simpinterior{t}
				\big\}
		\text{ .}
	\end{align*}
	%
	En particular, por el lema \ref{thm:subdivisionequivaleaparticion},
	$L'$ es una subdivisi\'{o}n de $L$. Dicho de otra manera, toda
	subdivisi\'{o}n de un complejo subdivide a todo subcomplejo
	tambi\'{e}n. Notemos que $L'$ es el \'{u}nico subcomplejo de $K'$ con
	esta propiedad. Lo llamamos \emph{la subdivisi\'{o}n inducida por %
	$K'$}.
\end{obsSubdividirUnSubcomplejo}

\begin{obsSubdividirUnParSimplicial}\label{obs:subdividirunparsimplicial}
	?`Vale m\'{a}s en general? Dada una transformaci\'{o}n simplicial
	$\varphi:\,L\rightarrow K$ y una subdivisi\'{o}n $K'$ de $K$,
	?`existe una subdivisi\'{o}n $L'$ de $L$ y una transformaci\'{o}n
	simplicial $\varphi':\,L'\rightarrow K'$ compatibles con la
	subdivisi\'{o}n $K'$ y la transformaci\'{o}n $\varphi$?
	La compatibilidad podr\'{\i}a estar dada ``naturalmente'' en
	t\'{e}rminos de las realizaciones geom\'{e}tricas de los complejos
	y de los morfismos, es decir, en t\'{e}rminos del funtor
	$\geom{\cdot}$. La pregunta es la siguiente: dados un par simplicial
	$\varphi:\,L\rightarrow K$ y una subdivisi\'{o}n $K'$ de $K$, ?`existe
	un par simplicial de la forma $\varphi':\,L'\rightarrow K'$ tal que
	$L'$ sea una subdivisi\'{o}n de $L$ y tal que
	\begin{center}
		\begin{tikzcd}
			\geom{L'} \arrow[r,"\geom{\varphi'}"] \arrow[d,"g"'] &
				\geom{K'} \arrow[d,"f"] \\
			\geom{L} \arrow[r,"\geom{\varphi}"'] & \geom{K}
		\end{tikzcd}
	\end{center}
	sea un diagrama conmutativo de espacios topol\'{o}gicos y funciones
	continuas? Las funciones $f:\,\geom{K'}\rightarrow\geom{K}$ y
	$g:\,\geom{L'}\rightarrow\geom{L}$ son los homeomorfismos asociados
	a las subdivisiones: est\'{a}n dadas en los v\'{e}rtices por
	$f(\alpha_{v'})=v'$ para todo v\'{e}rtice $v'$ de $K'$ y por
	$g(\beta_{w'})=w'$ para todo v\'{e}rtice $w'$ de $L'$.

	Supongamos que existe un par simplicial con estas caracter\'{\i}sticas
	y sean
	\begin{align*}
		W\,=\,\vertices{L} & \quad\text{,}\quad W'\,=\,\vertices{L'}
			\text{ ,} \\
		V\,=\,\vertices{K} & \quad\text{,}\quad V'\,=\,\vertices{K'}
		\text{ .}
	\end{align*}
	%
	Entonces $W'\subset\geom{L}$ y
	$\varphi\big(W'\big)\subset V'\subset\geom{K}$. Si $w'\in W'$,
	entonces
	\begin{align*}
		f(\geom{\varphi'}\beta_{w'}) & \,=\,f(\alpha_{\varphi'w'})
			\,=\,\varphi'w'\quad\text{y} \\
		\geom{\varphi}\big(g(\beta_{w'})\big) & \,=\,
			\geom{\varphi}w'
		\text{ .}
	\end{align*}
	%
	Es decir, la transformaci\'{o}n simplicial $\varphi'$ verifica,
	en los v\'{e}rtices de $L'$, la igualdad
	\begin{align*}
		\varphi'w' & \,=\,\geom{\varphi}w'
		\text{ .}
	\end{align*}
	%
	Recordando que toda transformaci\'{o}n simplicial est\'{a} determinada
	por su valor en los v\'{e}rtices, se deduce que, fijada la
	subdivisi\'{o}n $L'$ (que, por definici\'{o}n, se construye con puntos
	de la realizaci\'{o}n $\geom{L}$), existe a lo sumo una
	transformaci\'{o}n simplicial $\varphi':\,L'\rightarrow K'$ tal que
	$f\circ\geom{\varphi'}=\geom{\varphi}\circ g$. Notemos que el conjunto
	$W'$ de v\'{e}rtices de $L'$ est\'{a} (casi) determinado por $\varphi$
	y por la subdivisi\'{o}n $K'$: si $w'\in W'$ entonces
	$\varphi(w')\in V'$ debe ser un v\'{e}rtice de la subdivisi\'{o}n $K'$.

	Otra observaci\'{o}n que puede llegar a ser importante es que la
	subdivisi\'{o}n $L'$ se puede definir en cada uno de los
	s\'{\i}mplices $t\in L$ y cada uno de los espacios $\geom{t}$ es
	compacto y homeomorfo a un s\'{\i}mplice de $\bb{R}^{q+1}$ para
	alg\'{u}n $q\geq 0$. En particular, si $v'\in V'$ es un v\'{e}rtice
	de $K'$, entonces
	\begin{math}
		\geom{t}\cap\big(\geom{\varphi}^{-1}\big(\{v'\}\big)\big)
	\end{math}
	es igual a un subespacio af\'{\i}n intersecado con $\geom{t}$ y,
	por lo tanto, algo que tiene ``forma de s\'{\i}mplice'': existe un
	conjunto finito $t'\subset\geom{t}$ tal que
	\begin{align*}
		\convexa{t'} & \,=\,\geom{t}\,\cap\,
			\big(\geom{\varphi}^{-1}\big(\{v'\}\big)\big)
	\end{align*}
	%
	y minimal con esta propiedad. Este conjunto $t'$ tiene necesariamente
	la propiedad de que, si llamamos $T'$ al complejo cuyos v\'{e}rtices
	son los puntos de $t'$ y cuyos s\'{\i}mplices son todos los
	subconjuntos no vac\'{\i}os de $t'$
	(es decir, $T'=\partes(t')\setmin\{\varnothing\}$), entonces $T'$,
	junto con la funci\'{o}n lineal que en los v\'{e}rtices (los puntos
	de $t'$) es la identidad, es una triangulaci\'{o}n de $\convexa{t'}$.
	Es de esperar que $L'$ tenga como v\'{e}rtices a la uni\'{o}n
	$\bigcup_{t\in L,v'\in V'}\,t'$. En caso de existir otros v\'{e}rtices
	en $L'$, \'{e}stos deber\'{a}n pertenecer a alguno de los conjuntos
	$\convexa{t'}$. Podr\'{\i}an existir distintas subdivisiones de $L$
	que cumplan lo pedido.

	Diremos que un par simplicial $\varphi':\,L'\rightarrow K'$
	subdivide un par simplicial $\varphi:\,L\rightarrow K$, si
	$L'$ es una subdivisi\'{o}n de $L$, $K'$ es una subdivisi\'{o}n de $K$
	y $f\circ\geom{\varphi'}=\geom{\varphi}\circ g$, donde $g$ y $f$
	son los hoemomorfismos asociados a las subdivisiones.

	Usando el lema \ref{thm:subdividirunsimplice}, se puede ver
	inductivamente que existe una subdivisi\'{o}n $L'$ de $L$, una
	subdivisi\'{o}n $K''$ de $K'$ y una transformaci\'{o}n simplicial
	$\varphi'':\,L'\rightarrow K''$ tales que $\varphi''$ subdivide a
	$\varphi$ (tal vez, eligiendo el punto $w_{0}$ en alguna de las
	``caras'' colapsadas de $L$ se pueda demostrar que
	\ref{thm:subdividirunsimplice} sigue valiendo y que no resulta
	necesario subdividir $K'$).
\end{obsSubdividirUnParSimplicial}

\begin{obsTriangularPorSubdivisiones}\label{obs:triangularporsubdivisiones}
	Sea $\varphi$ ($\varphi:\,L\rightarrow K$) una triangulaci\'{o}n de
	$h$ ($h:\,A\rightarrow X$) y sea
	$(\psi',\psi):\geom{\varphi}\rightarrow h$ el homeomorfismo de pares
	correspondiente. Si $\varphi'$ ($\varphi':\,L'\rightarrow K'$) es una
	subdivisi\'{o}n de $\varphi$, entonces $\varphi'$ es una
	triangulaci\'{o}n $h$, v\'{\i}a el homeomorfismo de pares
	$(\psi'\circ f,\psi\circ g)$.
	\begin{center}
		\begin{tikzcd}
			\geom{L'} \arrow[r,"\geom{\varphi'}"] \arrow[d,"g"'] &
				\geom{K'} \arrow[d, "f"] \\
			\geom{L} \arrow[r,"\geom{\varphi}"] \arrow[d,"\psi"'] &
				\geom{K} \arrow[d,"\psi'"] \\
			A \arrow[r,"h"'] & X
		\end{tikzcd}
	\end{center}
\end{obsTriangularPorSubdivisiones}

\begin{obsSubdivisionesYEstrellas}\label{obs:subdivisionesyestrellas}
	Sea $K'$ una subdivisi\'{o}n de un complejo simplicial $K$.
	Sean $v$ y $v'$ v\'{e}rtices de $K$ y de $K'$, respectivamente.
	Entonces $v'\in\estrella[K]{v}$ (estrella en $K$, pues $v$ podr\'{\i}a
	(deber\'{\i}a) ser un v\'{e}rtice de $K'$, tambi\'{e}n), si y s\'{o}lo
	si $\estrella[K']{v'}\subset\estrella[K]{v}$.
\end{obsSubdivisionesYEstrellas}

%
\section{La subdivisi\'{o}n baric\'{e}ntrica}
\theoremstyle{plain}
\newtheorem{lemaSubdividirUnSimplice}{Lema}[section]
\newtheorem{propoBaricentricaEsSubdivision}[lemaSubdividirUnSimplice]%
	{Porposici\'{o}n}
\newtheorem{propoSubcomplejoEsRetracto}[lemaSubdividirUnSimplice]%
	{Porposici\'{o}n}

\theoremstyle{remark}
\newtheorem{obsBaricentricaDeSubcomplejoEsPlena}[lemaSubdividirUnSimplice]%
	{Observaci\'{o}n}

%-------------

En esta secci\'{o}n introducimos una forma can\'{o}nica de subdividir un
complejo.

\begin{lemaSubdividirUnSimplice}\label{thm:subdividirunsimplice}
	Sea $s\in K$ un s\'{\i}mplice en un complejo $K$. Sea $K'$ una
	subdivisi\'{o}n del subcomplejo $\carasp{s}$ de caras propias de $s$.
	Entonces, dado un punto arbitrario $w_{0}\in\simpinterior{s}$,
	el complejo $K'*w_{0}$ (ver ejemplo \ref{ejemplo:sumadecomplejos}) es
	una subdivisi\'{o}n de $\caras{s}$.
\end{lemaSubdividirUnSimplice}

\begin{proof}
	El complejo $K'*w_{0}$ es la suma de los complejos $K'$ y
	$\{\{w_{0}\}\}$ (es decir, el complejo que tiene a $w_{0}$ como
	\'{u}nico v\'{e}rtice). Los s\'{\i}mplices de este compejo son de la
	forma: $s'\in K'$, el s\'{\i}mplice puntual $\{w_{0}\}$ o
	$s'\sqcup\{w_{0}\}$ con $s'\in K'$. Entonces se cumplen \emph{(i)}
	lso v\'{e}rtices de $K'* w_{0}=V'\sqcup\{w_{0}\}$ son puntos de
	$\geom{\caras{s}}$ y \emph{(ii)} los s\'{\i}mplices de $K'*w_{0}$
	est\'{a}n contenidos en la realizaci\'{o}n de alg\'{u}n s\'{\i}mplice
	de $\caras{s}$ (todos est\'{a}n contenidos en $\geom{s}$). Del lema
	\ref{thm:conosobrelascaraspropias} (o de su demostraci\'{o}n) se
	deduce que los puntos de la realizaci\'{o}n
	$\geom{s}=\geom{\caras{s}}$ o bien son iguales a $w_{0}$, o bien
	pertenecen a $\geom{\carasp{s}}$, o bien pertenecen a
	$\simpinterior{s'\sqcup\{w_{0}\}}$, para un \'{u}nico s\'{\i}mplice
	$s'\in K'$. Esto implica que \emph{(iii')} los s\'{\i}mplices
	abiertos de $K'*w_{0}$ determinan una partici\'{o}n finita de
	$\geom{s}$. Precisamente, si $s_{1}\in\caras{s}$, entonces el conjunto
	\begin{math}
		\big\{\simpinterior{t'}\,:\,t'\in K'*w_{0},\,
			\simpinterior{t'}\subset\simpinterior{s_{1}}\big\}
	\end{math}
	es una partici\'{o}n finita de $\simpinterior{s_{1}}$. Por el lema
	\ref{thm:subdivisionequivaleaparticion}, $K'*w_{0}$ es una
	subdivisi\'{o}n de $\caras{s}$.
\end{proof}

Sea $K$ un complejo simplicial y sea $\subdiv{K}$ la colecci\'{o}n de
conjuntos finitos y no vac\'{\i}os de baricentros de s\'{\i}mplices de $K$
que est\'{a}n totalmente ordenados por inclusi\'{o}n de una cara en un
s\'{\i}mplice m\'{a}s grande. Es decir, un elemento de $\subdiv{K}$ es un
conjunto de la forma
\begin{align*}
	& \{\bari{s_{0}},\,\dots,\,\bari{s_{q}}\}
	\text{ ,}
\end{align*}
%
donde $\lista[0]{s}{q}\in K$ y $s_{i-1}$ es una cara (propia) de $s_{i}$.
En general, asumiremos que los elementos de un conjunto perteneciente a
$\subdiv K$ est\'{a}n numerados de esta manera. La colecci\'{o}n $\subdiv{K}$
constituye un complejo simplicial cuyos v\'{e}rtices son los conjuntos
puntuales de baricentros de s\'{\i}mplices de $K$.

De la definici\'{o}n del complejo $\subdiv{K}$, se deduce que
los v\'{e}rtices de $\subdiv{K}$ son puntos de $\geom{K}$ y que,
si $s'=\{\bari{s_{0}},\,\dots,\,\bari{s_{q}}\}\in\subdiv{K}$, entonces
$s'\subset\geom{s_{q}}$. En particular, $\subdiv{K}$ y $K$ satisfacen
las propiedades \emph{(i)} y \emph{(ii)} de la definici\'{o}n de
subdivisi\'{o}n. Se puede ver tambi\'{e}n que, si $L\subset K$ es un
subcomplejo, entonces $\subdiv{L}$ es un subcomplejo de $\subdiv{K}$ y que,
si $s'\subset\geom{s_{q}}$ y $\bari{s_{q}}\in s'$, entonces $s_{q}$ es el
s\'{\i}mplice m\'{a}s chico de $K$ que contiene a $s'$ (tal que
$s'\subset\geom{s_{q}}$).

\begin{propoBaricentricaEsSubdivision}\label{thm:baricentricaessubdivision}
	$\subdiv K$ es una subdivisi\'{o}n de $K$.
\end{propoBaricentricaEsSubdivision}

\begin{proof}
	Todo lo que hay que ver es que se verifica la condici\'{o}n
	\emph{(iii')}. Sea $s\in K$. Por la observaci\'{o}n
	\ref{obs:definicionsubdivisiones}, si $K'$ es un complejo
	simplicial cuyos v\'{e}rtices son puntos de $\geom{K}$, $s'\in K'$
	y $s\in K$ es el s\'{\i}mplice m\'{a}s chico tal que
	$s'\subset\geom{s}$, entonces
	$f\big(\simpinterior{s'}\big)\subset\simpinterior{s}$. De esto y
	de los comentarios anteriores, se deduce que
	\begin{align*}
		\big\{s'\in\subdiv K\,:\,
			\simpinterior{s'}\subset\simpinterior s\big\} & \,=\,
			\big\{s'\in\subdiv K\,:\,\bari s
				\text{ es el \'{u}ltimo v\'{e}rtice de } s'
				\big\} \\
		& \,=\,\big\{s'\in\subdiv {\caras s}\,:\,
			\simpinterior{s'}\subset\simpinterior s\big\}
		\text{ .}
	\end{align*}
	%
	Veamos que para todo s\'{\i}mplice $s\in K$ se cumple que
	$\subdiv{\caras s}$ es una subdivisi\'{o}n de $\caras s$ y que
	estamos en las condiciones del lema
	\ref{thm:subdivisionequivaleaparticion}. Si $\dim\,s=q=0$, entonces
	$\subdiv{\caras s}=\caras s$. Si $q>0$ y asumimos que
	$\subdiv{\caras{s_{1}}}$ es una subdivisi\'{o}n de $\caras{s_{1}}$
	para todo s\'{\i}mplice $s_{1}$ de dimensi\'{o}n menor, entonces,
	$\subdiv{\caras{s_{1}}}$ es una subdivisi\'{o}n de $\caras{s_{1}}$
	para toda cara propia $s_{1}\in\carasp s$. En particular, se deduce
	que $\subdiv{\carasp s}$ es una subdivisi\'{o}n de $\carasp s$.
	Finalmente, como
	\begin{math}
		\subdiv{\caras s}=\big(\subdiv{\carasp s}\big)*\bari s
	\end{math}~,
	por el lema \ref{thm:subdividirunsimplice}, concluimos que
	$\subdiv{\caras s}$ es una subdivisi\'{o}n de $\caras s$.
\end{proof}

Llamamos \emph{subdivisi\'{o}n baric\'{e}ntrica de $K$} a la subdivisi\'{o}n
$\subdiv K$ de $K$.

\begin{obsBaricentricaDeSubcomplejoEsPlena}%
	\label{obs:baricentricadesubcomplejoesplena}
	Sea $L\subset K$ un subcomplejo y sean $\subdiv L$ y $\subdiv K$ las
	subdivisiones baric\'{e}ntricas correspondientes a $L$ y a $K$.
	Entonces $\subdiv L$ es un subcomplejo de $\subdiv K$. Supongamos que
	$\{\bari{s_{0}},\,\dots,\,\bari{s_{q}}\}$ es un s\'{\i}mplice de
	$\subdiv K$ cuyos v\'{e}rtices pertenecen a $\subdiv L$. Entonces,
	por definici\'{o}n, $s_{i-1}$ es una cara propia de $s_{i}$ y cada
	$s_{i}$ pertenece a $L$. En particular,
	$\{\bari{s_{0}},\,\dots,\,\bari{s_{q}}\}\in\subdiv L$ y
	$\subdiv L\subset\subdiv K$ es un subcomplejo pleno.
\end{obsBaricentricaDeSubcomplejoEsPlena}

Sea $L\subset K$ un subcomplejo y consideremos el par de espacios
$(\geom{K},\geom{L})$ (un \emph{par poliedral}. En general, un par poliedral
es un par $h:\,A\rightarrow X$ (no necesariamente un subespacio) que admite
una triangulaci\'{o}n $(\varphi:\,L\rightarrow K, (f',f))$ (no necesariamente
un subcomplejo)).

\begin{propoSubcomplejoEsRetracto}\label{thm:subcomplejoesretracto}
	El subespacio $\geom{L}\subset\geom{K}$ es un retracto por
	deformaci\'{o}n fuerte de un entorno de $\geom{L}$ en $\geom{K}$.
\end{propoSubcomplejoEsRetracto}

\begin{proof}
	Por la observaci\'{o}n \ref{obs:baricentricadesubcomplejoesplena},
	podemos asumir que $L$ es un subcomplejo pleno de $K$. Sea $N\subset K$
	el complemento de $L$ en $K$. Notemos que $\geom{K}\setmin\geom{N}$
	es abierto en $\geom{K}$ y contiene a $\geom{L}$, por el lema
	\ref{thm:subcomplejopleno}. Demostraremos que $\geom{L}$ es un
	retracto por deformaci\'{o}n fuerte de este abierto.

	Sea $\alpha\in\geom{K}\setmin\geom{N}$. Entonces, o bien
	$\alpha\in\geom{L}$, o bien existen v\'{e}rtices
	$\lista[0]{v}{p}\in L$ y $\lista[p+1]{v}{q}\in N$ tales que
	$\alpha\in\simpinterior{\{\lista[0]{v}{q}\}}$. En el segundo caso,
	\begin{align*}
		\alpha & \,=\,\sum_{i=0}^{q}\,\alpha^{i}\,\alpha_{v_{i}}
		\text{ ,}
	\end{align*}
	%
	donde $\alpha^{i}>0$ para todo $i\in[\![0,q]\!]$. Definimos
	\begin{align*}
		a & \,=\,\sum_{i=0}^{p}\,\alpha^{i}\text{ ,} \\
		\alpha' \,=\,\sum_{i=0}^{p}\,
			\frac{\alpha^{i}}{a}\,\alpha_{v_{i}}
			& \quad\text{y}\quad
		\alpha'' \,=\,\sum_{i=p+1}^{q}\,
			\frac{\alpha^{i}}{1-a}\,\alpha_{v_{i}}
		\text{ .}
	\end{align*}
	%
	Entonces $\alpha = a\,\alpha'+(1-a)\,\alpha''$, $\alpha' \in\geom{L}$
	y $\alpha''\in\geom{N}$. Sea
	\begin{align*}
		F & \,:\,\big(\geom{K}\setmin\geom{N}\big)\,\times\,\intervalo
			\,\rightarrow\,\big(\geom{K}\setmin\geom{N}\big)
	\end{align*}
	%
	la funci\'{o}n dada por
	\begin{align*}
		F(\alpha,t) & \,=\,
			\begin{cases}
				\alpha & \quad\text{si }\alpha\in\geom{L} \\
				t\,\alpha'+(1-t)\,\alpha &
					\quad\text{si }\alpha\in
					\geom{K}\setmin\big(
						\geom{N}\cup\geom{L}\big)
			\end{cases}
		\text{ .}
	\end{align*}
	%
	Entoncs $F$ es continua y es una retracci\'{o}n por deformaci\'{o}n
	fuerte de $\geom{K}\setmin\geom{N}$ en $\geom{L}$. La continuidad de
	$F$ se deduce de que $F$ es continua en $\geom{L}$, por ser constante,
	y de que $F$ es continua en cada subconjunto de la forma
	$\geom{s'\cup s''}\cap\big(\geom{K}\setmin\geom{N}\big)$, por ser
	una homotop\'{\i}a lineal. Adem\'{a}s ambas definiciones se pegan
	bien: en la clausura de
	\begin{math}
		\geom{s'\cup s''}\cap\big(\geom{K}\setmin\geom{N}\big)
	\end{math}
	dentro de $\geom{K}\setmin\geom{N}$, $\alpha'=\alpha$ y ambas
	definiciones coinciden.
\end{proof}

%
\section{Refinamientos}
\theoremstyle{plain}
\newtheorem{lemaAcotarLaMetricaEnSimplices}{Lema}[section]
\newtheorem{lemaAcotarLaMetricaEnComplejos}[lemaAcotarLaMetricaEnSimplices]%
	{Lema}
\newtheorem{teoRefinarPorTriangulaciones}[lemaAcotarLaMetricaEnSimplices]%
	{Teorema}

\theoremstyle{remark}

%-------------

Todo complejo simplcial $K$ admite una m\'{e}trica, pero, en general,
la topolog\'{\i}a inducida no coincide con la topolog\'{\i}a coherente
en $\geom{K}$. De hecho esto es as\'{\i} si y s\'{o}lo si $K$ es localmente
finito. Diremos que una m\'{e}trica en $\geom{K}$ es \emph{lineal}, si
coincide con la m\'{e}trica inducida por una realizaci\'{o}n de $K$ dentro de
un espacio vectorial topol\'{o}gico $E$ cuya topolog\'{\i}a sea compatible con
una m\'{e}trica. M\'{a}s espec\'{\i}ficamente, diremos que una m\'{e}trica en
$\geom{K}$ es lineal, si es la m\'{e}trica inducida por una norma
(cualquiera) en alg\'{u}n espacio $\bb{R}^{n}$ v\'{\i}a una realizaci\'{o}n
de $K$ en $\bb{R}^{n}$. Notemos que todo complejo finito admite una
m\'{e}trica lineal. Tambi\'{e}n es cierto que, dada una m\'{e}trica lineal
en $\geom{K}$ y una subdivisi\'{o}n $K'$ de $K$, entonces la misma m\'{e}trica
es lineal en $\geom{K'}$. Dada una m\'{e}trica (arbitraria) en $\geom{K}$,
se define la \emph{densidad} (o \emph{apertura}, \emph{fineza}) de $K$ como
el supremo de los di\'{a}metros de sus s\'{\i}mplices con respecto a
esta m\'{e}trica:
\begin{align*}
	\mesh K & \,=\,\sup\,\{\diam\geom{s}\,:\,s\in K\}
	\text{ .}
\end{align*}
%

\begin{lemaAcotarLaMetricaEnSimplices}\label{thm:acotarlametricaensimplices}
	Dada una m\'{e}trica lineal en un $m$ s\'{\i}mplice $s$ y dado un
	s\'{\i}mplice $s'\in\subdiv{\caras s}$ de la subdivisi\'{o}n
	baric\'{e}ntrica, se cumple que
	\begin{align*}
		\diam\geom{s'} & \,\leq\,\frac{m}{m+1}\,
			\diam\geom{s}
		\text{ .}
	\end{align*}
	%
\end{lemaAcotarLaMetricaEnSimplices}

\begin{lemaAcotarLaMetricaEnComplejos}\label{thm:acotarlametricaencomplejos}
	Si $K$ es un complejo de dimensi\'{o}n $m$, dada una m\'{e}trica
	lineal en $\geom{K}$, se cumple que
	\begin{align*}
		\mesh{\big(\subdiv K\big)} & \,\leq\,\frac{m}{m+1}\,
			\mesh K
		\text{ .}
	\end{align*}
	%
\end{lemaAcotarLaMetricaEnComplejos}

Sea $X$ un espacio topol\'{o}gico y sea $(K,f)$, $f:\,\geom{K}\rightarrow X$,
una triangulaci\'{o}n de $X$ por un complejo simplicial $K$. Sea $\cal{U}$ un
cubrimiento de $X$ por abierto. Se dice que la triangulaci\'{o}n
\emph{es m\'{a}s fina que $\cal{U}$} o que \emph{refina a $\cal{U}$}, si,
para todo v\'{e}rtice $v$ de $K$, existe un abierto $U\in\cal{U}$ tal que
$f\big(\estrella v\big)\subset U$. Si $X=\geom{K}$ y
$f:\,\geom{K}\rightarrow\geom{K}$ es la identidad de $\geom K$, decimos
tambi\'{e}n que el complejo $K$ refina el cubrimiento $\cal U$. Si $K'$ es
una subdivisi\'{o}n de $K$ y $f:\,\geom{K'}\rightarrow\geom{K}$ es el
homeomorfismo dado por $\alpha_{v'}\mapsto v'$ en los v\'{e}rtices, tambi\'{e}n
decimos que $K'$ refina el cubrimiento $\cal U$.

\begin{teoRefinarPorTriangulaciones}\label{thm:refinarportriangulaciones}
	Sea $\cal U$ un cubrimiento por abiertos de un espacio triangulable
	compacto $X$. Entonces existe una triangulaci\'{o}n de $X$ m\'{a}s
	fina que $\cal U$.
\end{teoRefinarPorTriangulaciones}

La siguiente definici\'{o}n aprecer\'{a} en la demostraci\'{o}n del teorema.
Denominamos \emph{subdivisiones baric\'{e}ntricas iteradas} de un complejo
simplicial $K$ a las subdivisiones definidas recursivamente de la siguiente
manera:
\begin{align*}
	\subdiv[0]{K} & \,=\, K\text{ ,}\\
	\subdiv[n]{K} & \,=\,\subdiv{\big(\subdiv[n-1]{K}\big)}
		\quad\text{si } n\geq 1
	\text{ .}
\end{align*}
%

\begin{proof}
	Sea $(K,f)$ una triangulaci\'{o}n arbitraria de $X$. Entonces,
	como $f:\,\geom{K}\rightarrow X$ es un homeomorfismo y $X$ es
	compacto, por el corolario \ref{thm:complejofinitoespaciocompacto},
	$K$ es un complejo finito y, por el teorema
	\ref{thm:inmersiondeuncomplejo}, admite una realizaci\'{o}n en
	un $\bb{R}$-espacio de dimensi\'{o}n finita. La m\'{e}trica inducida
	por la norma euclidea de este espacio, por ejemplo, es lineal en
	$\geom{K}$.

	Supongamos que el espacio $\geom{K}$ viene dado con una m\'{e}trica
	lineal. Sea $\epsilon >0$ un n\'{u}mero de Lebesque para el
	cubrimiento $\big\{f^{-1}(U)\,:\,U\in\cal U\big\}$ de $\geom{K}$
	respecto de esta m\'{e}trica. Esto quiere decir que, si
	$A\subset\geom{K}$ tiene di\'{a}metro menor que $\epsilon$, entonces
	$f(A)$ est\'{a} contenido en $U$ para alg\'{u}n abieto $U$ del
	cubrimiento $\cal U$. Sea $m=\dim\,K$ y sea $N\geq 1$ tal que
	\begin{align*}
		\Big(\frac{m}{m+1}\Big)^{N}\cdot\mesh K & \,\leq\,
			\epsilon / 2
		\text{ .}
	\end{align*}
	%
	Para $n\geq N$, $\mesh{\subdiv[n]{K}}\leq\epsilon/2$. Si
	$v'$ es un v\'{e}rtice de $\subdiv[n]{K}$, entonces el conjunto
	$\estrella{v'}$ tiene di\'{a}metro acotado por
	\begin{align*}
		\diam{\estrella{v'}} & \,\leq\,2\cdot
			\mesh{\subdiv[n]{K}} \,\leq\,\epsilon
		\text{ .}
	\end{align*}
	%
	Entonces, para $n\geq N$, vale que
	\begin{align*}
		f\big(\estrella{v'}\big)\,\subset\,U
	\end{align*}
	%
	para alg\'{u}n abierto $U\in\cal U$, con lo que
	$(\subdiv[n]{K},f)$ es refina el cubrimiento $\cal U$.
\end{proof}

% ?`Es cierto este resultado para pares triangulables? ?`Es cierto para espacios
% triangulables no compactos?

%
\section{Aproximaci\'{o}n simplicial}
\theoremstyle{plain}
\newtheorem{teoExistenciaDeAproximaciones}{Teorema}[section]
\newtheorem{lemaDefinicionAproximacionSimplicial}%
	[teoExistenciaDeAproximaciones]{Lema}
\newtheorem{lemaAproximacionEsHomotopica}%
	[teoExistenciaDeAproximaciones]{Lema}
\newtheorem{teoCaracterizacionAproximacionesSimpliciales}%
	[teoExistenciaDeAproximaciones]{Teorema}
\newtheorem{coroAproximacionInducidaEnSubcomplejos}%
	[teoExistenciaDeAproximaciones]{Corolario}
\newtheorem{propoCaracterizacionDeAproximables}%
	[teoExistenciaDeAproximaciones]{Proposici\'{o}n}
\newtheorem{coroCaracterizacionDeAproximacionesPorSubdivision}%
	[teoExistenciaDeAproximaciones]{Corolario}

\theoremstyle{remark}
\newtheorem{obsDefinicionAproximacionSimplicial}%
	[teoExistenciaDeAproximaciones]{Observaci\'{o}n}
\newtheorem{obsAproximacionDeParesEsDePares}%
	[teoExistenciaDeAproximaciones]{Observaci\'{o}n}
\newtheorem{obsComposicionDeAproximacionesEsAproximacion}%
	[teoExistenciaDeAproximaciones]{Observaci\'{o}n}

%-------------

Sean $h_{1}:\,X_{1}\rightarrow X_{1}'$ y $h_{2}:\,X_{2}\rightarrow X_{2}'$
pares de espacios topol\'{o}gicos y sea $(f',f):\,h_{1}\rightarrow h_{2}$ un
morfismo de pares. Supongamos que existen transformaciones simpliciales
$\varphi_{1}:\,K_{1}'\rightarrow K_{1}$ y
$\varphi_{2}:\,K_{2}'\rightarrow K_{2}$ y triangulaciones
$(\varphi_{1},(\psi_{1}',\psi_{1}))$ de $h_{1}$ y
$(\varphi_{2},(\psi_{2}',\psi_{2}))$ de $h_{2}$.
\begin{center}
\begin{tikzcd}
	\geom{K_{1}}\arrow[dr,"\varphi_{1}"]\arrow[dd,"\psi_{1}"'] & &
		\geom{K_{2}}\arrow[dr,"\varphi_{2}"]\arrow[dd,"\psi_{2}"] & \\
	& \geom{K_{1}'}\arrow[dd,crossing over,near start,"\psi_{1}'"'] & &
		\geom{K_{2}'}\arrow[dd,"\psi_{2}'"] \\
	X_{1}\arrow[dr,"h_{1}"']\arrow[rr,near end,"f"'] & &
		X_{2}\arrow[dr,"h_{2}"'] & \\
	& X_{1}'\arrow[rr,"f'"'] & & X_{2}'
\end{tikzcd}
\end{center}
La pregunta que surge es si existe una transformaci\'{o}n simplicial
$\varphi_{1}\rightarrow\varphi_{2}$ cuya realizaci\'{o}n haga conmutar el
diagrama anterior.

No hablaremos de pares por el momento. Sean $K_{1}$, $K_{2}$ complejos
simpliciales y sea $f:\,\geom{K_{1}}\rightarrow\geom{K_{2}}$ una funci\'{o}n
continua. Una transformaci\'{o}n simplicial $\varphi:\,K_{1}\rightarrow K_{2}$
se dice que es una \emph{aproximaci\'{o}n simplicial de $f$}, si, dados
$\alpha\in\geom{K_{1}}$ y $s_{2}\in K_{2}$, $f(\alpha)\in\simpinterior{s_{2}}$
implica que $\geom{\varphi}\alpha\in\geom{s_{2}}$ (o, lo que es equivalente
(por que esto es cierto para todo $s_{2}\in K_{2}$),
$\geom{\varphi}\alpha\in\geom{s_{2}}$ si $f(\alpha)\in\geom{s_{2}}$).

\begin{obsDefinicionAproximacionSimplicial}%
	\label{obs:definicionaproximacionsimplicial}
	Sea $f:\,\geom{K_{1}}\rightarrow\geom{K_{2}}$ una funci\'{o}n continua
	y sea $\varphi:\,K_{1}\rightarrow K_{2}$ una aproximaci\'{o}n
	simplicial de $f$. Si $v$ es un v\'{e}rtice de $K_{1}$ y
	$f(\alpha_{v})$ se corresponde con un v\'{e}rtice de $K_{2}$, entonces
	$\geom{\varphi}\alpha_{v}=f(\alpha_{v})$ y, por lo tanto,
	$f(\alpha_{v})=\beta_{\varphi(v)}$. ($\alpha_{v}$ denota la
	caracter\'{\i}stica de $v$ y $\beta_{\varphi(v)}$ denota la
	caracter\'{\i}stica de $\varphi(v)$). Notemos que, en principio,
	$f(\alpha_{v})$ no est\'{a} forzada a ser un punto correspondiente a
	un v\'{e}rtice de $K_{2}$. Incluso si este fuese el caso, $f$
	podr\'{\i}a no ser simplicial\dots El siguiente lema muestra que el
	caso en que $f$ es ``simplicial'' no es muy interesante.
\end{obsDefinicionAproximacionSimplicial}

\begin{lemaDefinicionAproximacionSimplicial}%
	\label{thm:definicionaproximacionsimplicial}
	Sea $f:\,\geom{K_{1}}\rightarrow\geom{K_{2}}$ una funci\'{o}n continua
	y sea $L_{1}\subset K_{1}$ un subcomplejo. Supongamos que existe una
	transformaci\'{o}n simplicial $\psi:\,L_{1}\rightarrow K_{2}$ tal que
	\begin{align*}
		f|_{\geom{L_{1}}} & \,=\,\geom{\psi}
		\text{ .}
	\end{align*}
	%
	Si $\varphi:\,K_{1}\rightarrow K_{2}$ es una aproximaci\'{o}n
	simplicial de $f$, entonces
	\begin{align*}
		\geom{\varphi}|_{\geom{L_{1}}} & \,=\,\geom{\psi}
		\text{ ,}
	\end{align*}
	%
	es decir, $\varphi|_{L_{1}}=\psi$. En particular, existe una
	\'{u}nica aproximaci\'{o}n simplicial a una funci\'{o}n continua
	de la forma $\geom{\varphi}$ y dicha transformaci\'{o}n es $\varphi$.
\end{lemaDefinicionAproximacionSimplicial}

En otras palabras, si una funci\'{o}n continua $f$ est\'{a} inducida por
una transformaci\'{o}n simplicial en alg\'{u}n subcomplejo del dominio,
entonces toda aproximaci\'{o}n simplicial de $f$ est\'{a} forzada a
coincidir con dicha transformaci\'{o}n en dicho subcomplejo.

\begin{proof}
	Sean $f$ ,$\psi$ y $\varphi$ como en el enunciado. Sea
	$\alpha\in\geom{L_{1}}$ y supongamos que $\alpha=\alpha_{v}$ con
	$v$ un v\'{e}rtice de $L_{1}$. Por hip\'{o}tesis, como
	$f|_{L_{1}}=\geom{\psi}$, se cumple
	\begin{align*}
		f(\alpha_{v}) & \,=\,\geom{\psi}\alpha_{v}\,=\,\beta_{\psi(v)}
	\end{align*}
	%
	y, por otro lado, si $\varphi$ es una aproximaci\'{o}n simplicial de
	$f$, entonces, por lo visto en la observaci\'{o}n
	\ref{obs:definicionaproximacionsimplicial},
	\begin{align*}
		\geom{\varphi}\alpha_{v} & \,=\,f(\alpha_{v})
		\text{ .}
	\end{align*}
	%
	En particular, $\geom{\varphi}\alpha_{v}=\geom{\psi}\alpha_{v}$,
	$\beta_{\varphi(v)}=\beta_{\psi(v)}$ y $\varphi(v)=\psi(v)$ para
	todo v\'{e}rtice de $L_{1}$.
\end{proof}

Repetimos la definici\'{o}n de aproximaci\'{o}n simplicial: si
$f:\,\geom{K_{1}}\rightarrow\geom{K_{2}}$ es una funci\'{o}n continua, lo que
se requiere de una transformaci\'{o}n $\varphi:\,K_{1}\rightarrow K_{2}$
para que sea una aproximaci\'{o}n simplicial de $f$ es que, para todo
punto $\alpha\in\geom{K_{1}}$, si su imagen $f(\alpha)$ est\'{a} contenida
en (el interior de) un s\'{\i}mplice $s_{2}\in K_{2}$, entonces
$\geom{\varphi}\alpha$ no puede estar muy lejos: de hecho debe pertenecer
al mismo s\'{\i}mplice (a lo sumo pertenece a la clausura de
$\simpinterior{s_{2}}$).

\begin{lemaAproximacionEsHomotopica}\label{thm:aproximacioneshomotopica}
	Sea $\varphi:\,K_{1}\rightarrow K_{2}$ una aproximaci\'{o}n
	simplicial de $f:\,\geom{K_{1}}\rightarrow\geom{K_{2}}$. Sea
	$A\subset\geom{K_{1}}$ el conjunto $\{\geom{\varphi}=f\}$. Entonces
	$\geom{\varphi}\simeq f\,(\rel A)$.
\end{lemaAproximacionEsHomotopica}

\begin{proof}
	Sea $F:\,\geom{K_{1}}\times\intervalo\rightarrow\geom{K_{2}}$ la
	homotop\'{\i}a
	\begin{align*}
		F(\alpha,t) & \,=\,t\,f(\alpha)+(1-t)\,\geom{\varphi}(\alpha)
		\text{ .}
	\end{align*}
	%
	Esta funci\'{o}n est\'{a} bien definida: si $\alpha\in\geom{K_{1}}$ y
	$s_{2}\in K_{2}$ es tal que $f(\alpha)\in\simpinterior{s_{2}}$,
	entonces $\geom{\varphi}\alpha\in\geom{s_{2}}$ y la combinaci\'{o}n
	convexa de estos dos puntos est\'{a} definida y es un punto de
	$\geom{s_{2}}$. Restringiendo a cada s\'{\i}mplice de $K_{1}$,
	la imagen por $f$ y la imagen por $\geom{\varphi}$ son compactas
	en $\geom{K_{2}}$. Por el corolario
	\ref{thm:compactocontenidoenfinitossimplices}, si $s\in K_{1}$, existe
	una cantidad finita de s\'{\i}mplices $\lista{t}{r}\in K_{2}$ tales que
	\begin{align*}
		f(s) & \,\subset\,\bigcup_{i=1}^{r}\,\simpinterior{t_{i}}
		\text{ .}
	\end{align*}
	%
	Dado que $\varphi$ es una aproximaci\'{o}n simplicial, se cumple que
	\begin{align*}
		\geom{\varphi}(s) & \,\subset\,
			\bigcup_{i=1}^{r}\,\geom{t_{i}}
		\text{ .}
	\end{align*}
	%
	Esta uni\'{o}n, por ser finita, est\'{a} contenida en alg\'{u}n
	subcomplejo finito $L_{2}\subset K_{2}$. Por finitud de $L_{2}$,
	$\geom{L_{2}}=\geom{L_{2}}_{d}$ y la topolog\'{\i}a en $\geom{L_{2}}$
	est\'{a} dada por la m\'{e}trica euclidea en sus s\'{\i}mplices. Pero
	la topolog\'{\i}a en $\geom{s}$ tambi\'{e}n est\'{a} dada por la
	m\'{e}trica euclidea. Como $\geom{L_{2}}\subset\geom{K_{2}}$ es un
	subespacio, las funciones
	$f,\geom{\varphi}:\,\geom{s}\rightarrow\geom{L_{2}}$ son continuas.
	En particular, son continuas respecto de la m\'{e}trica. Sea
	$\alpha\in\geom{s}$. Dado $\epsilon>0$, existe $\delta>0$ tal que,
	si $\alpha'\in\geom{s}$ y $d(\alpha,\alpha')<\delta$, entonces, si
	bien sus im\'{a}genes pueden pertenecer a distintos s\'{\i}mplices
	en $K_{2}$ las mismas (por $f$ o por $\geom{\varphi}$) pertenecen a
	$\geom{L_{2}}$ y
	\begin{align*}
		d\big(f(\alpha),f(\alpha')\big)\,<\,\epsilon
			& \quad\text{y}\quad
		d\big(\geom{\varphi}\alpha,\geom{\varphi}\alpha'\big)
			\,<\,\epsilon
	\end{align*}
	%
	En cuanto a $F$, si $t,t'\in\intervalo$, entonces
	$F(\alpha,t),F(\alpha',t')\in\geom{L_{2}}$ y
	\begin{align*}
		d\big(F(\alpha,t),F(\alpha',t')\big) & \,\leq\,
			d\big(F(\alpha,t),F(\alpha,t')\big) \,+\,
			d\big(F(\alpha,t'),F(\alpha',t')\big)
		\text{ .}
	\end{align*}
	%
	Por un lado,
	\begin{align*}
		d\big(F(\alpha,t),F(\alpha',t)\big)^{2} & \,=\,
			\sum_{w\in L_{2}}\,
				\big|F(\alpha,t)(w)-F(\alpha',t)(w)\big|^{2} \\
		& \,=\,\sum_{w\in L_{2}}\,
			\big|t\,\big(f(\alpha) w-f(\alpha')w\big)+
				(1-t)\,\big(\geom{\varphi}\alpha w-
				\geom{\varphi}\alpha' w\big)\big|^{2} \\
		& \,\leq\,\sum_{w\in L_{2}}\,
			t\,\big|f(\alpha)w-f(\alpha')w\big|^{2} +
			(1-t)\,\big|\geom{\varphi}\alpha w
				\geom{\varphi}\alpha' w\big|^{2} \\
		& \,\leq\,t\,d\big(f(\alpha),f(\alpha')\big)^{2}\,+\,
			(1-t)\,d\big(\geom{\varphi}\alpha,
				\geom{\varphi}\alpha'\big)^{2}
	\end{align*}
	%
	Por otro,
	\begin{align*}
		d\big(F(\alpha',t),F(\alpha',t')\big)^{2} & \,=\,
			\sum_{w\in L_{2}}\,\big|F(\alpha',t)(w)
				-F(\alpha',t')(w)\big|^{2} \\
		& \,=\,\sum_{w\in L_{2}}\,\big|(t-t')\,f(\alpha')w
			+(t'-t)\,\geom{\varphi}\alpha'w\big|^{2} \\
		& \,=\,|t-t'|^{2}\sum_{w\in L_{2}}\,\big(f(\alpha')w
			+\geom{\varphi}\alpha'w\big)^{2} \\
		& \,=\,|t-t'|^{2}\Big(
			\sum_{w\in L_{2}}\,\big(f(\alpha')w\big)^{2}
			\,+\,
			\sum_{w\in L_{2}}\,2\,f(\alpha')w\,
				\geom{\varphi}\alpha'w \\
		&\qquad\qquad\qquad\,+\,
			\sum_{w\in L_{2}}\,\big(\geom{\varphi}\alpha'w\big)^{2}
			\Big) \\
		& \,\leq\,|t-t'|^{2}\,\Big(
			\sum_{w\in L_{2}}\,f(\alpha')w
			\,+\,
			2\,\sum_{w\in L_{2}}\,f(\alpha')w\,
				\geom{\varphi}\alpha'w \\
		&\qquad\qquad\qquad\,+\,
			\sum_{w\in L_{2}}\,\geom{\varphi}\alpha'w
			\Big) \\
		& \,\leq\,4\,|t-t'|^{2}
	\end{align*}
	%
	Esto demuestra que, si $\alpha,\alpha'\in\geom{s}$ verifican
	$d(\alpha,\alpha')<\delta$ y $|t-t'|<\sqrt{\epsilon}/2$ entonces
	$d\big(F(\alpha,t),F(\alpha',t')\big)<2\epsilon$ y $F$ es
	continua en $\geom{s}\times\intervalo$.

	Como $F|_{\geom{s}\times\intervalo}$ es continua para todo
	$s\in K_{1}$, $F$ es continua, por el teorema
	\ref{thm:homotopiasdecomplejos}. Dado que $F(\alpha,t)=f(\alpha)$,
	si $\alpha\in A$, concluimos que $F$ es una homotop\'{\i}a
	que realiza $\geom{\varphi}\simeq f\,(\rel{A})$.
	% de $f$ en $\geom{\varphi}$ relativa a $A$.
\end{proof}

\begin{teoCaracterizacionAproximacionesSimpliciales}%
	\label{thm:caracterizacionaproximacionessimpliciales}
	Sea $\varphi:\,K_{1}\rightarrow K_{2}$ una aplicaci\'{o}n
	definida \'{u}nicamente en los v\'{e}rtices
	($\varphi:\,V_{1}\rightarrow V_{2}$) y sea
	$f:\,\geom{K_{1}}\rightarrow\geom{K_{2}}$ una funci\'{o}n continua.
	Entonces $\varphi$ determina una aproximaci\'{o}n simplicial de $f$,
	si y s\'{o}lo si, para todo v\'{e}rtice $v$ de $K_{1}$, se cumple que
	\begin{align*}
		f\big(\estrella v\big) &\,\subset\,\estrella{\varphi(v)}
		\text{ .}
	\end{align*}
	%
\end{teoCaracterizacionAproximacionesSimpliciales}

La funci\'{o}n $\varphi$ est\'{a} \'{u}nicamente definida en los v\'{e}rtices.
En particular, no se asume que es una transformaci\'{o}n simplicial.

\begin{proof}
	Supongamos que $\varphi:\,K_{1}\rightarrow K_{2}$ es una
	aproximaci\'{o}n simplicial de $f$ y sean $\alpha\in\geom{K_{1}}$ y
	$s_{2}\in K_{2}$ tales que $f(\alpha)\in\simpinterior{s_{2}}$.
	Como $\varphi$ es una aproximaci\'{o}n simplicial,
	$\geom{\varphi}\alpha\in\geom{s_{2}}$. Por otra parte,
	$\alpha\in\estrella v$ para cierto v\'{e}rtice $v$ de $K_{1}$.
	Entonces $\alpha(v)\not =0$ y $\geom{\varphi}\alpha(\varphi(v))$
	es distinto de cero, tambi\'{e}n. En particular, el v\'{e}rtice
	$\varphi(v)$ peretenece a $s_{2}$ y
	$f(\alpha)\in\estrella{\varphi(v)}$. Como $\alpha\in\estrella v$
	era arbitrario, se verifica
	\begin{align*}
		f\big(\estrella v\big) & \,\subset\,\estrella{\varphi(v)}
		\text{ .}
	\end{align*}
	%
	Rec\'{\i}procamente, si la funci\'{o}n en v\'{e}rtices $\varphi$
	verifica esta condici\'{o}n para todo v\'{e}rtice de $K_{1}$,
	entonces, dado $s=\{\lista[0]{v}{q}\}\in K_{1}$, por el corolario
	\ref{thm:simplicedesubcomplejo}, la intersecci\'{o}n
	$\bigcap_{i=0}^{q}\,\estrella{v_{i}}$ es no vac\'{\i}a. Pero esto
	implica que
	\begin{align*}
		\bigcap_{i=0}^{q}\,\estrella{\varphi(v_{i})} & \,\supset\,
			\bigcap_{i=0}^{q}\,f\big(\estrella{v_{i}}\big)
				\,\supset\,
			f\Big(\bigcap_{i=0}^{q}\,\estrella{v_{i}}\Big)
			\,\not=\,\varnothing
		\text{ .}
	\end{align*}
	%
	Apelando de nuevo al corolario \ref{thm:simplicedesubcomplejo},
	los v\'{e}rtices $\{\varphi(v_{0}),\,\dots,\,\varphi(v_{q})\}$
	son los v\'{e}rtices de un s\'{\i}mplice en $K_{2}$. Es definitiva,
	$\varphi:\,K_{1}\rightarrow K_{2}$ es simplicial.

	Finalmente, para ver que $\varphi$ es una aproximaci\'{o}n simplicial
	de $f$, sea $\alpha\in\geom{K_{1}}$ y sea $s_{2}\in K_{2}$ tal que
	$f(\alpha)\in\simpinterior{s_{2}}$. Sea $s\in K_{1}$ tal que
	$\alpha\in\simpinterior{s}$ y sea $v\in s$ un v\'{e}rtice cualquiera
	de $s$. Entonces, por hip\'{o}tesis, como $\alpha\in\estrella v$,
	se cumple que $f(\alpha)\in\estrella{\varphi(v)}$. En particular,
	$\varphi(v)$ es un v\'{e}rtice de $s_{2}$. Como $\varphi$ es
	simplicial, $\varphi(s)\subset s_{2}$ y, por lo tanto,
	$\geom{\varphi}\big(\geom{s}\big)\subset\geom{s_{2}}$. As\'{\i},
	$\geom{\varphi}\alpha\in\geom{s_{2}}$.
\end{proof}

Sean $\varphi_{1}:\,L_{1}\rightarrow K_{1}$ y
$\varphi_{2}:\,L_{2}\rightarrow K_{2}$ dos pares simpliciales y sea
$(f',f):\,\geom{\varphi_{1}}\rightarrow\geom{\varphi_{2}}$ una funci\'{o}n
continua en pares ($f'\circ\geom{\varphi_{1}}=\geom{\varphi_{2}}\circ f$).
Supongamos que existe una transformaci\'{o}n de pares simpliciales
$(\psi',\psi):\,\varphi_{1}\rightarrow\varphi_{2}$ tal que
$\psi':\,K_{1}\rightarrow K_{2}$ es una aproximaci\'{o}n simplicial de $f'$.
Si $\beta\in\geom{L_{1}}$ y $t_{2}\in L_{2}$ es tal que
$f(\beta)\in\simpinterior{t_{2}}$, entonces
\begin{align*}
	f'\big(\geom{\varphi_{1}}\beta\big) & \,=\,
		\geom{\varphi_{2}}f(\beta) \,\in\,
		\geom{\varphi_{2}}\big(\simpinterior{t_{2}}\big) \,\subset\,
		\simpinterior{\varphi_{2}t_{2}}
\end{align*}
%
($\varphi_{2}$ es simplicial, con lo que $\varphi_{2}t_{2}\in K_{2}$). Como
$f'$ es aproximada por $\psi'$, vale que
\begin{align*}
	\geom{\psi'}\big(\geom{\varphi_{1}}\beta\big) & \,\in\,
		\geom{\varphi_{2}t_{2}}
	\text{ .}
\end{align*}
%
Pero
\begin{align*}
	\geom{\psi'}\geom{\varphi_{1}} & \,=\,\geom{\psi'\varphi_{1}}\,=\,
		\geom{\varphi_{2}\psi}\,=\,\geom{\varphi_{2}}\geom{\psi}
	\text{ .}
\end{align*}
%
Entonces se deduce que
\begin{align*}
	\geom{\varphi_{2}}\big(\geom{\psi}\beta\big) & \,\in\,
		\geom{\varphi_{2}}\big(\geom{t_{2}}\big)
	\text{ .}
\end{align*}
%
No parece haber raz\'{o}n para esperar que $\psi$ sea una aproximaci\'{o}n
simplicial de $f$. Supongamos, por otro lado, que $w_{0}$ es un v\'{e}rtice de
$L_{1}$. Sabemos, porque $\psi$ es simplicial, que $\psi(w_{0})$ es un
v\'{e}rtice de $L_{2}$.
% Supongamos que $f(\beta)\not\in\estrella{\psi(w_{0})}$.
Supongamos que $\beta\in\estrella w_{0}$.
Entonces
\begin{align*}
	\big(\geom{\varphi_{1}}\beta\big)(\varphi_{1}w_{0}) & \,=\,
		\sum_{\varphi_{1}w=\varphi_{1}w_{0}}\,\beta(w) \,>\,0
\end{align*}
%
y $\geom{\varphi_{1}}\beta\in\estrella{\varphi_{1}(w_{0})}$. De esto podemos
deducir que $f'\big(\geom{\varphi_{1}}\beta\big)$ pertenece a
$\estrella{\psi'(\varphi_{1}w_{0})}$. Pero esto significa que
\begin{align*}
	\geom{\varphi_{2}}\circ f(\beta) & \,\in\,
		\estrella{\varphi_{2}(\psi w_{0})}
	\text{ .}
\end{align*}
%
De nuevo, si $\varphi_{2}$ identifica s\'{\i}mplices distintos en $L_{2}$,
no hay raz\'{o}n para esperar que $\psi$ sea una aproximaci\'{o}n simplicial
de $f$. El siguiente corolario garantiza que, cuando los pares $(K_{j},L_{j})$
son subcomplejos, toda aproximaci\'{o}n simplicial de
$f':\,K_{1}\rightarrow K_{2}$ induce una aproximaci\'{o}n de
$f=f'|_{\geom{L_{1}}}$ en los subcomplejos.

\begin{coroAproximacionInducidaEnSubcomplejos}%
	\label{thm:aproximacioninducidaensubcomplejos}
	Sea $f:\,\geom{K_{1}}\rightarrow\geom{K_{2}}$ una funci\'{o}n
	continua y sean $L_{1}\subset K_{1}$ y $L_{2}\subset K_{2}$
	subcomplejos. Supongamos que
	$f\big(\geom{L_{1}}\big)\subset\geom{L_{2}}$ y que $f$ admite
	una aproximaci\'{o}n simplicial $\varphi:\,K_{1}\rightarrow K_{2}$.
	Entonces $\varphi\big(L_{1}\big)\subset L_{2}$ y
	$\varphi|_{L_{1}}:\,L_{1}\rightarrow L_{2}$ es una aproximaci\'{o}n
	simplicial de $f|_{\geom{L_{1}}}$.
\end{coroAproximacionInducidaEnSubcomplejos}

Notemos que, si bien $\varphi$ se restringe a una aplicaci\'{o}n de los
v\'{e}rtices de $L_{1}$ en los v\'{e}rtices de $L_{2}$, como no se asume
que ni $L_{2}$ ni $L_{1}$ sean subcomplejos plenos, no es inmediato que
la restricci\'{o}n de $\varphi$ sea simplicial de $L_{1}$ \emph{en} $L_{2}$.

\begin{proof}
	Por el teorema \ref{thm:caracterizacionaproximacionessimpliciales},
	alcanzar\'{a} con demostrar que, dado un v\'{e}rtice $w$ de $L_{1}$,
	entonces $\varphi(w)$ es un v\'{e}rtice de $L_{2}$ y que se verifica
	que
	\begin{align*}
		f\big(\estrella w\,\cap\,\geom{L_{1}}\big) & \,\subset\,
			\big(\estrella{\varphi(w)}\big)\,\cap\,\geom{L_{2}}
		\text{ .}
	\end{align*}
	%
	Por un lado, por hip\'{o}tesis,
	$f\big(\estrella w\big)\subset\estrella{\varphi(w)}$. Por otro lado,
	como $f\big(\geom{L_{1}}\big)\subset\geom{L_{2}}$, existe un
	s\'{\i}mplice $t_{2}\in L_{2}$ tal que $f(w)\in\simpinterior{t_{2}}$.
	En particular, $\geom{\varphi}(\alpha_{w})\in\geom{t_{2}}$ y
	$\varphi(w)$ es un v\'{e}rtice ($\varphi$ es simplicial) de $t_{2}$.
	Entonces $\varphi(w)$ es un v\'{e}rtice de $L_{2}$ y
	\begin{align*}
		f\big(\estrella w\,\cap\,\geom{L_{1}}\big) & \,\subset\,
			f\big(\estrella w\big)\,\cap\,\geom{L_{2}} \,\subset\,
			\big(\estrella{\varphi(w)}\big)\,\cap\,\geom{L_{2}}
		\text{ .}
	\end{align*}
	%
\end{proof}

\begin{obsAproximacionDeParesEsDePares}\label{obs:aproximaciondeparesesdepares}
	Sean $L_{1}\subset K_{1}$ y $L_{2}\subset K_{2}$ subcomplejos y
	sea
	\begin{math}
		f:\,(\geom{K_{1}},\geom{L_{1}})\rightarrow
			(\geom{K_{2}},\geom{L_{2}})
	\end{math}
	una funci\'{o}n continua de pares de espacios topol\'{o}gicos. Por
	el corolario \ref{thm:aproximacioninducidaensubcomplejos}, toda
	aproximaci\'{o}n simplicial $\varphi:\,K_{1}\rightarrow K_{2}$ es
	est\'{a} forzada a ser una transformaci\'{o}n de pares
	$\varphi:\,(K_{1},L_{1})\rightarrow (K_{2},L_{2})$.

	Una pregunta que surge de esto es, si $\psi'$ es una aproximaci\'{o}n
	de $f'$ y $\psi$ es una aproximaci\'{o}n de $f$ (donde
	$(f',f'):\,\geom{\varphi_{1}}\rightarrow\geom{\varphi_{2}}$ es un
	morfismo de pares entre los espacios de los pares de complejos
	$\varphi_{1}$ y $\varphi_{2}$) ?`es cierto que
	$\psi'\varphi_{1}=\varphi_{2}\psi$?, es decir, ?`es $(\psi',\psi)$ una
	transformaci\'{o}n simplicial de pares? En general, parece ser que
	la respuesta es negativa: lo que se puede deducir es que el punto
	\begin{align*}
		\geom{\varphi_{2}}f(\beta_{w}) & \,=\,
			f'\big(\geom{\varphi_{1}}\beta_{w}\big)
	\end{align*}
	%
	($w$ un v\'{e}rtice de $L_{1}$) pertenece a la intersecci\'{o}n
	\begin{align*}
		& \estrella{\psi'(\varphi_{1}w)}\,\cap\,
			\geom{\varphi_{2}}\big(\estrella{\psi(w)}\big)
		\text{ .}
	\end{align*}
	%

	Volviendo al caso de los subcomplejos, si
	$\varphi:\,(K_{1},L_{1})\rightarrow (K_{2},L_{2})$ es una
	aproximaci\'{o}n simplicial de
	\begin{math}
		f:\,(\geom{K_{1}},\geom{L_{1}})\rightarrow
			(\geom{K_{2}},\geom{L_{2}})
	\end{math}~,
	entonces la homotop\'{\i}a $f\simeq\geom{\varphi}$ dada por el lema
	\ref{thm:aproximacioneshomotopica} es, en realidad, una homotop\'{\i}a
	de pares, pues la imagen de $\geom{L_{1}}\times\intervalo$ debe
	mantenerse dentro de $\geom{L_{2}}$.
\end{obsAproximacionDeParesEsDePares}

\begin{obsComposicionDeAproximacionesEsAproximacion}%
	\label{obs:composiciondeaproximacionesesaproximacion}
	Sean $g:\,\geom{K_{2}}\rightarrow\geom{K_{3}}$ y
	$f:\,\geom{K_{1}}\rightarrow\geom{K_{2}}$ funciones continuas y sean
	$\psi:\,K_{2}\rightarrow K_{3}$ y $\varphi:\,K_{1}\rightarrow K_{2}$
	aproximaciones simpliciales. Entonces, dado un v\'{e}rtice $v$ de
	$K_{1}$,
	\begin{align*}
		g\circ f\big(\estrella v\big) & \,\subset\,
			g\big(\estrella{\varphi(v)}\big) \,\subset\,
			\estrella{\psi\circ\varphi(v)}
		\text{ .}
	\end{align*}
	%
	Por lo tanto, $\psi\circ\varphi$ es una aproximaci\'{o}n
	simplicial de $g\circ f$.
\end{obsComposicionDeAproximacionesEsAproximacion}

\begin{propoCaracterizacionDeAproximables}%
	\label{thm:caracterizaciondeaproximables}
	Sea $f:\,\geom{K_{1}}\rightarrow\geom{K_{2}}$ una funci\'{o}n
	continua. Entonces $f$ admite una aproximaci\'{o}n simplicial,
	si y s\'{o}lo si $K_{1}$ refina el cubrimiento por abiertos
	\begin{align*}
		\cal U & \,=\, \big\{f^{-1}\big(\estrella v'\big)\,:\,
			v'\text{ v\'{e}rtice de }K_{2}\big\}
		\text{ .}
	\end{align*}
	%
\end{propoCaracterizacionDeAproximables}

\begin{proof}
	Supongamos que existe una aproximaci\'{o}n simplicial
	$\varphi:\,K_{1}\rightarrow K_{2}$ de $f$. Entonces, dado un
	v\'{e}rtice $v$ de $K_{1}$, $\varphi(v)=v'$ es un v\'{e}rtice de
	$K_{2}$ y
	\begin{align*}
		\estrella v & \,\subset\,f^{-1}\big(\estrella v'\big)
		\text{ .}
	\end{align*}
	%
	En particular, $K_{1}$ refina el cubrimiento $\cal U$ de
	$\geom{K_{1}}$. Rec\'{\i}procamente, si para todo v\'{e}rtice $v$ de
	$K$ existe alg\'{u}n v\'{e}rtice $v'$ tal que $\estrella v$ est\'{e}
	contenido en el abierto $f^{-1}\big(\estrella v'\big)$, entonces
	$\varphi:\,v\mapsto v'$ define una aplicaci\'{o}n de los v\'{e}rtices
	de $K_{1}$ en los de $K_{2}$ que verifica las hip\'{o}tesis del
	teorema \ref{thm:caracterizacionaproximacionessimpliciales}.
\end{proof}

\begin{coroCaracterizacionDeAproximacionesPorSubdivision}%
	\label{thm:caracterizaciondeaproximacionesporsubdivision}
	Sea $K'$ una subdivisi\'{o}n de un complejo simplicial $K$ y sea
	$\varphi$ una aplicaci\'{o}n definida en los v\'{e}rtices de $K'$
	con imagen en los v\'{e}rtices de $K$. Entonces $\varphi$ determina
	una aproximaci\'{o}n simplicial de la identidad
	$\geom{K'}\rightarrow\geom{K}$, si y s\'{o}lo si
	$v'\in\estrella{\varphi(v')}$ para todo v\'{e}rtice $v'$ en $K'$.
\end{coroCaracterizacionDeAproximacionesPorSubdivision}

\begin{proof}
	Este resultado es consecuencia de la observaci\'{o}n
	\ref{obs:subdivisionesyestrellas} y del teorema
	\ref{thm:caracterizacionaproximacionessimpliciales}.
\end{proof}

Notemos que esto implica la existencia de aproximaciones simpliciales de la
identidad $\geom{K'}\rightarrow\geom{K}$ para toda subdivisi\'{o}n $K'$ de $K$.
Llegamos al teorema central de existencia de aproximaciones simpliciales.

\begin{teoExistenciaDeAproximaciones}\label{thm:existenciadeaproximaciones}
	Sea $(K_{1},L_{1})$ un par simplicial finito y sea
	\begin{math}
		f:\,(\geom{K_{1}},\geom{L_{1}})\rightarrow
			(\geom{K_{2}},\geom{L_{2}})
	\end{math}
	una funci\'{o}n continua. Existe $N\geq 1$ tal que, si $n\geq N$,
	entonces $f$ admite una aproximaci\'{o}n simplicial
	\begin{math}
		\varphi:\,(\subdiv[n]{K_{1}},\subdiv[n]{L_{1}})\rightarrow
			(K_{2},L_{2})
	\end{math}~.
\end{teoExistenciaDeAproximaciones}

\begin{proof}
	Ver el corolario \ref{thm:aproximacioninducidaensubcomplejos},
	la proposici\'{o}n \ref{thm:caracterizaciondeaproximables},
	el corolario \ref{thm:caracterizaciondeaproximacionesporsubdivision} y
	el teorema \ref{thm:refinarportriangulaciones}.
	El valor de $N$ depender\'{a} del codominio (ver el ejemplo
	\ref{ejemplo:aproximacionesenelcirculo}).
\end{proof}

%
\section{Contig\"{u}idad}
\theoremstyle{plain}
\newtheorem{teoHomotopicasAdmitenAproximacionesContiguas}{Teorema}[section]
\newtheorem{lemaAproximacionesSonContiguas}%
	[teoHomotopicasAdmitenAproximacionesContiguas]{Lema}
\newtheorem{teoHomotopicasPorContiguas}%
	[teoHomotopicasAdmitenAproximacionesContiguas]{Teorema}
\newtheorem{coroCodominioNumerableHomotopicasNumerables}%
	[teoHomotopicasAdmitenAproximacionesContiguas]{Corolario}

\theoremstyle{remark}
\newtheorem{obsComposicionesDeContiguasSonContiguas}%
	[teoHomotopicasAdmitenAproximacionesContiguas]{Observaci\'{o}n}
\newtheorem{obsContiguasSonHomotopicas}%
	[teoHomotopicasAdmitenAproximacionesContiguas]{Observaci\'{o}n}
\newtheorem{obsLimiteDirectoDeSubdivisiones}%
	[teoHomotopicasAdmitenAproximacionesContiguas]{Observaci\'{o}n}

%-------------

En esta secci\'{o}n definimos una relaci\'{o}n an\'{a}loga a la relaci\'{o}n
de homotop\'{\i}a entre pares de espacios topol\'{o}gicos. Nos concentraremos
en pares $(K,L)$ de complejos simpliciales donde $L\subset K$ es un
subcomplejo. El objetivo es entender la relaci\'{o}n, valaga la redundancia,
entre estas dos nociones de equivalencia.

Sean $K_{1},K_{2}$ complejos simpliciales y sea $L_{1}\subset K_{1}$ y
$L_{2}\subset K_{2}$ subcomplejos. Dos transformaciones simpliciales
$\varphi,\varphi':\,(K_{1},L_{1})\rightarrow (K_{2},L_{2})$ se dicen
\emph{contiguas}, si,
\begin{itemize}
	\item[(i)] dado un s\'{\i}mplice $s\in K_{1}$, la uni\'{o}n
		$\varphi(s)\cup\varphi'(s)$ es un s\'{\i}mplice de $K_{2}$ y
	\item[(ii)] si $s\in L_{1}$, entonces $\varphi(s)\cup\varphi'(s)$
		es un s\'{\i}mplice de $L_{2}$.
\end{itemize}
%
Esto no define una relaci\'{o}n de equivalencia, pero s\'{\i} sim\'{e}trica y
reflexiva en el conjunto de transformaciones simpliciales
$(K_{1},L_{1})\rightarrow (K_{2},L_{2})$. Para definir una relaci\'{o}n de
equivalencia, transitivizamos la relaci\'{o}n: decimos que dos transformaciones
simpliciales $\varphi,\varphi'$ est\'{a}n relacionadas, o que
\emph{pertenecen a la misma clase de contig\"{u}idad}, si existe una
cantidad finita de transformaciones simpliciales $\lista{\varphi}{r}$
tales que $\varphi_{1}=\varphi$, $\varphi_{r}=\varphi'$ y
$\varphi_{i-1}$ y $\varphi_{i}$ son contiguas. Escribimos
$\varphi\sim\varphi'$ para denotar que $\varphi$ y $\varphi'$ pertenecen a la
misma clase de contig\"{u}idad. Denotamos el conjunto de estas clases por
$\contiguas{K_{1},L_{1}}{K_{2},L_{2}}$ y la clase de una transformaci\'{o}n
simplicial $\varphi$ por $\clase{\varphi}$.

\begin{obsComposicionesDeContiguasSonContiguas}%
	\label{obs:composicionesdecontiguassoncontiguas}
	Si $\varphi,\varphi':\,(K_{1},L_{1})\rightarrow (K_{2},L_{2})$
	son contiguas y
	$\psi,\psi':\,(K_{2},L_{2})\rightarrow (K_{3},L_{3})$ son contiguas,
	entonces $\psi\varphi$ y $\psi'\varphi'$ son contiguas: dado un
	s\'{\i}mplice $s\in K_{1}$,
	\begin{align*}
		\psi\varphi(s)\,\cup\,\psi'\varphi'(s) & \,\subset\,
			\psi\big(\varphi(s)\cup\varphi'(s)\big)\,\cup\,
			\psi'\big(\varphi(s)\cup\varphi'(s)\big)
		\text{ .}
	\end{align*}
	%
	Entonces, como la uni\'{o}n de la derecha es, por hip\'{o}tesis, un
	s\'{\i}mplice de $K_{3}$, el subconjunto de la izquierda tambi\'{e}n
	lo es; si $s\in L_{1}$, entonces lo mismo es cierto reemplazando los
	s\'{\i}mplices $K_{i}$ por los $L_{i}$.

	De esto se deduce que, si, m\'{a}s en general, $\varphi\sim\varphi'$
	y $\psi\sim\psi'$, entonces $\psi\varphi\sim\psi'\varphi'$ y la
	composici\'{o}n de clases de contig\"{u}idad est\'{a} bien
	definida como la clase de las composiciones:
	\begin{align*}
		\clase{\psi}\circ\clase{\varphi} & \,=\,
			\clase{\psi\circ\varphi}
		\text{ .}
	\end{align*}
	%
\end{obsComposicionesDeContiguasSonContiguas}

\begin{obsContiguasSonHomotopicas}\label{obs:contiguassonhomotopicas}
	Sean
	\begin{math}
		\varphi,\varphi':\,(K_{1},L_{1})\rightarrow (K_{2},L_{2})
	\end{math}
	transformaciones simpliciales contiguas que coinciden en alg\'{u}n
	subcomplejo $L\subset K_{1}$. Sea
	\begin{align*}
		F & \,:\,\big(\geom{K_{1}}\times\intervalo,
			\geom{L_{1}}\times\intervalo\big)\,\rightarrow\,
				(\geom{K_{2}},\geom{L_{2}})
	\end{align*}
	%
	la homotop\'{\i}a de $\geom{\varphi}$ en $\geom{\varphi'}$ dada por
	\begin{align*}
		F(\alpha,t) & \,=\,(1-t),\geom{\varphi}\alpha + 
					t\,\geom{\varphi'}\alpha
		\text{ .}
	\end{align*}
	%
	Para $\alpha\in\geom{L_{1}}$, como $s\in L_{1}$ implica
	$\varphi(s)\cup\varphi'(s)\in L_{2}$, $F(\alpha,t)\in\geom{L_{2}}$.
	Para $\alpha\in\geom{L}$, el resultado no depende de $t$. En
	definitiva, si $\varphi$ y $\varphi'$ son contiguas y coinciden en
	alg\'{u}n subcomplejo del dominio (independiente de $L_{1}$ y de
	$L_{2}$), entocnes
	\begin{math}
		\geom{\varphi}\simeq\geom{\varphi'}\big(\rel{\geom{L}}\big)
	\end{math}~.

	En particular, $\varphi\sim\varphi'$ implica
	$\geom{\varphi}\simeq\geom{\varphi'}$.
\end{obsContiguasSonHomotopicas}

De las \'{u}ltimas dos observaciones se deduce que, en primer lugar,
las clases de contig\"{u}idad de transformaciones simpliciales determinan
una categor\'{\i}a cuyos objetos son los pares simpliciales, cuyos morfismos
son las clases de contig\"{u}idad de transformaciones simpliciales con
la composici\'{o}n dada por la clase de las composiciones. Los objetos son
los mismos que en la categor\'{\i}a de pares simpliciales. En segundo lugar,
las aplicaciones $(K,L)\mapsto (\geom{K},\geom{L})$ y
$\clase{\varphi}\mapsto\clase{\geom{\varphi}}$ determinan un funtor de la
categor\'{\i}a \emph{de contig\"{u}idad} de pares de complejos simpliciales
en la categor\'{\i}a de homotop\'{\i}a de pares de espacios topol\'{o}gicos.

\begin{lemaAproximacionesSonContiguas}\label{thm:aproximacionessoncontiguas}
	Sean
	\begin{math}
		\varphi,\varphi':\,(K_{1},L_{1})\rightarrow (K_{2},L_{2})
	\end{math}
	dos aproximaciones simpliciales de una misma funci\'{o}n continua
	\begin{math}
		f:\,(\geom{K_{1}},\geom{L_{1}})\rightarrow
			(\geom{K_{2}},\geom{L_{2}})
	\end{math}~.
	Entonces $\varphi$ y $\varphi'$ son contiguas.
\end{lemaAproximacionesSonContiguas}

\begin{proof}
	Sea $s=\{\lista[0]{v}{q}\}\in K_{1}$ un s\'{\i}mplice. Por
	el corolario \ref{thm:simplicedesubcomplejo},
	$\bigcap_{i=0}^{q}\,\estrella v_{i}\not=\varnothing$. Por el teorema
	\ref{thm:caracterizacionaproximacionessimpliciales},
	\begin{align*}
		\bigcap_{i=0}^{q}\,\big(
			\estrella{\varphi(v_{i})}\cap\estrella{\varphi'(v_{i})}
			\big) & \,\supset\,
			\bigcap_{i=0}^{q}\,f\big(\estrella{v_{i}}\big)
				\,\supset\,
			f\Big(\bigcap_{i=0}^{q}\,\estrella{v_{i}}\Big)
				\,\not=\,\varnothing
		\text{ .}
	\end{align*}
	%
	Por el corolario \ref{thm:simplicedesubcomplejo}, la uni\'{o}n
	\begin{math}
		\varphi(s)\cup\varphi'(s)=
			\{\varphi(v_{i})\}_{i}\cup\{\varphi'(v_{i})\}_{i}
	\end{math}
	es el conjunto de v\'{e}rtices de un s\'{\i}mplice en $K_{2}$. Si
	asumimos que $s\in L_{1}$, entonces esta uni\'{o}n es un subconjunto
	de v\'{e}rtices de $L_{2}$ tales que la intersecci\'{o}n de las
	estrellas correspondientes es no vac\'{\i}a y, por lo tanto,
	constituyen el conjunto de v\'{e}rtices de un s\'{\i}mplice en $L_{2}$.
	En definitiva, $\varphi$ y $\varphi'$ son contiguas.
\end{proof}

Transformaciones simpliciales que definen funciones continuas homot\'{o}picas
en los espacios de los complejos pueden no pertenecer a la misma clase de
contig\"{u}idad. Aun as\'{\i}, en el caso de que el dominio sea un cimplejo
finito, es posible subdividir este complejo de forma tal que transformaciones
que inducen funciones homot\'{o}picas son aproximables en la subdivisi\'{o}n
por transformaciones en la misma clase de contig\"{u}idad.

\begin{teoHomotopicasAdmitenAproximacionesContiguas}%
	\label{thm:homotopicasadmitenaproximacionescontiguas}
	Sea $K_{1}$ un complejo simplicial finito. Sean
	\begin{align*}
		f,f' & \,:\,(\geom{K_{1}},\geom{L_{1}})\,\rightarrow\,
			(\geom{K_{2}},\geom{L_{2}})
	\end{align*}
	funciones homot\'{o}picas. Entonces existe $N\geq 1$ y aproximaciones
	simpliciales $\varphi$ de $f$ y $\varphi'$ de $f'$
	\begin{align*}
		\varphi,\varphi' & \,:\,(\subdiv[N]{K_{1}},\subdiv[N]{L_{1}})
			\,\rightarrow\,(K_{2},L_{2})
	\end{align*}
	%
	en la misma clase de contig\"{u}idad.
\end{teoHomotopicasAdmitenAproximacionesContiguas}

\begin{proof}
	Sea
	\begin{align*}
		F & \,:\,\big(\geom{K_{1}}\times\intervalo,
				\geom{L_{1}}\times\intervalo\big)
			\,\rightarrow\, (\geom{K_{2}},\geom{L_{2}})
	\end{align*}
	%
	una homotop\'{\i}a de $f$ en $f'$. Al ser $\geom{K_{1}}$ compacto y
	\begin{math}
		\big\{F^{-1}\big(\estrella{v'}\big)\,:\,
			v'\text{ v\'{e}rtice en }K_{2}\big\}
	\end{math}
	un cubrimiento por abiertos de $K_{1}$, existen
	$0=t_{0}<t_{1}<\cdots<t_{r}=1$ tales que, para todo $\alpha\in K_{1}$,
	$F(\alpha,t_{i-1})$ y $F(\alpha,t_{i})$ pertenezcan a un mismo
	abierto $\estrella{v'}$ de $K_{2}$.

	Para cada $i=0,\,\dots,\,r$, sea
	\begin{align*}
		f_{i} & \,:\,(\geom{K_{1}},\geom{L_{1}})\,\rightarrow\,
			(\geom{K_{2}},\geom{L_{2}})
	\end{align*}
	%
	la funci\'{o}n dada por $f_{i}(\alpha)=F(\alpha,t_{i})$ y sea,
	para $i\geq 1$, $\cal{U}_{i}$ el cubrimiento
	\begin{align*}
		\cal{U}_{i} & \,=\,\big\{
			f_{i}^{-1}\big(\estrella{v'}\big)\cap
				f_{i-1}^{-1}\big(\estrella{v'}\big)\,:\,
			v'\in K_{2}\big\}
	\end{align*}
	%
	por abiertos de $K_{1}$. Por el teorema
	\ref{thm:refinarportriangulaciones}, existe $N\geq 1$ tal que la
	subdivisi\'{o}n $\subdiv[N]{K_{1}}$ sea m\'{a}s fina que todos los
	cubrimientos $\lista{\cal{U}}{r}$. Esto significa que, para cada
	$i=1,\,\dots,\,r$, si $v$ es un v\'{e}rtice de $\subdiv[N]{K_{1}}$,
	entonces existe $U\in\cal{U}_{i}$ tal que $\estrella v\subset U$.
	Dicho de otra manera, para cada v\'{e}rtice $v$ de la subdivisi\'{o}n,
	existe al menos un v\'{e}rtice $v'$ de $K_{2}$ tal que
	\begin{align*}
		\estrella v & \,\subset\,
			f_{i}^{-1}\big(\estrella{v'}\big)\,\cap\,
			f_{i-1}^{-1}\big(\estrella{v'}\big)
		\text{ .}
	\end{align*}
	%
	Eligiendo, debe existir una funci\'{o}n $\varphi_{i}$ definida en
	los v\'{e}rtices de $\subdiv[N]{K_{1}}$ con imagen en los
	v\'{e}rtices de $K_{2}$ tal que
	\begin{align*}
		f_{i}\big(\estrella v\big),\cup\,
			f_{i-1}\big(\estrella v\big) & \,\subset\,
			\estrella{\varphi_{i}(v)}
	\end{align*}
	%
	para todo v\'{e}rtice $v$ de la subdivisi\'{o}n. En particular, por
	el teorema \ref{thm:caracterizacionaproximacionessimpliciales},
	cada una de las funciones $\varphi_{i}$ determina una
	transformaci\'{o}n simplicial
	\begin{align*}
		\varphi_{i} & \,:\,(\subdiv[N]{K_{1}},\subdiv[N]{L_{1}})
			\,\rightarrow\,(K_{2},L_{2})
	\end{align*}
	%
	que es, a la vez, aproximaci\'{o}n de
	$f_{i}$ y de $f_{i-1}$. Del lema \ref{thm:aproximacionessoncontiguas},
	se deduce que $\varphi_{i}$ y $\varphi_{i+1}$ son contiguas.
	En particular, $\varphi_{1}\sim\varphi_{r}$. Pero $\varphi_{1}$ es
	una aproximaci\'{o}n de $f_{0}=f$ y $\varphi_{r}$ es una
	aproximaci\'{o}n de $f_{r}=f'$.
\end{proof}

Este resultado no es cierto en general, si $K_{1}$ no es finito (ver el
ejemplo \ref{ejemplo:homotopicasnocontiguas}).

\begin{obsLimiteDirectoDeSubdivisiones}\label{obs:limitedirectodesubdivisiones}
	Sea $K_{1}$ un complejo simplicial y sea $L_{1}$ un subcomplejo. Por
	el corolario \ref{thm:caracterizaciondeaproximacionesporsubdivision},
	existen aproximaciones simpliciales
	\begin{align*}
		\varphi & \,:\,(\subdiv{K_{1}},\subdiv{L_{1}})\,\rightarrow\,
			(K_{1},L_{1})
	\end{align*}
	%
	de la identidad
	\begin{align*}
		& (\geom{\subdiv{K_{1}}},\geom{\subdiv{L_{1}}})\,\rightarrow\,
			(\geom{K_{1}},\geom{L_{1}})
		\text{ .}
	\end{align*}
	%
	Dos aproximaciones de la identidad son, por
	\ref{thm:aproximacionessoncontiguas}, contiguas. Teniendo en cuenta
	esto, si
	\begin{math}
		\lambda:\,(\subdiv{K_{1}},\subdiv{L_{1}})\rightarrow
			(K_{1},L_{1})
	\end{math}
	es una aproximaci\'{o}n simplicial de la identidad y
	$\varphi:\,(K_{1},L_{1})\rightarrow (K_{2},L_{2})$ es una
	transformaci\'{o}n simplicial, la aplicaci\'{o}n
	\begin{align*}
		\subdiv{\clase{\varphi}} & \,=\,\clase{\varphi\circ\lambda}
			\,=\,\clase{\varphi}\circ\clase{\lambda}
	\end{align*}
	%
	est\'{a} bien definida y define una funci\'{o}n
	\begin{align*}
		\subdiv{\null} & \,:\,
			\contiguas{K_{1},L_{1}}{K_{2},L_{2}}\,\rightarrow\,
			\contiguas{\subdiv{K_{1}},\subdiv{L_{1}}}{K_{2},L_{2}}
		\text{ .}
	\end{align*}
	%
	Iterando este procedimiento, se obtiene una sucesi\'{o}n
	\begin{align*}
		\subdiv[n,n+1]{\null} & \,:\,
			\contiguas{\subdiv[n]{K_{1}},\subdiv[n]{L_{1}}}%
				{K_{2},L_{2}}\,\rightarrow\,
			\contiguas{\subdiv[n+1]{K_{1}},\subdiv[n+1]{L_{1}}}%
				{K_{2},L_{2}}
	\end{align*}
	%
	para $n\geq 0$. Para cada entero no negativo, elegimos alguna
	aproximaci\'{o}n
	\begin{align*}
		\lambda_{n+1,n} & \,:\,
			(\subdiv[n+1]{K_{1}},\subdiv[n+1]{L_{1}})
			\,\rightarrow\,
			(\subdiv[n]{K_{1}},\subdiv[n]{L_{1}})
	\end{align*}
	%
	de la identidad
	\begin{align*}
		\id & \,:\,
			(\geom{\subdiv[n+1]{K_{1}}},\geom{\subdiv[n+1]{L_{1}}})
			\,\rightarrow\,
			(\geom{\subdiv[n]{K_{1}}},\geom{\subdiv[n]{L_{1}}})
		\text{ .}
	\end{align*}
	%
	Entonces
	\begin{align*}
		\subdiv[n,n+1]{\clase{\varphi}} & \,=\,
			\clase{\varphi\circ\lambda_{n+1,n}}\,=\,
			\clase{\varphi}\circ\clase{\lambda_{n+1,n}} \,=\,
			\clase{\lambda_{n+1,n}}^{*}\big(\clase{\varphi}\big)
		\text{ .}
	\end{align*}
	%
	Notemos que, como $\lambda_{n+1,n}$ es una aproximaci\'{o}n
	simplicial de la identidad, la realizaci\'{o}n es homot\'{o}pica a
	la identidad: $\geom{\lambda_{n+1,n}}\simeq\id$. En particular, las
	clases de homotop\'{\i}a
	\begin{align*}
		\clase{\geom{\varphi\circ\lambda_{n+1,n}}} & \,=\,
			\clase{\geom{\varphi}}
	\end{align*}
	%
	coinciden.

	Dados $m\geq n$, sea
	\begin{align*}
		\lambda_{m,n} & \,:\,(\subdiv[m]{K_{1}},\subdiv[m]{L_{1}})
			\,\rightarrow\,(\subdiv[n]{K_{1}},\subdiv[n]{L_{1}})
	\end{align*}
	%
	la composici\'{o}n
	\begin{align*}
		\lambda_{m,n} & \,=\,\lambda_{n+1,n}\circ\lambda_{n+2,n+1}
			\circ\cdots\circ\lambda_{m,m-1}
	\end{align*}
	%
	y sea
	\begin{align*}
		\subdiv[n,m]{\null} & \,:\,
			\contiguas{\subdiv[m]{K_{1}},\subdiv[m]{L_{1}}}%
				{K_{2},L_{2}} \,\rightarrow\,
			\contiguas{\subdiv[n]{K_{1}},\subdiv[n]{L_{1}}}%
				{K_{2},L_{2}}
	\end{align*}
	%
	la composici\'{o}n
	\begin{align*}
		\subdiv[n,m]{\null} & \,=\,\subdiv[n,n+1]{\null}\circ
			\subdiv[n+1,n+2]{\null}\circ\cdots\circ
			\subdiv[m-1,m]{\null}
		\text{ .}
	\end{align*}
	%
	Entonces, por un lado,
	\begin{align*}
		\subdiv[n,m]{\clase{\varphi}} & \,=\,
			\clase{\varphi\circ\lambda_{m,n}}
		\text{ .}
	\end{align*}
	%
	Por otro lado, como $\lambda_{m,n}$ es una aproximaci\'{o}n de la
	identidad, $\geom{\lambda_{m,n}}\simeq\id$. Si
	\begin{math}
		\varphi:\,(\subdiv[n]{K_{1}},\subdiv[n]{L_{1}})\rightarrow
			(K_{2},L_{2})
	\end{math}~,
	entonces
	\begin{align*}
		\clase{\geom{\varphi\circ\lambda_{m,n}}} & \,=\,
			\clase{\geom{\varphi}}
		\text{ .}
	\end{align*}
	%
	
	Tomando el l\'{\i}mite directo, se obtiene un funtor
	\begin{align*}
		& \lim_{\to}\,\contiguas{\subdiv[n]{K_{1}},\subdiv[n]{L_{1}}}%
				{K_{2},L_{2}}
		\text{ ,}
	\end{align*}
	%
	contravariante en $(K_{1},L_{1})$ y covariante en $(K_{2},L_{2})$.
\end{obsLimiteDirectoDeSubdivisiones}

\begin{teoHomotopicasPorContiguas}\label{thm:homotopicasporcontiguas}
	Sea $K_{1}$ un complejo simplicial finito. Entonces existe una
	isomorfismo natural
	\begin{align*}
		\lim_{\to}\,\contiguas{\subdiv[n]{K_{1}},\subdiv[n]{L_{1}}}%
				{K_{2},L_{2}} & \,\simeq\,
			\homotopicas{\geom{K_{1}},\geom{L_{1}}}%
				{\geom{K_{2}},\geom{L_{2}}}
		\text{ .}
	\end{align*}
	%
\end{teoHomotopicasPorContiguas}

\begin{proof}
	Para definir una transformaci\'{o}n natural, es necesario definir,
	para cada par de subcomplejos $(K_{1},L_{1})$ y $(K_{2},L_{2})$,
	una funci\'{o}n del l\'{\i}mite directo en el conjunto de clases
	de homotop\'{\i}a de funciones de
	$(\geom{K_{1}},\geom{L_{1}})$ en $(\geom{K_{2}},\geom{L_{2}})$. Una
	funci\'{o}n definida en el l\'{\i}mite directo, equivale a una
	sucesi\'{o}n de funciones
	\begin{align*}
		f_{n} & \,:\,\contiguas{\subdiv[n]{K_{1}},\subdiv[n]{L_{1}}}%
				{K_{2},L_{2}}\,\rightarrow\,
			\homotopicas{\geom{K_{1}},\geom{L_{1}}}%
				{\geom{K_{2}},\geom{L_{2}}}
		\text{ ,}
	\end{align*}
	%
	para $n\geq 0$, tales que
	\begin{align*}
		f_{n} & \,=\,f_{n+1}\circ\subdiv[n,n+1]{\null}
		\text{ .}
	\end{align*}
	%
	Definimos $f_{n}$ como la funci\'{o}n
	\begin{align*}
		f_{n}\big(\clase{\varphi}\big) & \,=\,\clase{\geom{\varphi}}
		\text{ ,}
	\end{align*}
	%
	si
	\begin{math}
		\varphi:\,(\subdiv[n]{K_{1}},\subdiv[n]{L_{1}})\rightarrow
			(K_{2},L_{2})
	\end{math}~.
	Esta funci\'{o}n est\'{a} bien definida, por la observaci\'{o}n
	\ref{obs:contiguassonhomotopicas}. Entonces, por la observaci\'{o}n
	\ref{obs:limitedirectodesubdivisiones},
	\begin{align*}
		f_{n+1}\circ\subdiv[n,n+1]{\null}\big(\clase{\varphi}\big)
			& \,=\,\clase{\geom{\varphi\circ\lambda_{n+1,n}}}
			\,=\,\clase{\geom{\varphi}}\,=\,
			f_{n}\big(\clase{\varphi}\big)
		\text{ .}
	\end{align*}
	%
	Las funciones $f_{n}$ son naturales en el par $(K_{1},L_{1})$:
	dada $\psi_{1}:\,(K_{1},L_{1})\rightarrow (K'_{1},L'_{1})$,
	el diagrama
	\begin{center}
		\begin{tikzcd}
			\contiguas{\subdiv[n]{K'_{1}},\subdiv[n]{L'_{1}}}%
				{K_{2},L_{2}} \arrow[r,"f_{n}"]
					\arrow[d,"\clase{\psi}^{*}"'] &
			\homotopicas{\geom{K'_{1}},\geom{L'_{1}}}%
				{\geom{K_{2}},\geom{L_{2}}}
					\arrow[d,"\clase{\geom{\psi}}^{*}"] \\
			\contiguas{\subdiv[n]{K_{1}},\subdiv[n]{L_{1}}}%
				{K_{2},L_{2}} \arrow[r,"f_{n}"'] &
			\homotopicas{\geom{K_{1}},\geom{L_{1}}}%
				{\geom{K_{2}},\geom{L_{2}}}
		\end{tikzcd}
	\end{center}
	conmuta: dada $\varphi'$ una representante de una clase de
	contig\"{u}idad en el extremo superior izquierdo,
	\begin{align*}
		f_{n}\circ\clase{\psi}^{*}\big(\clase{\varphi'}\big) & \,=\,
			f_{n}\big(\clase{\varphi'\circ\psi}\big) \,=\,
			\clase{\geom{\varphi'\circ\psi}}
		\quad\text{y} \\
		\clase{\geom{\psi}}^{*}\circ f_{n}\big(\clase{\varphi'}\big)
			& \,=\,\clase{\geom{\psi}}^{*}
				\big(\clase{\geom{\varphi'}}\big) \,=\,
			\clase{\geom{\varphi'}}\circ\clase{\geom{\psi}}
		\text{ .}
	\end{align*}
	%
	De manera similar, se puede ver que la definici\'{o}n de $f_{n}$ es
	natural en $(K_{2},L_{2})$. Queda determinada, entonces, una
	transformaci\'{o}n natural
	\begin{align*}
		f & \,=\, \{f_{n}\}_{n\geq 0} \,:\,
			\lim_{\to}\,
			\contiguas{\subdiv[n]{K_{1}},\subdiv[n]{L_{1}}}%
				{K_{2},L_{2}} \,\rightarrow\,
			\homotopicas{\geom{K_{1}},\geom{L_{1}}}%
				{\geom{K_{2}},\geom{L_{2}}}
		\text{ .}
	\end{align*}
	%

	Veamos que $f$ es una biyecci\'{o}n natural. Sea
	\begin{math}
		g:\,(\geom{K_{1}},\geom{L_{1}})\rightarrow
			(\geom{K_{2}},\geom{L_{2}})
	\end{math}
	una funci\'{o}n continua y sea
	\begin{math}
		\varphi:\,(\subdiv[n]{K_{1}},\subdiv[n]{L_{1}})\rightarrow
			(K_{2},L_{2})
	\end{math}
	una aproximaci\'{o}n simplicial de $g$, cuya existencia est\'{a}
	garantizada por el teorema \ref{thm:existenciadeaproximaciones}.
	Entonces
	\begin{align*}
		f_{n}\big(\clase{\varphi}\big) & \,=\,\clase{\geom{\varphi}}	
			\,=\,\clase{g}
		\text{ .}
	\end{align*}
	%
	En particular, $f$ es sobreyectiva. Sean ahora
	\begin{align*}
		\varphi & \,:\,(\subdiv[n]{K_{1}},\subdiv[n]{L_{1}})
			\,\rightarrow\,(K_{2},L_{2}) \\
		\varphi' & \,:\,(\subdiv[n']{K_{1}},\subdiv[n']{L_{1}})
			\,\rightarrow\,(K_{2},L_{2})
	\end{align*}
	%
	transformaciones simpliciales tales que
	$\geom{\varphi}\simeq\geom{\varphi'}$, es decir,
	\begin{align*}
		f_{n}\big(\clase{\varphi}\big) & \,=\,
			\clase{\geom{\varphi}} \,=\,
			\clase{\geom{\varphi'}} \,=\,
			f_{n'}\big(\clase{\varphi'}\big)
	\end{align*}
	%
	Por el teorema \ref{thm:homotopicasadmitenaproximacionescontiguas},
	existe $m\geq n,n'$ y aproximaciones
	\begin{align*}
		\psi,\psi' & \,:\,(\subdiv[m]{K_{1}},\subdiv[m]{L_{1}})
			\,\rightarrow\,(K_{2},L_{2})
	\end{align*}
	%
	de $\varphi$ y, respectivamente, de $\varphi'$ que pertenecen a la
	misma clase de contig\"{u}idad \textit{\{revisar la demo del teo\}}.
	Ahora bien, como $\varphi$ es una aproximaci\'{o}n de $\geom{\varphi}$
	y $\lambda_{m,n}$ es una aproximaci\'{o}n de la identidad, la
	composici\'{o}n $\varphi\circ\lambda_{m,n}$ es, tambi\'{e}n, seg\'{u}n
	la observaci\'{o}n \ref{obs:composiciondeaproximacionesesaproximacion},
	una aproximaci\'{o}n simplicial de $\geom{\varphi}$. En particular,
	por el lema \ref{thm:aproximacionessoncontiguas}, las transformaciones
	$\varphi\circ\lambda_{m,n}$ y $\psi$ son contiguas. An\'{a}logamente,
	$\varphi'\circ\lambda_{m,n'}$ y $\psi'$ tambi\'{e}n son contiguas.
	Pero entonces
	\begin{align*}
		\subdiv[m-n]{\clase{\varphi}} & \,=\,
			\clase{\varphi\circ\lambda_{m,n}} \,=\,
			\clase{\varphi'\circ\lambda_{m,n'}} \,=\,
			\subdiv[m-n']{\clase{\varphi'}}
	\end{align*}
	%
	en $\contiguas{\subdiv[m]{K_{1}},\subdiv[m]{L_{1}}}{K_{2},L_{2}}$.
	Concluimos, as\'{\i}, que $f$ es inyectiva.
\end{proof}

\begin{coroCodominioNumerableHomotopicasNumerables}%
	\label{thm:codominionumerablehomotpicasnumerables}
	Sea $X$ un espacio topol\'{o}gico compacto y sea $Y$ el espacio de
	un complejo simplicial a lo sumo numerable. Sean $A\subset X$ y
	$B\subset Y$ subespacios. Entonces el conjunto de clases de
	homotop\'{\i}a de pares $\homotopicas{X,A}{Y,B}$ es a lo sumo
	numerable.
\end{coroCodominioNumerableHomotopicasNumerables}

%
\section{Ejemplos}
\theoremstyle{definition}
\newtheorem{ejemploVacio}{Ejemplo}[section]
\newtheorem{ejemploSubconjuntosFinitos}[ejemploVacio]{Ejemplo}
\newtheorem{ejemploCarasDeUnSimplice}[ejemploVacio]{Ejemplo}
\newtheorem{ejemploCarasPropias}[ejemploVacio]{Ejemplo}
\newtheorem{ejemploQEsqueleto}[ejemploVacio]{Ejemplo}
\newtheorem{ejemploNervioDeUnaFamilia}[ejemploVacio]{Ejemplo}
\newtheorem{ejemploSumaDeComplejos}[ejemploVacio]{Ejemplo}
\newtheorem{ejemploSegmentosEnteros}[ejemploVacio]{Ejemplo}
\newtheorem{ejemploReticulado}[ejemploVacio]{Ejemplo}
\newtheorem{ejemploSubcomplejoDeCaras}[ejemploVacio]{Ejemplo}
\newtheorem{ejemploSubcomplejosUnionInterseccion}[ejemploVacio]{Ejemplo}
\newtheorem{ejemploSubcomplejoDelNervio}[ejemploVacio]{Ejemplo}
\newtheorem{ejemploBolaYEsfera}[ejemploVacio]{Ejemplo}
\newtheorem{ejemploAproximacionesEnElCirculo}[ejemploVacio]{Ejemplo}
\newtheorem{ejemploAproximacionesEnElCirculoCont}[ejemploVacio]{Ejemplo}
\newtheorem{ejemploFuncionesNulhomotopicasEntreEsferas}[ejemploVacio]{Ejemplo}
\newtheorem{ejemploHomotopicasNoContiguas}[ejemploVacio]{Ejemplo}

%-------------

\begin{ejemploVacio}[El complejo vac\'{\i}o]%
	\label{ejemplo:vacio}
	El conjunto vac\'{\i}o visto como un conjunto vac\'{\i}o de
	s\'{\i}mplices es un complejo. Lo denotamos $\varnothing$ y lo
	denominamos \emph{complejo vac\'{\i}o}.
\end{ejemploVacio}

\begin{ejemploSubconjuntosFinitos}[El complejo de subconjuntos finitos]%
	\label{ejemplo:subconjuntosfinitos}
	Dado un conjunto $A$, el conjunto de subconjuntos finitos
	no vac\'{\i}os de $A$ constituye un complejo.
\end{ejemploSubconjuntosFinitos}

\begin{ejemploCarasDeUnSimplice}[Las caras de un s\'{\i}mplice]%
	\label{ejemplo:carasdeunsimplice}
	Sea $K$ un complejo simplicial y sea $s\in K$ un s\'{\i}mplice.
	El conjunto de caras de $s$ constituye un complejo, \emph{el %
	complejo de caras de $s$}, denotado $\caras{s}$.
\end{ejemploCarasDeUnSimplice}

\begin{ejemploCarasPropias}[Las caras propias]%
	\label{ejemplo:caraspropias}
	El conjunto de caras propias de un s\'{\i}mplice $s$ tambi\'{e}n
	constituye un complejo, lo denotamos $\carasp{s}$.
\end{ejemploCarasPropias}

\begin{ejemploQEsqueleto}[El $q$-esqueleto de un complejo]%
	\label{ejemplo:qesqueleto}
	Sea $K$ un complejo simplicial. El \emph{$q$-esqueleto de $K$}
	es el complejo, denotado $\qesq{q}{K}$, cuyos s\'{\i}mplices son todos
	los $p$-s\'{\i}mplices de $K$ con $p\leq q$. El $q$-esqueleto de
	un complejo $K$ es un subcomplejo de $K$.
\end{ejemploQEsqueleto}

\begin{ejemploNervioDeUnaFamilia}[El nervio de una familia de subconjuntos]%
	\label{ejemplo:nerviodeunafamilia}
	Sea $X$ un conjunto y sea $\cal{W}$ una familia de subconjuntos
	de $X$. El \emph{nervio de $\cal{W}$}, denotado $\nerv{\cal{W}}$,
	es el complejo simplicial cuyos s\'{\i}mplices son los subconjuntos
	finitos de $\cal{W}$ cuya intersecci\'{o}n sea no vac\'{\i}a.
	En particular, los v\'{e}rtices de $\nerv{\cal{W}}$ son exactamente
	los elementos de $\cal{W}$.
\end{ejemploNervioDeUnaFamilia}

\begin{ejemploSumaDeComplejos}[La suma de dos complejos]%
	\label{ejemplo:sumadecomplejos}
	Sean $K_{1},K_{2}$ dos complejos simpliciales. La \emph{suma de %
	$K_{1}$ con $K_{2}$}, denotada $K_{1}*K_{2}$, es el complejo
	simplicial cuyos s\'{\i}mplices son los s\'{\i}mplices del
	complejo $K_{1}$, los de $K_{2}$ y las uniones disjuntas de
	un s\'{\i}mplice de $K_{1}$ con uno de $K_{2}$. De esta manera,
	el conjunto de v\'{e}rtices de $K_{1}*K_{2}$ es igual a la uni\'{o}n
	disjunta del conjunto de v\'{e}rtices de $K_{1}$ con el de los de
	$K_{2}$. En s\'{\i}mbolos,
	\begin{align*}
		K_{1}*K_{2} & \,=\,K_{1}\,\sqcup\,K_{2}\,\cup\,
			\big\{s_{1}\sqcup s_{2}\,:\,s_{1}\in K_{1},\,
						s_{2}\in K_{2}\big\}
		\text{ .}
	\end{align*}
	%
\end{ejemploSumaDeComplejos}

\begin{ejemploSegmentosEnteros}[Segmentos enteros]%
	\label{ejemplo:segmentosenteros}
	Sea $V=\bb{Z}$ y sea $K$ la familia
	\begin{align*}
		K & \,=\,\big\{\{n\}\,:\,n\in\bb{Z}\big\}\,\cup\,
			\big\{\{n,n+1\}\,:\,n\in\bb{Z}\big\}
		\text{ .}
	\end{align*}
	%
	Entonces $K$ es un complejo simplicial cuyos v\'{e}rtices son
	los n\'{u}meros enteros y cuyos $1$-s\'{\i}mplices son los
	intervalos enteros $\{n,n+1\}$. El complejo $K$ no posee
	s\'{\i}mplices de mayor dimensi\'{o}n.
\end{ejemploSegmentosEnteros}

\begin{ejemploReticulado}[Puntos en un reticulado]%
	\label{ejemplo:reticulado}
	Sea $n\geq 1$ un entero fijo. Consideramos el conjunto de $n$-tuplas
	de enteros $\bb{Z}^{n}$ con el orden parcial dado por comparar
	las coordenadas: $(\lista*{x}{n})\leq (\lista*{y}{n})$, si
	$x^{i}\leq y^{i}$ para todo $i=1,\,\dots,\,n$. Sea $K$ la familia de
	conjuntos finitos no vac\'{\i}os y totalmente ordenados de
	$\bb{Z}^{n}$, $s=\{x_{0}\leq x_{1}\leq\dots\leq x_{q}\}$, que
	verifican $x_{q}^{i}-x_{0}^{i}=0\text{ o }1$ para todo
	$i=1,\,\dots,\,n$. Entonces $K$ es un complejo simplicial cuyo
	conjunto de v\'{e}rtices es $V=\bb{Z}^{n}$.
\end{ejemploReticulado}

\begin{ejemploSubcomplejoDeCaras}\label{ejemplo:subcomplejodecaras}
	Dado un complejo $K$ y un s\'{\i}mplice $s\in K$,
	el complejo de caras $\caras{s}\subset K$ es un subcomplejo de $K$
	y el complejo de caras propias $\carasp{s}\subset K$, tambi\'{e}n lo
	es. Tambi\'{e}n vale que $\carasp{s}\subset\caras{s}$ es un
	subcomplejo.
\end{ejemploSubcomplejoDeCaras}

\begin{ejemploSubcomplejosUnionInterseccion}
	\label{ejemplo:subcomplejosunioninterseccion}
	Dada una familia de subcomplejos $\{L_{i}\}_{i}$ de un complejo
	$K$, la uni\'{o}n (conjunt\'{\i}stica de los conjnuntos de
	s\'{\i}mplices) $\bigcup_{i}\,L_{i}$ y la intersecci\'{o}n (lo
	mismo) $\bigcap_{i}\,L_{i}$ son subcomplejos de $K$.
\end{ejemploSubcomplejosUnionInterseccion}

\begin{ejemploSubcomplejoDelNervio}\label{ejemplo:subcomplejodelnervio}
	Sea $X$ un conjunto, $A\subset X$ un subconjunto y sea
	$\cal{W}$ una familia de subconjuntos de $X$. Sea $\nerv{\cal{W}}$
	el nervio de la familia $\cal{W}$. Entonces, seg\'{u}n el ejemplo
	\ref{ejemplo:nerviodeunafamilia},
	\begin{align*}
		\nerv{\cal{W}} & \,=\,
			\Big\{ \{\lista{W}{r}\}\,:\,
				W_{i}\in\cal{W},\,
				W_{1}\cap\cdots\cap W_{r}\not=\varnothing
			\Big\}
		\text{ .}
	\end{align*}
	%
	Sea, entonces, $\nerv[A]{\cal{W}}$ el subconjunto de $\nerv{\cal{W}}$
	dado por
	\begin{align*}
		\nerv[A]{\cal{W}} & \,=\,
			\Big\{ \{\lista{W}{r}\}\,:\,
				W_{i}\in\cal{W},\,
				A\cap\big(W_{1}\cap\cdots\cap W_{r}\big)
					\not=\varnothing
			\Big\}
		\text{ .}
	\end{align*}
	%
	Entonces $\nerv[A]{\cal{W}}$ es un subcomplejo de $\nerv{\cal{W}}$.
\end{ejemploSubcomplejoDelNervio}

\begin{ejemploBolaYEsfera}[La bola y la esfera]\label{ejemplo:bolayesfera}
	Sea $n\geq 1$, sea $\mathrm{B}^{n+1}$ la bola unitaria en
	$\bb{R}^{n+1}$ y sea $\esfera{n}$ la esfera unitaria. Si $s$ es un
	$n+1$-s\'{\i}mplice (en alg\'{u}n complejo), existe un homeomorfismo
	entre el par $(\mathrm{B}^{n+1},\esfera{n})$ y el par
	$(\geom{\caras{s}},\geom{\carasp{s}})$.
\end{ejemploBolaYEsfera}

\begin{ejemploSegmentosEnteros}\label{ejemplo:segmentosenterostriangulacion}
	Sea $K$ el complejo del ejemplo \ref{ejemplo:segmentosenteros} y sea
	$f:\,\geom{K}\rightarrow\bb{R}$ una funci\'{o}n tal que
	$f|_{\geom{\{n\}}}=n$ y $f|_{\geom{\{n,n+1\}}}$ sea un homeo con el
	intervalo $[n,n+1]\subset\bb{R}$. Entonces $f$ es una triangulaci\'{o}n
	de $\bb{R}$.
\end{ejemploSegmentosEnteros}

\begin{ejemploReticulado}\label{ejemplo:reticuladotriangulacion}
	Sea $K$ el complejo definido en el ejemplo \ref{ejemplo:reticulado}.
	Sea $f:\,\geom{K}\rightarrow\bb{R}^{n}$ la funci\'{o}n
	$f(\alpha)^{i}=\sum_{x\in\bb{Z}^{n}}\,\alpha(x)\,x^{i}$.
	Entonces $f$ es una traingulaci\'{o}n\dots
\end{ejemploReticulado}

\begin{ejemploAproximacionesEnElCirculo}%
	\label{ejemplo:aproximacionesenelcirculo}
	Sea $s$ un $2$-s\'{\i}mplice y sea $\carasp{s}$ el complejo de sus
	caras propias. Entonces $\geom{\carasp{s}}$ es homeomorfo a
	$\esfera{1}$. En particular, las clases homotop\'{\i}a de funciones
	$\geom{\carasp{s}}\rightarrow\geom{\carasp{s}}$ son infinitas.
	Pero, para cada $n\geq 0$ entero no negativo, existen a lo sumo
	finitas transformaciones simpliciales
	$\subdiv[n]{\carasp{s}}\rightarrow\carasp{s}$. En
	consecuencia, fijado $n$, existen funciones
	$\geom{\carasp{s}}\rightarrow\geom{\carasp{s}}$ que no admiten
	aproximaciones definidas en $\subdiv[n]{\carasp{s}}$.
\end{ejemploAproximacionesEnElCirculo}

\begin{ejemploAproximacionesEnElCirculoCont}%
	\label{ejemplo:aproximacionesenelcirculocont}
	Sean $s$ y $\carasp{s}$ como en el ejemplo
	\ref{ejemplo:aproximacionesenelcirculo}. Sean $v_{0},\,v_{1},\,v_{2}$
	los v\'{e}rtices de $\carasp{s}$ (de $s$). Sean
	\begin{align*}
		v_{0} \,=\,\bari{v_{0}} & \quad\text{,}\quad
		v_{1} \,=\,\bari{v_{1}} \quad\text{y}\quad
		v_{2} \,=\,\bari{v_{2}} \text{ ,} \\
		w_{0} \,=\,\bari{\{v_{2},\,v_{0}\}} & \quad\text{,}\quad
		w_{1} \,=\,\bari{\{v_{0},\,v_{1}\}} \quad\text{y}\quad
		w_{2} \,=\,\bari{\{v_{1},\,v_{2}\}}
	\end{align*}
	%
	los baricentros del complejo y sea
	$f:\,\geom{\carasp{s}}\rightarrow\geom{\carasp{s}}$ la funci\'{o}n
	lineal dada por
	\begin{align*}
		v_{0}\,\mapsto\,w_{1}\,\mapsto\,v_{1}\,\mapsto\,w_{2}
			\,\mapsto\,v_{2}\,\mapsto\,w_{0}\,\mapsto\,v_{0}
	\end{align*}
	%
	en los v\'{e}rtices de $\subdiv{\carasp{s}}$. Entonces $f$ es
	homot\'{o}pica a la identidad de $\geom{\carasp{s}}$, pero
	no admite una aproximaci\'{o}n simplicial de la forma
	$\carasp{s}\rightarrow\carasp{s}$. Sin embargo, existen exactamente
	ocho aproximaciones de la forma
	$\subdiv{\carasp{s}}\rightarrow\carasp{s}$ determinadas por lo que
	valen en los v\'{e}rtices $v_{0},\,v_{1},\,v_{2}$.

	En cuanto a la existencia de la homotop\'{\i}a, una posibilidad es
	desandar gradualmente la misma funci\'{o}n $f$. En cuanto a la
	segunda afirmaci\'{o}n, si $\varphi:\,\carasp{s}\rightarrow\carasp{s}$
	es una transformaci\'{o}n simplicial, entonces
	$f(v_{0})=w_{1}$ implica que $\geom{\varphi}v_{0}$ pertenece
	a la cara generada por $v_{0}$ y $v_{1}$. Como $\varphi(v_{0})$ es
	un v\'{e}rtice,
	\begin{align*}
		\varphi(v_{0}) & \,\in\,\{v_{0},v_{1}\}
		\text{ .}
	\end{align*}
	%
	An\'{a}logamente,
	\begin{align*}
		\varphi(v_{1}) & \,\in\,\{v_{1},v_{2}\}\quad\text{y} \\
		\varphi(v_{2}) & \,\in\,\{v_{2},v_{0}\}
		\text{ .}
	\end{align*}
	%
	Notemos que esta observaci\'{o}n tambi\'{e}n es v\'{a}lida si el
	dominio de $\varphi$ es cualquier subdivisi\'{o}n de $\carasp{s}$.
	Por otro lado, $w_{1}=\frac{1}{2}\,v_{0}+\frac{1}{2}\,v_{1}$ implica
	\begin{align*}
		\geom{\varphi}w_{1}& \,=\,
			\frac{1}{2}\,\varphi(v_{0})
			+ \frac{1}{2}\,\varphi(v_{1})
	\end{align*}
	%
	y $f(w_{1})=v_{1}$ implica que $\geom{\varphi}w_{1}=v_{1}$,
	tambi\'{e}n. Esto fuerza
	\begin{align*}
		\varphi(v_{0}) & \,=\,\varphi(v_{1})=v_{1}
		\text{ .}
	\end{align*}
	%
	Pero, de manera similar, se deduce que debe cumplirse
	$\varphi(v_{2})=\varphi(v_{0})=v_{0}$. Lo que es absurdo.

	En cuanto a la existencia de las aproximaciones en
	$\subdiv{\carasp{s}}$, sabemos, por lo visto en el p\'{a}rrafo
	anterior, que de existir una aproximaci\'{o}n
	$\varphi:\,\subdiv{\carasp{s}}\rightarrow\carasp{s}$ para $f$,
	$\varphi(v_{i})\in\{v_{i},v_{(i+1\mod 3)}\}$, hay dos opciones para
	$\varphi(v_{i})$, para cada $i=0,1,2$. La diferencia con el caso
	anterior es que los baricentros no se escriben como
	combinaciones propias de v\'{e}rtices de $\subdiv{\carasp{s}}$,
	\emph{son} v\'{e}rtices del complejo subdividido. Al igual que antes,
	$f(w_{i})=v_{i}$, con lo que $\varphi(w_{i})$ est\'{a} forzada a
	tomar el valor $v_{i}$, para cada $i=0,1,2$. Esto no impone condiciones
	sobre los valores de $\varphi$ en los $v_{i}$ y, cualquiera sea la
	elecci\'{o}n de dichos valores queda determinada una transformaci\'{o}n
	simplicial $\varphi:\,\subdiv{\carasp{s}}\rightarrow\carasp{s}$ que
	es una aproximaci\'{o}n simplicial de $f$.
\end{ejemploAproximacionesEnElCirculoCont}

\begin{ejemploFuncionesNulhomotopicasEntreEsferas}%
	\label{ejemplo:funcionesnulhomotopicasentreesferas}
	Sea $n\geq 1$ y sea $m<n$. Toda funci\'{o}n
	$\esfera{m}\rightarrow\esfera{n}$ es homot\'{o}pica a una constante.
	Sea $s_{1}$ un $(m+1)$-s\'{\i}mplice y sea $s_{2}$ un
	$(n+1)$-s\'{\i}mplice. Entonces $\esfera{m}$ es homeomorfa a
	$\geom{\carasp{s_{1}}}$ y $\esfera{n}$ es homeomorfa a
	$\geom{\carasp{s_{2}}}$. Sea
	$f:\,\geom{\carasp{s_{1}}}\rightarrow\geom{\carasp{s_{1}}}$ una
	funci\'{o}n continua. Porque $\esfera{m}$ es compacta, el teorema
	de existencia de aproximaciones, \ref{thm:existenciadeaproximaciones},
	implica que para $i$ suficientemente grande, existe una
	aproximaci\'{o}n simplicial
	$\varphi:\,\subdiv[i]{\carasp{s_{1}}}\rightarrow\carasp{s_{2}}$ de
	$f$ y, por el lema \ref{thm:aproximacioneshomotopica},
	$\geom{\varphi}\simeq f$. Entonces, para demostrar que toda $f$ es
	homot\'{o}pica a una constante, ser\'{a} suficiente probar que, para
	toda aproximaci\'{o}n $\varphi$, la funci\'{o}n $\geom{\varphi}$ lo es.

	Como la dimensi\'{o}n del complejo $\subdiv[i]{\carasp{s_{1}}}$ es
	$m$, la imagen por una transformaci\'{o}n simplicial $\varphi$ en
	$\carasp{s_{2}}$ est\'{a} incluida en el $m$-esqueleto de
	$\carasp{s_{2}}$. En particular, como $m<n$, existe alg\'{u}n punto
	$\alpha\in\geom{\carasp{s_{2}}}$ que no pertenece a la imagen
	$\geom{\varphi}\big(\geom{\subdiv[i]{\carasp{s_{1}}}}\big)$. Pero
	esto implica que $\geom{\varphi}$ tiene imagen en el espacio
	$\geom{\carasp{s_{2}}}\setmin\{\alpha\}$, que es homeomorfo a la
	esfera $\esfera{n}$ sin un punto, que, a su vez, es homeomorfa a
	$\bb{R}^{n}$, que es contr\'{a}ctil. En definitiva, $\geom{\varphi}$
	es homot\'{o}pica a una constante.
\end{ejemploFuncionesNulhomotopicasEntreEsferas}

Un espacio topol\'{o}gico $X$ se dice \emph{$n$-conexo} ($n\geq 0$), si toda
funci\'{o}n continua $f:\,\esfera{k}\rightarrow X$ ($k\leq n$) se puede
extender de manera continua a una funci\'{o}n definida en la bola
$\disco{k+1}$. El ejemplo anterior muestra que la esfera $\esfera{n}$ es
$(n-1)$-conexa. En particular, si $n>1$, entonces $\esfera{n}$ es simplemente
conexa. Se deduce entonces que toda funci\'{o}n continua
$f:\,\esfera{n}\rightarrow\esfera{1}$ ($n>1$) se factoriza por el revestimiento
$\exp{\null}:\,\bb{R}\rightarrow\esfera{1}$. Como $\bb{R}$ es contr\'{a}ctil,
concluimos que toda funci\'{o}n continua de $f$ debe ser homot\'{o}pica a una
constante.

\begin{ejemploHomotopicasNoContiguas}\label{ejemplo:homotopicasnocontiguas}
	Sea $K_{1}=K_{2}$ el complejo del ejemplo
	\ref{ejemplo:segmentosenteros}. Entonces
	$\geom{K_{1}}=\geom{K_{2}}=\bb{R}$. Sea
	$\varphi:\,K_{1}\rightarrow K_{2}$ la identidad de complejos
	simpliciales y sea $\varphi':\,K_{1}\rightarrow K_{2}$ la
	transformaci\'{o}n constante dada por $\varphi'(n)=0$ en todo
	v\'{e}rtice $n\in\bb{Z}$ de $K_{1}$. Como el espacio del complejo
	$K_{2}$, $\bb{R}$, es contr\'{a}ctil, vale que
	$\geom{\varphi}\simeq\geom{\varphi'}$. Ahora bien, si $K_{1}'$ es una
	subdivisi\'{o}n de $K_{1}$ y $\psi,\psi':\,K_{1}'\rightarrow K_{2}$
	son transformaciones simpliciales tales que $\psi$ es una
	aproximaci\'{o}n de $\geom{\varphi}$ (la identidad) y $\psi'$ es una
	aproximaci\'{o}n de $\geom{\varphi'}$ (la funci\'{o}n constante), por
	un lado, en v\'{e}rtices, $\psi:\,\bb{Z}\rightarrow\bb{Z}$ es
	suryectiva y $\psi':\,\bb{Z}\rightarrow\bb{Z}$ es constante
	($\psi'(n)=0$ para todo $n\in\bb{Z}$). Pero, por otro lado, si
	$\psi$ y $\psi'$ fuesen contiguas, los conjuntos
	$\{\psi(n),\psi'(n)\}$ deber\'{\i}an ser s\'{\i}mplices de $K_{2}$.
	En particular --y esto es cierto en general, no s\'{o}lo en este
	ejemplo--, la imagen del conjunto de v\'{e}rtices por $\psi$
	deber\'{\i}a ser finita, si y\'{o}lo si la imagen por $\psi'$ lo
	fuese. En este caso, sin embargo, una es infinita numerable (es
	sobreyectiva) y la otra es finita (consiste en un \'{u}nico punto).
	Por lo tanto, $\psi$ y $\psi'$ no pueden pertenecer a la misma
	clase de contig\"{u}idad.
\end{ejemploHomotopicasNoContiguas}

%
%

%--------

\begin{thebibliography}{9}
\bibitem{SpanierAlgTop}
Spanier; \textit{Algebraic Topology}


\end{thebibliography}

\end{document}
